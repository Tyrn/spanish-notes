\chapter{Our Fellow Human Beings}

As we go about our daily lives, we are often put to the task of
describing our fellow human beings. And the vast range of personalities these humans represent requires an equally vast vocabulary of descriptive terms. Rare is it indeed to encounter a person whose behavior
can be wrapped up in so simple a concept as ``good" or ``evil." Instead,
human behavior runs the gamut from charming to wicked, honest to
malicious, overbearing to submissive, and happy to forlorn---sometimes even on the same day. For each of these many states and traits,
you will need words in Spanish.

Of course, learning all of the epithets employed to describe human personalities would be the intellectual equivalent of memorizing
a thesaurus. You could do that, naturally, but since you'll be wanting
to save a little brain space for such things as nouns and verbs, it's better to concentrate on a few common ways of referring to other individuals. Thus what follows is by no means an exhaustive list of all Spanish words describing people but rather a collection of some of the most
useful ones---ones you will probably hear, and may even be called, in
the company of Spanish speakers.

Where appropriate, rough English equivalents are provided.
The translations serve more to highlight the Spanish word's relative
strength and connotations than to provide a literal translation. As in
any language, you should choose your words carefully when referring
to other people. When in doubt, a good rule is to use them for the first
time when the person being described is not within hearing range.
Outside of a language-learning context, of course, this is called gossip.
But since you're still learning, it's allowed.

\section{The good}

Unless you are at a particularly low point in a mood swing,
you will probably concur that when all is said and done, there are still
a few good people out there. And with luck and perseverance, you may
even meet one of them. When you do, it's important to be ready with
the appropriate verbal match for your smile and that warm feeling in
your belly.

A good way to convey this sensation is by telling a person you
like them. Conveying this in Spanish, though, requires careful attention to differences in degree and intensity of the words. If you guide
yourself by the dictionary, for instance, you will think that \emph{querer}
means ``to like." This is true---up to a point---but what a dictionary
often doesn't tell you is just how strong the ``liking" reflected by
\emph{querer} really is. The distinction is important, since saying \emph{te quiero} to
someone whom you just ``like" could provoke a considerably stronger
reaction than you bargained for. \emph{Te quiero} means ``I love you"---in
some contexts even ``I want you"---and is virtually synonymous with
the magic words \emph{Te amo}. Suppose someone asks, ``What do you think
of my boyfriend?" If you try to say ``I like him" with a response like \emph{Lo
quiero}, expect to be misunderstood: what you are really saying is ``I
want him," as in ``I want him for myself." Such slips of the tongue
could prove fatal.

So how do you say ``Hey, you're not such a bad bloke" in Spanish? The simplest formula throughout the Spanish-speaking world is to
use \emph{caer bien}---literally ``to fall well." \emph{Ese señor me cae bien} is the common, colloquial, and non-intimate way of saying ``I like that fellow." It
is equally safe in direct speech. \emph{¿Sabes? me caes bien} = ``You know, I
like you." This useful phrase will never fail to get your message across
loudly, clearly, and without confusion and misinterpretation.

But, you say, my dictionary also says to use gustar for ``to
like," as in \emph{Me gustan las tortillas}. True---up to that point, again. Gustar with people is a slightly different matter. \emph{Me gusta Paco} does mean
``I like Paco," but it carries many of the connotations and commitments implied in \emph{Quiero a Paco}. ``I fancy Paco" might be a good English equivalent. If you're male and you say \emph{Me gusta Paco}, expect to
receive funny looks. \emph{Paco me cae bien} is presumably what you want
to say.

As long as you're looking through the dictionary, you may also
discover \emph{agradar}. This verb does work for ``I like you" (\emph{Me agradas})
without the romantic overtones, but it can sound a little forced or
formal.

Presumably your descriptions of others will occasionally go
beyond the fact that you like them or dislike them. You'll want to
describe, for the benefit of your listener, what kind of person Juan or
Juanita is. If Juan is a ``nice guy," for instance, you could say \emph{es un
buen tipo, es buena persona}, or \emph{es buena gente} (closer to ``he's good
people"). These work with Juanita as well, with one important exception: when referring to women, note that \emph{tipa} is derogatory. Thus \emph{Es
una buena tipa} sounds inherently contradictory---kind of like saying
``She's a nice bitch"---and should be avoided. Regional expressions
of general approval abound. In Mexican slang a likable person will often be described by saying simply \emph{Es buena onda}, roughly ``He's (or
she's) cool."

Once you've mastered how to say you like a person, you'll
want to delve deeper and learn words for specific personality traits. For
the purposes of learning some of these, we'll divide positive personality traits into four entirely arbitrary groups: the \emph{amables}, or ``nice"
people; the \emph{simpáticos}, or ``cool" people; the \emph{listos}, or ``sharp" people;
and the \emph{serios}, or ``solid" people. Try to learn at least one description
from each, and you'll be well on your way to explaining why exactly a
given person ``falls well to you."

Note as you go that all of these words, unless otherwise stated,
should be used with \emph{ser}. In fact, using some of these words with \emph{estar}
can lead to confusion and may change the meaning altogether. Where
this is the case, it is noted.

\section{The \emph{amables}}

Let's face it: some people are just downright nice. God knows
how they got that way or how they manage to stay that way, but the
fact is that they are kind, warm, and caring. And---why not admit
it?---we like them. On the other hand, they may never become our best
friends, and our dealings with them may never transcend the simply
social. In Spanish such a person could be called \emph{amable}, equating with
``kind." \emph{Amable} is a useful, generic, and somewhat bland word, perhaps most common in the stock phrase \emph{gracias, muy amable}. \emph{Gentil}
is used in much the same way, though it rings more formal. Note also
that \emph{gentil} is a trickster (see Chapter 3); it can mean ``gentle," but it often covers much more.

Here are other commonly used favorable descriptions that fall
roughly in this category:

\subsection{\emph{generoso}}

%GENEROSO
Works safely as ``generous."

\subsection{\emph{bondadoso}}

%BONDADOSO
Comes closer to ``charitable" and ``giving."

\subsection{\emph{desprendido}}

%DESPRENDIDO
A nice, multisyllabic word for ``generous" in
the sense of ``disinterested."

\subsection{\emph{atento}}

%ATENTO
Means ``thoughtful."

\subsection{\emph{detallista}}

%DETALLISTA
Means ``thoughtful," too, but goes further and is
used for the sort of people who send thank---you notes and tasteful, personalized gifts. It's the same in the masculine and feminine.

\subsection{\emph{cortés}}

%CORTÉS
Means ``courteous" or ``polite." A common maxim
in Spanish is \emph{Lo cortés no quita lo valiente}, which, very loosely translated, means something like ``Real men can be polite, too."

\subsection{\emph{una persona considerada}}

%UNA PERSONA CONSIDERADA
A very safe cognate for ``a considerate person."

\subsection{\emph{una persona comprensiva}}

%UNA PERSONA COMPRENSIVA
Similar to \emph{una persona considerada} but distinct, referring more to a very ``understanding" or ``compassionate" person.

\subsection{\emph{dulce}}

%DULCE
Means ``sweet," a good word for some people.

\subsection{\emph{es un alma de Dios} and \emph{es un pan de Dios}}

%ES UN ALMA DE DIOS and ES UN PAN DE DIOS
Expressions
for extremely good-hearted people, of the sort who have never had a
harsh word for anyone.

\bsk

Also falling into the \emph{amable} category are a number of words
and phrases typical of polite society:

\subsection{\emph{una fina persona}}

%UNA FINA PERSONA
Suggests a kind, considerate person.
Turned around, \emph{una persona fina} is a well-bred, ``fine" person.

\subsection{\emph{educado}}

%EDUCADO
Often used much like \emph{una persona fina}. Remember in passing that \emph{educado} has a far broader meaning in Spanish than
``educated" does in English (see Chapter 2). Sometimes \emph{educado} comes
closer to ``classy" or ``a class act" in English.

\subsection{\emph{una persona culta}}

%UNA PERSONA CULTA
Goes a step further, describing a dignified, tasteful individual.

\section{The \emph{simpáticos}}

This is the category that would include your best friends, and
the most descriptive word for them is \emph{simpático}. This word is also a
trickster, looking like ``sympathetic" but meaning something quite different. A single English word doesn't really do it justice; probably you
would say something like ``He (or she) is great!" The younger set might
find ``cool" a close equivalent. Easier than translating the epithet is
imagining the sort of person who would be worthy of it: a happy,
friendly, attractive, charming, witty, altogether likable person.

About the only quirk to be noted about \emph{simpático} is its use
with the verb \emph{estar}, especially in reference to members of the opposite
sex and babies. In these contexts \emph{está simpático} can sound like a backhanded compliment. For babies, it would be like saying, ``What an
interesting-looking child." For members of the opposite sex, especially
``eligible" ones about your own age, it's on a par with saying ``Well, he
(or she) certainly has a good personality." Perhaps a parallel English
word would be ``cute," in cases when it's understood that you're deliberately not using a more flattering word. If you're trying to be nice, use
simpatico with \emph{ser}.

If you want to pin down a \emph{simpático}'s personality further,
there is a wealth of words at your disposal:

\subsection{\emph{alegre}}

%ALEGRE
Covers ``happy," ``happy-go-lucky," ``outgoing," and
the like.

\subsection{\emph{amigable}}

%AMIGABLE
A safe cognate for ``friendly," but the Spanish
word is less common and more specific.

\subsection{\emph{atractivo}}

%ATRACTIVO
A straightforward cognate of ``attractive." Like
the English word, it stresses the person's physical charms.

\subsection{\emph{encantador}}

%ENCANTADOR
A close fit for ``charming."

\subsection{\emph{relajado}}

%RELAJADO
Describes a simpático who is ``relaxed" or ``laid-back."

\subsection{\emph{sociable}}

%SOCIABLE
Means ``sociable" and is used to describe the outgoing, partying type.

\subsection{\emph{genial}}

%GENIAL
A useful word, though it almost ranks as a trickster.
It usually means ``clever" or ``great" when referring to things or ideas,
and ``a character" or ``a wild-and-crazy guy (or gal)" when referring to
people. It is usually a favorable assessment. For ``genial," rely on \emph{amable} or \emph{amigable}.

\section{The \emph{listos}}

Some people rank high in our esteem because they are notably
``clever," ``sharp," or ``bright." Calling them that presupposes some intelligence on their part, but it's not the same thing as calling them
``smart" or ``intelligent." ``Witty" might come closer, in some cases. A
catch-all Spanish word for these people is \emph{listo}, used with \emph{ser} (with \emph{estar} it means ``ready"). This is the word for people who appeal to you on
an intellectual level, people who enlighten, challenge, and entertain,
people you generally like to have around.
Not all bright and witty people are likable, of course, but the
ones who are can merit additional words:

\subsection{\emph{gracioso} and \emph{chistoso}}

%GRACIOSO and CHISTOSO
Stress the humorous aspect of the
person's personality.

\subsection{\emph{astuto} and \emph{ágil}}

%ASTUTO and ÁGIL
Emphasize sheer mental acumen.

\subsection{\emph{despierto}}

%DESPIERTO
A close fit for ``bright," suggesting one part ``brilliant" and one part ``bright-eyed and bushy-tailed," ``lively," ``energetic." Used with \emph{estar}, of course, \emph{despierto} simply means ``awake."

\section{The \emph{serios}}

Some people we like simply because they seem to be ``good
people." They are honest, forthright, hard-working, and guileless. What
you see is what you get. These people tend to be our most esteemed
co-workers and our counselors in times of trouble, and the best word
for them in Spanish is \emph{serio}, used, as usual, with \emph{ser}. \emph{Es una persona
seria} = ``He (or she) is a real straight-shooter" or ``He (or she) has his
(or her) act together" or even is a ``together person." \emph{Serio} does not
necessarily convey the idea of a droopy-faced soul or a rigid, unyielding
sort (although with \emph{estar} it can). Instead, it refers to a reliable, trustworthy,solid individual. \emph{Formal} means much the same, but is more, well, formal.

Since \emph{serio} is often used to describe the ideal qualities in a
professional colleague, many words in this category pop up frequently
in reference to the workplace:

\subsection{\emph{capaz}}

%CAPAZ
Means ``capable." (Remember that \emph{capable} is a dangerous trickster; see Chapter 3.)

\subsection{\emph{trabajador}}

%TRABAJADOR
Means ``hard-working."

\subsection{\emph{responsable} and \emph{dedicado}}

%RESPONSABLE and DEDICADO
These are our old friends ``responsible" and ``dedicated."

\subsection{\emph{cumplido}}

%CUMPLIDO
An extremely useful word for ``reliable" or ``competent." Saying \emph{Es muy cumplida} about someone means ``She gets the job done."

\bsk

Some other qualifiers are slippery and should be used with care:

\subsection{\emph{muy vivo}}

%MUY VIVO
Sounds very complimentary---especially considering the alternative---but this somewhat slangy expression hints that
the person is ``clever," ``nobody's fool," and perhaps even a tad ``shady,"
``underhanded" or ``crooked." \emph{Despierto}, noted above, is a safer choice.

\subsection{\emph{movido}}

%MOVIDO
Suffers the same fate as \emph{muy vivo}. It refers to a very
active person---a real ``go-getter"---but on the downside it can imply
that you're never really sure where this person is ``moving," with
whom, or according to whose rules.

\bsk

Other ``serious" words stress moral rectitude and solidness of
judgment:

\subsection{\emph{recto}}

%RECTO
Highlights the person's basic decency and honesty.

\subsection{\emph{honesto}}

%HONESTO
Includes ``honest" but often covers a wider range
of moral traits; ``decent" might be a better translation. For simply
``honest," \emph{honrado} is more accurate.

\subsection{\emph{una persona de confianza}}

%UNA PERSONA DE CONFIANZA
A common way to say a person can be trusted or confided in, either in the sense of personal honesty or competence.

\subsection{\emph{sensato}}

%SENSATO
Means ``sensible" and is a good word to describe
someone's sound judgment and solid character.

\section{The bad}

All of us know that if you don't have anything nice to say
about a person, you shouldn't say anything at all. None of us practices
this belief, though, which is why this section is needed. For unless you
lead an extraordinarily blessed existence, you will probably find yourself in need of one or more of the following words sooner or later. After
all, even if you don't take pleasure in verbally thrashing your fellow
human being, you'll still need a few good words to describe those
who do.

Spanish, alas, is a rich language for belittling others. The range
of words, phrases, and expressions that can be employed to this end is
practically infinite and, furthermore, varies widely from region to region. Not all of these epithets fall within the range of dignified abuse,
of course, and some are so harsh and vulgar that they can reflect as
poorly on the person using them as on the person for whom they are
intended. For crude, harsh, and undignified abuse, flip ahead to
Chapter 10.

In this section, we'll concern ourselves with descriptions for
people whom you simply don't like. These people, in Spanish, can be
said to ``fall badly to you"; \emph{te caen mal}. The converse of \emph{caer bien},
\emph{caer mal} is the safest and most universal way of expressing dislike in
Spanish. A watered-down version is \emph{no caer bien}. \emph{Pedro no me cae
bien} = ``I not too fond of Pedro." A ``watered-up" version is \emph{no caer
nada bien}. \emph{Pedro no me cae nada bien} = ``I don't like Pedro at all." In
many countries, local and colloquial modifications of \emph{caer mal} have
been invented. In Mexico it is common to hear that so-and-so \emph{me cae
gordo}: ``falls fat to me," literally. \emph{Me cae pesado}, or ``he falls heavy to
me," is the same idea. Remember to use the correct gender depending
on who is ``falling to you": \emph{él me cae pesado} but \emph{ella me cae pesada}.

As when expressing your likes, you are on shaky ground using
the verb \emph{querer} to convey dislike. \emph{No lo quiero} means ``I don't love
him" or ``I don't want him"---leaving a lot to the imagination of the
listener. \emph{Gustar} is also dubious. \emph{No me gusta} suggests ``He (or she) is
not for me" or ``is not my type."

To capture what it is about a person that ``falls badly" to you,
you will need to enter the world of words for negative personality
traits. Again, we can divide them into four arbitrary groups: the \emph{pesados}, or ``obnoxious" people; the \emph{imbéciles}, or ``jerks"; the \emph{malvados},
or ``mean" people; and the \emph{cochinos}, or ``slobs." Remember that while
these words are less likely overall to provoke a brawl than their
four-lettered cousins in Chapter 10, you should still use them advisedly.
Pay attention to such intangibles as tone and situation. The same word
said lazily to a friend can offend or inflame a stranger. Unless otherwise noted, these epithets are best used with \emph{ser}.

\section{The \emph{pesados}}

This group covers the obnoxious boor, and \emph{pesado} is the word
we've chosen to carry the banner. \emph{Una persona pesada} can be anyone
from a nagger to an intolerable snob. What all \emph{pesados} have in common is that they get on our nerves after a while---in many cases after
a very short while. They are not really malicious; they're just underequipped in the personality department.

\emph{Pesados} tell crude jokes when no one wants to hear them,
make repeated passes at the same woman, use foreign words that nobody understands, drop names, sneer at your wardrobe, flaunt theirs,
gossip a lot, argue about everything, and never accept their mistakes.
Thank God we're not like them! In a lighter vein, you can use pesado
among friends or family to tell someone to behave themselves or to
``lighten up." \emph{No seas pesado}, said without much conviction, carries
the same message as ``Don't be a pain."

Other words explore the \emph{pesado} phenomenon in more detail:

\subsection{\emph{pedante}}

%PEDANTE
For intellectually snobbish people (the ones who
use those foreign words). When we were kids, we called these people
``know-it-alls." Some people may refrain from correcting your Spanish
for fear of appearing \emph{pedante}. The word can be generalized to cover
anyone who has a higher opinion of himself or herself than others do.
``Stuck-up" comes to mind in English.

\subsection{\emph{presumido}}

%PRESUMIDO
The word you've been searching for to say
``show-off." Kids who stick their neat toys in your face but don't let
you play with them are \emph{presumidos}. Ditto for adults with fancy cars.

\subsection{\emph{prepotente}}

%PREPOTENTE
Implies a powerful person who abuses his or
her authority to the detriment of others. It's a good word for cops,
judges, politicians, and bosses, providing they don't overhear you. A
good English equivalent doesn't really exist.

\subsection{\emph{arrogante}}

%ARROGANTE
Safe for ``arrogant," ``excessively proud."

\subsection{\emph{vanidoso}}

%VANIDOSO
Means ``vain" but is more commonly used than
its English equivalent.

\subsection{\emph{snob} or \emph{esnob}}

%SNOB or ESNOB
This term, for better or worse, is creeping
into Spanish. You may hear it, but try to use a Spanish equivalent to
avoid being taken for one yourself.

\subsection{\emph{creerse}}

%CREERSE
This verb, along with an appropriate adjective, will
deflate a pretentious person in a hurry. Someone who tells bad jokes
\emph{se cree chistoso}; someone who wears fancy clothes \emph{se cree elegante};
someone who uses foreign phrases \emph{se cree inteligente}; and so on. \emph{Se
cree mucho}, by itself, covers the lot of these people. In other words,
they believe themselves to be ``great shakes," but this opinion is not
widely held.

\bsk

Regional words and slang expressions to cover snootiness are
frequent and often the most colorful. In Mexico \emph{sangrón} is widely if
somewhat slangily used to convey ``obnoxious." In Mexico, too, you
can say of a snob, \emph{Le echa mucha crema a sus tacos}, or ``He (or she)
puts a lot of cream on his (or her) tacos." Once I heard, in reference to
a particularly stuck-up young man, that \emph{se cree la última Pepsi del
desierto} (``he thinks he's the last Pepsi in the desert"). Keep your ears
open, wherever you are, and you should be able to add fun new descriptions to the stock you acquire here.

\section{The \emph{imbéciles}}

As a rule, none of us likes stupid people. It's not a matter of
their I.Q.---in fact, they may be quite intelligent---but we call them
``stupid" nonetheless. These are people who do stupid things and make
stupid comments. Sometimes they make us wonder how on earth anyone can be so downright, undeniably, irretrievably STUPID!
When you reach that point in your regard of a fellow human
being, the word you want is \emph{imbécil}. In truth, \emph{estúpido} exists and is
widely used as well, but it lacks the punch that \emph{imbécil} packs. Say it
with heavy, accentuated, spitting scorn on the middle syllable: \emph{im-BE-cil}. Now throw your hand in the air as you say it, as if casting this
person from the planet: \emph{¡im-BE-cil!} Isn't this fun?
\emph{Imbécil}, like ``stupid," doesn't necessarily mean that the object of your scorn is unintelligent. This is a good thing, because it
means you can use it on an even wider range of people: college graduates, college professors, or even college presidents, if you like. An English equivalent might be ``jerk" or perhaps ``stupid jerk" or even ``You
stupid jerk!" depending on the amount of spit and scorn you want to
add. It is a multipurpose, and strong, pejorative. I once watched as an
unseemly man pestered a young, well-dressed blonde woman on the
subway in a major Latin American city; taking her for a tourist, he
kept asking her ``Whey a you from?" in barely pidgin English. Finally
the woman got fed up with this harassment, turned on the man, and
spit \emph{De aquí, ¡imbécil!} in his face. The man slinked off, humbled, and
the crowd was delighted. \emph{Estúpido} just wouldn't have worked.
The \emph{imbécil} category includes a number of other useful descriptions for those who make a distinctly bad impression on us:

\subsection{\emph{idiota}}

%IDIOTA
Similar to \emph{imbécil} but perhaps a shade weaker. Still,
native English speakers should be warned that this word is not used
nearly as casually in Spanish as in English---as in ``Oh, don't be an
idiot" or ``What a blunder! I feel like an idiot!" Same in masculine and
feminine: \emph{él es un idiota, ella es una idiota}.

\subsection{\emph{infeliz}}

%INFELIZ
Less a ``jerk" and more a ``klutz" or a ``schmuck." It
is used to describe a sort of hapless clod whose luck is mostly bad, at
least in part because of a lack of willpower to make it better. An \emph{infeliz}
is not necessarily a scoundrel or even an entirely unpleasant person,
but you probably wouldn't want to have one as a friend. To soften the
blow, qualify it with \emph{pobre}. \emph{Un pobre infeliz} = ``a poor sap," ``a loser," a real ``sad sack."

\subsection{\emph{tonto}}

%TONTO
Works as ``silly" or a watered-down ``fool."

\subsection{\emph{payaso}}

%PAYASO
Means ``clown" and is a stronger word for ``fool,"
particularly a clumsy, oafish one.

\subsection{\emph{baboso}}

%BABOSO
Less harsh than payaso but no less clear in its implications (it comes from \emph{babear}, ``to slobber").

\subsection{\emph{tarado}}

%TARADO
A handy word suggesting that something crucial to
cerebral functioning may be missing; ``moron" might fit.

\subsection{\emph{tarugo}}

%TARUGO
A term that means ``a block of wood." Applied to
people, it equates nicely with ``blockhead."

\subsection{\emph{burro}}

%BURRO
Meaning ``donkey," the word calls to mind ``dumbbell" or ``dunce" and, like those words, is a favorite among schoolkids.

\subsection{\emph{bobo}}

%BOBO
In the same class as burro, suggesting ``dummy."

\subsection{\emph{ignorante} and \emph{cretino}}

%IGNORANTE and CRETINO
Safe cognates that are heard on occasion.

\subsection{\emph{simple}}

%SIMPLE
Mostly a false cognate, especially when talking
about people; it means ``simple-minded" or ``simpleton" more than
``simple." If ``simpleton" is the word you seek, \emph{simplón} is even more
expressive.

\subsection{\emph{torpe}}

%TORPE
Means ``clumsy oaf" or ``klutz."

\subsection{\emph{necio}}

%NECIO
A useful word that refers more to stubbornness than
stupidity, though the two often go hand in hand; ``jackass" may be a
good equivalent.

\section{The \emph{malvados}}

In this group we lump people whom we consider mean, offensive, malicious, or simply ``bad" people. \emph{Malvado} is a good catchall to
describe these people, especially if they have done us some harm. \emph{Maldito} is stronger, though without being crudely so, and is invariably the
word chosen in subtitled films to translate ``bastard." (In Spanish \emph{bastardo} is usually reserved for persons born out of wedlock and thus
works poorly as a general-purpose insult.) Here are some other useful
descriptions of this type:

\subsection{\emph{mal parido} and \emph{mal nacido}}

%MAL PARIDO and MAL NACIDO
Meaning literally ``born bad,"
these phrases convey the same ignominy as \emph{malvado} but sound a bit
snazzier.

\subsection{\emph{sinvergüenza}}

%SINVERGÜENZA
Literally meaning ``without shame" or
``scoundrel," the word suggests that this person actually enjoys being
offensive. Very commonly used. Same form for masculine and feminine.

\subsection{\emph{canalla}}

%CANALLA
Refers generally to men, often a man who mistreats women. It works well for ``lout." 1I Same form for masculine and
feminine.

\subsection{\emph{desgraciado}}

%DESGRACIADO
A common term of generic opprobrium, generally meaning an unpleasant person who has tried to do us some
wrong. It also implicitly tries to write the person off as trivial and unworthy of genuine scorn. It equates fairly well with ``wretch" or ``sap"
in English.

\subsection{\emph{infame}}

%INFAME
A good strong word used to describe someone who,
by his or her actions, has earned everlasting infamy---at least in our
eyes. Murderous tyrants are \emph{infames}; so, by extension (and slight exaggeration), is anyone who causes you genuine harm with malicious
intent.

\subsection{\emph{mentiroso}}

%MENTIROSO
A strong word, stronger even than ``liar." It suggests the person is a habitual liar and uses lies to defame and gain advantage. Be careful using it.

\subsection{\emph{maléfico}}

%MALÉFICO
Describes a real evil sort, a pernicious ``ne'er-do-well."

\subsection{\emph{malviviente}}

%MALVIVIENTE
Literally meaning ``bad-liver" (one who lives
badly), the word suggests an incorrigible rogue with a long police
record.

\subsection{\emph{delincuente}}

%DELINCUENTE
Much the same as \emph{malviviente}, the term
means ``delinquent" but stresses the criminal lifestyle that such a person has chosen. As a result of increasing drug use and the crime that
goes with it, words like \emph{mariguano} are coming into vogue to mean the
same thing. To an English speaker the word calls to mind a fading hippie, but in Spanish it describes a dangerous ``druggie" and is becoming
the common word for scandalous and vandalous teenagers.

\section{The \emph{cochinos}}

For uncouth, classless slobs, \emph{cochino} is the word. It is one of
four common Spanish words for ``pig" (the animal)---\emph{cerdo, puerco},
and \emph{marrano} are the others---and all work well for the human equivalent. If you dislike someone because he or she throws trash out of car
windows, takes up three train seats, or burps at the dinner table, this is
the category you should check. (You should be aware that the word
\emph{marrano} has a long anti-Semitic history, though it is rarely used that
way today.)

\subsection{\emph{grosero}}

%GROSERO
The catchall word for ``rude." \emph{Se portó bastante
grosero} would be the common way of saying ``He acted quite rude,"
``He treated us badly."

\subsection{\emph{vago}}

%VAGO
Describes an unkempt person or ``bum." (It is also the
word for ``vague"; see Chapter 3.)

\subsection{\emph{bajo}}

%BAJO
Widely used in the phrase \emph{un tipo de lo más bajo}, a
strong denunciation that translates well as ``a low-life," ``creep," or
even ``scum." Note, however, that \emph{Es un tipo bajo} usually just means
``He's a short guy."

\bsk

A number of other common words refer to general ``low-class"
behavior. Though most modern-day English speakers are not accustomed to think of behavioral traits in class terms, in much of the
Spanish-speaking world the class distinction is still a strong one.
Words like \emph{corriente, vulgar}, and \emph{ordinario} denote little more than
``common" or ``characteristic of the masses," but all three connote
``uncouth," ``slobbish," or ``classless" (as in ``he has no class") when
used to refer to people. \emph{Un hombre corriente} can be expected to meet
his dinner guests unshaven, in boxer shorts and scratching his paunch.
\emph{Inculto} falls into this group as well. It means ``uncultured" but is
much broader and more frequently heard than its English cognate; an
\emph{inculto} not only doesn't appreciate Mozart but plays his radio too loud.

\section{The indifferent}

We have looked at the human race from both sides now. We
have met the lowest of the low, the snobs and scoundrels, and learned
what to call them. We have also been introduced to the creme de la
creme, the well-bred gentlefolk, and have had a few charmed words
with them. But what about the rest of the world's inhabitants---that
gray, faceless mass that hovers uncertainly between good and evil?
What do you say to describe a ``nowhere type," a fair-to-middling sort,
a person who's nothing to write home about and just like the next guy,
only less so? What do you say about people when there isn't much to
say about them one way or the other?

A number of unspectacularly adequate Spanish words rise
humbly to the task. \emph{Regular, normal}, and \emph{común} all make the grade,
though you should be alert to their idiosyncrasies. \emph{Regular} with \emph{estar},
for example, is a notch or two below ``regular" on the scale of descriptive tags; whereas in English it means ``average," in Spanish it is closer
to ``fair" or even ``not so hot," especially when referring to objects.
\emph{Esta regular el camino} does not mean ``It's an average road" but rather
``It's a pretty lousy road." ``Regular" in the sense of ``consistent" is
\emph{regular} with \emph{ser}. Be alert to the distinction: \emph{Es un cliente regular}
would be ``He's a steady customer," while \emph{Ese cliente está regular} suggests the customer is a lousy tipper or otherwise ``fair." Using \emph{asiduo}
instead of \emph{regular} to describe the steady customer can avoid a potential
misunderstanding and win you brownie points for using a good Spanish word.

\emph{Normal} is the best all-around word for the ``nothing-special"
category. \emph{Mediocre} carries a clear connotation of insufficiency, as does
its cognate in English. Other words---specifically \emph{corriente} and \emph{ordinario}---are listed in dictionaries as synonyms for ``normal" or ``ordinary,"
but in fact can be charged with negative meaning when used to label
people, as noted above. \emph{Común} works in the expression \emph{un hombre común}, meaning ``common folk" or ``the man on the street." \emph{Común y
corriente} is a good translation for ``run-of-the-mill," without the negative connotations of \emph{corriente} by itself. Still, your safest bet in almost
every circumstance involving people is normal.

In situations where physical appearance is being commented
on, a different set of words takes over. A person whose looks are ``nothing special" could be described as \emph{pasadero} or \emph{pasable}, meaning ``acceptable" or ``tolerable" in the sense of ``if nothing better comes
along." A useful, if regional, slang phrase for ``not bad" (or ``not great")
is \emph{dos tres}; it works equally well for people, places, and things and
sounds, to my ear, a lot better than \emph{así así}, which every textbook will
tell you means ``so-so." \emph{Más o menos} (technically, ``more or less") is
also commonly used in Spanish, even when there's no indication of
what is being compared: \emph{¿Qué tal la película? Más o menos}. = ``How
was the film?" ``Not bad." Sometimes in slang \emph{más o menos} gets
shortened to \emph{ma' o meno'} in this usage.

Some fun expressions for saying ``nothing special" employ neither-nor constructions. A Peruvian, for instance, might say someone is
\emph{ni chicha ni limoná}, where \emph{chicha} is a local alcoholic drink and \emph{limoná} is ``lemonade." The idea is that the person is neither alcoholic
nor nonalcoholic-that is, isn't anything well-defined. ``Neither here
nor there" comes out as \emph{ni de aquí ni de allá}. \emph{Ni pinta ni da color}
means that someone (or something) ``neither paints nor adds color,"
suggesting that there is no real reason for this person or thing to exist
at all.

Another common way of saying ``no great shakes" is \emph{nada del
otro mundo} (literally, ``nothing from the other world". Another handy
phrase is \emph{Es cosa de cada domingo} (``It's an every-Sunday thing"),
equating with ``a dime a dozen" and showing that you are distinctly
unimpressed. \emph{No es nada fuera de lo común} is good but a bit stilted
for ``It's nothing out of the ordinary." When you are completely unimpressed by something or someone, but still unwilling to turn it or
them down, resort to \emph{peor es nada}---the correct rendering for ``better
than nothing."

\section{Temporary states}

Since we've spent so much of this chapter name-calling, it's
only decent of us to finish by noting that most bad traits are only
temporary and that most people, if we'd only give them time and get
to know them, would soon return to being their real sweet, lovable
selves. With this is mind, you'll need the proper vocabulary to describe
these warm and caring individuals who just happen to be acting like
miserable worms at the moment.

One way to get this across is by using \emph{estar}, the Verb of Temporary States, but not simply as a substitute for \emph{ser}. Instead, by inserting \emph{de} between estar and your adjective, you change ``He's an S.O.B.
(or whatever)!" to ``He's being an S.O.B.!" Thus \emph{Es un grosero} describes
a permanently rude, coarse, and unpleasant man; \emph{Está de grosero} describes a man who is acting rudely. \emph{Es un presumido} = ``He's a showoff"; \emph{Está de presumido} = ``He's showing off."

This construction is especially useful for capturing moods. For
instance, \emph{Está de pesado} = ``He's in a lousy mood." Two other key
phrases for moods employ the generic terms \emph{buenas} and \emph{malas}. \emph{Está
de buenas} means someone is ``in a good mood." \emph{Está de malas} is the
dreaded opposite.

\emph{Estar de} can also be used instead of \emph{ser} for ironic effect. For
instance, \emph{Está de generoso} suggests that someone who usually isn't
generous is for some reason suddenly giving away the store. Saying
that a child \emph{está de obediente} suggests that the child's obedient behavior is indeed most unusual. \emph{Estar de} can likewise be used to explain
away behavior as a temporary aberration: \emph{Estoy de tarado} = ``I'm acting a little stupid." It should be noted that some adjectives by their
very meaning resist the \emph{estar de} construction; it is hard, for instance,
to imagine someone being \emph{infame} for only a few minutes or so.


%%% Local Variables:
%%% mode: latex
%%% TeX-master: "main-keenan-breaking-out"
%%% End:
