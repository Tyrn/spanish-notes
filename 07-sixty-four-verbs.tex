\chapter{Sixty-four Verbs, Up Close and Personal}

Pocket dictionaries will generally give you simple, one-word
equivalents for Spanish verbs. Better dictionaries will give you a list of
other possible meanings and maybe some examples. But rarely is the
dictionary reader given guidance on what usages are common---i.e.,
worth the bother of learning---and which are poetic or archaic and thus
irrelevant. And besides, who wants to read a dictionary?

What follows is a pared-down list of sixty-four Spanish verbs
whose basic meaning you probably already know, but whose inner secrets and common usage go far beyond that. The list cuts through the
clutter and highlights unexpected usages that a native English speaker
may not be on the lookout for or that are ``out of character" for a given
verb. Each section also explains a chosen few idiomatic expressions,
selected for their frequency in everyday spoken Spanish. There's a lot
to absorb here, but the alphabetical listing will allow for hour after
hour of repeated consultation. Take heart---it could be worse. You
could be reading the dictionary!

\section{\emph{acabar}}

This is a synonym for \emph{terminar} and, like that word, is a good
equivalent for most uses of ``to end" and ``to finish" in English. Used
with \emph{con} as an intensifier, both verbs work well for ``to finish off":
\emph{Acabé con la leche}. Often you will hear \emph{acabarse} in the reflexive to
mean ``to run out of." \emph{Se nos acabó el dinero} = ``We've run out of
money," \emph{Se acabo} by itself, meanwhile, means either ``It's over" or
``I'm out of it," You'll hear it a lot in stores, at newsstands, and the like
when what you want to buy is no longer in stock. \emph{Terminar} is likewise
used this way; \emph{agotar} is also heard, especially in the phrase \emph{Está agotado} (``We're out of it"). \emph{Está acabado} usually isn't used in this sense,
since its meaning is closer to ``It's finished" or, colloquially, ``He's
washed up." \emph{Acabar} has one other very common use that \emph{terminar}
doesn't have: in the present tense, with \emph{de}, it means ``to have just,"
as in Acabo de comer (``I've just eaten"). Used in the imperfect, it becomes ``had just." \emph{Acababa de comer cuando llegaste} = ``I had just
eaten when you arrived."

\section{\emph{amanecer}}

This somewhat uncommon word, which means ``to dawn," is
included here because of one expression that has perplexed generations
of language students, especially those who live for a time with a
Spanish-speaking family when studying abroad. The expression is
\emph{¿Cómo amaneciste?}---which translates literally as ``How did you
dawn?" but which means ``How did you sleep?" The answer is (usually) \emph{Bien, gracias}. A fun Spanish expression for ``to die in one's sleep"
uses this verb: \emph{Amaneció muerto}; if it translates at all, it would have
to be rendered ``He woke up all dead."

\section{\emph{andar}}

Dictionaries say it means ``to walk," which of course it does,
but that won't help you when you hear \emph{anda corriendo} or \emph{anda en
coche} for the first time. In English we would probably be inclined
to say ``go around" for most uses of \emph{andar}. \emph{Pedro anda gritando tu
nombre} = ``Pedro's going around shouting your name." \emph{Andar} also
covers slangy expressions like ``to hang out" or ``to hang around." \emph{Ya
no ando con ellos} = ``I don't hang around with them anymore." \emph{¿Por
dónde andas?} works well for ``Whereabouts are you?" or the colloquial
``Where are you at?" And in some countries \emph{anda} lends itself to the
common idiomatic expressions \emph{¡Ándale!} and \emph{¡Anda!} Said with vigor,
they mean ``Let's get a move on!" or ``Way to go!" Said in passing, they
mean the same as ``okay" or ``all right." In Mexico, for instance, you'll
hear \emph{ándale} all the time for ``that's fine," ``that's right," and even
``good-bye": \emph{Nos vemos mañana. Ándale}. Throw a \emph{pues} on the end
and you'll be saying nothing at all but will sound very fluent: \emph{Ándale
pues} (``Have a nice day"). Remember as well the use of \emph{andar} for ``to
run" or ``to work" in reference to objects. \emph{¿Qué tal anda tu coche?} =
``How's your car running?" (literally, ``How's your car walking?").
Some wags have even argued that the different conception of time in
Spanish-speaking countries is due to the fact that in Spanish clocks
walk rather than run!

\section{\emph{antojar}}

Used generally as a reflexive (\emph{antojarse}), this is an exceptionally common verb and one that you should get to know well. To translate it, dictionaries offer ``to long for" or ``to desire earnestly," but its
use in Spanish covers a lot more ground than that. Closer to the mark
would be ``to get a hankering for" or simply ``to feel like." An \emph{antojo}
is a ``craving" or an ``urge" and covers both intense longings (like the
kind that make pregnant women eat pickles) and simpler pleasures.
You'll probably run across the verb frequently in these same situations. Some examples: \emph{Se me antoja una pizza} = ``I'm dying for (could
go for) a pizza." \emph{¿Por qué no vas a ir al cine? Porque no se me antoja}.
= ``Why aren't you going to the movies?" ``Because I don't feel like it."
\emph{Déjame una dona; luego se me va a antojar} = ``Leave me a doughnut;
I'll probably feel like having one later."

\section{\emph{bajar}}

This verb means ``to go down," ``to put down," ``to get off,"
and so on. Most of its uses are predictable, but a few that may not be
include ``to go downstairs," ``to get out of a car (bus, train, etc.)," and
``to lose weight" (\emph{bajar de peso}). It also means ``to get (something)
down," as when you ask someone to get your suitcase down off the
rack (\emph{¿Me baja la maleta, por favor?}). \emph{Bájale}, by itself, is usually ``Turn
it down," referring to the volume or the general noise level; in the
right context, it can also mean ``Slow down."

\section{\emph{caber}}

An irregular verb that you should learn. It means ``to fit," but
only in the sense of ``fit into" or ``fit onto." It is not used for clothing.
In the first-person present, it's \emph{quepo}, and you're likely to hear it in
\emph{¿Quepo yo?}---meaning ``Will I fit?" or ``Is there room for one more?"
Otherwise, you may run into it in set expressions like \emph{cabe decir} (``it's
worth mentioning") and \emph{no cabe duda} (``there's no doubt").

\section{\emph{caer}}

It means ``to fall," of course, and ``to drop" when used reflexively (\emph{caerse}): \emph{Se me cayó el vaso} = ``I dropped the glass," It's also
very frequently heard in the phrases \emph{caer bien} and \emph{caer mal} to express
likes and dislikes (see Chapter 4). You may also run across \emph{caer} for ``to
visit unexpectedly" or ``to drop in on." \emph{Te caigo en la tarde} is an informal way of saying ``I'll drop in on you in the afternoon." Sometimes
it's used to suggest that someone's arrival was not only unexpected but
also unwelcome. ``What are your in-laws doing here?" might be answered by \emph{Es que me cayeron} (``They just kind of showed up").

\section{\emph{cambiar}}

Meaning ``to change" as well as ``to make change" in the sense
of ``Can you change a twenty?" (\emph{¿Me puede cambiar un billete de a
veinte?}), \emph{cambiar} also crops up in a number of common expressions.
These include \emph{cambiar de idea} or \emph{cambiar de opinión} (``to change
one's mind"), \emph{cambiar de ropa} (``to change one's clothes"), and \emph{cambiar de casa} (``to move").

\section{\emph{coger}}

This is one of those words that many dictionaries handle with
more discretion than clarity. The simple fact is that \emph{coger} is a vulgar
term for ``to fornicate" in several countries (Argentina, Mexico, Uruguay, and others), where as a result it is rarely used in proper company
(see Chapter 10). That said, it is also one of the most commonly used
verbs in some other countries (especially Spain). What's a poor student
to do when faced with the choice? That will depend on where you are
learning the language and with whom you expect to be communicating. But if you want to use substitutes for coger right from the start---in the sense of ``to get," ``to take," ``to grab"---it may not be such a bad
idea. The word that usually replaces it is \emph{tomar}, as in \emph{tomar el tren}. In
Mexico, particularly, \emph{agarrar} is often heard. Both substitutes are 
understood even where \emph{coger} is used, and both can save you considerable
embarrassment.

\section{\emph{conocer}}

This verb is often confused with \emph{saber} by students of Spanish;
both mean ``to know," but \emph{conocer} is used in the sense of ``to be familiar with." A limited rule of thumb: use \emph{conocer} for proper nouns and
all specific people, places, and things; use saber for everything else. Do
you know Paris? \emph{Conocer}. Do you know the French Quarter? \emph{Conocer}.
Do you know the old lady there who sells flowers on the street? \emph{Conocer}. Do you know her name? \emph{Saber}. Do you know what flowers she
sells? \emph{Saber}. Do you know what she's saying about you? \emph{Saber}.

\emph{Conocer} also means ``to meet," but keep in mind that it only
works for the first time you meet someone. English speakers use ``to
meet" to describe routine encounters, such as ``I met my mother at
the train station." They also tend to say \emph{la primera vez que lo conocí}
to convey ``the first time I met him" (instead of the less redundant
\emph{cuando lo conocí}). In Spanish you can only \emph{conocer} someone once, and
needless to say it would be difficult (not to mention dramatic) to \emph{conocer} your mother at a train station. Finding the right Spanish word for
``to meet" in these other situations will give you an idea of how overworked this poor verb is in English. When you are referring to a first
meeting, as noted, use \emph{conocer}. \emph{Lo conocí en París} = ``I met him (for
the first time) in Paris." All chance encounters after that are handled
by \emph{encontrar}. \emph{La encontré en el cine} = ``I met (ran into) her at the
movies." Meeting a plane or a train would be covered by \emph{recibir}. \emph{Me
recibieron en la estación} = ``They met me at the station." For a
planned get-together you would use \emph{quedar en verse con} or \emph{quedar en
encontrarse con}: \emph{Quedé en verme con unas amigas en el centro} = ``I
met (up with) some friends downtown."

\section{\emph{creer}}

It means ``to believe," but it is also almost always the word
you want for ``to think." Beginning students, incorrectly, often prefer
\emph{pensar} (see below). The distinction is subtle, but it works out roughly
as follows: if the emphasis is specifically on the thought process or the
act of thinking, use \emph{pensar}. If you're stating a personal belief or opinion, use \emph{creer}. Thus \emph{Creó que tienes la razón} = ``I think you're right."
\emph{Creer} is also used in many interjections and phrases, such as \emph{¿Qué
crees?} (``Guess what?"), \emph{Créeme} (``Trust me"), \emph{¿Tú crees?} (``You really
think so?"), and so on. A good phrase to learn is \emph{ni creas}, which could
be translated as ``don't expect" or ``no way." \emph{Ni creas que te voy a ayudar} = ``You're crazy if you think I'm going to help you" or ``Don't expect me to help you," \emph{Creo que sí} and \emph{Creo que no}, finally, should be
on the tip of your tongue for ``I think so" and ``I don't think so."

\section{\emph{cuidar}}

Technically, this means ``to care for," though in English we
would generally use some other word to translate it. \emph{Ana se quedó
para cuidar a los niños} = ``Ana stayed home to watch (take care of)
the kids." \emph{Cuide su cartera en ese barrio} = ``Watch your wallet in that
neighborhood." \emph{Cuídate} is ``Take care of yourself" or, slangily, ``Take it
easy"; it is sometimes used as a parting comment. \emph{Cuidar} in general is
the word you want for asking someone to ``watch over" or ``to keep an
eye on" something, as when you want to leave your luggage in the bus
station for a few minutes (not a recommended practice). \emph{¿Me puede
cuidar la maleta unos minutos?} = ``Can you keep an eye on my bag
for a few minutes?" See also \emph{guardar}.

\section{\emph{dar}}

``To give" and much more. A common additional meaning of
dar is ``to hit," giving rise to a number of expressions. \emph{Dar en el blanco}
is ``to hit the bull's-eye" and is often heard for ``to guess right," ``to hit
the nail on the head." \emph{Dar en la torre} is a common idiomatic expression that covers physical beatings as well as more metaphorical thrashings. ``How did the Cubs do against the Mets?" \emph{Le dieron en la torre}
(``They beat the pants off `em"). \emph{Dale}, by itself, is ``Hit it" (or ``Hit
him," ``Hit her"); you can let off steam by shouting it at boxing matches.
More metaphorically, it means ``Give it your best" or ``Give `em hell."
Often, in this sense, it's paired with \emph{duro} and said to give encouragement: \emph{¡Dale duro, Juan!} (``Give `em hell, Juan!"). By the same token,
you can use \emph{dándole duro} to convey the intensity of an action or an
effort. ``How are you coming with the term paper?" \emph{¡Dándole duro!}

Here are other dar expressions you'll want to know:

\bsk

\indu \emph{dar a la calle} (\emph{al patio, a la alberca}, etc.) = ``to face the street
(the patio, the pool, etc.)"

\indu \emph{da igual} or \emph{da lo mismo} = ``it's all the same to me" or ``it
doesn't matter"

\indu \emph{dar coraje} = ``to make mad"

\indu \emph{dar (la) lata} = ``to pester" or ``to be a pain"

\bsk

\emph{Dar} is also used in some parts of the Spanish-speaking world for
``what's on"---in the sense of showings on television or at the movie
theater. \emph{Están dando la segunda parte de} Arma Mortal = ``They're
showing the second part of \emph{Lethal Weapon}." (See also pasar.) Finally,
though you won't find it in the dictionary, an increasingly common expression in American Spanish is \emph{dar chance} for ``to give a break." \emph{Vamos, oficial, denos chance} = ``Come on, officer, give us a break." Purists will tell you that this is a horrible barbarism and that you should
say \emph{denos una oportunidad} instead. But purists should consider cutting their losses, since more and more speakers of slang are already bypassing \emph{dar chance} in favor of \emph{dar un break}, which is more barbaric yet.

\emph{Darse}, the reflexive form, is also used for a few handy phrases,
such as \emph{darse cuenta de}, which means ``to realize." \emph{Perdón, no me di
cuenta de que estaba estacionado en su pie} =: ``Sorry, I didn't realize
I was parked on your foot." \emph{Darse por vencido} is the phrase you want
for ``to give up," ``to surrender." To say ``I give up" in response to a
riddle, for instance, you can even use \emph{Me doy} (more formally, \emph{Me rindo}) all by itself.

\section{\emph{decir}}

``To tell" or ``to say." You'll probably also be using it a lot in
the phrase \emph{querer decir}, or ``to mean." \emph{¿Qué quiere decir esa palabra?}
= ``What does that word mean?" \emph{Decir} also pops up in a lot of cute
little phrases, such as \emph{No me digas} (``You don't say), \emph{Dime} (``Tell me"
or simply ``Yes?"), and \emph{¿Qué decías?} (``What were you saying?"---after
an interruption). You should also be alert to \emph{Dile} (and \emph{Dígale}) as a typical preface to an indirect command, and thus the subjunctive. \emph{Dile que
venga} = ``Tell him to come here." If you're stating a fact instead of
issuing a command, it takes the indicative. \emph{Dile que estamos aquí} =
``Tell him (or her) we're here."

\section{\emph{dejar}}

Meaning ``to leave" or ``to let," this verb is often found in
other expressions and is not always a reliable vehicle for English ``let"
phrases. For example, \emph{Déjenos entrar} means ``Let us in," but \emph{Entremos}
(or \emph{Vamos a entrar}) means ``Let's go in." \emph{Dejar} is heard in such phrases
as \emph{Déjame en paz} (``Leave me alone"), \emph{Déjalo} (``Leave it" or ``Drop it"
or ``Skip it," often referring to a sensitive topic or one best dealt with
later), and \emph{Deja ver} or \emph{Déjame ver} = ``Let me see."

\section{\emph{disfrutar}}

``To enjoy." Only mentioned here to dissuade you from saying
\emph{Disfrútense} for ``Enjoy yourselves." It does mean this, but probably not
in the way you intend. ``Have fun with yourselves" or ``Take pleasure
in yourselves" would probably be a more accurate translation. Instead,
you should say \emph{Que lo disfruten}, usually in reference to a specific
event. For the best translation of ``Enjoy yourselves," forget about \emph{disfrutar} altogether and use \emph{divertirse}: \emph{Diviértanse} or \emph{Que se diviertan}.

\section{\emph{dormir}}

\emph{Dormir} is ``to sleep" while \emph{dormirse} is ``to fall asleep." Just a
reminder: ``sleep," the noun, has to be expressed by \emph{sueño} (see \emph{soñar}).

\section{\emph{durar}}

This is one of the words you will need to ask how long things
take (movies, bus rides, flights, speeches, etc.). You can think of \emph{durar}
as a generally safe translation of ``to last." \emph{La fiesta duró toda la noche}
= ``The party lasted all night." Generally \emph{durar} is used for things that
have a specific duration---or \emph{dura}tion, if it helps you remember. (See
also \emph{tardar} and \emph{hacer}.)

\section{\emph{echar}}

\emph{Echar} is one of those words that take up about three pages in
the dictionary. Not all of those expressions are that common or useful,
though. Almost all of them have to do with a forcible casting out or
expulsion. File away this division of labor: \emph{meter} is for putting in, \emph{sacar} for taking out, and \emph{echar} for kicking out, more or less. Buses \emph{echan}
smoke, teachers \emph{echan} students, and so on. An unexpected use of
\emph{echar} is for ``to pour," as with a liquid into a glass or from a pitcher.
Some of the idiomatic expressions using \emph{echar} are very handy. \emph{Echar
de menos a} is ``to miss (someone)" (though in the Americas you are
more likely to hear extrañar). \emph{Echar a perder} is ``to spoil," be it children or food in your refrigerator. \emph{Echar ganas} is a very common
expression for ``to show enthusiasm" or ``to give it a good effort: ``Mom,
I can't understand my math homework." \emph{Vamos, échale ganas, hijo}.
\emph{Echar una mano} is the expression you will need for ``to lend a hand,"
and \emph{Échame la mano} is ``Gimme a hand." \emph{Echar un ojo} is ``to have a
quick look," and \emph{echar la culpa} is ``to blame."

\section{\emph{encargar}}

This verb, meaning ``to entrust" or ``to commission," is far
more common than these awkward English translations suggest. Almost always it conveys some notion of ``charge"---to take charge, to be
in charge, to charge, and so on. With \emph{de}, for instance, it means ``to take
charge of" or ``to put someone in charge of" something: \emph{Yo me encargo
de la ensalada} = ``I'll be in charge of (take care of) the salad." \emph{El encargado} is the all-purpose word for ``the guy in charge." Without the
\emph{de}, \emph{encargar} means ``to entrust" but often with more colloquial English equivalents. For example, if someone's going off to the store, you
might say \emph{¿Te encargo unas aspirinas?} (``Can you get me some aspirin?"). Many usages could almost be translated ``to order." \emph{Le encargué
dos litros al lechero} = ``I ordered two liters from the milkman." Sometimes, what you're ordering or entrusting is not spelled out, and ``to
count on" would make for a better fit. If someone offers to fix your car
by nightfall and you plan to leave town that very night, you might say
to the mechanic, \emph{Se lo encargo mucho} (``I'm really counting on you").

\section{\emph{equivocar}}

Used as a reflexive (\emph{equivocarse}), this is the common verb for
``to make a mistake" (although \emph{estar equivocado} works as well). Usually used with the preposition \emph{de}, it means ``to get something wrong."
For instance, when you've dialed the wrong number, you would say,
\emph{Perdón, me equivoque de teléfono} (or \emph{número}). If Pedro shows up on
the wrong day for a party, you would tell him, \emph{Te equivocaste de día}.
And so on. As a subtle distinction, \emph{estar equivocado} generally suggests someone is mistaken; \emph{equivocarse} means he or she has made a
mistake.

\section{\emph{esperar}}

Meaning both ``to hope" and ``to wait," the verb will usually
take the subjunctive in both senses. \emph{Espero que venga} = ``I hope he (or
she) comes." \emph{Estoy esperando que regrese} = ``I'm waiting for him (or
her) to get back." Unless the context makes it clear, you would use the
preposition \emph{a} with \emph{esperar} to say ``to wait for." Thus, \emph{Espero a que
regrese} = ``I'll wait for him (or her) to get back." (\emph{Esperar por} or \emph{esperar para} should never be used for ``to wait for.") \emph{Esperar} is frequently
heard in the imperative for ``Wait a minute" or ``Hold on." \emph{Espérese,
por favor} is the form you're most likely to hear. In familiar usage it's
\emph{Espérate}, which often comes out sounding like \emph{'pérate}.

\section{\emph{estar}}

The other verb for ``to be"---the one that covers transitory
states. The \emph{ser} versus \emph{estar} confrontation was covered in detail in
Chapter 5. So here it's enough to glance at one use of the verb you
might not have been expecting---about the only one that is out of character for it from the standpoint of an English speaker---\emph{estar} when
used to ask ``What's today's date?": \emph{¿A qué estamos hoy?} or \emph{¿A cuántos
estamos?} The answer is phrased \emph{Estamos a} and the number: \emph{Estamos
a 25}, for example. Remember too that \emph{estar + de} is used for moods
and inclinations. \emph{El está de malas} = ``He's in a bad mood." (This use
is covered in Chapter 4.)

\section{\emph{estrenar}}

This is not the commonest of verbs, nor is it one that you desperately need to learn. It means to use or display something for the
first time---to ``debut" something, as it were. \emph{Juan está estrenando su
nuevo coche} = ``Juan is trying out his new car." \emph{Voy a estrenar la camisa que me regalaste} = ``I'm going to wear (for the first time) the
shirt that you gave me." \emph{Un estreno}, referring to films and theater, is a
``premiere" or ``opening"; referring to an artist, it would be a ``debut."

\section{\emph{guardar}}

Meaning ``to guard" or ``to save," this is the common verb to
use for telling someone to ``hold on to" something or ``put (something)
away." \emph{Guardar} is usually used when you give someone something
that you want them to put away until you need it again. If you're going
swimming and don't want to take your traveler's checks into the pool
with you, you might hand them to a friend and say \emph{¿Me los guardas?}
\emph{Guardar} is also commonly used by parents to tell their children, ``Put
it away" (\emph{Guárdalo}), be it in their closet or in their pocket. \emph{Guárdame
un poco} is ``Save a little for me," said of a favorite foodstuff, for instance. (See also ``Save" in Chapter 11.) The word is a little tricky to
use in the sense of its English cognate. \emph{¿Me guarda la maleta?} could be
``Keep an eye on my suitcase" but also something like ``Put my suitcase away"; so if you tell someone to ``guard" it and they walk off with
it, you'll know why. Use \emph{cuidar} (see above) for ``to keep an eye on,"
``to watch."

\section{\emph{haber}}

This is one of the megaverbs, with more uses than you might
ever be inclined to learn. Some are indispensable, though. \emph{Hay}, of
course, is the way to say ``There is\ldots{}" and ``There are\ldots{}" Say it
with a lilt and it becomes the questions ``Are there\ldots{}?" and ``Is
there\ldots{}?" \emph{No hay} is the typical curt response to questions about
availability: \emph{¿Hay café? No hay. ¿Hay cuartos? No hay}. \emph{Hay que} is the
impersonal way of saying ``to have to"---that is, when it's not obvious
exactly who has to do something. \emph{Hay que ir a España para aprender
español} = ``One has to go to Spain to learn Spanish." The imperfect
and preterit forms of \emph{haber} are \emph{había} and \emph{hubo}, respectively. \emph{Había}
is the more commonly used of the two. \emph{Había veinte personas en el
coche} = ``There were twenty people in the car." \emph{Hubo} is for something that was there all at once or not for long. \emph{Hubo un choque en la
carretera} = ``There was an accident on the highway." Remember never
to use \emph{habían} or \emph{hubieron} for ``there were," regardless of the number
of persons or things involved: \emph{Había una monja en la lancha} and \emph{Había dos monjas en la lancha}.

\section{\emph{hablar}}

A straightforward word for ``to talk" or ``to speak," but keep
it in mind for use on the telephone, where its use is rampant. It's not
hard to imagine a phone conversation going something like this:
(R-i-i-i-ing.) \emph{Hola. ¿Quién habla? ¿Con quién quiere hablar? Habla
Juan. ¿Puedo hablar can Fred? El habla. ¡Ah! ¡Hablas español!} Of
these, the \emph{El habla} (or \emph{Ella habla}) response is the most important to
get straight. It will spare you countless episodes of saying \emph{Soy él} (or
\emph{ella}) or \emph{Hablando}, both of which are incorrect when you mean ``This
is he (or she)" or ``Speaking."

\section{\emph{hacer}}

Hacer means ``to make" or ``to do," but you know that already.
Where English speakers have to remember to use \emph{hacer} is in the many
weather-related expressions that in English are covered by ``to be."
Some of the things that ``make" in Spanish are \emph{frío, calor, viento}, and
\emph{sol} (``cold," ``hot," ``windy," ``sunny"). \emph{Hacer} is also the way to say
``ago" in Spanish: \emph{Hace dos años nació mi hijo} (``My son was born two
years ago"). When did they leave? \emph{Hace un rato} (``A little while ago").
To say ``We did it!" or ``We made it!" in Spanish, you say \emph{¡Lo hicimos!}
With \emph{la} instead of \emph{lo} and intensified with \emph{ya}, it becomes a colloquial
\emph{¡Ya la hicimos!} (``We've got it made!"). Don't trouble your mind
searching for an antecedent to la---there is none. A very useful phrase
with \emph{hacer}, finally, is \emph{Hazte de cuenta} (or \emph{Haz de cuenta}), which introduces a thought with ``Pretend\ldots{}" or ``Let's say\ldots{}" (See also
Chapter 8 under ``\emph{Haz de cuenta que}.")

\emph{Hacerse}, in the reflexive, also turns up in a lot of unexpected
expressions. It's a common way of translating ``to become" and is also
common idiomatically for ``to make like" or ``to act like." \emph{El se hace el
payaso} = ``He's acting like a clown." \emph{Se hace el loco para no ir a la
cárcel} = ``He's pretending to be crazy to avoid going to prison." You'll
often encounter this expression in the negative as an exhortation.
\emph{No te hagas tonto} = ``Don't play stupid." \emph{No te hagas la víctima} =
``Don't play the victim." In Mexico especially you'll encounter \emph{No te
hagas} all by itself, with the predicate understood. This is a good translation for common interjections like ``Come off it!" or ``Don't gimme
that!" or ``Cut it out!"

\section{\emph{ir}}

The verb for ``to go" comes into play in many situations that
parallel English usages, but you'll have to be on the lookout for them
to learn to use them well. It's widely used, for instance, for negative
imperatives, otherwise known as warnings, equating approximately
with the English ``Don't go\ldots{}": \emph{No te vayas a meter en líos} =
``Don't go getting yourself in trouble." \emph{No le vayas a decir} = ``Now
don't go telling him" or just ``Don't tell him." In many compound verb
forms, \emph{ir} is extremely useful to get your point across. It's used, as in
English, in the future (\emph{Voy a llamar} = ``I'm going to call") and in the
imperfect (\emph{Iba a llamar} = ``I was going to call"). \emph{}And \emph{Vamos a} means
``Let's\ldots{}" Other common expressions with \emph{ir} include \emph{irle a} (``to root
for," as in \emph{Yo le voy a los Orioles}), \emph{ir por} (``to go get" or ``to go for," as
in \emph{Voy por el coche}), and \emph{Ahí te va} (meaning ``Catch" or ``Your turn").
\emph{Vaya} by itself is ``All right" or ``Omigosh," depending on tone and context. \emph{Vaya, vaya, vaya} is the common way to say ``Well, well, well," as
in ``What have we here?" \emph{Vaya} plus a noun is the equivalent of the
sarcastic comment ``Some\ldots{}!" If you return home to find the plumber
has managed to flood the entire basement, you might say sarcastically
\emph{¡Vaya plomero!} (``Some plumber!").

\section{\emph{lograr}}

This is the word you want for ``to manage" when used with
another verb in the infinitive. \emph{Logré reparar la tele} = ``I managed to
fix the television." \emph{Si logro salir de esta reunión, estaré en casa en
media hora}, = ``If I manage to get out of this meeting, I'll be home in
half an hour."

\section{\emph{llevar}}

\emph{Llevar} means ``to carry" or ``to take" and is often used for
``to bring," as we saw in Chapter 5. Here we'll concern ourselves with
\emph{llevar} in those expressions you'll want to have handy in your daily
doings. One common one is \emph{llevar} for expressions of time. A good way
to answer the inevitable question heard abroad, ``How long have you
been here," is to say \emph{Llevo} + the number + \emph{meses} (\emph{años, días}) \emph{aquí}.
The question, in fact, will often be phrased \emph{¿Cuánto tiempo lleva
(usted) aquí?} Once you get a feel for this usage, you'll find yourself
needing it more and more. \emph{Llevo dos días en cama} is much smoother
and more colloquial than \emph{He estado en cama durante} (or \emph{desde hace})
\emph{dos días} for ``I've spent two days in bed." \emph{Llevar} is also useful for ``to
wear" or ``to have on." \emph{Es el señor que lleva 1entes} = ``He's the man
with the glasses on."

\emph{Llevarse}, the reflexive, is also handy for a couple of expressions. A common one is for ``to get along." \emph{Ella y yo no nos llevamos}
= ``She and I don't get along." \emph{No me llevo con ellos} = ``I don't get
along with them." Another one you'll need, especially for shopping,
is \emph{llevarse} for ``to take" something (after paying for it, naturally). It
means the same as llevar essentially, but the reflexive is added for emphasis. \emph{¿Dos mil pesos? Me lo llevo} = ``Two thousand pesos? It's a
deal." \emph{Me llevo dos} = ``I'll take two." \emph{Llévatelo} is ``Take it," pure and
simple. Any usage that translates as ``to take along" gets the reflexive
as well: \emph{Me llevo un suéter por si hace frío} = ``I'll take a sweater in
case it gets cold."

\section{\emph{mandar}}

Meaning ``to send" or ``to order," this word can cause problems because of other words that can get the same message across.
Most uses of ``to send," for instance, work better with \emph{enviar}, and ``to
order" in a restaurant is \emph{pedir} or (increasingly) \emph{ordenar}. \emph{Mandar} is
used for real orders, of the sort that generals and bosses give, and you'll
come across it a lot in conjunction with a second verb. In these cases
it works as ``to send out for" or ``to have done." \emph{Manda hacer unas copias} = ``Have some copies made." It usually implies an order to an underling, so don't use it freely unless you are in a position either to give
orders or to take orders. If you go to Mexico, your first encounter with
\emph{mandar} may be the ubiquitous expression \emph{¿Mande?} for ``What?" or
``You called?" It's considered polite, but many foreigners (especially
from other Spanish-speaking countries) seem to think it servile and demeaning. If you feel that way, you can substitute \emph{¿Cómo?} or even the
brusque \emph{¿Qué?}---but you'll hear \emph{¿Mande?} just the same.

\section{\emph{meter}}

Meter, meaning ``to put (something) in," is used in a far wider
range of circumstances. It's the common way of saying ``to go inside"
(\emph{Vamos a meternos} = ``Let's go inside") or ``to go in" (\emph{Vamos a meternos al agua} = ``Let's go in the water"). It can even be used for ``to get
in" a car (as in \emph{Métete al coche}), but \emph{subirse} is preferred. The reflexive
form \emph{meterse} is also a good translation for ``to get involved in" or ``to
get mixed up with." \emph{No te metas} = ``Don't get involved." \emph{No te metas
can mi hermana} = ``Don't mess around with my sister." \emph{Meterse en
líos} takes the idea further and means ``to get mixed up in problems,"
``to get into trouble." If you get caught in the middle of a family
squabble and find opposing sides of the squabble looking to you for
support, you might throw up your hands and say \emph{Yo no me meto} (``I'm
not getting involved in this").

\section{\emph{notar}}

This verb is worth learning in the reflexive form (\emph{notarse}) to
express ``I can see that" or ``It shows." It's a nice, dry comment that
says that you, too, can perceive the obvious. If someone in the midst of
a downpour reminds you that it's the rainy season, you might respond
\emph{Se nota} (``I figured that out" or ``But of course"). Adding the personal
pronouns \emph{me, te}, or \emph{se} personalizes the phrase. Your friend, screaming,
tells you she's angry. You say, \emph{Se te nota} (``So I see," literally ``One
notes that in you.") \emph{Notar}, incidentally, is not a good word for ``to
note something down" or ``to make a note of." For that, use \emph{anotar}
or \emph{apuntar}.

\section{\emph{parar}}

``To stop." \emph{Pare el mundo, quiero bajarme} = ``Stop the world,
I want to get off." \emph{¿Dónde para el tren?} = ``Where does the train stop?"
And so on. \emph{Párale} is sometimes employed to say ``Stop it" or ``Cut it
out," as when someone is talking too much or the kids are screaming.
In most of the Americas (and even parts of Spain), the reflexive \emph{pararse}
means ``to stand up," and \emph{parado} is ``standing up." Travelers will
sometimes ask if there is room on a train or bus and be told \emph{Si quiere
ir parado} (``If you want to travel standing up"). \emph{Párate que nos vamos}
would be a colloquial way of saying ``Get up, we're going." Learn to
distinguish between \emph{parar} and \emph{pararse} for ``to stop." The reflexive
form is appropriate for stopping unassisted, whereas \emph{parar} suggests
something stopping something else. \emph{Paré el coche} is ``I stopped the
car." \emph{El coche se paró} is ``The car stopped."

\section{\emph{parecer}}

Meaning ``to seem," this verb is more frequently encountered
than its English equivalent. One of the most common ways of conveying likes and dislikes in Spanish, in fact, is with \emph{parecer}. Here are
some typical usages: \emph{¿Qué te parece?} = ``What do you think?" or
``How does that strike you?" \emph{¿Te parece?} = ``Is that okay with you?"
\emph{No me parece} = ``I don't like it." \emph{Me parece bien} = ``Fine with me."
The reflexive \emph{parecerse} is ``to look like" or ``to resemble." \emph{Me parezco a mi madre}
 = ``I take after my mother." ``To look like" in the sense of
 ``to look as if" requires \emph{parece que}, not the reflexive. \emph{Parece que va a
llover} = ``It looks like (as if) it's going to rain."

\section{\emph{pasar}}

One of several options for ``to happen" (see Chapter 11), \emph{pasar}
is also usually safe for most uses of ``to pass," although it's a bit slangy
at the dinner table (\emph{Pásame la sal}). One use you may not be expecting:
in some Latin American countries, \emph{pasar} is the verb to use when asking ``what's on" television or at the local theater. \emph{¿Qué están pasando
en el 8?} = ``What's on Channel 8?" \emph{Están pasando la nueva película
de de Niro en el Cine Colón} = ``They're showing the new de Niro
movie at the Cine Colon." In South American countries the verb of
choice here would be \emph{dar}, usually in the phrase \emph{estar dando}.

As a colloquial greeting, both \emph{¿Qué pasa?} and \emph{¿Qué pasó?} are
used, though individual countries tend to have a preferred form. In
most situations, \emph{¿Qué pasa?} implies that something is wrong or abnormal; this is the question to ask when there are police cars parked in
front of your house. In Mexico, where \emph{¿Qué pasó?} is the preferred
greeting, saluting someone with \emph{¿Qué pasa?} may prompt raised eyebrows and the question \emph{¿Por qué?} So much for starting a pleasant
conversation.

Using \emph{pasar} can conveys ``to happen to" in the sense of ``to
become of." \emph{¿Qué pasó con Juan?} = ``What's become of Juan?"---that
is, why is he late? In the present tense \emph{¿Qué pasa?} with an indirect
object pronoun (\emph{me, te, le, nos}, or \emph{les}) is like asking ``What's (my, your,
his, her, our, their) problem?" In fact, \emph{¿Qué te pasa?} is roughly equivalent to ``What's bugging you?" In the past tense, this same construction means ``What happened to (me, you, him, her, us, them)?": \emph{¿Qué
le pasó?} = ``What happened to him?"---for example, why are they carrying him off on a stretcher?

\emph{Pasarse}, the reflexive form, plus the preposition \emph{de} is very
handy in expressions meaning ``to go too far," figuratively speaking. \emph{Se
pasó de listo} means someone ``was too clever" or ``was too sneaky,"
and implies that the person got caught at it. Sometimes it can be translated as ``to get carried away." When a person \emph{se pasa de listo (lista)}
with another person, it can mean that he or she is making unwelcome
sexual advances. \emph{Ese señor se pasó de listo con María} = ``That guy
made a (rude) pass at Maria." The formula \emph{pasarse de} can be used with
almost any adjective or quality, positive or negative. \emph{Usted se pasa de
generosa} = ``You are being overly generous." \emph{Te pasaste de imbécil} =
``You were even stupider than usual."

\section{\emph{pedir}}

``To ask," ``to ask for," and the correct verb for ``to order" in
a restaurant (although \emph{ordenar} is gaining ground). \emph{Pedir} should make
you think of indirect commands and the subjunctive: \emph{Pídele que se
vaya} (``Ask him to leave"). It is also used in a number of stock phrases,
quite a few of which you are liable to need in the course of your dealings in Spanish. Some common ones include \emph{pedir permiso} (``to ask
permission"), \emph{pedir perdón} (``to apologize"), \emph{pedir informes} (``to ask for
information"), and \emph{pedir ayuda} (``to ask for help"). \emph{Pedir prestado} is
the correct phrase for ``to borrow," but you'll often find \emph{prestar} (see below) handier. A rule of life for some in the Spanish-speaking world is
\emph{Es más fácil pedir perdón que pedir permiso}, or ``It's easier to ask forgiveness than permission."

\section{\emph{pensar}}

The verb for ``to think," though often \emph{creer} (see above) is preferred. To the extent that there is a rule for distinguishing them, use
\emph{pensar} when you might use ``to have been thinking" or ``to be thinking
about" in English. If it's just a simple statement of opinion, use \emph{creer}.
\emph{Pienso que debes irte} = ``It is my feeling that you should go." \emph{Creo
que debes irte} = ``You should probably go." Sometimes the two are
interchangeable. A usage of \emph{pensar} you should become very familiar
with is \emph{pensar} plus the infinitive to mean ``to plan on" or ``to intend."
\emph{Pienso irme mañana} = ``I plan to go tomorrow." \emph{Pienso quedarme
unos días} = ``I intend to stay a couple of days." \emph{Pensar + en} is ``to
think of" or ``to have in mind." \emph{Pensar + sobre} (or \emph{acerca de}) works as
``to think about" or ``to consider" something. \emph{Pensar + de} is ``to think
of," in the sense of an opinion, and is a lot like \emph{creer}. \emph{Estoy pensando en nuestras vacaciones}
= ``I'm thinking of (remembering, daydreaming about) our vacation." \emph{Todavía estoy pensando sobre} (or \emph{acerca de})
\emph{nuestras vacaciones} = ``I'm still thinking about (considering, analyzing) our vacation." \emph{¿Qué piensas de nuestras vacaciones?} = ``What do
you think of our vacation (so far)? Stock phrases with \emph{pensar} include
\emph{¡ni pensarlo!} (``no way," ``it's out of the question"), \emph{pensándolo bien}
(``on second thought"), and \emph{sin pensar} (``without thinking,"
``unintentionally").

\section{\emph{poder}}

Meaning ``to be able," this is in general a predictable word.
Aside from its quirks in the past tenses (see Chapter 5), it's just a matter of mastering a few stock phrases to get a hold on \emph{poder}. One such
phrase is \emph{poder + con}, which means something like ``to handle" or ``to
deal with." Examples will be useful here. A student who has trouble
learning biology might lament, \emph{No puedo con la biología}. In the sports
pages, you'll often come across headlines saying things like \emph{Los Leones
No Pudieron con Los Toros} (``The Lions Couldn't Handle the Bulls"---
that is, the Bulls beat the Lions, pure and simple). A Cuban postrevolutionary chant intones \emph{Fidel, Fidel, ¿qué tiene Fidel, que los americanos no pueden con él}, which means ``Fidel, Fidel, what does Fidel
have, that the Americans can't handle (defeat) him?" Depending on the
context, \emph{poder + con} can also mean ``to tolerate," and in this sense is
nearly synonymous with the verb \emph{aguantar}. \emph{¿Ay, no puedo con mi hermano!} = ``Arrgh, I just can't stand my brother!" A fun way of describing an extremely irritating person is to say \emph{No puede ni consigo mismo}
(``He can't even stand himself"). Other phrases using poder that you'll
want on the tip of your tongue: \emph{¿Se puede?} (``May I"), \emph{Puede ser}
(``Could be" or ``Maybe"), and \emph{Puede que} plus the subjunctive (``It
could be that\ldots{}" or ``Maybe\ldots{}").

\section{\emph{prestar}}

\emph{Prestar}, ``to lend," works for just about anything you might
want to borrow, just as ``to lend" does in English. Thus \emph{Préstame tu
pluma} = ``Lend me your pen." (For ``lending a hand," though, you
would probably use dar or echar: \emph{Oye, échame una mano}.) What is
difficult for many English speakers is to switch between ``borrow"
phrases and ``lend" phrases in Spanish. This sometimes leads to convoluted constructions with \emph{pedir prestado}, like \emph{"Puedo pedir prestado tu
pluma?} Much more natural in Spanish is to turn it around (i.e., saying
``you lend" instead of ``I borrow") and use \emph{prestar} by itself. If \emph{Préstame}
is too tactless for your tastes, say \emph{¿Me presta?} or \emph{¿Me prestas?}: \emph{¿Me
prestas tu pluma?} In very slangy speech you might hear \emph{Presta para
acá} or \emph{Presta pa'cá} for ``Hand it over" or ``Give it up." The latter example has sexual overtones, as it does in English.
\emph{Prestar} is also used for ``to pay attention," which, if you think
about it, is much more realistic than the English concept (we don't really ``pay" attention, we just lend it out). \emph{¿Niños, presten atención!} =
``Children, pay attention!" \emph{Prestarse}, the reflexive form, can be quite
useful for ``to lend oneself," though it sounds much more idiomatic in
Spanish than in English. \emph{¿Tú crees que Juan nos deje copiar en el examen? No, él no se presta a eso}. = ``Do you think Juan will let us copy
his exam?" ``No, he doesn't lend himself to that."

\section{\emph{quedar}}

A megaverb that you'll want to have on your side as quickly as
possible. Its most straightforward uses revolve around ``to stay" or ``to
remain." \emph{Aquí me quedo} is ``I'm staying here" and is sometimes used
as a name for cantinas. \emph{Quédate aquí} = ``Stay here." Other uses
require ``to have left" in English. \emph{Sólo me quedan treinta dólares} = ``I
only have thirty dollars left."

A host of other expressions with \emph{quedarse}, the reflexive, are
better covered in English by ``to keep." \emph{Me quedé con treinta dólares}
= ``I kept thirty dollars." \emph{Quédese con el cambio} would be ``Keep the
change." \emph{Quédatelo} = ``Keep it." For use in shopping, \emph{quedarse} is a
lot like \emph{llevarse} (see above). \emph{Me quedo con el azul} = ``I'll take the blue
one." Often quedarse suggests a final or resultant state of affairs. \emph{Me
quedé helado} is, literally, ``I was left frozen" and suggests you were frozen with fear. \emph{Me quedé en blanco} is to say ``I ended up blank" or ``I
didn't understand that at all." If someone asks you whether you understood an explanation of the theory of relativity, you could answer, \emph{Para
nada. Me quedé en blanco}. In English slang the equivalent might
even be"I spaced." A stock phrase you should remember for personal
dealings is \emph{¿En qué quedamos?} to mean something like ``What's the
agreement, then?" or ``So what's the deal?" Use it toward the end of
conversations to establish clearly the next step, be it the signing of a
multimillion-dollar merger agreement or a date to sip margaritas under
the stars.

Along these same lines of final or resultant states are the everyday expressions \emph{quedar bien} and \emph{quedar mal}. Like many of the expressions using \emph{quedar}, these seem to defy a simple English translation,
but the idea is ``to end up well (or badly) with someone." Their use is
similar to \emph{caer bien} and \emph{caer mal}, and often they can be translated
with ``impress," though that's a little strong. ``To get on someone's
good side" might come closer for \emph{quedar bien}. \emph{Se puso corbata para
quedar bien con los suegros} = ``He put on a tie to get on his in-laws'
good side." For \emph{quedar mal}, an example will be more helpful than an
English equivalent. \emph{Quedé mal con él porque no lo saludé} = ``I've
gotten on his bad side (i.e., he's mad at me) because I didn't say hello
to him."

Finally, quedar has a couple of common uses that you should
be alert to since they don't fall into the ``stay" or ``remain" categories.
It is the common word for ``to fit," for clothes and everything else. Remember to use it with the indirect object pronouns \emph{me, te, 1e}, etc. \emph{Este
saco no me queda} = ``This coat doesn't fit." Also, and more natural to
a Spanish speaker, \emph{Este saco me queda grande} (or \emph{chico}) = ``This coat
is too big (or small) for me." \emph{Quedar} also comes into play for describing locations. As a tourist, especially, you will hear it (and can even
use it!) a lot. \emph{Perdón, ¿donde queda la plaza? Adelante, a tres cuadras}.
= ``Excuse me, where is the plaza?" ``Three blocks up." \emph{Queda cerca}
and \emph{queda lejos} are both handy phrases for travelers. \emph{¿Queda cerca la
plaza? No, queda lejos}.

\section{\emph{querer}}

An absolutely vital word, and one this book gives a lot of space
to so you can get it right. It means ``to like" or ``to want" and, with
people, ``to love" or ``to want" (see Chapter 4). Like \emph{poder} and \emph{saber},
this verb acts a little strangely in the past tenses (see Chapter 5). Otherwise it's trustworthy, though it's worth rehashing a few tips for using
it well. \emph{Querer} performs many interesting tricks when paired with an
adverb and in the subjunctive. Before you faint, remember that we already went over that (Chapter 6) and you survived it quite nicely---\emph{cuando quieras, donde quieras, como quieras}, and so on, We've also
gone over how to say you ``want" something (using \emph{traer}) without
dragging \emph{yo quiero} into your speech habits (Chapter 2). Another option
is to use \emph{quisiera} (``I'd like"). \emph{Querer} is also, in the phrases \emph{con querer}
and \emph{sin querer}, your best ticket for handling ``on purpose" and ``by
accident" (Chapter 12). \emph{Mamá, metí al gato en la piscina. ¿Fue con
querer o sin querer?} = ``Mommy, I put the cat in the swimming pool."
``Was it on purpose or by accident?"

\section{\emph{repetir}}

``To repeat," of course, but also ``to burp" or ``to provoke
burps," The proper word for ``to burp" is \emph{eructar}, which covers most
every burp, while \emph{repetir} is for those little, barely perceptible, good-eatin' burps. Just thought you'd want to know.

\section{\emph{romper}}

Remember that romper is ``to break intentionally": \emph{Rompí
el vaso tirándolo contra la pared} = ``I broke the glass by throwing it
against the wall." \emph{Romperse} is ``to break" in the sense of an accidental
act: \emph{Se me rompió el vaso cuando lo estaba lavando} = ``The glass
broke (on me) when I was washing it," In this construction the literal
meaning is ``such-and-such broke itself to me (or you, him, her, us,
them)." The distinction is not dogma, but you should try to stick to
it. \emph{Romper con} is ``to break up with" in the sense of lonely hearts and
whatnot.

\section{\emph{saber}}

A sometimes complicated verb, \emph{saber} bears watching for its
trickery in past tenses (Chapter 5) and in contrast to \emph{conocer} (see
above). An imperfect but useful rule of thumb: use \emph{conocer} with
proper and specific nouns and \emph{saber} or \emph{saber de} with the rest of them
and most clauses. \emph{¿Conoces París?} but \emph{¿Sabes dónde comen los parisinos?} and \emph{¿Sabes de su historia?} An exception to this rule are the
names of languages, which take \emph{saber}: \emph{¿Sabes inglés?} Another more
sweeping but also far-from-perfect rule: more often than not the word
you want for ``to know" is \emph{saber}. \emph{Saber} is frequently followed by verb
infinitives; \emph{conocer} never is. \emph{Saber} carries with it the idea of ``to know
how," so you don't have to say \emph{saber como}. \emph{¿Sabes esquiar?} \emph{No, pero
se caerme}. = ``Do you know how to ski?" ``No, but I know how to
fall down."
A few of the many stock expressions using \emph{saber} include
\emph{¿Sabes qué?} (``Know what?"), \emph{No sé} (``I don't know"), \emph{¿Quién sabe?}
(``Who knows?"), \emph{¡De haberlo sabido!} (``If I had only known!"), \emph{¿Yo
qué sé?} (``What do I know?" or ``Don't ask me!"). \emph{Un sabelotodo} is ``a
know-it-all," and a useful phrase for ``as far as I know" is \emph{que yo sepa}.
Finally, and just possibly to confuse you further, \emph{saber} is also
the word for ``to taste," as in how something tastes to you. \emph{La sopa
sabe bien} is ``The soup tastes good." In the first person (for use after
kissing or among cannibals), the correct form is \emph{sé}, but colloquially
you might hear \emph{sepo}. \emph{¿Que tal sé (sepo)? Sabes a pepinillos agrios}.
= ``How do I taste?" ``You taste like pickles." For the transitive ``to
taste"---that is, to taste something---you need to use probar.

\section{\emph{seguir}}

``To follow," yes, but also frequently ``to continue" or ``to keep
(on)." The most common formula is \emph{seguir} plus the infinitive: \emph{Sigue
viniendo} (``He keeps coming"), \emph{Sigues comiendo} (``You keep eating")
\emph{Sigo llorando} (``1 keep crying"), and so on. \emph{Seguir} also works to translate a lot of the uses of ``still" in English. In fact, using \emph{seguir} frequently sounds more natural in Spanish than using \emph{todavía}, which
is what native English speakers tend to resort to: \emph{Sigue creyendo en
Santa Claus}. = ``She still believes in Santa Claus." \emph{Sigo enfermo} =
``I'm still sick." For the negative, you can use \emph{seguir + sin} and sound
very ``Spanish" indeed: \emph{¿Sigues sin creer en Dios?} = ``Do you still not
believe in God?" \emph{Seguir sin} and \emph{seguir con} are also useful constructions with a noun tacked on instead of the infinitive. \emph{Sigue sin trabajo}
= ``He still doesn't have a job." A few odds and ends to take note of:
\emph{Síguele} is a handy phrase for ``Keep it up" in both its genuine and
ironic senses; \emph{¿Quién sigue?} = ``Who's next?"; \emph{¿Cómo sigues?} as a
greeting means ``How are you getting along?" and implies that a person
has been sick or afflicted by some sort of trouble, even if it's only Spanish grammar exercises.

\section{\emph{sentir}}

Sentir means ``to feel" and is a transitive verb used with direct
objects---that is, things you feel. \emph{Sentirse}, the reflexive, is used with
adjectives to express how you feel. Thus \emph{Siento frío} is ``I feel cold" but
\emph{Me siento bien} is ``I feel fine." \emph{Siento asco} is ``I feel nauseous," and
\emph{Me siento mal del estomago} is ``I feel sick to my stomach." \emph{Lo siento},
keep in mind, is ``I'm sorry," and \emph{Lo siento mucho} is ``I'm very sorry."
In case you need to explain further, you need only add \emph{haber} and a past
participle, dropping the \emph{lo}. Try to learn a couple of these by heart and
keep them at the ready: \emph{Siento haber llegado tarde} = ``Sorry I'm
late." \emph{Siento mucho no haber podido ir} = ``I'm very sorry I couldn't
come (or go)."

\section{\emph{ser}}

Except when you need \emph{estar} (Chapter 5), \emph{ser} is used for ``to
be." It is the verb of permanent states, of the way things are, of telling
it like it is. There are a billion or so expressions using \emph{ser}, but most
of them are predictable and translate as ``to be" in English. A few to
watch: \emph{¿De quién es?} = ``Whose is it?" \emph{¿De qué es?} = ``What is it
made of?" Like \emph{querer}, \emph{ser} in its subjunctive forms can be employed
to form ``-ever" words or to say ``any"; \emph{cuando sea, donde sea}, and so
forth. This is covered in Chapter 6, but a few examples here can't hurt.
Usually you employ \emph{que sea} after someone has already fed you the antecedent. It's better not to repeat the antecedent but think fast to get
the right gender. If someone asks, ``What brand of beer do you want?"
you can answer \emph{La (marca) que sea}. ``In what restaurant do you want
to eat?" \emph{En el (restaurante) que sea}. It's worth a little extra work to get
these expressions down pat, since otherwise you'll be inclined to say
ghastly things like \emph{Cualquier hora que quieres}, instead of the pure and
lilting \emph{Cuando sea}, for ``Anytime."

\section{\emph{servir}}

``To serve," yes, but how often do you say ``to serve" in an average day? ``After I serve dinner, if it serves you, m'lord, I'll serve you
poisoned coffee and it'll serve you right!" In Spanish \emph{servir} is much
more commonly heard for ``to work" in the sense of ``to function."
\emph{No sirve mi teléfono} is ``My phone doesn't work" \emph{¿Para qué sirve?} is
``What is it used for?" or even ``What good is it?" When \emph{servir} is used
for ``to serve," it is often dressed up in the stock phrase \emph{¿En qué le
puedo servir?} (``May I help you?"). \emph{Sírvase} (or \emph{Sírvete}), finally, can be
used for ``Help yourself," but a Spanish speaker would probably say
\emph{Tome lo que quiera} instead.

\section{\emph{soler}}

There is no English equivalent for this word among the verbs,
which is odd considering how often you'll find yourself needing it. \emph{Soler} (meaning ``to be in the habit of," ``to be accustomed to") is useful to
describe something you usually do and is stuck before another verb
in the infinitive, making it a snap to use. \emph{Suelo comer a las dos} = ``I
usually eat at two." \emph{Suele ir al cine saliendo del trabajo} = ``He usually goes to the movies after work" See Chapter 12 (under ``Usually")
for other ways to handle this concept.

\section{\emph{sonar}}

``To sound" or ``to ring." \emph{Suena el timbre} = ``The doorbell's
ringing," \emph{Eso suena dudoso} = ``That sounds dubious." \emph{Sonar} is also
the verb to use to say ``to ring a bell," as in what happens in your
memory when something sounds familiar. ``Do you know Juan Pérez?"
\emph{No, pero el nombre me suena} (``No, but the name rings a bell"). \emph{Sonarse}, incidentally, is ``to blow one's nose," and \emph{sonarse a alguien} is
``to smack someone."

\section{\emph{soñar}}

Meaning ``to dream," the verb is used with \emph{con}. \emph{Sueño con serpientes} = ``I dream about snakes." Here's something worth remembering: \emph{sueño}, the noun form, means both ``dream" and ``sleep." \emph{Una
falta de sueño} = ``A lack of sleep." \emph{Soñado}, the adjective form, is a
fun word for ``dreamy" or ``ideal": \emph{la playa soñada} = ``the beach of
your dreams."

\section{\emph{subir}}

The polar opposite of \emph{bajar} (see above), \emph{subir} works conversely
for gaining weight, getting on a bus, and so on.

\section{\emph{tardar}}

\emph{Tardar} works well for ``to take" in time expressions. \emph{Durar}
(see above), on the other hand, generally works better for ``to last."
\emph{Tarda el tiempo que quieras} = ``Take as much time as you want."
\emph{¿Cuánto tardará en venir?} = ``How long will he take in coming?"
Since we sometimes use ``to take" and ``to last" loosely in English,
there can be confusion in the correct Spanish choice. \emph{El avión tarda
media hora en venir de allá a acá}. = ``The plane takes half an hour
to get here from there." \emph{El vuelo dura media hora} = ``The flight lasts
half an hour." As a rule, use \emph{durar} whenever ``to last" could work, and
use \emph{tardar} otherwise. \emph{Tardar} also means ``to take time" with the suggestion of ``to take too much time," ``to dally," ``to be late." \emph{Tardé en
llegar porque había mucho tráfico} = ``I took a long time getting here
(I'm late) because there was a lot of traffic." \emph{El tren está tardando en
llegar} = ``The train is taking too much time (is late) arriving."

\section{\emph{tener}}

A megaverb, \emph{tener} is almost worthy of a chapter of its own.
Many of its uses are predictable renditions of ``to have" in English, including the indispensable \emph{tener que} for ``to have to": \emph{Tengo que irme}
= ``I have to go." It can also be used by itself in the negative to say ``I
don't have any" when the antecedent is understood. \emph{Dame dinero. No
tengo}. = ``Give me money." ``I don't have any." \emph{¿Dónde está su boleto?
No tengo}. = ``Where's your ticket?" ``I don't have one." Another common use of tener with an implied complement is the question \emph{¿Qué
tienes?} (or \emph{¿Qué tiene?}), which can translate as ``What's your (or his,
her) problem?" or ``What's the matter with you (or him, her, it)?" Another example: ``I don't like the house they've picked out for us." \emph{¿Qué
tiene?} (``What's wrong with it?")

Where you will have to pay special attention to \emph{tener} is in the
thousand and one expressions in Spanish that use \emph{tener} plus a noun for
what in English would be ``to be" plus an adjective. In English, for instance, you ``are cold," whereas in Spanish you ``have cold." Common
examples of this construction include \emph{tener hambre} (``to be hungry"),
\emph{tener sed} (``to be thirsty"), \emph{tener frío} (``to be cold"), \emph{tener calor} (``to be
hot"), \emph{tener sueño} (``to be sleepy"), \emph{tener paciencia} (``to be patient"),
\emph{tener cuidado} (``to be careful"), \emph{tener razón} (``to be right"), \emph{tener prisa}
(``to be in a hurry").
There are also another thousand and one expressions, many of
them quite colloquial, using \emph{no tener}. Here are some you should learn:

\bsk

\indu \emph{No tiene sentido}. = ``It doesn't make sense."

\indu \emph{No tiene caso}. = ``There's no point" or ``What's the point?"
(indicating futility)

\indu \emph{No tiene chiste}. = ``It's boring" or ``What's the point?" (indicating insipidness)

\indu \emph{No tiene (nada) que ver}. = ``That has nothing to do with it"
``That's irrelevant."

\indu \emph{No tiene vergüenza}. = ``He (or she) is shameless."

\indu \emph{No tiene lógica}. = ``It's illogical" or ``It doesn't make sense."
(indicating incredulity)

\indu \emph{No tiene ni pies ni cabeza}. = ``I can't make heads or tails of it."

\indu \emph{No tiene en donde caerse muerto}. = ``He (or she) is flat
broke."
(literally, ``doesn't even have anywhere to drop dead")

\indu \emph{No tiene remedio}. = ``There's no way out" or ``There's nothing to be done."

\indu \emph{No tiene} (\emph{ni la menor} or \emph{ni la más mínima}) idea. = ``He (or
she) hasn't got the faintest idea" or ``He (or she) hasn't
got a clue."

\section{\emph{tirar}}

The common verb for ``to throw," though \emph{aventar} and \emph{arrojar}
are also used quite a lot regionally. \emph{Tirar} also has the implication of
``to throwaway" or ``to throw out." \emph{Tira esa basura} = ``Throw that
worthless thing out." It is also used for ``to knock over," as in \emph{Tiré el
vaso} (``I knocked over the glass"). To convey ``to toss," as in ``Toss me
a pen," you could use \emph{aventar} (in the Americas) or \emph{echar} or even \emph{tirar},
but mostly you wouldn't use anything because it's considered rude
in the Spanish-speaking world to toss things (see Chapter 2). Unless
you're especially keen on seeing a particular object in flight, stick to
\emph{Préstame} (see \emph{prestar} above).

\section{\emph{tocar}}

``To touch," of course, but also ``to play" (a musical instrument) and ``to knock" or ``to ring" (at someone's door). In Casablanca
Bogie would have told the pianist, \emph{Tócala, Sam}. An extremely common use of \emph{tocar} that is often glossed over in textbooks is for ``to experience" or ``to be one's turn." A simple and good translation is elusive,
but examples will get the point across. \emph{Me toca}. = ``My turn." \emph{¿A
quién le toca?} = ``Whose turn is it?" \emph{A mi no me toca decirle}. = ``It's
not up to me to tell him." \emph{Al dueño le toca arreglar la casa}. = ``It's
up to the owner to fix the house." Sometimes the best translation involves ``to get": \emph{¿A quién le toca la última rebanada?} = ``Who gets the
last slice?" \emph{A ti te tocó la más guapa de las hermanas}. = ``You got the
prettiest of the sisters." \emph{A mi me tocó el más feo de los hermanos} =
``I got (stuck with) the ugliest brother." \emph{No me ha tocado verlo en concierto} = ``1 haven't had (gotten) a chance to see him in concert." And
so on. A final note on \emph{tocar}: its past participle, \emph{tocado}, is a common
synonym for \emph{loco}---as in the English sense of being slightly ``touched"
in the head. Children often adapt it to \emph{toca-toca}, translating perhaps as
``cuckoo."

\section{\emph{traer}}

\emph{Traer} is straightforward for ``to bring," except when ``take"
and ``bring" (\emph{llevar} and \emph{traer}) get mixed up (see Chapter 5).

\section{\emph{tratar}}

``To treat," of course, but far more commonly encountered
with \emph{de} and meaning ``to try." For some reason, though, many Spanish-English dictionaries refuse to acknowledge this fact, leaving the student to choose among \emph{ensayar, procurar, intentar}, and \emph{pretender}.
These are all worthy and acceptable verbs, of course, and someday you
might even want to learn them. But for now, remember: \emph{tratar de} =
``to try." \emph{Traté de dormir} = ``I tried to sleep." \emph{Tratamos de llamarte}
= ``We tried to call you." \emph{Trata de venir antes de las once} = ``Try to
come before eleven." Only when you're using ``to try" in the sense of
``to sample" or ``to test" should you abandon \emph{tratar de}; the correct verb
here is \emph{probar}. An awkward but illustrative example: \emph{Trata de probar
el vino blanco 1985}. = ``Try to try (i.e., make an effort to sample) the
1985 white wine."

\emph{Tratarse}, the reflexive form, is a useful verb for ``to have to do
with" or ``to be about." \emph{¿De qué se trata?} is the common way of asking
``What's it about?" (in reference to a film, a book, a scuffle, an argument, and the like). ``To treat" in the sense of ``to pay for someone
else" is usually handled by \emph{invitar}. \emph{Yo invito} = ``I'm treating."

\section{\emph{valer}}

Meaning ``to be worth," this verb is frequently encountered
in the stock phrases \emph{vale la pena} (``it's worth it") and \emph{no vale la pena}
(``it's not worth it"). In addition, you may find \emph{valer} handy for asking
prices. \emph{¿Cuánto vale?} = ``How much is it?" Another good use of valer,
preceded by \emph{más}, is to translate English phrases that use ``better" or
``had better." \emph{Más te vale irte} = ``You'd better get out of here." \emph{Más
vale preguntar} = ``We'd better ask." And there's the old standby \emph{Más
vale tarde que nunca} (``Better late than never"). In Spain and less so
elsewhere, \emph{vale} by itself is a common interjection for ``all right" or
``okay." In Mexico, especially, \emph{valer} with an indirect object pronoun
is a somewhat crude way of saying ``couldn't care less." \emph{Me vale} = ``I
couldn't care less." \emph{Le vale} = ``He doesn't give a damn." Its crudeness
comes from the fact that it's a shortened and therefore euphemistic
form of another phrase, which you'll have to read about in Chapter 10.

\section{\emph{venir}}

Meaning ``to come," naturally, and sometimes difficult to distinguish from ``to go" (see Chapter 5). This verb has some common
but unexpected uses, as in \emph{No viene al caso} (``That's beside the point").
\emph{Que viene} is especially worth learning in stock phrases like \emph{la semana
que viene} (``next week") and \emph{el año que viene} (``next year"). \emph{Venirse},
the reflexive form, is usually an innocent intensifier for \emph{venir}, but it
has sexual overtones in some countries and should be used with care.

\section{\emph{ver}}

As ``to see," this is a pretty straightforward verb. It can sometimes be confused with \emph{mirar}, since \emph{ver} also works as ``to look at" in
many cases where you might be tempted to use \emph{mirar}: \emph{Ese señor se me
queda viendo} = ``That man keeps looking at me." \emph{Estoy viendo tus
discos} = ``I'm looking at your records." \emph{Mirar} would work fine in the
first example, but in the second would suggest you are gazing at the
records as if waiting for them to do something. An extremely common
expression is \emph{A ver}, which is simply ``Let's see\ldots{}" but which is used
mainly to buy time while you think of a clever response. \emph{Vamos a
ver} means the same but is less common as an interjection or ``crutch
word." \emph{Tener que ver con} is the easiest way to translate ``to have to do
with." \emph{No tengo nada que ver con el asunto} = ``I have nothing to do
with this business." \emph{No tiene que ver} = ``That's irrelevant."

\emph{Ver} is sometimes used to express an opinion, in the sense of
how you ``see" or ``size up" a problem. \emph{La veo difícil} = ``It looks difficult to me." \emph{Verse}, the reflexive, covers almost all uses of ``to look"
that refer to the appearance of something or someone. You should etch
the phrase \emph{se ve} onto the end of your tongue and have it ready for such
common utterances as \emph{Se ve bien} (``It/he/she looks good") \emph{Se ve difícil}
(``It looks difficult"), \emph{Se ve bonito} (``It looks nice"), and so on. \emph{Se ve que}
plus a clause is an easy way to communicate ``You can tell that\ldots{}" or
``It's obvious that\ldots{}" \emph{Se ve que no han cambiado el agua en la piscina} = ``You can tell they haven't changed the water in the pool." \emph{Se
ve que son grandes amigos} = ``You can tell that they're great friends."

\section{\emph{volver}}

In the sense of ``to return" or ``to come back," \emph{volver} is interchangeable with \emph{regresar}. \emph{Volver} has another common use that you
will want to learn, though---one that \emph{regresar} does not share. \emph{Volver}
plus \emph{a} plus an infinitive is frequently found as a substitute for ``to repeat" or ``to do again." It will take some practice before you start to
use it properly---and as an alternative to \emph{otra vez}---but it's worth the
extra effort. Some typical examples of when to substitute: \emph{Gracias,
vuelvo a llamar más tarde} = ``Thanks, I'll call back (again) later." \emph{Si
vuelves a pedírmelo, no te lo voy a dar} = ``If you ask me for it again,
I'm not going to give it to you." \emph{Vuelve a intentar} = ``Try again."
\emph{Volver} cannot be used transitively---to ``return" a book to the library, for
instance, or ``to give back" a borrowed item. Use either \emph{regresar} or \emph{devolver} in these situations. \emph{Devuélveme a mi chica} (``Give me back my
girl"), for instance, was the name of a pop song and movie of a few seasons ago. \emph{Volverse}, finally, is one of the common ways to handle ``to
become" (or ``to get"). Skip ahead to Chapter 11 for details.


%%% Local Variables:
%%% mode: latex
%%% TeX-master: "main-keenan-breaking-out"
%%% End:
