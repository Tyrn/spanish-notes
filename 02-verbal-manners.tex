\chapter{Minding Your Verbal Manners}

Being polite is something many people, especially the young,
associate with visits to Grandma. In daily life, only bootlickers and
dweebs make a special effort to be polite; the rest of us are as we
are---take us or leave us.

Actually much of what goes for politeness is implicit in our
behavior and requires no special effort. Society has carved on our
minds the notion that if we don't follow certain preestablished, communal norms, it will use harsh and unfriendly epithets to describe us
behind our backs. So we take it for granted that you should open the
door for the elderly, avoid using expletives in public places, and refrain
from cutting in front of people on the exit ramp. We don't think of it as
being polite; we just do it because society says so.

Spanish-speaking society has its own set of unspoken norms
that you, as an outsider, won't have had beaten into your head from
birth. This means that you will have to pay attention to them and actually work at being polite. In addition, you will want to master the
subtleties of verbal manners that you now unthinkingly control in English. Consider these examples. Do you say the same sweet-sounding
phrases to a mean-faced bureaucrat as you do to a pleasant cashier? Of
course not. Nor do you use the same words or tone with an elderly
person that you use with someone more your age. Getting a feel for
subtleties in Spanish requires getting a handle on the language that is
,used to express manners.

In Spanish, there is really no good translation for "polite."
\emph{Cortés, amable}, and \emph{pulido} all come close but are better translated by
their English cognates: ``courteous," ``amiable" (or ``nice"), and ``polished." Instead, a Spanish speaker will talk about someone's \emph{educación}, which goes beyond a person's schooling to cover upbringing in
general and manners in particular. \emph{Es una persona educada} means that
so-and-so is a person who has good manners and is polite in dealing
with others. The person in question could be a grease monkey in the
neighborhood lube shop or a physics professor, a kindergarten dropout
or a triple Ph.D.; \emph{educado} simply means that the person has decent
manners. ``Well bred," though somewhat out of favor in modern egalitarian societies, conveys the right idea.

In passing, it's worth noting that \emph{rudo} is not the equivalent of
"rude," nor is it a good opposite for \emph{educado}. A Spanish-speaker would
probably use \emph{mal educado, sin educación}, or \emph{de poca educación}, or
would resort to \emph{grosero}. This last word, in context, refers to a foulmouthed individual, but in a more general sense it comes closer to
"rude." \emph{Se portó muy grosero con nosotros} = "He was very rude
to us."

Being polite, of course, is more than simply uttering elegant
phrases at key points in a conversation. To achieve the rank of \emph{educado} and skirt all that is associated with \emph{grosero}, you'll need one part
proper language and one part common sense. We've already addressed
the rudiments of good manners in our brief review of greetings and
good-byes (see Chapter 1). The phrases \emph{buenos días, buenas tardes},
and \emph{buenas noches}, in conjunction with \emph{gracias} and \emph{hasta luego},
will get you through 90 percent of your daily encounters. But that's
about all they'll do. If your goal is to go beyond the point of just getting by, you'll want a more in-depth look at the universe of Spanish
formalities.

\section{Meetings and greetings}

Politeness begins upon meeting a person. You meet 'em, you
greet 'em. The question is, how?

The answer depends on the person you're meeting. If there's
little or no chance of ever seeing the person again, you're safe and sufficient with the \emph{buenos días\ldots{} hasta luego} formula. If you think the
person may figure in your future life, or if the meeting is the result of
an introduction, you're expected to go beyond that. From a second encounter onward, except in the case of repeated encounters with employees (shop clerks, the doorman, waiters, the gardener), you should
usually employ a more personal greeting than just "good day" or "good
night."

Most students of Spanish have been drilled in the basic forms
of greeting, but they're worth a quick review. On being introduced to a
person, you have at your disposal a number of stock responses: \emph{mucho
gusto, tanto gusto}, and \emph{encantado} (or \emph{encantada}) in roughly descending order of frequency. For less formal introductions and situations
\emph{¿qué tal?} and \emph{hola} work well. Save \emph{muchísimo gusto} for someone
you've been dying to meet.

Once you've been introduced to a person, you'll naturally be
expected to greet this individual at all future encounters, be it on the
street or at a party. How you do this reflects (a) who the person is and
(b) who you are in relation to that person. Here are some of the common options, in more or less descending order of formality:

\bsk

1. \emph{¿Cómo está?} or \emph{¿Como está usted?}

2. \emph{¿Cómo le va?}

3. \emph{¿Qué tal?}

4. \emph{¿Cómo estamos?}

5. \emph{¿Cómo estás?}

6. \emph{¿Qué hay de nuevo?}

7. \emph{¿Qué pasó?} or \emph{¿Qué pasa?} (varies by country)

8. \emph{¿Qué me cuentas?} or \emph{¿Qué me dices?}

9. \emph{¿Qué onda?} (Mexico) or \emph{¿Quiúbole?} (mostly Mexico and
the Caribbean)

10. Slangy variations of the preceding, such as \emph{¿Qué pasotes?}
or \emph{¿Qué pasión?} (from \emph{¿Qué pasó?}), \emph{¿Qué hongos?} (from ¿Qué onda?),
and so on. Save these for the people whose street gang you're looking
to join.

\bsk

It bears noting that all these expressions---except \emph{¿Cómo está?}
and \emph{¿Cómo está usted?}---imply some level of friendliness. Said another way, if you're not on especially friendly terms with the person,
stick to the first expression listed above. It is the only appropriate
form for greeting a person whose social, familial, occupational, or political position warrants your respect, and thus is the safe choice for
those who aren't, strictly speaking, your buddies. \emph{¿Cómo estamos?} has
paternalistic overtones and is often used by older people to greet
younger ones---even if the younger ones are thirty or forty years old.
This greeting is also a safe one when you're on good terms with the
person but aren't sure whether to use tu or usted (more on that bugaboo in a bit).

A common way of sprucing up any greeting is to use the person's name, title, or both. The commonest titles are \emph{Don} and \emph{Doña}
(for older people) and professional titles like \emph{Doctor, Contador} (accountant or C.P.A.), \emph{Ingeniero, Profesor, Maestro} (any teacher or craftsperson and sometimes even mechanics and plumbers), and the ubiquitous \emph{Licenciado} (virtually anybody who wears a tie). These are used far
more often than their English equivalents, especially in the workplace.

As an example of how greetings work, let's take the case of
Juan Doe, assistant director in charge of flange production, arriving at
his office. For simplicity's sake, let's presume all of the males in his
workplace are named Alberto Alvarez and all the females Teresa Ruiz.
Juan parks his car on the street and walks toward the office building. In
order, he meets and greets the following:

\bsk

\inda The eighty-year-old doorman:

\indu \emph{Buenos días, Don Alberto.}

\inda The security guard:

\indu \emph{Buenos días.}

\inda The sixty-year-old elevator operator:

\indu \emph{¿Cómo le va, Doña Tere?}

\inda The receptionist:

\indu \emph{Hola, Tere. ¿Cómo estás?}

\inda A same-aged colleague in the hall:

\indu \emph{¿Qué tal, Alberto?}

\inda A younger colleague at her desk:

\indu \emph{Buenos días, Tere. ¿Qué hay de nuevo?}

\inda A visiting branch manager:

\indu \emph{Buenos días, Señora Ruiz. ¿Cómo le va?}

\inda A co-worker and best friend:

\indu \emph{¿Quiúbole, Beto?}

\inda The immediate boss:

\indu \emph{Hola, Alberto. ¿Cómo estás?}

\inda An older co-worker:

\indu \emph{¿Qué me cuenta, Don Alberto?}

\inda An employee:

\indu \emph{Buenos días, Alberto. ¿Cómo estamos?}

\inda The division director:

\indu \emph{Muy buenos días, Señor Alvarez. ¿Cómo le va?}

\inda The office boy:

\indu \emph{¿Qué pasó, Beto?}

\inda The factory owner and CEO:

\indu \emph{Buenos días, Don Alberto. ¿Cómo está usted?}

\inda The secretary:

\indu \emph{Buenos días, Tere. ¿Qué tal?}

\bsk

By now, as you might imagine, Juan is exhausted and it's time for his
coffee break.

A couple of general tips on greetings are in order. First, note
that when greeting so many people, you will naturally gravitate toward
new ways of saying the same thing. That's because saying \emph{buenos días}
to twenty-five consecutive people can be extremely boring.

Second, use nicknames only if the person is accustomed to being called that. In other words, pay attention to whether others call a
certain Jose "Pepe" before you call him that. Use generic nicknames---such as \emph{viejo, compadre}, and \emph{jovenazo}---only when you feel certain
that the person won't be offended by your informality.

Third, if you're a male, avoid affectionate pet names for female
friends, employees, and co-workers. In Latin America it is common to
hear men calling women co-workers and employees things like \emph{linda}
and \emph{cariño}. To most North Americans this treatment is patronizing at
best and at worst borders on sexual harassment. Men in Latin America
have been slow about concerning themselves with these matters, but
that's no reason for you to imitate them.

Fourth, if you're a woman, stick to more formal modes of address until you're sure that your friendliness won't be taken as encouragement by the wolfish male mind. It's unfortunate that you have to
consider this issue, but that doesn't make it any less real.

And fifth, greet everyone possible, especially when meeting
a group of people. If you've met the people before, you are expected
to take the trouble of greeting each of them individually. Not to do
so can be interpreted as an offense. The same goes for saying goodbye. If you're in too much of a hurry or there are simply too many
people involved, make sure you issue an all-encompassing \emph{¿Qué tal,
cómo están?} or \emph{Hasta luego}, and make sure it's interpreted as all-encompassing.

\section{\emph{Usted} versus \emph{tú}}

From the moment you greet a person, you will start to think
about whether so-and-so is an "\emph{usted} person" or a "\emph{tú} person." Native
Spanish speakers make this decision instinctively; you will have to
think it through, and repeatedly. It is a concept that doesn't have an
easy English equivalent, but it is usually not that hard to keep straight.
Perhaps the most functional system for converting the concept into
English is to use usted in Spanish with anyone you would address with
"Mr.," "Mrs.," "Ms.," or "Miss" in English. Mr. Brown is your neighbor, so you use \emph{usted} with him. Once you get to know him and call
him "Fred," you can switch to \emph{tú}. Your lawyer is Ms. Smith, so she's
an \emph{usted} person; if you call your lawyer "Betty," then you would also
probably use \emph{tú} with her. And so on.

This rule of thumb works even when you don't actually know
the person's name. Your waiter's name is Juan Perez, but you probably
wouldn't know that. If you did, though, would you call him "Juan" or
"Mr. Perez?" That will generally depend upon his age, your age, and so
on. If it would feel awkward calling a twenty-two-year-old "Mr. Perez,"
go ahead and use \emph{tú} with him. Then again, if the twenty-two-year-old
happens to be an undersecretary for tax policy in the finance ministry---or a traffic cop---you would probably call him "Mr. Perez" and
thus use \emph{usted}.

As a rule, people aren't bashful about telling you to use \emph{tú}
with them if they feel it's appropriate. Almost no one will tell you, to
use \emph{usted} when you're using tu---it's the equivalent of putting you in
your place. So when in doubt, you're far safer using \emph{usted} and waiting
until you're told to do otherwise.

The best way to choose the right form is to listen to the conversation around you. Let's say you're meeting a group of friends at a
restaurant. Upon arriving, someone you don't know is sitting with
your friends. You are introduced to this person as Betty (your name),
and she to you as Yolanda (her name). This is your first clue: almost
certainly you are expected to \emph{tutear} (use \emph{tú} with) this person. To play
it safe, though, you can return the greeting with a noncommittal \emph{mucho gusto} and keep your ears open. If Sam asks Yolanda, \emph{¿Dónde estás
trabajando ahora?} then you can feel safe using \emph{tú} yourself. Likewise, if
you're addressed with \emph{usted}, you should respond with \emph{usted}.

That said, there are a few cases in which the \emph{tú-usted} relationship is not reciprocal---that is, when you will use \emph{tú} with the person
and he or she will use \emph{usted} with you, or vice versa. Almost always
this is the result of a considerable age difference. You might use \emph{usted}
with your friends' parents, for instance, and they will likely use \emph{tú}
with you. (This, incidentally, fits under the "Mr. and Mrs." rule.) Turn
the formula around for your children's friends.

\section{Magic words}

Besides greeting people, expressing thanks is probably the
daily act that most requires a modicum of civility. \emph{Gracias} is the
obligatory comment, of course, but you can spice it up with a \emph{muchas}
before it or a \emph{muy amable} after it or both, as noted in Chapter 1. By
doing so, you'll sound both more polite and more fluent. \emph{Muy gentil} is
also used, but it sounds somewhat strained.

When being thanked, you can respond with \emph{por nada, de
nada, no hay por que}, or \emph{no hay de que}. They're all about the same
and mean "You're welcome." You will hear \emph{para servirle} a lot from
clerks and waiters, but don't use it yourself unless it is genuinely
your job or duty to serve the person, or unless you're feeling especially subservient.

Another common linguistic nicety is asking permission. In
Spanish, as in English, the most typical way of asking permission is
essentially to excuse oneself for having the audacity to ask. Thus we
ask a person's "permission" to squeeze by them in an aisle. Here are
some options for communicating your humility while asking someone
to move over or let you by, again in descending order of formality:

\bsk

1. \emph{Con permiso}

2. \emph{¿Me permite?}

3. \emph{Perdón}

4. \emph{¿Se puede?}

5. \emph{Comper'} (a slangy version of con permiso)

6. \emph{Hágase un poco para allá, por favor}

7. \emph{Abreme espacio} or \emph{Abreme cancha}

8. \emph{Hazte pa'llá}

\bsk

The first five expressions are formal enough for just about any occasion. \emph{¿Me permite?} is sometimes used ironically, as when someone is
clearly blocking the way; say it very innocently for greatest effect. \emph{¿Se
puede?} is also the common way to ask to see something in a store; it
presumes you will wait for an affirmative response before, say, taking a
painting down off the wall or an earring out of a glass case. Unless the
object you wish to see is obvious (you're pointing at it, for instance),
you should use the full phrase: \emph{¿Se puede ver?}

The last three expressions on the list convey informality or
rudeness, depending on the person you are speaking to and your
tone of voice. \emph{Hazte pa'llá}, for instance, can be used for either "Scoot
over a little" (with a friend) or "Get out of the way" (With a stranger).
\emph{Abreme cancha} is very slangy.

The phrase for "Coming through!"---as when carrying a two-hundred-pound sofa down the hall---is \emph{¡Golpe avisa! Excúsame}, incidentally, does exist as a Spanish expression, but it doesn't mean "Excuse me." Use something (anything!) else.

Here's a cultural tip. One nicety that many foreigners
have trouble learning is to say gracias when someone sends \emph{saludos}
through them to another person: \emph{Salúdame a tu esposa} ("My regards to your wife") is something you will hear constantly if you are
married, for instance. Your English-speaking reflex will be to answer
"Okay, I will," but in Spanish it is customary to thank the person for
this \emph{detalle} ("thoughtfulness" or "consideration").

As \emph{gracias} is the universal word for "thank you," \emph{por favor} is
all you will ever need for "please." Always remember to use it when
asking for something. If you're tired of it and want to flex some Spanish muscles, use instead \emph{si es tan amable}, generally placed after your
request: \emph{Un café, si es tan amable}. It means \emph{por favor}.

\section{Sugar versus saccharine}

Spanish speakers are famous for going a bit overboard with
their floridness and politeness, and it is a matter of opinion whether
foreigners learning the language should leap in after them. Use your
own judgment. For instance, Spanish speakers will often refer to their
house as \emph{su casa} or \emph{tu casa}---"your house"---meaning that now that
you know them, you should consider their abode to be yours. This can
get pretty confusing at times: someone will be giving you directions to
their home and at the end they will point to the map they've improvised and say, "And here is your house." You'll be tempted to respond,
"But my house is nowhere near there!"---until you recall this subtle
gesture of hospitality. It can get even more confusing when a nonnative speaker, from whom such a gesture is generally not expected,
tries to communicate it.

Slightly less silly-sounding in the mouths of foreigners is the
statement \emph{Está usted en su casa} when someone comes to visit. It's really nothing more than a way of saying "Make yourself at home" and
shouldn't be made to sound more grandiose than that. To use it really
correctly, save it for when a houseguest makes a simple request, such
as "Can I use the phone?" To this you reply, with a wave of the hand,
\emph{Estás en tu casa}.

A word on homes is in order at this point. In much of the
Spanish-speaking world, homes are considered private reserves, and it
is not especially common to receive an invitation to visit someone's
house. So first of all, don't expect to receive such an invitation. And
don't be surprised if your offers of hospitality---"Hey, Pedro, how about
popping by after work for a beer?"---are taken as a nice gesture instead
of as a genuine invitation. Furthermore, never drop in unannounced on
a friend in the Spanish-speaking world. Finally, you may be told while
visiting someone's house that some object you admire \emph{es tuya} ("it's
yours," "take it"). Not only shouldn't you accept such offers, you
should be careful about making them. Sooner or later a literal-minded
guest might take you up on it!

Other sweet-sounding phrases that border on the sickly sweet
include solemn declarations of humility (saying \emph{su servidor} instead of
"I"), exaggerated requests for cooperation (\emph{Tenga usted la bondad de
traerme un café}), and overly formal greetings (\emph{Me es grato tener la
oportunidad de conocerlo}). All sound like you're reading out of a
phrasebook---and a phrasebook written for royal weddings, at that.

\section{Asking and getting}

Politeness is particularly important when asking someone for
something, since otherwise you may not get it. Bicultural lore is full of
anecdotes about Party A getting something in twenty minutes that
took Party B three weeks to get, simply because Party A asked politely
and Party B was viewed as rude. Whether you plan to use your Spanish
to get government authorizations or good directions, your command of
these niceties is critical.

It's easy to sound rude or clumsy when making a special or
even routine request in Spanish, especially since most students of the
language are taught the imperative as the sole way of asking for things:
Thus many students of Spanish will tell their hostess, \emph{Tráigame un café} ("Bring me a coffee"), thinking that's the correct, formal, and polite way to petition one. A good hostess will bring you one anyhow,
but on some interior level she's thinking, "Sure, here's one in your lap,
schmuck." Fortunately, it's quite easy to sound natural when asking
for something, but most Spanish texts neglect to mention the most frequently used form of the imperative in daily life: the indicative. That
is, most people don't say \emph{Tráigame un café, por favor} but \emph{¿Me trae un
café, por favor?} An added advantage to this form is that you don't have
to worry about those strange imperative forms and can stick to the
tried-and-true indicative instead.

The \emph{¿me trae? + por favor} formula is all you'll ever need for
restaurants and the like. But this use of the indicative also works for
most other situations and can employ a number of different verbs in
addition to \emph{traer}, especially \emph{permitir, dar, prestar}, and \emph{regalar}. \emph{Permitir} is the most formal: \emph{¿Me permite un cigarro?} will get a cigarette off
your future father-in-law. \emph{Prestar} and \emph{regalar} are the least formal, and
the latter implies you're going to keep what you're given: \emph{¿Me prestas
tu pluma?} means "Lend me your pen." \emph{¿Me regalas tu pluma?} is "Can
I have your pen (forever)?" In colloquial use, \emph{pasar} is common and
equates with \emph{prestar}: \emph{¿Me pasas el cenicero?} means "Pass the ashtray
over here, would ya?" Another common formal way of asking for
things is equivalent to "May I" in English: \emph{¿Puedo tomar un cigarro?}
("May I have a cigarette?").

\section{Etc.}

Phone Spanish is generally even more polite than face-to-face
Spanish. Listening to it, you will hear a lot of phrases like \emph{si es tan
amable} and \emph{si no es mucha molestia} tacked on to simple requests.
Once you get the hang of it, you'll be doing the same thing. To ask for
someone on the phone, you can be formal---\emph{¿Me puede comunicar can
el Sr. So-and-so?}---or informal---\emph{¿Está por ahí So-and-so?}. It will depend on whom you're calling. To say "Speaking" when someone calls
and asks for you, just say \emph{Él} (or \emph{ella}) \emph{habla}.

Interrupting is considered bad form in any language, of course,
but some foreigners seem to do it more when the conversation they are
interrupting is in a foreign language---i.e., one they don't understand
that well. Be aware of this, and be conspicuously polite when you do
need to interrupt, directing yourself to the party whom you are momentarily cutting out of the picture. Act, in other words, as if you were
cutting in on a dance. \emph{¿Me permite un momento?} you can ask.

Certain situations call for specific graces from you. Some are
verbal graces and some aren't, but all come under the heading of \emph{buena educación}.

\bsk

1. When you pass by someone who is eating, and presuming
you are at least vaguely acquainted with the person, wish him or her
\emph{provecho} or \emph{buen provecho} ("bon appetit," as we'd say in English).
Don't say it to total strangers in restaurants, though.

2. North Americans like to toss things. "Here's the pencil you
asked for," they'll say as they wing a fine-pointed no. 2 lead pencil by
your ear. "This is no good," they'll say as they crumple a sheet of paper
and lob it at the wastebasket. In the Spanish-speaking world, this behavior is considered barely short of barbaric.

3. Ditto for pointing at people.

4. Be conscious of not "giving your back" to people. Many
people from non-Latin cultures do it without intending offense. But in
Spanish-speaking cultures it's common to see people realize there is
someone behind them listening, do a half-turn, say \emph{perdón}, and continue speaking in a way that includes the previously excluded person.

\section{Sweet sorrow}

Departures are easily handled in Spanish by any of a number
of words, but \emph{adiós} and \emph{hasta luego} are sufficient for almost every circumstance. \emph{Adiós}, as you may have been taught, is generally used for
more lasting farewells. \emph{Nos vemos} is also a common colloquial send-off, equating with "See you later." \emph{Ciao} (or \emph{chao}) and \emph{bye} are making
rapid headway into Spanish from their respective languages.

Slightly more formal leave-taking makes use of phrases like
\emph{Que le vaya bien}, used only when the person you say it to is doing the
leaving. More fancy is \emph{Vaya con Dios}, but unless you're a nun or a
priest, it comes close to the saccharine category. When taking leave of
a group or passing by people on your way out, it's considered nice
manners to say \emph{Con permiso} to the people who are staying on. Usually it is
used for people whom you didn't actually say hello to on your way in.
Leaving the dentist's office, for instance, you say \emph{Gracias} to the dentist
and the receptionist and \emph{Con permiso} to the people sitting in the waiting room.

For less formal departures and with the younger set, you can
say things like \emph{Cuídate} (roughly, "Take it easy") and \emph{Pórtate bien} ("Behave yourself") to people when they are leaving. If it's nighttime and
the person is presumably leaving to go home to sleep, you can say \emph{Que
descanses} ("Rest up"). As a rule, try to respond with a farewell that is
different from the one used by the other person. If someone says \emph{Hasta
luego}, you say \emph{Nos vemos}; if they say \emph{Nos vemos}, answer \emph{Hasta luego}.

