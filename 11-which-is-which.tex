\chapter{Which Is Which?}

For the foreigner, the path to fluency in Spanish is marked by
constant dilemmas. Ser or estar? Imperfect or preterit? Subjunctive
or indicative? Do you want the word on the left or the word on the
right? Do you want to keep the word you have, or do you want to trade
it for the word behind Curtain Number One? Or do you want to swap
it all for the word Linda is bringing down the aisle in a lovely giftwrapped box?
For foreigners learning English, one of the greatest bugaboos
is the tendency for one English word to have three or four (or a dozen)
different meanings. The problem is worse with verbs, which team up
with prepositions in a million-odd ways to produce new meanings altogether. As an exasperated Mexican friend once noted, "What can you
expect from a language where first you chop a tree down and then you
chop the same tree up?"
The converse of these difficulties are your difficulties, as a native English speaker, in learning Spanish. Every single English word, it
seems, requires a bundle of totally different foreign words to describe
it. And worse, you're expected to learn them all!
In this chapter, we look at some of the English words that
don't have a single Spanish equivalent. Instead, you must pause in your
palaver and search for the correct Spanish word. And while no book
can present and resolve all possible word-choice dilemmas, this chapter can help you sort through the choices so you will know which ones
you need to learn.

\section{\emph{back}}

%BACK
The body part, as you probably know already, is la espalda..
But what about all those other uses of "back?" For these, you'll need
to study tras, atras, and detras, plus de regreso and de vuelta and, last
but far from least, fonda (or al fonda).
Generally, to express "in back of" or "behind," atras and detras are the words you'll use most. Tras, for all intents and purposes, is
just like detras and can be safely discarded for the time being. As for
the difference between atras and detras, native speakers will tell you
they distinguish between these two words "by which one sounds right."
That is true, but of little help to you, the student. So we must turn to
rules, though they are far from perfect, to clarify the situation.
Here's the main rule to remember: use detras de instead of
atras de, and you'll usually be right. In fact, it is hard to think of a case
where atras de would be correct and detras de wrong. So you are free
to forget about atras de altogether. Or almost. One exception is in reference to someone coming behind or after someone else. Here, detras
de suggests "in pursuit of"; atras de just means "back there somewhere" or even "later." iD6nde esta tu mama~ Viene atras de mi.
"Where's your mom?" "She's coming behind me (later)." Change atras
de mi to detras de mi and you have "She's coming after me"-in
hot pursuit.
The underlying intuitive difference between atras and detras
is hard to pin down. In most cases, detras suggests a spatial relationship, or one thing "behind" another or being blocked by another. Atras
is much vaguer, meaning "in back/' "back there somewhere" or "behind (the speaker)." You might say, for instance, that there is a mountain detras de mi casa. But if someone inside the house asks where the
mountain is, you might say alla atras.
In fact, if we accept detras de as (almostl always the correct
choice, then we could invent a second, equally imperfect rule to cover
atras: when you don't need the de, use atras. That is, when standing
alone as an adverb, atras is correct and sufficient. This usually covers
backward motion, as in va bacia atras or va para atras. In keeping
with its vague "backness/, atras is usually the word of choice in figurative expressions. Se esta ecbando para atras = "He's backing out."
No mires atras = "Don't look back."
For other uses of "back/' de regreso and de vuelta both mean
"back" in the sense of "returned." Both are used with estar. Estd de
vuelta despues de su viaje = "He's back now after his trip." Of course,
regresar and volver convey the same thing.
Fonda, or al fonda, is one of the handiest words for "back"
WHICH IS WHICH? 137
and one you should try to learn. EI fonda, as a noun, refers to the
"back" of a bus, movie theater, or building (as well as the "bottom"
of a glass, swimming pool, and so forth). You'll hear al fonda for "to
the back" or "all the way in the back." Ask the bus driver or theater
usher if there are any seats and the answer may be Sf, al fonda. Atras,
especially with cars, can be used this way as well. It means something
like "in the back" as opposed to "at the very back," which is al fonda.
Al fonda is extremely common when giving directions, particularly
indoors. Ask where the bathroom is in a restaurant and the reply may
be, for instance, Al fonda ala derecha ("All the way back on your
right").

\section{\emph{become}}

%BECOME
This is perhaps the gold-medal winner for troublesomeness
in the English-to-Spanish olympics, especially when "to become" includes the sense of "to get." As a rule, ponerse is the handiest word for
"to become," but you should learn at least some of the others. Ponerse
in general is for fleeting states of mind or conditions, as in ponerse furioso ("to get/become furious"), ponerse viejo ("to get/become old"),
ponerse nervioso ("to get/become nervous"), and so on.
For longer-term, usually nonreversible conditions, hacerse is
often the verb you want. Me hice rico = "1 became (got) rich"; EI penique se esta haciendo inutil =' "The penny is becoming useless"j EI
nuevo baile se esta hacienda, popular ("The new dance is becoming
popular"). The distinction is' subt,le, sometimes to the point of near in-
.visibility. Thus one can also hacerse viejo, but it suggests "becoming
an old person" or "becoming elderly," whereas ponerse viejo hints at
"starting to feel (or look) old."
Volverse, another word for "to become," is perhaps best translated as "to turn into." Se ha vuelto una verdadera molestia = "He's
become (turned into) a real nuisance." It involves a more sudden and
unexpected change than hacerse, though it shares with that word a
change into a more or less lasting state. Se volvi6 rico, for instance,
can also be used to say "He got rich," but it gives the impression of
overnight wealth-winning the lottery, perhaps, or inheriting a fortune-and also carries an intimation of criticism, as if the change were
undeserved as well as unexpected. Volverse is also the only phrase to
be used for "to go crazy" in the permanent sense: volverse loco.
Both transformarse and convertirse can also be rendered
"turned into," but usually in connection with a more thoroughgoing,
physical change. EI agua se transforma en vapor = "Water becomes
(turns into) steam." Clark Kent se convierte en Superman = "Clark
Kent becomes (turns into) Superman." Sometimes these phrases can
138 BREAKING OUT OF BEGINNER'S SPANISH
cover more figurative conversions: Se ha convertido en un buen escritor = "He's become a good writer." Llegar a ser, meanwhile, works
well as "to turn out to be," as in La formula llego a ser muyutil =
"The formula turned out to be very useful." This phrase also covers
"to become" in the sense of "worked his (or her) way up to." Despues
de muchos esfuerzos, llego a ser medico. = "After a lot of hard work,
he became a doctor." In many cases, an easier construction can be
made with the verb resultar, which also means "to turn out." Este libro resulto ser muy interesante. = "This book turned out to be very
interesting."
A very common but often overlooked way of saying "to become" in Spanish is through the use of reflexive verbs, especially those
that are little more than an en- or a- prefix attached to an adjective or
noun: acalorarse ("to get/become hot"), enfriarse ("to get/become
cold"), entristecerse ("to get/become sad"), acercarse ("to get/become
close"J, and so on. Sometimes no prefix is needed with the adjective,
as in mo;arse ("to get/become wet"J and cansarse ("to get/become
tired"J. Pay attention to which nouns and adjectives can be made into
verbs and how, and you'll be well on your way to conquering the "get/
become" problem.

\section{\emph{burn}}

%BURN
In English, this is a fairly simple if destructive processj in
Spanish it's a little more complex. Arder refers to the actual act of
burningj think of it as "on fire" or "burning up." Thus houses and
candles arden, and so do fever victims: Estoy ardiendo en calentura.
Out of context, Estoy ardiendo can sound like double entendre-"I'm
on fire, baby, sweep me off my feet." Figuratively, arder comes into
play in the expression esta que arde, meaning someone is "burning
mad" or a situation is "boiling over."
Incendiar, another choice for "to burn," is usually used for
things that should not be burning: forests, buildings, the lawn, the
lawn furniture, and so on. Think of it as "to set on fire." Even in the
case of pasion incendiaria, the implication is that the fire has gotten
out of hand and may end up being destructive.
Quemar also translates as "to burn," but generally brings to
mind (in the reflexive quemarse) "to burn up" or "to burn down," (another case of contradictory prepositions in English saying almost exactly the same thing). If you go away for the weekend and come back
to find your toolshed in ashes, a neighbor might explain simply, Se
quemo. Quema in its nomeflexive form is the usual way to convey the
transitive "to burn (something)." Estas quemando el bistec = "You're
WHICH IS WHICH? 139
burning the steak." (If you were to use incendiar here, you would be
saying, "You're setting fire to the steak.") The general idea behind quemar is "to consume (or damage) by fire." It is the correct verb for when
one thing "burns" another: the sun quema your skin, the stove quema
your fingers, and so on. The painful and unappealing marks left on
your skin to mark these mishaps are called quemaduras ("burns"). To
say "I burned myself," whether in the sun or the kitchen, you would
say Me queme.

\section{\emph{corner}}

%CORNER
Ninety-five percent of the time, just use esquina for "corner."
For that matter, use it for "intersection" too, although intersecci6n
is increasingly heard these days. For a "crossroads," or intersection
of highways, use either entronque, encrucijada, or cruce de caminos.
Dictionaries will give rinc6n for "corner" as well, but it generally refers to an inside corner where two walls meet-the corner of a room,
in other words. By extension, it is used to mean a "nook" or "cranny,"
and is a favorite word to assign to cocktail lounges, piano bars, and
the like.

\section{\emph{date}}

%DATE
Here you'll have to slow down and analyze what exactly you're
trying to say. Do you mean "the date," as in July 13, 1984? Then the
Spanish word you want is fecha. If you mean "date" as in "appointment," you have a couple of words to choose from. A "date" of the sort
that thrills teens (and, let's be frank, some postteens as well) is almost
universally a cita. A cita is also what you have with a doctor or lawyer, however-in other words, an "appointment." That is, in Spanish
you can have a cita with your doctor without upsetting your spouse. A
compromiso is also an "appointment," usually an unspecified one. It
can almost mean simply "something to do" and implies an obligation
to be present at a certain time-but no more. Thus, to escape from a
dreary luncheon, you can excuse yourself with Perd6n, tengo un compromiso. All you're really saying is "Sorry, gotta go."

\section{\emph{fail}}

%FAIL
The best all-purpose word for "to fail" is fracasar, though one;
1952 edition of a heavyweight Spanish-English dictionary I consulted
140 BREAKING OUT OF BEGINNER'S SPANISH
defines it as "to crumble, to break into pieces: applied commonly to
ships." A ship that crumbles into bits would certainly meet my definition of "failure," but fracasar in common usage covers-much more
ground than that. Fallar is also used for many general senses of "to
fail," though it seems a little kinder and less permanent than fracasar.
Fa1l6 en su intento suggests "He did not succeed"; Fracas6 en su intento hints "He was a total flop." Often you'll hear fallar in the form
estar fallando to mean "not to be working well," "to be on the blink,"
in reference to some appliance or gizmo. When someone or something
"fails" you in the sense of not living up to your expectations, either
fallar, defraudar, or decepcionar can be used. To "fail" an exam or a
course, reprobar is much used in the Americas, while suspender is
more frequent in Spain. Thus un suspenso on your exam in Madrid is
beyond suspense: it's an F. In Mexico City receiving an F would be expressed as reprobar e1 examen or ser reprobado par e1 profesor.

\section{\emph{front}}

%FRONT
This word opens an unruly can of worms on its journey into
Spanish. The basic problem is the excess of believable cognates, which
then have to be sorted through. For instance, you can say there is a
fountain a1 frente de 1a casa, enfrente de 1a casa, frente ala casa, or
de frente a 1a casa, and you won't necessarily be saying the same
thing. What's the difference?
A1 frente de is usually the expression for the English "in front
of." It means "at (in) the front part of (something)." In the earlier example, a fountain that is a1 frente de 1a casa is on the property at the
front of the house. Next come a host of prepositional expressions that,
in many circumstances, mean "in front of" but generally only in the
sense of "facing." Enfrente de is one of these expressions, and it is better regarded as a trickster (see Chapter 3). Think of it as asynonym for
"facing" or "across the way." A fountain that is enfrente de 1a casa is
likely to be across the street from the house. Frente a means much the
same and should be viewed as just as unreliable as enfrente de in most
cases. Sometimes the act of "facing" is implicit. Lo tienes frente a los
ojos = "You have it in front of your eyes"-that is, "staring you in the
face." De frente a, finally, states very clearly that something is "face
to face," literally "with its front facing (something)." Estoy parada de
frente a1 sol = "I'm standing facing the sun."
Perhaps the safest word of all for "in front of" is de1ante, usually followed by de. Generally, whenever you can use "ahead" or
"ahead of," you should probably be using de1ante or de1ante de. But it
also covers a wide range of "in front of" situations. The fountain can
WHICH IS WHICH? 141
be delante de la casa, for instance. Bob se sienta delante de mi en la
c1ase = "Bob sits ahead of me in class."
The classroom example may prove, well, instructive. Ana's
seat is in the third row; Bob sits delante de ella. Pedro sits in the front
row, delante de Bob and al frente de la fila ("at the front of the row")
and frente ala maestra ("in front of-i.e., facing-the teacher); when
Ana gets up to read her book report, she goes al frente del salon and
stands facing the class, or de frente a la c1ase. Ana's sister is in the
classroom across the hall, or enfrente del salon de Ana.
Finally, whereas atrds is "in back" (see "Back"), adelante is
usually "up front"-in the front seat of a car, for instance. To refer to
"the front" or "the front part" of something, el frente can usually be
used. El frente de la tienda estd sucio = "The front of the store is
dirty." La frente, in the feminine, refers exclusively to "the forehead."

\section{\emph{funny}}

%FUNNY
Does it make you laugh or wonder? That's the first question
you'll need to ask (and answer). If it's the former, you can choose from
chistoso, gracioso, and comico. If the latter, stick to extraiio and raro.
Pick your favorite and use it; they are virtually interchangeable. Anything having to do with "fun," incidentally, should be carefully distinguished from things that are "funny," since what's fun isn't always
funny. "Fun" needs either the adjective divertido or the noun diversion to describe it.

\section{\emph{happen}}

%HAPPEN
"To happen" can almost always be covered by some form of
pasar in Spanish. Other counterparts exist-including acontecer, ocurrir, suceder, and acaecer-but there's really no need to learn them
other than to recognize them when you hear them. You will have to
learn some of their noun forms, however, since pasar doesn't have one.
Acontecimiento, hecho, and suceso are three of the most frequently
encountered words for "happening," "occurrence," and "event."
Evento will be understood-and is gradually being incorporated in this
sense-but technically it means something that happened by chance
or with no advance planning.

\section{\emph{here}}

%HERE
Spanish is funny. Even when you know you're "here," you
have to know which "here" to use. Two words, acd and aqui, cover
142 BREAKING OUT OF BEGINNER'S SPANISH
the situation. Which is which? Textbooks generally promote the idea
that aca is the correct word for verbs of motion, as in Ven aca ("Come
here"), while aqui is the word of choice for stationary presence in a
given place, as in Estoy aqui ("I'm here"). After hearing one person on
the street shout Estoy aca, though, you might wonder why grammarians even bother with these distinctions.
The comparative forms of aca come into play when giving
guidance to, say, someone backing a truck down an alley. Mas aca
means "A little more this way," No tan aca means "Not so much this
way," and j¥a! means "Just right, hold it." jPraaac! ("Crunch!") means
you won't be asked to help truck drivers back down alleys anymore.
Finally, a very common use of "here" in English is to say
"Here you are" or "Here you go" when giving someone something (for
example, paying the cab driver the fare). In Spanish you can say pretty
much the same thing with aqui: Aqui esta and Aqui tiene both work,
as does Tome ("Take"). A somewhat archaic expression that still creeps
into written Spanish is He aqui, and some dictionaries will tell you
that it means "Here is ...." Actually, it's closer to "Behold!" Before
trying it out on the cabbie, try to imagine yourself saying "Behold, the
ten pesos!" in English. Then go back to Aqui tiene.

\section{\emph{hot, warm}}

%HOT, WARM
Native English speakers aspiring to fluency in Spanish should
be aware of the latter language's hierarchy of "heat" adjectives. Apart
from the several words representing various degrees of heat, different
words are used depending on the object in question. A soup fresh off
the fire, for instance, will drop in temperature from hirviendo ("boiling
hot" or "scalding") to caliente ("hot") to calientita ("pretty hot") to
tibia ("lukewarm") to temp1ada (not hot but not cold either). A cold
beverage goes from he1ada to fria, fresca, a1 tiempo ("room temperature"), tibia, and caliente. For weather, you can describe the day in order of ascending temperature: he1ado, frio, fresco, temp1ado, calientito, caliente, and ca1uroso. A good word for warm days is calido, often
in the sense of "unseasonably (or pleasantly) warm." As expressive alternatives for ca1uroso you can say Hace un calor barbaro (or tremendo, bochomoso, sofocante) or simply Hace un ca1m6n.
"Hot" in its many figurative senses in English can be a dangerous concept to convey in Spanish. Caliente in reference to other
people, especially of the opposite sex, is risky. Think of caliente as
"in heat" rather than "hot-looking," and you'll get the general idea. As
for the common novice mistake of saying Estoy caliente instead of
WHICH IS WHICH? 143
Tengo calor ("I'm hot"), those of you who have used it know the consequences: usually belly laughs all around, sometimes an unwanted offer.

\section{\emph{hurry}}

%HURRY
Several choices present themselves for the concept of "to
hurry," and mostly it's a matter of choosing one and using it. There are
some differences, though. Your choices are (the envelope, please): tener
prisa, apurarse, and darse prisa. The first means "to be in a hurry,"
and Tengo prisa is a basic Neurotic-Speak expression. Apurarse is less
common except in the phrase iApurate! (or iApurese!), meaning "Hurry
up!" and then only in the Americasj apresurarse replaces apurarse east
of the Atlantic. The adjective apurado is often heard, but implies more
"harried" or "frantic" than just "in a hurry." Darse prisa is sort of like
"to make oneself hurry" or "to get a move on," and iDate prisa! is
practically interchangeable with iApurate! for urging someone to get
the lead out. When you're in such a hurry that you can't even stop to
say "Hurry up," appropriate grunts include iYa! iAprisa! iDeprisa! and
in Mexico iAndale! and iOrale!

\section{\emph{look}}

%LOOK
"To look" in English covers a lot of ground. "To look for" is
not the same as "to look at," for instance, and "to look good" is a different concept from "to look like." As usual, one English verb can
combine with a score of prepositions to produce dozens of distinct
meanings-each of which in Spanish may require a different verb. The
most common Spanish words for the concept "to look" are buscar ("to
look for") and mirar ("to look at").
"To look" by itself is verse, a verb you should learn to use. Te
ves bien = "You look good." Se ve horrible = "It (or he or she) looks
horrible." ieOmO me veot = "How do I look?" And so on.
"To look like"-with the meaning "to resemble" or "to remind one of"-is either parecer or parecerse a. Ese sefior se parece a
mi padre = "That man looks like my father." When "to look like"
means "to look as if," parecer by itself is enough. Parece que va a
llover = "It looks like (as if it is going to) rain."

\section{\emph{miss}}

%MISS
Your husband goes away on a bus and you miss him. He
misses the bus, so you don't miss him. On his way to the bus, another
144 BREAKING OUT OF BEGINNER'S SPANISH
bus just misses him, causing him to miss the bus and you almost to
miss him very much. Still with me? There's obviously a lot of missing
going on in this world. And in Spanish, you must take care to differentiate what you do to your husband and what the bus does.
Let's deal first with the verb that expresses what you feel
when any loved one goes away. This sense of "to miss" in the Spanish
of the Americas is handled by extranar. In Spain, you'll hear echar de
menos. As in English, you can "miss" not only a person but your bed,
your favorite brand of candy bar, or any once-regular activity, like your
nightly mud bath. All are covered by extranar and echar de menos.
Now let's look at "to miss" in the context of missing a bus,
a plane, a party, a movie, a lecture, and so on. In Spanish, this idea is
handled by perder, which of course is also the word for "to lose." Perdimas el tren = "We missed the train." If you miss something that
you were expected to attend, you would use faltar a, as in Falte a la
c1ase de espanol ("I missed the Spanish class"). Think of it as "to
miss" in the sense of "to be absent from." When something is "missing" or "absent," faltar is also the word of choice. Faltan once d6lares
de mi cartera = "Eleven dollars is missing from my wallet."
Then there is "to miss" in the sense of the close call-"The
bus barely missed the parked truck"-which is a fairly tricky concept
to translate into Spanish. Some dictionaries might have you use errar,
no acertar, or no dar en el blanco, but these imply that the bus was
trying deliberately to hit the truck-and may even have gone back for
a second shot. These are all useful expressions, but they are best saved
for when your jump shot goes astray or when you guess something
wrong. To convey the idea of the bus "missing" the truck, you would
have to resort to pasar rozando ("to graze") or pasar cerca ("to go by
close"). To get the point across even more clearly, you would need to
trade in "miss" for "almost hit." Un autobus par poco (or casi) atropella ami marido = "A bus just missed (running over) my husband."

\section{\emph{next}}

%NEXT
At some point in your progress as a student, you will realize
that pr6ximo is not the only word for "next." Soon afterward, you will
realize that it's often not even a very good word for "next." What are
the other words, and when are they used? For starters, you're safe with
pr6ximo when "next" means some indefinite "next time." In fact
pr6xima vez ("next time") is one of its commonest uses. But pr6ximo
does not work for "the next day," which is el dia siguiente (or al dia
siguiente). It works again for "next week" (la pr6xima semana), and
fairly well for "next year" (el pr6ximo ana), but you are probably more
WHICH IS WHICH? 145
likely to hear la semana que entm (or entrante) and el ano que viene.
Somewhere someone has probably thought up an explanation for these
deviationsj ignore it and just learn the phrases.
Another slight deviation to be alert for is the use of otro for
"next," which you will run across constantly. Al otro dia can mean
"the next day," for instance. More common is la otm for "the next"
street, stoplight, or corner when being given directions. Esta en la otra
calle = "It's on the next street over." What tends to confuse beginners
is that the noun is often implied, not stated. De vuelta no en esta sino
en la otm means "Don't turn here (at this corner) but at the next one."
For "next to" in a physical sense, pr6ximo is flat-out wrongj
junto a or allado de is correct. La tienda esta allado de su casa or La
tienda esta junto a su casa = "The store is next to his house." Note
that as an adverb of place, junto does not change with the gender of
the subject. "The girl next door" is either la muchacha de junto or la
muchacha de allado. Colloquially and quite commonly you will hear
pegado (literally, "stuck to") for "right next door."

\section{\emph{office}}

%OFFICE
Such an easy concept-and such a wealth of Spanish words to
cover it. Let's start with-and discard-oficio, which applies only to
"office" in the symbolic, conceptual sense: "the office of vice president." For a room where someone works, several words could fit, depending on what you want to say. The most generic term for "office" is
. oficina, and this can be used safely about 90 percent of the time. Getting pickier, we can distinguish between the larger office under the
boss's control and the actual office where he or she works, which could
be called his or her privada ("private office"). Doctors and dentists
have special offices called consultas or consultorios. Lawyers dwell in
bufetes, which works for both the physical office and the "firm." A
despacho is a workplace, usually nongovernmental, that is often involved in accounting or administration, though lawyers can labor in
despachos as well. A taller is a "workshop" where manual labor is
done. Taller mecanico means "worskhop" or "garage" and has nothing
to do with the height of the mechanic who works there.

\section{\emph{old, older}}

%OLD, OLDER
The main problem with this concept is the word viejo, which
you should probably be taught from early on not to use in reference
to people.
146 BREAKING OUT OF BEGINNER'S SPANISH
The other problem is the concept of "older," which in English
can refer to a two-year-old in comparison with a one-year-old, In Spanish, viejo means "old" in the very specific sense (for humans) of having
been on the planet for more than fifty Of sixty years. Mas viejo, naturally, means "older," in the sense of having been among us for sixtyplus years. Therefore, when referring to your "older brother," you can't
say he's your hermano mas viejo unless he is, officially, a potential
Gray Panther. He is your hermano mayor-your "greater brother," in
age if in nothing else. A very common way of phrasing this idea is with
grande. iTienes hermanos mas grandest = "Do you have older siblings?" A "grown-up," in fact, can safely be translated as una persona
grande-"a big person."
I say "safely" because, as always when talking about age, there
are numerous ways to stick your foot in your mouth. Almost all of
them have to do with the word viejo, which, as noted, you would do
well to forget except in reference to buildings and such. With senior
citizens, the tendency in all cultures is to avoid calling them "old"-
thus terms like "senior citizens." There's nothing wrong with being
old-most of us take special precautions to become old ourselves-but
nobody seems to like being reminded that they've achieved this status.
So in Spanish you say things like Mi abue1a ya es muy grande, Es
una persona ya grande or Es una persona mayor instead of saying Mi
abue1a es vieja. Other words given for "old" by the dictionaries are
also borderline rude except when talking about, say, pyramids. These
include anciano and antiguo, especially.
Mi viejo and mi vieja, meanwhile, come in for frequent colloquial use. They can mean "the old man" (or father) and "the old lady"
(or mother), but more commonly they refer slangily to one's "guy"
(husband, boyfriend, or lover) or "gal" (wife, girlfriend, or lover). Viejo
(without the mil is also used slangily in many countries for "buddy" or
"man," regardless of the age of the person addressed. Vieja is not used
this way ever, though, since una vieja is a rude way of referring to a
woman, similar to "a broad" or even "a bitch."

\section{\emph{order}}

%ORDER
In this case, your choice is the same word but with different
genders: 1a orden and e1 orden. Which is which? Well, if you're traveling through Spanish-speaking areas, you should first learn the word for
"an order" of the sort that you submit to a waiter or waitress at a restaurant. It is 1a orden, the feminine form, and you should invent some
mnemonic gimmick to make this stick. E1 orden, the masculine form,
means "the order" of things-that is, their organization or sequence.
WHICH IS WHICH? 147
Ordenar as a verb, to mean "to order (in a restaurant)," is being used
increasingly in Latin America, but the proper word for this is pedir.
Ordenar really only means "to put one's things in order." "To order
(someone about)" would be mandar. A final note on "order": "in order
to" should be handled with para, never with an expression using
orden.

\section{\emph{piece}}

%PIECE
This is a tricky one, and even most Spanish speakers won't be
able to explain the subtle distinctions-though they will invariably
and instinctively use the correct form. First of all, banish pieza from
your mind for all but a few usages, usually having to do with a "part"
of a machine or car. Think of it as a self-contained "piece," and not so
much as an arbitrarily determined or broken-off "fragment" of some
larger whole.
For fragments, the two most common words are trozo and
pedazo. Sometimes, these two are interchangeable-with a "stretch"
of highway, for instance, either trozo or pedazo will work (though
tramo would be an even better word here)-and other times they're
not. The difference seems to reside in whether the piece in question is
a plainly three-dimensional chunk (trozo) or not (pedazo), whether the
piece in question is useful in itself (trozo) or not (pedazo), whether it's
an indistinct division between part and whole (trozo) or a fairly clear
one (pedazo), and so on. An object breaks into pedazos, and you noro mally would ask for a trozo of something at the dinner table.
In this last setting, however, bear in mind that many comestibles come in rebanadas ("slices"), and that this is the preferred word
for "pieces" of bread, cake, pizza, and so on. A cute word for a "little
piece," "smidgeon," or "tad" is nadita-"a little nothing." The word
taiada is also used widely to mean "piece" or "slice," but in many
areas it also carries the connotation of a "cut" or "percentage" of an
illegal or dirty business.

\section{\emph{plan}}

%PLAN
As a noun, "plan" equals plan pretty consistently, but keep in
mind the difference between plan and plano (a building's "floorplan"
or "design," a town "map"). To be in a certain plan is a colloquial way
of saying to be "in a phase," to be "on a kick." Estoy en mi plan limpiador = "I'm in my cleaning mood." For the verb "to plan" it is almost always better to forget about planear and use pensar instead, un-
148 BREAKING OUT OF BEGINNER'S SPANISH
less you are consciously planning a scheme or a strategy. Pensamos
irnos manana = "We're planning to leave tomorrow."

\section{\emph{quiet}}

%QUIET
In general, quieta translates better as "still"-as in "keep
still"-than as "quiet." So what's the right word for "quiet?" That
depends on what you're using it to say. "Keep quiet" is Gallate (for
a single noisemaker) or Gallense (for.a band of rowdy tykes). But be
careful with your tone: Gallate can convey the rudeness of an abrupt
"Shut up!" and in fact can be difficult to say without sounding rude.
With friends, a clever way around this problem is to address them in
formal tones: Gallese usted. Paradoxically, this more formal construction tones down the harshness of Gallate quite a bit. The formal construction is also commonly used on children, much as a parent might
say "John Joseph Doe, you quiet down" instead of "Be quiet, Johnny."
Where all else fails, "Shhhh" translates quite well as Shhhh.
Gallado as an adjective means "hushed" or "not answering,"
as in how a stadium crowd reacts when the visiting team scores a
touchdown. Gallar, the verb, can be used to convey "speechless." La
callaron = "They left him speechless." (This could also be rendered
Se qued6 sin palabras, or "He was left speechless.") A person who is
"quiet" by nature could be called una persona callada or una persona
de pocas palabras.
Many times, "quiet" in English is used to describe serenity,
calmness, peacefulness, and ease: "Let's just have a quiet night at
home, dear." For these cases, tranquilo is the correct choice. It also
covers "quiet" places. Busquemos un lugar tranquilo para platicar =
"Let's look for a quiet place to talk."

\section{\emph{save}}

%SAVE
Once you learn that salvar is rarely the word you want for "to
save," it becomes tempting to use ahorrar all the time. And it does
work for many usages: "to save time" (ahorrar tiempo), "to save
money" (ahorrar dinero), "to save (making) a trip" (ahorrar un viaje),
and So on. Still, it's not always the smoothest translation of these and
similar common English idioms. For "to save a trip," for instance, evitarse ("to avoid") could be used comfortably. Me evite un viaje ala
tienda cuando lleg6 con azucar = "I saved (myself) a trip to the store
when she (or he) showed up with sugar."
Another common use of "to save" in English is in the sense of
WHICH IS WHICH? 149
"to set something aside," and in these cases ahorrar should be avoided.
Instead, use guardar. Guardame un trozo para mas tarde = "Save me
a piece for later." Guardar is also the correct verb for "saving" ticket
stubs, receipts, and other items, and can often be translated "to hold
on to." Salvar is also a perfectly good word for "to save," of course, but
only in the very specific sense of "to rescue." "Lifesavers," both the
candy and the flotation device, are salvavidas.

\section{\emph{short}}

%SHORT
Ask yourself: "short" as in "not tall" or "short" as in "not
long"? In English, we lump these concepts together: "A short man
took a short trip." In Spanish, you have to distinguish between the
two: Un hombre bajo hizo un viaje corto. Breve can generally be substituted for corto, never for bajo.

\section{\emph{sign}}

%SIGN
As you travel through the Spanish-speaking world, you may
find yourself asking directions and, in the course of doing so, asking if
there are "signs" marking the way. After a period of traveling there, of
course, you'll realize the quaint innocence of this question in many
places, but in the beginning at least you will need a word for "signs."
.What shall it be? The best all-purpose word for "road sign" is letrero,
and that's what you would generally ask for: iHay letreros~The word
r6tulo is also used (in some countries more than others), but generally
this word refers to the sign over a shop.
What about signo, you wonder? For the most part, ignore itoutside of math and grammar it gets little use. You will find it in stock
expressions like signa de interrogaci6n ("question mark"), signa de
admiraci6n ("exclamation point"), signa negativo ("negative sign"),
signa (de) menos ("minus sign"), and so forth. A more all-purpose word
for "sign" in the sense of an intangible "signal" or "indication" is seiial.
Es una buena seiial = "It's a good sign."
Other words that you may want to incorporate as you gain fluency are huella ("sign" in the sense of "trace" or 'clue"), seiia (a hand
"signal" or "gesture," including but not limited to obscene ones), indicio (safe for "indication"), muestra ("sign" in the sense of "evidence
of something"l, marca ("mark"), and so on. There's much overlap in
the use of these words, and for the most part you're safe-if not excessively precise-with just letrero and seiial. "To sign," of course, is firmar, and una firma is "a signature."
150

\section{\emph{skip}}

%SKIP
In spoken English, "to skip" is used more than you might at
first think You can skip breakfast, skip to school, skip school, skip a
grade in school, skip ahead in math or, if you don't like math, skip it
altogether. Still, many dictionaries and textbooks neglect "skip" as,
well, not very dignified English. In Spanish, most figurative uses of
"to skip" have a good equivalent in the reflexive saltarse, which-like
"to skip"-is much used in daily speech. Me salte el desayuno = "I
skipped breakfast." Se ha saltado un rengl6n = "He's skipped a line."
"Skipping school" would call for some other expression, of which
many local variants exist-irse de pinta in Mexico, for instance.

\section{\emph{there}}

%THERE
As in the case of "here" (see above), the several words that are
used for "there" differ only slightly from each other. Ahi, the linguists
say, means "there" when "there" is "over there"-near the person
being addressed. Alli is "over there" when this is not near either the
speaker or the listener, but is generally within sight. Alla is "way over
there"-that is, far away, out of sight, yonder. As in the case of aea and
aqui, the linguists overlook the fact that most Spanish speakers tend
to use the three "theres" almost interchangeably.
Still, in certain expressions there is a correct "there" to use.
One of these is mas alla, which means "beyond" or "past." When you
tell the cab driver where you live, for instance, you might say Vivo
mas alla del parque ("I live past the park"J. When he stops at the start
of your street and looks back at you, you might say Mas alla ("Farther
on"). Don't confuse mas alld with el mas alld ("The Great Beyond"J,
though you might think that, from the way the cabbie is driving, that
is his ultimate destination.
Another note on alld: it is often used as a kind of shorthand
way of referring to a foreign country. A visitor to Mexico, for instance,
might be a little befuddled the first time he or she hears, completely
out of the blue, a question like iY tienen tequila allat ("Do they have
tequila there?"). On the other hand, alla is a handy substitute for
stuffy-sounding expressions like "where I come from "or "in my country." Where the context is clear, it works in a way as "back home."
Alld tengo esposa y tres hijos = "Back home I've got a wife and
three kids."
Finally, remember that in Spanish "there" is usually left out of
the traditional phone query, "Is So-and-so there?" Instead, you simply
ask iEsta Fulanot If you must get the "there" into your question, you
WHICH IS WHICH?
can ask tEsta par ahi Fulanol-though this can come off sounding
fairly informal.
151

\section{\emph{worker}}

%WORKER
What should you call your co-workers? That may depend on
your mood, but in Spanish it also depends on what kind of work they
do. Obrero may seem to work well as "worker," but it refers almost
exclusively to "laborers" doing manual work. Within this grouping,
there are albaiiiles ("construction workers"), labradores (usually,
"field hands"), and iornaleros ("day laborers"). Trabaiador is the generic term for "worker," though it too connotes some actual, sweatproducing labor. For "white-collar workers" or "office workers" (i.e.,
the sweat-free positions), empleado and oficinista are close fits. Note
that empleado means more than just "employee" and is by itself a
fairly respectable job description, often suggesting a job in the public
sector. In fact, many government forms offer empleado as an occupational category all by itself.

