\chapter{Which Is Which?}

For the foreigner, the path to fluency in Spanish is marked by
constant dilemmas. \emph{Ser} or \emph{estar}? Imperfect or preterit? Subjunctive
or indicative? Do you want the word on the left or the word on the
right? Do you want to keep the word you have, or do you want to trade
it for the word behind Curtain Number One? Or do you want to swap
it all for the word Linda is bringing down the aisle in a lovely gift-wrapped box?

For foreigners learning English, one of the greatest bugaboos
is the tendency for one English word to have three or four (or a dozen)
different meanings. The problem is worse with verbs, which team up
with prepositions in a million-odd ways to produce new meanings altogether. As an exasperated Mexican friend once noted, ``What can you
expect from a language where first you chop a tree down and then you
chop the same tree up?"

The converse of these difficulties are your difficulties, as a native English speaker, in learning Spanish. Every single English word, it
seems, requires a bundle of totally different foreign words to describe
it. And worse, you're expected to learn them all!

In this chapter, we look at some of the English words that
don't have a single Spanish equivalent. Instead, you must pause in your
palaver and search for the correct Spanish word. And while no book
can present and resolve all possible word-choice dilemmas, this chapter can help you sort through the choices so you will know which ones
you need to learn.

\section{\emph{back}}

%BACK
The body part, as you probably know already, is \emph{la espalda}.
But what about all those other uses of ``back?" For these, you'll need
to study \emph{tras, atrás}, and \emph{detrás}, plus \emph{de regreso} and \emph{de vuelta} and, last
but far from least, \emph{fondo} (or \emph{al fondo}).

Generally, to express ``in back of" or ``behind," \emph{atrás} and \emph{detrás} are the words you'll use most. \emph{Tras}, for all intents and purposes, is
just like \emph{detrás} and can be safely discarded for the time being. As for
the difference between \emph{atrás} and \emph{detrás}, native speakers will tell you
they distinguish between these two words ``by which one sounds right."
That is true, but of little help to you, the student. So we must turn to
rules, though they are far from perfect, to clarify the situation.

Here's the main rule to remember: use \emph{detrás de} instead of
\emph{atrás de}, and you'll usually be right. In fact, it is hard to think of a case
where \emph{atrás de} would be correct and \emph{detrás de} wrong. So you are free
to forget about \emph{atrás de} altogether. Or almost. One exception is in reference to someone coming behind or after someone else. Here, \emph{detrás
	de} suggests ``in pursuit of"; \emph{atrás de} just means ``back there somewhere" or even ``later." \emph{¿Dónde está tu mamá? Viene atrás de mí}.
``Where's your mom?" ``She's coming behind me (later)." Change \emph{atrás
	de mí} to \emph{detrás de mí} and you have ``She's coming after me"---in
hot pursuit.

The underlying intuitive difference between \emph{atrás} and \emph{detrás}
is hard to pin down. In most cases, \emph{detrás} suggests a spatial relationship, or one thing ``behind" another or being blocked by another. Atrás
is much vaguer, meaning ``in back," ``back there somewhere" or ``behind (the speaker)." You might say, for instance, that there is a mountain \emph{detrás de mi casa}. But if someone inside the house asks where the
mountain is, you might say \emph{allá atrás}.

In fact, if we accept \emph{detrás de} as (almost) always the correct
choice, then we could invent a second, equally imperfect rule to cover
\emph{atrás}: when you don't need the \emph{de}, use \emph{atrás}. That is, when standing
alone as an adverb, \emph{atrás} is correct and sufficient. This usually covers
backward motion, as in \emph{va hacia atrás} or \emph{va para atrás}. In keeping
with its vague ``backness," \emph{atrás} is usually the word of choice in figurative expressions. \emph{Se está echando para atrás} = ``He's backing out."
\emph{No mires atrás} = ``Don't look back."

For other uses of ``back," \emph{de regreso} and \emph{de vuelta} both mean
``back" in the sense of ``returned." Both are used with \emph{estar}. \emph{Está de
	vuelta después de su viaje} = ``He's back now after his trip." Of course,
\emph{regresar} and \emph{volver} convey the same thing.

\emph{Fondo}, or \emph{al fondo}, is one of the handiest words for ``back"
and one you should try to learn. \emph{El fondo}, as a noun, refers to the
``back" of a bus, movie theater, or building (as well as the ``bottom"
of a glass, swimming pool, and so forth). You'll hear \emph{al fondo} for ``to
the back" or ``all the way in the back." Ask the bus driver or theater
usher if there are any seats and the answer may be \emph{Sí, al fondo}. \emph{Atrás},
especially with cars, can be used this way as well. It means something
like ``in the back" as opposed to ``at the very back," which is \emph{al fondo}.
\emph{Al fondo} is extremely common when giving directions, particularly
indoors. Ask where the bathroom is in a restaurant and the reply may
be, for instance, \emph{Al fondo a la derecha} (``All the way back on your
right").

\section{\emph{become}}

%BECOME
This is perhaps the gold-medal winner for troublesomeness
in the English-to-Spanish olympics, especially when ``to become" includes the sense of ``to get." As a rule, \emph{ponerse} is the handiest word for
``to become," but you should learn at least some of the others. \emph{Ponerse}
in general is for fleeting states of mind or conditions, as in \emph{ponerse furioso} (``to get/become furious"), \emph{ponerse viejo} (``to get/become old"),
\emph{ponerse nervioso} (``to get/become nervous"), and so on.

For longer-term, usually nonreversible conditions, \emph{hacerse} is
often the verb you want. \emph{Me hice rico} = ``I became (got) rich"; \emph{El penique se está haciendo inútil} = ``The penny is becoming useless"; \emph{El
	nuevo baile se está haciendo popular} (``The new dance is becoming
popular"). The distinction is subtle, sometimes to the point of near
invisibility. Thus one can also \emph{hacerse viejo}, but it suggests ``becoming
an old person" or ``becoming elderly," whereas \emph{ponerse viejo} hints at
``starting to feel (or look) old."

\emph{Volverse}, another word for ``to become," is perhaps best translated as ``to turn into." \emph{Se ha vuelto una verdadera molestia} = ``He's
become (turned into) a real nuisance." It involves a more sudden and
unexpected change than \emph{hacerse}, though it shares with that word a
change into a more or less lasting state. \emph{Se volvió rico}, for instance,
can also be used to say ``He got rich," but it gives the impression of
overnight wealth---winning the lottery, perhaps, or inheriting a fortune---and also carries an intimation of criticism, as if the change were
undeserved as well as unexpected. \emph{Volverse} is also the only phrase to
be used for ``to go crazy" in the permanent sense: \emph{volverse loco}.

Both \emph{transformarse} and \emph{convertirse} can also be rendered
``turned into," but usually in connection with a more thoroughgoing,
physical change. \emph{El agua se transforma en vapor} = ``Water becomes
(turns into) steam." \emph{Clark Kent se convierte en Superman} = ``Clark
Kent becomes (turns into) Superman." Sometimes these phrases can
cover more figurative conversions: \emph{Se ha convertido en un buen escritor} = ``He's become a good writer." \emph{Llegar a ser}, meanwhile, works
well as ``to turn out to be," as in \emph{La fórmula llegó a ser muy útil} =
``The formula turned out to be very useful." This phrase also covers
``to become" in the sense of ``worked his (or her) way up to." \emph{Después
	de muchos esfuerzos, llegó a ser médico}. = ``After a lot of hard work,
he became a doctor." In many cases, an easier construction can be
made with the verb \emph{resultar}, which also means ``to turn out." \emph{Este libro resultó ser muy interesante}. = ``This book turned out to be very
interesting."

A very common but often overlooked way of saying ``to become" in Spanish is through the use of reflexive verbs, especially those
that are little more than an \emph{en-} or \emph{a-} prefix attached to an adjective or
noun: \emph{acalorarse} (``to get/become hot"), \emph{enfriarse} (``to get/become
cold"), \emph{entristecerse} (``to get/become sad"), \emph{acercarse} (``to get/become
close"), and so on. Sometimes no prefix is needed with the adjective,
as in \emph{mojarse} (``to get/become wet") and \emph{cansarse} (``to get/become
tired"). Pay attention to which nouns and adjectives can be made into
verbs and how, and you'll be well on your way to conquering the ``get/become" problem.

\section{\emph{burn}}

%BURN
In English, this is a fairly simple if destructive process; in
Spanish it's a little more complex. \emph{Arder} refers to the actual act of
burning; think of it as ``on fire" or ``burning up." Thus houses and
candles \emph{arden}, and so do fever victims: \emph{Estoy ardiendo en calentura}.
Out of context, \emph{Estoy ardiendo} can sound like \emph{double entendre}---``I'm
on fire, baby, sweep me off my feet." Figuratively, \emph{arder} comes into
play in the expression \emph{está que arde}, meaning someone is ``burning
mad" or a situation is ``boiling over."

\emph{Incendiar}, another choice for ``to burn," is usually used for
things that should not be burning: forests, buildings, the lawn, the
lawn furniture, and so on. Think of it as ``to set on fire." Even in the
case of \emph{pasión incendiaria}, the implication is that the fire has gotten
out of hand and may end up being destructive.

\emph{Quemar} also translates as ``to burn," but generally brings to
mind (in the reflexive \emph{quemarse}) ``to burn up" or ``to burn down," (another case of contradictory prepositions in English saying almost exactly the same thing). If you go away for the weekend and come back
to find your toolshed in ashes, a neighbor might explain simply, \emph{Se
	quemó}. \emph{Quemar} in its nonreflexive form is the usual way to convey the
transitive ``to burn (something)." \emph{Estás quemando el bistec} = ``You're
burning the steak." (If you were to use \emph{incendiar} here, you would be
saying, ``You're setting fire to the steak.") The general idea behind \emph{quemar} is ``to consume (or damage) by fire." It is the correct verb for when
one thing ``burns" another: the sun \emph{quema} your skin, the stove \emph{quema}
your fingers, and so on. The painful and unappealing marks left on
your skin to mark these mishaps are called \emph{quemaduras} (``burns"). To
say ``I burned myself," whether in the sun or the kitchen, you would
say \emph{Me quemé}.

\section{\emph{corner}}

%CORNER
Ninety-five percent of the time, just use \emph{esquina} for ``corner."
For that matter, use it for ``intersection" too, although \emph{intersección}
is increasingly heard these days. For a ``crossroads," or intersection
of highways, use either \emph{entronque, encrucijada}, or \emph{cruce de caminos}.
Dictionaries will give \emph{rincón} for ``corner" as well, but it generally refers to an inside corner where two walls meet---the corner of a room,
in other words. By extension, it is used to mean a ``nook" or ``cranny,"
and is a favorite word to assign to cocktail lounges, piano bars, and
the like.

\section{\emph{date}}

%DATE
Here you'll have to slow down and analyze what exactly you're
trying to say. Do you mean ``the date," as in July 13, 1984? Then the
Spanish word you want is \emph{fecha}. If you mean ``date" as in ``appointment," you have a couple of words to choose from. A ``date" of the sort
that thrills teens (and, let's be frank, some postteens as well) is almost
universally a \emph{cita}. A \emph{cita} is also what you have with a doctor or lawyer, however---in other words, an ``appointment." That is, in Spanish
you can have a \emph{cita} with your doctor without upsetting your spouse. A
\emph{compromiso} is also an ``appointment," usually an unspecified one. It
can almost mean simply ``something to do" and implies an obligation
to be present at a certain time---but no more. Thus, to escape from a
dreary luncheon, you can excuse yourself with \emph{Perdón, tengo un compromiso}. All you're really saying is ``Sorry, gotta go."

\section{\emph{fail}}

%FAIL
The best all-purpose word for ``to fail" is \emph{fracasar}, though one
1952 edition of a heavyweight Spanish-English dictionary I consulted
defines it as ``to crumble, to break into pieces: applied commonly to
ships." A ship that crumbles into bits would certainly meet my definition of ``failure," but \emph{fracasar} in common usage covers
much more
ground than that.

\emph{Fallar} is also used for many general senses of ``to
fail," though it seems a little kinder and less permanent than \emph{fracasar}.
\emph{Falló en su intento} suggests ``He did not succeed"; \emph{Fracasó en su intento} hints ``He was a total flop." Often you'll hear \emph{fallar} in the form
\emph{estar fallando} to mean ``not to be working well," ``to be on the blink,"
in reference to some appliance or gizmo. When someone or something
``fails" you in the sense of not living up to your expectations, either
\emph{fallar, defraudar}, or \emph{decepcionar} can be used. To ``fail" an exam or a
course, \emph{reprobar} is much used in the Americas, while \emph{suspender} is
more frequent in Spain. Thus \emph{un suspenso} on your exam in Madrid is
beyond suspense: it's an F. In Mexico City receiving an F would be expressed as \emph{reprobar el examen} or \emph{ser reprobado par el profesor}.

\section{\emph{front}}

%FRONT
This word opens an unruly can of worms on its journey into
Spanish. The basic problem is the excess of believable cognates, which
then have to be sorted through. For instance, you can say there is a
fountain \emph{al frente de la casa, enfrente de la casa, frente a la casa}, or
\emph{de frente a la casa}, and you won't necessarily be saying the same
thing. What's the difference?

\emph{Al frente de} is usually the expression for the English ``in front
of." It means ``at (in) the front part of (something)." In the earlier example, a fountain that is \emph{al frente de la casa} is on the property at the
front of the house. Next come a host of prepositional expressions that,
in many circumstances, mean ``in front of" but generally only in the
sense of ``facing." \emph{Enfrente de} is one of these expressions, and it is better regarded as a trickster (see Chapter 3). Think of it as a synonym for
``facing" or ``across the way." A fountain that is \emph{enfrente de la casa} is
likely to be across the street from the house. \emph{Frente a} means much the
same and should be viewed as just as unreliable as \emph{enfrente de} in most
cases. Sometimes the act of ``facing" is implicit. \emph{Lo tienes frente a los
	ojos} = ``You have it in front of your eyes"---that is, ``staring you in the
face." \emph{De frente a}, finally, states very clearly that something is ``face
to face," literally ``with its front facing (something)." \emph{Estoy parada de
	frente al sol} = ``I'm standing facing the sun."

Perhaps the safest word of all for ``in front of" is \emph{delante}, usually followed by \emph{de}. Generally, whenever you can use ``ahead" or
``ahead of," you should probably be using \emph{delante} or \emph{delante de}. But it
also covers a wide range of ``in front of" situations. The fountain can
be \emph{delante de la casa}, for instance. \emph{Bob se sienta delante de mí en la
	clase} = ``Bob sits ahead of me in class."

The classroom example may prove, well, instructive. Ana's
seat is in the third row; Bob sits \emph{delante de ella}. Pedro sits in the front
row, \emph{delante de} Bob and \emph{al frente de la fila} (``at the front of the row")
and \emph{frente a la maestra} (``in front of---i.e., facing---the teacher); when
Ana gets up to read her book report, she goes \emph{al frente del salón} and
stands facing the class, or \emph{de frente a la clase}. Ana's sister is in the
classroom across the hall, or \emph{enfrente del salón de Ana}.

Finally, whereas \emph{atrás} is ``in back" (see ``Back"), \emph{adelante} is
usually ``up front"---in the front seat of a car, for instance. To refer to
``the front" or ``the front part" of something, \emph{el frente} can usually be
used. \emph{El frente de la tienda está sucio} = ``The front of the store is
dirty." \emph{La frente}, in the feminine, refers exclusively to ``the forehead."

\section{\emph{funny}}

%FUNNY
Does it make you laugh or wonder? That's the first question
you'll need to ask (and answer). If it's the former, you can choose from
\emph{chistoso, gracioso}, and \emph{cómico}. If the latter, stick to \emph{extraño} and \emph{raro}.
Pick your favorite and use it; they are virtually interchangeable. Anything having to do with ``fun," incidentally, should be carefully distinguished from things that are ``funny," since what's fun isn't always
funny. ``Fun" needs either the adjective \emph{divertido} or the noun \emph{diversión} to describe it.

\section{\emph{happen}}

%HAPPEN
``To happen" can almost always be covered by some form of
\emph{pasar} in Spanish. Other counterparts exist---including \emph{acontecer, ocurrir, suceder}, and \emph{acaecer}---but there's really no need to learn them
other than to recognize them when you hear them. You will have to
learn some of their noun forms, however, since \emph{pasar} doesn't have one.
\emph{Acontecimiento, hecho}, and \emph{suceso} are three of the most frequently
encountered words for ``happening," ``occurrence," and ``event."
\emph{Evento} will be understood---and is gradually being incorporated in this
sense---but technically it means something that happened by chance
or with no advance planning.

\section{\emph{here}}

%HERE
Spanish is funny. Even when you know you're ``here," you
have to know which ``here" to use. Two words, \emph{acá} and \emph{aquí}, cover
the situation. Which is which? Textbooks generally promote the idea
that \emph{acá} is the correct word for verbs of motion, as in \emph{Ven acá} (``Come
here"), while \emph{aquí} is the word of choice for stationary presence in a
given place, as in \emph{Estoy aquí} (``I'm here"). After hearing one person on
the street shout \emph{Estoy acá}, though, you might wonder why grammarians even bother with these distinctions.

The comparative forms of \emph{acá} come into play when giving
guidance to, say, someone backing a truck down an alley. \emph{Mas acá}
means ``A little more this way," \emph{No tan acá} means ``Not so much this
way," and \emph{¡Ya!} means ``Just right, hold it." \emph{¡Praaac!} (``Crunch!") means
you won't be asked to help truck drivers back down alleys anymore.

Finally, a very common use of ``here" in English is to say
``Here you are" or ``Here you go" when giving someone something (for
example, paying the cab driver the fare). In Spanish you can say pretty
much the same thing with \emph{aquí}: \emph{Aquí está} and \emph{Aquí tiene} both work,
as does \emph{Tome} (``Take"). A somewhat archaic expression that still creeps
into written Spanish is \emph{He aquí}, and some dictionaries will tell you
that it means ``Here is\ldots{}" Actually, it's closer to ``Behold!" Before
trying it out on the cabbie, try to imagine yourself saying ``Behold, the
ten pesos!" in English. Then go back to \emph{Aquí tiene}.

\section{\emph{hot, warm}}

%HOT, WARM
Native English speakers aspiring to fluency in Spanish should
be aware of the latter language's hierarchy of ``heat" adjectives. Apart
from the several words representing various degrees of heat, different
words are used depending on the object in question. A soup fresh off
the fire, for instance, will drop in temperature from \emph{hirviendo} (``boiling
hot" or ``scalding") to \emph{caliente} (``hot") to \emph{calientita} (``pretty hot") to
\emph{tibia} (``lukewarm") to \emph{templada} (not hot but not cold either). A cold
beverage goes from \emph{helada} to \emph{fría, fresca, al tiempo} (``room temperature"), \emph{tibia}, and \emph{caliente}. For weather, you can describe the day in order of ascending temperature: \emph{helado, frío, fresco, templado, calientito, caliente}, and \emph{caluroso}. A good word for warm days is \emph{cálido}, often
in the sense of ``unseasonably (or pleasantly) warm." As expressive alternatives for \emph{caluroso} you can say \emph{Hace un calor bárbaro} (or \emph{tremendo, bochornoso, sofocante}) or simply \emph{Hace un calorón}.

``Hot" in its many figurative senses in English can be a dangerous concept to convey in Spanish. \emph{Caliente} in reference to other
people, especially of the opposite sex, is risky. Think of \emph{caliente} as
``in heat" rather than ``hot-looking," and you'll get the general idea. As
for the common novice mistake of saying \emph{Estoy caliente} instead of
\emph{Tengo calor} (``I'm hot"), those of you who have used it know the consequences: usually belly laughs all around, sometimes an unwanted offer.

\section{\emph{hurry}}

%HURRY
Several choices present themselves for the concept of ``to
hurry," and mostly it's a matter of choosing one and using it. There are
some differences, though. Your choices are (the envelope, please): \emph{tener
	prisa, apurarse}, and \emph{darse prisa}. The first means ``to be in a hurry,"
and \emph{Tengo prisa} is a basic Neurotic-Speak expression. \emph{Apurarse} is less
common except in the phrase \emph{¡Apúrate!} (or \emph{¡Apúrese!}), meaning ``Hurry
up!" and then only in the Americas; \emph{apresurarse} replaces \emph{apurarse} east
of the Atlantic. The adjective \emph{apurado} is often heard, but implies more
``harried" or ``frantic" than just ``in a hurry." \emph{Darse prisa} is sort of like
``to make oneself hurry" or ``to get a move on," and \emph{¡Date prisa!} is
practically interchangeable with \emph{¡Apúrate!} for urging someone to get
the lead out. When you're in such a hurry that you can't even stop to
say ``Hurry up," appropriate grunts include \emph{¡Ya! ¡Aprisa! ¡Deprisa!} and
in Mexico \emph{¡Ándale!} and \emph{¡Órale!}

\section{\emph{look}}

%LOOK
``To look" in English covers a lot of ground. ``To look for" is
not the same as ``to look at," for instance, and ``to look good" is a different concept from ``to look like." As usual, one English verb can
combine with a score of prepositions to produce dozens of distinct
meanings---each of which in Spanish may require a different verb. The
most common Spanish words for the concept ``to look" are \emph{buscar} (``to
look for") and \emph{mirar} (``to look at").

``To look" by itself is \emph{verse}, a verb you should learn to use. \emph{Te
	ves bien} = ``You look good." \emph{Se ve horrible} = ``It (or he or she) looks
horrible." ¿Cómo me veo? = ``How do I look?" And so on.

``To look like"---with the meaning ``to resemble" or ``to remind one of"---is either \emph{parecer} or \emph{parecerse a}. \emph{Ese señor se parece a
	mi padre} = ``That man looks like my father." When ``to look like"
means ``to look as if," \emph{parecer} by itself is enough. \emph{Parece que va a
	llover} = ``It looks like (as if it is going to) rain."

\section{\emph{miss}}

%MISS
Your husband goes away on a bus and you miss him. He
misses the bus, so you don't miss him. On his way to the bus, another
bus just misses him, causing him to miss the bus and you almost to
miss him very much. Still with me? There's obviously a lot of missing
going on in this world. And in Spanish, you must take care to differentiate what you do to your husband and what the bus does.

Let's deal first with the verb that expresses what you feel
when any loved one goes away. This sense of ``to miss" in the Spanish
of the Americas is handled by \emph{extrañar}. In Spain, you'll hear \emph{echar de
	menos}. As in English, you can ``miss" not only a person but your bed,
your favorite brand of candy bar, or any once-regular activity, like your
nightly mud bath. All are covered by \emph{extrañar} and \emph{echar de menos}.

Now let's look at ``to miss" in the context of missing a bus,
a plane, a party, a movie, a lecture, and so on. In Spanish, this idea is
handled by \emph{perder}, which of course is also the word for ``to lose." \emph{Perdimos el tren} = ``We missed the train." If you miss something that
you were expected to attend, you would use \emph{faltar a}, as in \emph{Falté a la
	clase de español} (``I missed the Spanish class"). Think of it as ``to
miss" in the sense of ``to be absent from." When something is ``missing" or ``absent," \emph{faltar} is also the word of choice. \emph{Faltan once dólares
	de mi cartera} = ``Eleven dollars is missing from my wallet."

Then there is ``to miss" in the sense of the close call---``The
bus barely missed the parked truck"---which is a fairly tricky concept
to translate into Spanish. Some dictionaries might have you use \emph{errar,
	no acertar}, or \emph{no dar en el blanco}, but these imply that the bus was
trying deliberately to hit the truck---and may even have gone back for
a second shot. These are all useful expressions, but they are best saved
for when your jump shot goes astray or when you guess something
wrong. To convey the idea of the bus ``missing" the truck, you would
have to resort to \emph{pasar rozando} (``to graze") or \emph{pasar cerca} (``to go by
close"). To get the point across even more clearly, you would need to
trade in ``miss" for ``almost hit." \emph{Un autobús par poco} (or \emph{casi}) \emph{atropella a mi marido} = ``A bus just missed (running over) my husband."

\section{\emph{next}}

%NEXT
At some point in your progress as a student, you will realize
that \emph{próximo} is not the only word for ``next." Soon afterward, you will
realize that it's often not even a very good word for ``next." What are
the other words, and when are they used? For starters, you're safe with
\emph{próximo} when ``next" means some indefinite ``next time." In fact
\emph{próxima vez} (``next time") is one of its commonest uses. But \emph{próximo}
does not work for ``the next day," which is \emph{el día siguiente} (or \emph{al día
	siguiente}). It works again for ``next week" (\emph{la próxima semana}), and
fairly well for ``next year" (\emph{el próximo año}), but you are probably more
likely to hear \emph{la semana que entra} (or \emph{entrante}) and \emph{el año que viene}.
Somewhere someone has probably thought up an explanation for these
deviations; ignore it and just learn the phrases.

Another slight deviation to be alert for is the use of \emph{otro} for
``next," which you will run across constantly. \emph{Al otro día} can mean
``the next day," for instance. More common is \emph{la otra} for ``the next"
street, stoplight, or corner when being given directions. \emph{Está en la otra
	calle} = ``It's on the next street over." What tends to confuse beginners
is that the noun is often implied, not stated. \emph{Dé vuelta no en ésta sino
	en la otra} means ``Don't turn here (at this corner) but at the next one."

For ``next to" in a physical sense, \emph{próximo} is flat-out wrong;
\emph{junto a} or \emph{al lado de} is correct. \emph{La tienda esta al lado de su casa} or \emph{La
	tienda está junto a su casa} = ``The store is next to his house." Note
that as an adverb of place, \emph{junto} does not change with the gender of
the subject. ``The girl next door" is either \emph{la muchacha de junto} or \emph{la
	muchacha de al lado}. Colloquially and quite commonly you will hear
\emph{pegado} (literally, ``stuck to") for ``right next door."

\section{\emph{office}}

%OFFICE
Such an easy concept---and such a wealth of Spanish words to
cover it. Let's start with---and discard---\emph{oficio}, which applies only to
``office" in the symbolic, conceptual sense: ``the office of vice president." For a room where someone works, several words could fit, depending on what you want to say. The most generic term for ``office" is
\emph{oficina}, and this can be used safely about 90 percent of the time. Getting pickier, we can distinguish between the larger office under the
boss's control and the actual office where he or she works, which could
be called his or her \emph{privada} (``private office"). Doctors and dentists
have special offices called \emph{consultas} or consultorios. Lawyers dwell in
\emph{bufetes}, which works for both the physical office and the ``firm." A
\emph{despacho} is a workplace, usually nongovernmental, that is often involved in accounting or administration, though lawyers can labor in
\emph{despachos} as well. A \emph{taller} is a ``workshop" where manual labor is
done. \emph{Taller mecánico} means ``workshop" or ``garage" and has nothing
to do with the height of the mechanic who works there.

\section{\emph{old, older}}

%OLD, OLDER
The main problem with this concept is the word \emph{viejo}, which
you should probably be taught from early on not to use in reference
to people.

The other problem is the concept of ``older," which in English
can refer to a two-year-old in comparison with a one-year-old. In Spanish, \emph{viejo} means ``old" in the very specific sense (for humans) of having
been on the planet for more than fifty Of sixty years. \emph{Más viejo}, naturally, means ``older," in the sense of having been among us for sixty-plus years. Therefore, when referring to your ``older brother," you can't
say he's your \emph{hermano más viejo} unless he is, officially, a potential
Gray Panther. He is your \emph{hermano mayor}---your ``greater brother," in
age if in nothing else. A very common way of phrasing this idea is with
\emph{grande}. \emph{¿Tienes hermanos más grandes?} = ``Do you have older siblings?" A ``grown-up," in fact, can safely be translated as \emph{una persona
	grande}---``a big person."

I say ``safely" because, as always when talking about age, there
are numerous ways to stick your foot in your mouth. Almost all of
them have to do with the word \emph{viejo}, which, as noted, you would do
well to forget except in reference to buildings and such. With senior
citizens, the tendency in all cultures is to avoid calling them ``old"---thus terms like ``senior citizens." There's nothing wrong with being
old---most of us take special precautions to become old ourselves---but
nobody seems to like being reminded that they've achieved this status.
So in Spanish you say things like \emph{Mi abuela ya es muy grande, Es
	una persona ya grande} or  \emph{Es una persona mayor} instead of saying \emph{Mi
	abuela es vieja}. Other words given for ``old" by the dictionaries are
also borderline rude except when talking about, say, pyramids. These
include \emph{anciano} and \emph{antiguo}, especially.

\emph{Mi viejo} and \emph{mi vieja}, meanwhile, come in for frequent colloquial use. They can mean ``the old man" (or father) and ``the old lady"
(or mother), but more commonly they refer slangily to one's ``guy"
(husband, boyfriend, or lover) or ``gal" (wife, girlfriend, or lover). \emph{Viejo}
(without the \emph{mí}) is also used slangily in many countries for ``buddy" or
``man," regardless of the age of the person addressed. \emph{Vieja} is not used
this way ever, though, since \emph{una vieja} is a rude way of referring to a
woman, similar to ``a broad" or even ``a bitch."

\section{\emph{order}}

%ORDER
In this case, your choice is the same word but with different
genders: \emph{la orden} and \emph{el orden}. Which is which? Well, if you're traveling through Spanish-speaking areas, you should first learn the word for
``an order" of the sort that you submit to a waiter or waitress at a restaurant. It is \emph{la orden}, the feminine form, and you should invent some
mnemonic gimmick to make this stick. \emph{El orden}, the masculine form,
means ``the order" of things---that is, their organization or sequence.

\emph{Ordenar} as a verb, to mean ``to order (in a restaurant)," is being used
increasingly in Latin America, but the proper word for this is \emph{pedir}.
\emph{Ordenar} really only means ``to put one's things in order." ``To order
(someone about)" would be \emph{mandar}. A final note on ``order": ``in order
to" should be handled with \emph{para}, never with an expression using
\emph{orden}.

\section{\emph{piece}}

%PIECE
This is a tricky one, and even most Spanish speakers won't be
able to explain the subtle distinctions---though they will invariably
and instinctively use the correct form. First of all, banish \emph{pieza} from
your mind for all but a few usages, usually having to do with a ``part"
of a machine or car. Think of it as a self---contained ``piece," and not so
much as an arbitrarily determined or broken-off ``fragment" of some
larger whole.
For fragments, the two most common words are \emph{trozo} and
pedazo. Sometimes, these two are interchangeable---with a ``stretch"
of highway, for instance, either \emph{trozo} or \emph{pedazo} will work (though
\emph{tramo} would be an even better word here)---and other times they're
not. The difference seems to reside in whether the piece in question is
a plainly three-dimensional chunk (\emph{trozo}) or not (\emph{pedazo}), whether the
piece in question is useful in itself (\emph{trozo}) or not (\emph{pedazo}), whether it's
an indistinct division between part and whole (\emph{trozo}) or a fairly clear
one (\emph{pedazo}), and so on. An object breaks into \emph{pedazos}, and you normally would ask for a \emph{trozo} of something at the dinner table.
In this last setting, however, bear in mind that many comestibles come in \emph{rebanadas} (``slices"), and that this is the preferred word
for ``pieces" of bread, cake, pizza, and so on. A cute word for a ``little
piece," ``smidgeon," or ``tad" is \emph{nadita}---``a little nothing." The word
\emph{tajada} is also used widely to mean ``piece" or ``slice," but in many
areas it also carries the connotation of a ``cut" or ``percentage" of an
illegal or dirty business.

\section{\emph{plan}}

%PLAN
As a noun, ``plan" equals \emph{plan} pretty consistently, but keep in
mind the difference between \emph{plan} and \emph{plano} (a building's ``floorplan"
or ``design," a town ``map"). To be in a certain plan is a colloquial way
of saying to be ``in a phase," to be ``on a kick." \emph{Estoy en mi plan limpiador} = ``I'm in my cleaning mood." For the verb ``to plan" it is almost always better to forget about \emph{planear} and use \emph{pensar} instead,
unless you are consciously planning a scheme or a strategy. \emph{Pensamos
	irnos mañana} = ``We're planning to leave tomorrow."

\section{\emph{quiet}}

%QUIET
In general, \emph{quieto} translates better as ``still"---as in ``keep
still"---than as ``quiet." So what's the right word for ``quiet?" That
depends on what you're using it to say. ``Keep quiet" is \emph{Cállate} (for
a single noisemaker) or \emph{Cállense} (for.a band of rowdy tykes). But be
careful with your tone: \emph{Cállate} can convey the rudeness of an abrupt
``Shut up!" and in fact can be difficult to say without sounding rude.
With friends, a clever way around this problem is to address them in
formal tones: \emph{Cállese usted}. Paradoxically, this more formal construction tones down the harshness of \emph{Cállate} quite a bit. The formal construction is also commonly used on children, much as a parent might
say ``John Joseph Doe, you quiet down" instead of ``Be quiet, Johnny."
Where all else fails, ``Shhhh" translates quite well as \emph{Shhhh}.

\emph{Callado} as an adjective means ``hushed" or ``not answering,"
as in how a stadium crowd reacts when the visiting team scores a
touchdown. \emph{Callar}, the verb, can be used to convey ``speechless." \emph{Lo
	callaron} = ``They left him speechless." (This could also be rendered
\emph{Se quedó sin palabras}, or ``He was left speechless.") A person who is
``quiet" by nature could be called \emph{una persona callada} or \emph{una persona de pocas palabras}.

Many times, ``quiet" in English is used to describe serenity,
calmness, peacefulness, and ease: ``Let's just have a quiet night at
home, dear." For these cases, \emph{tranquilo} is the correct choice. It also
covers ``quiet" places. \emph{Busquemos un lugar tranquilo para platicar} =
``Let's look for a quiet place to talk."

\section{\emph{save}}

%SAVE
Once you learn that \emph{salvar} is rarely the word you want for ``to
save," it becomes tempting to use \emph{ahorrar} all the time. And it does
work for many usages: ``to save time" (\emph{ahorrar tiempo}), ``to save
money" (\emph{ahorrar dinero}), ``to save (making) a trip" (\emph{ahorrar un viaje}),
and So on. Still, it's not always the smoothest translation of these and
similar common English idioms. For ``to save a trip," for instance, \emph{evitarse} (``to avoid") could be used comfortably. \emph{Me evité un viaje a la
	tienda cuando llegó con azúcar} = ``I saved (myself) a trip to the store
when she (or he) showed up with sugar."

Another common use of ``to save" in English is in the sense of
``to set something aside," and in these cases \emph{ahorrar} should be avoided.
Instead, use \emph{guardar}. \emph{Guárdame un trozo para más tarde} = ``Save me
a piece for later." \emph{Guardar} is also the correct verb for ``saving" ticket
stubs, receipts, and other items, and can often be translated ``to hold
on to." \emph{Salvar} is also a perfectly good word for ``to save," of course, but
only in the very specific sense of ``to rescue." ``Lifesavers," both the
candy and the flotation device, are \emph{salvavidas}.

\section{\emph{short}}

%SHORT
Ask yourself: ``short" as in ``not tall" or ``short" as in ``not
long"? In English, we lump these concepts together: ``A short man
took a short trip." In Spanish, you have to distinguish between the
two: \emph{Un hombre bajo hizo un viaje corto}. \emph{Breve} can generally be substituted for \emph{corto}, never for \emph{bajo}.

\section{\emph{sign}}

%SIGN
As you travel through the Spanish-speaking world, you may
find yourself asking directions and, in the course of doing so, asking if
there are ``signs" marking the way. After a period of traveling there, of
course, you'll realize the quaint innocence of this question in many
places, but in the beginning at least you will need a word for ``signs."
What shall it be? The best all-purpose word for ``road sign" is \emph{letrero},
and that's what you would generally ask for: \emph{¿Hay letreros?}The word
\emph{rótulo} is also used (in some countries more than others), but generally
this word refers to the sign over a shop.

What about \emph{signo}, you wonder? For the most part, ignore it---outside of math and grammar it gets little use. You will find it in stock
expressions like \emph{signo de interrogación} (``question mark"), \emph{signo de
	admiración} (``exclamation point"), \emph{signo negativo} (``negative sign"),
\emph{signo (de) menos} (``minus sign"), and so forth. A more all-purpose word
for ``sign" in the sense of an intangible ``signal" or ``indication" is \emph{señal}.
\emph{Es una buena señal} = ``It's a good sign."

Other words that you may want to incorporate as you gain fluency are \emph{huella} (``sign" in the sense of ``trace" or ``clue"), \emph{seña} (a hand
``signal" or ``gesture," including but not limited to obscene ones), \emph{indicio} (safe for ``indication"), \emph{muestra} (``sign" in the sense of ``evidence
of something"), \emph{marca} (``mark"), and so on. There's much overlap in
the use of these words, and for the most part you're safe---if not excessively precise---with just \emph{letrero} and \emph{señal}. ``To sign," of course, is \emph{firmar}, and \emph{una firma} is ``a signature."

\section{\emph{skip}}

%SKIP
In spoken English, ``to skip" is used more than you might at
first think You can skip breakfast, skip to school, skip school, skip a
grade in school, skip ahead in math or, if you don't like math, skip it
altogether. Still, many dictionaries and textbooks neglect ``skip" as,
well, not very dignified English. In Spanish, most figurative uses of
``to skip" have a good equivalent in the reflexive \emph{saltarse}, which---like
``to skip"---is much used in daily speech. \emph{Me salté el desayuno} = ``I
skipped breakfast." \emph{Se ha saltado un renglón} = ``He's skipped a line."
``Skipping school" would call for some other expression, of which
many local variants exist---\emph{irse de pinta} in Mexico, for instance.

\section{\emph{there}}

%THERE
As in the case of ``here" (see above), the several words that are
used for ``there" differ only slightly from each other. \emph{Ahí}, the linguists
say, means ``there" when ``there" is ``over there"---near the person
being addressed. \emph{Allí} is ``over there" when this is not near either the
speaker or the listener, but is generally within sight. \emph{Allá} is ``way over
there"---that is, far away, out of sight, yonder. As in the case of \emph{acá} and
\emph{aquí}, the linguists overlook the fact that most Spanish speakers tend
to use the three ``theres" almost interchangeably.

Still, in certain expressions there is a correct ``there" to use.
One of these is \emph{mas allá}, which means ``beyond" or ``past." When you
tell the cab driver where you live, for instance, you might say \emph{Vivo
	mas allá del parque} (``I live past the park"). When he stops at the start
of your street and looks back at you, you might say \emph{Mas allá} (``Farther
on"). Don't confuse \emph{mas allá} with \emph{el mas allá} (``The Great Beyond"),
though you might think that, from the way the cabbie is driving, that
is his ultimate destination.

Another note on \emph{allá}: it is often used as a kind of shorthand
way of referring to a foreign country. A visitor to Mexico, for instance,
might be a little befuddled the first time he or she hears, completely
out of the blue, a question like \emph{¿Y tienen tequila allá?} (``Do they have
tequila there?"). On the other hand, \emph{allá} is a handy substitute for
stuffy-sounding expressions like ``where I come from ``or ``in my country." Where the context is clear, it works in a way as ``back home."
\emph{Allá tengo esposa y tres hijos} = ``Back home I've got a wife and
three kids."

Finally, remember that in Spanish ``there" is usually left out of
the traditional phone query, ``Is So-and-so there?" Instead, you simply
ask \emph{¿Está Fulano?} If you must get the ``there" into your question, you
can ask \emph{¿Está por ahí Fulano?}---though this can come off sounding
fairly informal.

\section{\emph{worker}}

%WORKER
What should you call your co-workers? That may depend on
your mood, but in Spanish it also depends on what kind of work they
do. \emph{Obrero} may seem to work well as ``worker," but it refers almost
exclusively to ``laborers" doing manual work. Within this grouping,
there are \emph{albañiles} (``construction workers"), \emph{labradores} (usually,
``field hands"), and \emph{jornaleros} (``day laborers"). \emph{Trabajador} is the generic term for ``worker," though it too connotes some actual, sweat-producing labor. For ``white-collar workers" or ``office workers" (i.e.,
the sweat-free positions), \emph{empleado} and \emph{oficinista} are close fits. Note
that \emph{empleado} means more than just ``employee" and is by itself a
fairly respectable job description, often suggesting a job in the public
sector. In fact, many government forms offer \emph{empleado} as an occupational category all by itself.


%%% Local Variables:
%%% mode: latex
%%% TeX-master: "main-keenan-breaking-out"
%%% End:
