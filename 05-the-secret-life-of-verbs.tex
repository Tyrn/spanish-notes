\chapter{The Secret Life of Verbs}

Many students of Spanish can still recall their first encounter
with Spanish verbs. Often the reaction was something like, "What do
you mean it has different endings in every tense? What on earth for?
Aaaarrgh!"

Alas, for verb endings there is little alternative but to buckle
down and memorize them. But even when you've managed to separate
\emph{-aban} from \emph{-aron}, another unappetizing task awaits: using each ending
and each tense and mode in the right situation. To English speakers
the idea of an imperfect tense is unfamiliar, while the notion of a frequently used subjunctive mode can seem downright perverse. Can't
verbs be made just a little easier?

The answer is yes. There are several pointers on the use of the
tenses that will make learning them a less doleful task. And there are
a couple of obvious pitfalls that anyone wandering into the world of
tenses should be alert to. This chapter presumes you have some background knowledge on the use of tenses in Spanish; even so, I'll try to
summarize some of the basics of tense usage to refresh your memory.

\section{The present}

The present is the most straightforward of tenses in Spanish
and corresponds almost perfectly to the present tense in English: \emph{bailo}
is "I dance," \emph{estoy bailando} is "I am dancing," and so on. For the English compound present ("I do dance"), in Spanish you can slip a \emph{sí}
("yes") into the phrase. "He does eat a lot" = \emph{El sí come mucho}. "But,
honey, I do love you" = \emph{Pero cariño, sí te quiero}. Note that in English
the present progressive is used far more than in Spanish. Thus "She's
going" should almost always be rendered \emph{Se va} and only rarely \emph{Se esta
yendo}.

One tip on the present tense: it is used much more in Spanish
for future events than its English equivalent. For instance, the common way to ask "Are you coming tomorrow?" is simply \emph{¿Vienes mañana?} "We'll see you later" often gets rendered \emph{Nos vemos}---literally,
"We see each other." This trick works even when you're specifying a
future time or date. \emph{Te lo doy el martes} = "I'll give it to you Tuesday."
In Spanish the context makes it clear that the future is being referred
to. Compare these examples:

\bsk

\indu \emph{Te cuento}. = "I'll tell you (now)."

\indu \emph{Te contaré}. = "I'll tell you (someday)."

\indu \emph{Mañana te cuento}. = "I'll tell you tomorrow."

\indu \emph{Mañana te cantaré}. = "I shall tell you tomorrow."

\section{The future}

The future is not a particularly hard tense to learn, and when
to use it is pretty obvious. Still, it can be simplified considerably by
remembering the following rule: ignore it.

What? Just forget about the future? Well, maybe not altogether. There are a few uses for the future in common spoken Spanish.
But most of the time you can avoid it outright, and you'll even sound
more fluent by doing so.

The two most frequent substitutes for the future tense in
Spanish are the "present-as-future," discussed above, and the compound future using the verb \emph{ir} ("to go"), just as in English. \emph{Mañana voy
a llamar a mi hermano} = "Tomorrow I'm going to call my brother."
In common Spanish usage, one of these two constructions almost always replaces the future tense, although the future can be used and is
certainly understood: \emph{Mañana llamaré a mi hermano}. Sometimes,
though, the future sounds formal, stiff, and even awkward. \emph{Vuelvo en
seguida} = "I'll be right back." \emph{Volveré en seguida} = "I will return
promptly."

One special use of the future that anyone aspiring to fluency
must learn is what is called the "future of uncertainty." It is a very
common construction that has no real equivalent in English---which is
why many students shudder at the mere thought of it. The best way to
get a grip on it is by example. You often hear it, for instance, when
someone knocks unexpectedly on the door. "Who could that be?"
you'd blurt out in English. \emph{¿Quien será?} you'd say in Spanish. Other
examples: \emph{¿Habrá más?} = "Might there be more?" \emph{¿Estará en casa?} =
"Do you suppose he's at home?" And so on.

The future of uncertainty, applied to the past, uses the compound future perfect tense. That sounds difficult, but really it works
out to sticking \emph{habrá} or an equivalent in front of the past participle of
your choice. \emph{¿Quién habrá sido?} ("Who might that have been?"), you
might wonder on hearing a strange voice on your answering machine.
\emph{¿No habrán querido asustarme?} = "You don't suppose they wanted to
scare me, do you?"

\section{The conditional}

The conditional is a very predictable tense, conveying
thoughts in Spanish that in English rely on the auxiliary "would." It's
an easy tense to understand intuitively. "I'd like to eat now" is expressed as \emph{Me gustaría comer ahora}. "You would have liked it" = \emph{Te habría gustado}."

One special warning applies for the conditional: note that in
English we occasionally use "would" for repeated past actions, as in
"His father would eat every night at seven sharp." This construction
in Spanish calls for the imperfect---in fact, it's a textbook example of
when the imperfect is needed: \emph{Su padre cenaba todas las noches a las
siete en punto}. If you think about it long enough, it becomes clear why
this is not a true conditional, in either English or Spanish. But if you're
simply translating your "woulds," it's easy to make this mistake.

\section{The preterit versus the imperfect}

This face-off is one of the trickiest in Spanish, mostly because
in English we often gloss over the distinction. We do have the progressive past, of course, in such constructions as "He was flying a kite,"
but not all "ongoing" past activities use it. In many cases, you'll have
to slow your translating computer down a few megas and think about
exactly what kind of past action you are describing.

Basically the Spanish imperfect covers two constructions:

\subsection{}

\textbf{The progressive past}, including actions that are taking
place over a period of time in the past, usually in relation to some
other action that happened suddenly. For example, "He was sleeping
when the alarm went off." Or "She was overseas when the new president was elected." Keep your eye (and your mind's eye, if you're speaking)
out for these juxtapositions; they represent one of the most common uses of the imperfect. Because of the explicit juxtaposition, these
are also the easiest situations to recognize as an opportunity for the
imperfect.

\subsection{}

\textbf{Actions that took place over a period of time in the past},
often keyed to the English constructions "used to" and, on occasion,
"would" (see "The Conditional" above). This second common use of
the imperfect is indispensable in describing the way things were: "I
used to work nights" is expressed as \emph{Yo trabajaba de noche}. The key
words aren't always present, though. "When I was a kid, the teachers
beat the students" would be \emph{Cuando yo era niño los maestros golpeaban a los estudiantes}.

The problem of the disappearing key words can be seen in the
following sentences. Just as you could say "My father ate at seven every night," "My father used to eat at seven," or "My father would eat
at seven," you could just as easily state, pure and simple, "My father
ate at seven." Grammatically this last example is correct, but it is confusing without a clarifying context. Did he eat at seven o'clock just
once, or did he always eat at seven? Spanish lets you make the distinction in the verb itself: \emph{cenó a las siete} is clear in communicating that
he ate at seven on a certain occasion; \emph{cenaba a las siete}, using the
imperfect, indicates that it was his custom to eat at seven night after
night.

Understanding this distinction will clear up a lot of the conflicts between imperfect and preterit, but incorporating that knowledge
into your storytelling skills will take some time and practice. Often,
when relating a story, you'll have to jump nimbly back and forth from
imperfect to preterit, and this requires analyzing each action as it
pops up.

Let's say this father usually ate at nine, but on one particular
night he ate at seven; while he was eating, he found a fly in his soup
and fainted. What tenses will make this meaning clear in Spanish?
First, we have to explain that the father used to eat or usually ate (\emph{cenaba}) at nine; on the night in question, though, he ate (\emph{cenó}) at seven.
While he was eating (\emph{cenaba}---the imperfect again), he found a fly in
his soup (\emph{encontró una mosca en la sopa}---preterit) and fainted (\emph{se desmayó}---also preterit).

Sometimes, the differences between past and imperfect are
subtle to the point of near invisibility. In these cases, native Spanish
speakers will sense the distinction, but they have a hard time explaining it you. Let's say you answer the phone and your boyfriend is on the
other end. He could ask, \emph{¿Qué estabas haciendo?} Or he might ask,
\emph{¿Qué estuviste haciendo?} What's the difference? In the first example,
using the imperfect in the auxiliary, he is asking what you were doing
in relation to a sudden action---presumably, the ringing of the phone.
That is, "What are you doing right now (besides talking on the telephone)?" In the second example, he is asking what you were doing,
with the idea that you may have finished by now. "What have you
done all day?" or "What have you been up to?" is closer. Of course, if
he wanted to be completely clear, he could ask \emph{¿Qué hacías?} or \emph{¿Qué
hiciste hoy?} or \emph{¿Qué has estado haciendo?} and so on. But where's the
romance in that?

The important thing in these borderline cases is not to learn
the "right" form. (In fact, your boyfriend would probably just ask \emph{¿Qué
haces?}) The important thing is to search for the distinctions in these
close cases to get a feel for past tenses that will serve to guide you
when no key words or juxtapositions are there to help you along.

Let's take a final borderline example. You were at home last
night. But do you use the imperfect or the preterit to convey that to
your listener? Again, it depends. Both could be correct, but \emph{estaba en
casa} suggests you were there for the duration and something sudden or
specific happened during this time: \emph{Estaba en casa cuando se fue la
luz} ("I was at home when the lights went out"). The juxtaposition may
be implied, not stated, but you create the expectation of one by using
the imperfect. In other words, \emph{Estaba en casa anoche} in effect prompts
the question "And what happened?" \emph{Estuve en casa} is a much more
basic, self-contained, declarative sentence. It just says you were at
home, period. In this sense, it's a safer, more all-purpose choice than
\emph{estaba}.

\section{Special cases}

In Spanish the tense of certain verbs is almost as important as
word choice in getting your point across correctly, and students of the
language are generally unaware of the subtle twists they can give these
verbs by their choice of tense. Yet by manipulating the tenses well,
you can discover ways to express familiar English phrases in correct,
colloquial Spanish.

For instance, \emph{querer} ("to want") changes its meaning on the
trip from preterit to imperfect. Consider the case of \emph{quisieron} (preterit)
and \emph{querían} (imperfect). Both mean "they wanted," but a native Spanish speaker hears a difference. The former suggests that they wanted to
do something (and they did it). Thus their "wanting" came to an end,
at least for a while, so the verb goes into the preterit as a done deal.
\emph{Querían}, the imperfect, suggests they wanted to do something and,
evidently, they still want to. That is, they wanted but were unable to
do something. Thus, \emph{Quisieron ir al cine} and \emph{Querían ir al cine} both
mean "They wanted to go to the movies," but with a difference. The
preterit conveys the idea that they wanted to go the movies, so they
went. The imperfect suggests that they wanted to go the movies but
didn't---maybe after seeing the ticket prices.

The imperfect can also be used to say that they wanted to do
something and perhaps eventually did do it, but not before something
else intervened making that possible. For example, \emph{Querían ir al cine
y los mandé}, ("They wanted to go to the movies, so I sent them"). The
use of \emph{quisieron} implies that they would have gone without waiting
for someone to send them. As in the earlier example about being at
home,an imperfect on its own almost seems to raise the question \emph{¿Y
qué pasó?} It is, so to speak, the tense in which the other shoe is always
waiting to drop.

In the negative, the preterit-imperfect distinction is sharper
still. \emph{No quería ir a la fiesta} implies "I didn't want to go to the party
(but I went along anyway)," perhaps to avoid offending the host. \emph{No
quise ir a la fiesta} says bluntly "I didn't want to go to the party (so I
didn't)." Very often, \emph{querer} in the preterit and negative translates best
as "to refuse." \emph{No quisieron dar sus nombres} = "They refused to give
their names."

Similar changes occur to the verb \emph{poder} ("to be able") when it
makes the switch from preterit to imperfect. \emph{Mi hermano podía romperme la cabeza} suggests "My brother had the physical force necessary
to bust my head," while \emph{Mi hermano pudo romperme la cabeza} suggests that on the occasion in question he put this theoretical force to
the test. In the preterit, in other words, \emph{poder} means you not only
could do something but actually did do it. \emph{El podía nombrar todas las
capitales del mundo} ("He could name all of the world's capitals")
might be said of a smart lad. \emph{El pudo nombrar todas las capitales del
mundo} means the tyke was actually put to the test---and succeeded
("He was able to name all of the world's capitals").

As with \emph{querer}, the distinction is sharper in the negative. \emph{No
podía caminar} might mean that your feet hurt after a long hike and
you spent a day cuddled up on the couch watching sitcoms. \emph{No pude
caminar} suggests you actually tried to walk and fell on your face. In
general, the preterit with \emph{no poder} implies a fairly formidable obstacle
and suggests that an unsuccessful effort was at least made. The imperfect with \emph{no poder} is much less definite and does not hint that an effort was made. To take an extreme example, \emph{No podía volar del techo
de la casa} is stating the obvious: "I wasn't able to fly off the roof of my
house." \emph{No pude volar del techo de la casa} are words that, if uttered
at all, would most likely be uttered in an ICU ward by a sheepish lunatic---that is, after trying to fly and failing.

The last common verb whose meaning varies significantly
between the imperfect and the preterit is \emph{saber} ("to know"). In fact, this
variation causes students of Spanish no end of confusion. It seems unjust that such everyday statements as "I didn't know that!" and "Did
you know he was coming?" require a brain-racking choice of tense.

In fact, with \emph{saber} the difference is usually quite clear-cut.
Here's a rule of thumb that works well most of the time: use the imperfect. In the preterit \emph{saber} usually means "to find out." Some examples are in order. \emph{No sabia eso} = "I didn't know that." \emph{Sabia que
llegarías} = "I knew you'd show up." \emph{¿Sabías hablar español cuando
llegaste aquí} = "Did you know how to speak Spanish when you got
here?" In the preterit, in contrast, \emph{saber} generally refers to sudden
knowledge about a specific event. \emph{Supe que habías llegado} = "I heard
(I found out, word reached me) that you had arrived." \emph{¿Supiste\ldots{} ?} in
particular is almost always used for "Did you hear\ldots{} ?" and implies
some new gossip, revelation, or fast-breaking news. \emph{¿Supiste que gané
la lotería?} = "Did you know (hear) that I won the lottery?"

\section{\emph{ser} versus \emph{estar}}

In Spanish the question is not so much "to be or not to be?"
but "to be (\emph{ser}) or to be (\emph{estar})?" These two verbs are the source of
constant headaches and frequent errors for even intermediate and
advanced students of Spanish. Native Spanish speakers intuitively
choose the correct form without so much as a thought. You should
be so lucky.

One modern Spanish dictionary, in its introduction, makes
this very point (and rather smugly, I thought): "[Foreigners] should
know, so that they realize that the distinction between ser and estar is
clear and precise and that it is just a matter of managing to penetrate
the distinct nature of both verbs, that Spaniards, even the most uncultured ones, never use them wrong."

As a foreigner, of course, you will use them wrong, and about
10 percent of the cases will still seem mystifying to you even years
after you learn the common usages. But in at least 90 percent of the
cases the distinction between \emph{ser} and \emph{estar} is "clear and precise"---or
at least pretty easy to guess. As for that other 10 percent, well, you
gotta leave something to learn as you get older!

\section{The easy ones}

\emph{Ser} is the verb "to be" for things that are That Way, period.
They're not that way in relation to something else, or at certain times
of day, or in the spring or the fall, or only in election years. They are
that way because they were born that way and they will presumably
remain that way until the day they die. Ser is a solid, upstanding
verb---one that you can rely on to give you the same answer time
and time again.

\emph{Estar}, in comparison, is a flake. It is the variable, flighty, here-
today-gone-tomorrow verb "to be." \emph{Estar} covers personality traits that
are ephemeral and ethereal. It describes things that change from one
minute to the next. It's an all-over-the-place, outta-control kind of
verb. It's untrustworthy. It's slippery. You would never buy a used car
from a verb like \emph{estar}.

Let's take an example. Say your boss is a fool. \emph{Es una tonta},
you might say (though perhaps not to her face). But let's also say that
she spent all morning collecting mud samples and is now absolutely
filthy. \emph{Es una tonta and esta mugrienta}. What's more, in a moment of
inspired honesty, you told her that she looked like something that just
crawled out from under a rock, and now she's mad \emph{está enojada}. Just
like her to get so upset about a casual observation, you think. She's so
sensitive---\emph{es tan sensible}.

As you can see, we're getting a good picture of your boss: \emph{es
una tonta} and \emph{es muy sensible} (all the time), and \emph{esta mugrienta} and
\emph{esta enojada} (this afternoon). Probably not the best time to be kicking
back and reading a book, come to think of it.

Some words flat out change in meaning depending on whether
they are governed by \emph{ser} or \emph{estar}, and they can help us "penetrate"
those "distinct natures" we've been told so much about. Here's a rule
that can be applied in most cases: if you can add a "now" or "at the moment" to your description, you should be using \emph{estar}. If not, leave it to
\emph{ser}. \emph{Es un borracho}, for instance, means that someone is a "drunkard"---
a habitual drunk or a wino. \emph{Está borracho}, on the other hand, means
"He is drunk (at the moment)." \emph{Es callado} refers to a man who is
"quiet," not at any given moment but as a way of life---it's his nature;
he is a person who keeps to himself and speaks softly and rarely. To say
of another man \emph{está callado} means something quite different: he is
quiet---now. We are given no insight to his overall personality; we just
know that in this place and at this time, he's keeping his mouth shut.

Learning and reviewing examples is a good way to absorb the
essential difference between \emph{ser} and \emph{estar}. But a few other specific tips
may be helpful as well.

\subsection{}

Use \emph{ser} for general, permanent physical appearance: tall,
dark, handsome, short, light-skinned, ugly. (An exception will be dealt
with in a moment.) Use \emph{estar} for any temporary physical condition:
pale, flushed, disheveled, unshaven, and so on.

\subsection{}

For quantities, numerical or otherwise, always use \emph{ser}: \emph{somos veinte personas, es mucha, era poco, son dos}.

\subsection{}

For possession, use \emph{ser}: \emph{es mío, es de él, son de las señoras,
son suyos}.

\subsection{}

Location is always the province of \emph{estar}. This might throw
you if you think of the location of, say, a building as fairly unchanging.
But location is an implicit recognition of an object's relation to other
things---not a reflection of its indelible self---and thus is a job for
\emph{estar}.

\subsection{}

With all adverbs, adverbial expressions, and present participle forms or gerunds (the "-ing" form), use \emph{estar}: \emph{está bien, están en
buenas condiciones, está lloviendo, estoy nadando.}

\subsection{}

With all nouns, use \emph{ser}. If you have trouble recognizing
nouns, a good device is to key on the presence of the indefinite article
(\emph{un, una, unos, unas}). Thus \emph{es un doctor, eres una tonta, es un santo},
and so on. When the article is missing, as it often is in Spanish (that is,
\emph{es doctor}), you'll just have to remember that it could be used in that
situation and therefore it's a noun that requires \emph{ser}.

\section{Getting tricky: the past participles}

%GETTING TRICKY: THE PAST PARTICIPLES
With past participles (the "-ed" form in English, the \emph{-ado} and
\emph{-ido} forms in Spanish), things start to get tricky. Both \emph{estar} and \emph{ser}
can be used, but they mean different things. With \emph{estar} the participle
is generally being used as an adjective and to describe a passing state.
\emph{Estaba agotado} means "He was worn out (at that moment)."

With \emph{ser} the past participle is generally used to form a passive
construction or a predicate noun. In the case of the passive, you implicitly ask (and often must explicitly state) "whodunnit?"---that is,
who or what caused the action. \emph{Fue agotado}, for instance, means "He
was worn out," meaning something or someone wore him out. A few
past participles are used with \emph{ser} without any explicit causal agent, including \emph{conocido, sabido, tardado}, and \emph{parecido}.

Some examples may help clarify the distinction between \emph{ser}
and \emph{estar}. Say you went on an expedition to a remote patch of rainforest. When you got there, though, you found that it had recently been
bulldozed. On your return, someone may ask, "How was the forest?"
You could reply using either \emph{Estaba destruido}, referring to its destroyed state, or \emph{Fue destruido}, meaning essentially "It has been destroyed" and calling attention to the fact that someone or something
destroyed it. In English both senses can be covered by "It was destroyed." Spanish makes a finer distinction.

Another example: you can say both \emph{Las tiendas son cerradas a
las nueve} and \emph{Las tiendas están cerradas a las nueve}. What's the difference? With \emph{ser} you are saying that the stores are physically closed
by someone at nine 'o clock sharp. That is when the doors are shut and
the keys turn in the locks. With \emph{estar} you are saying that if you go to
the commercial district at nine you will find the stores closed. They
may have been closed at eight, or at six, or at five minutes to nine, but
in any case you will find that they are closed at nine. In other words,
\emph{Son cerradas a las nueve} = "They close at nine"; \emph{Están cerradas a las
nueve} = "They are closed by nine."

Finally, an example you will want to study assiduously if you
are of the married persuasion: \emph{soy casado} versus \emph{estoy casado}. Some
will argue that there's no big difference here. Others will say there's a
world of difference. Basically, \emph{soy casado} is "I'm a married man." It
describes a permanent state. \emph{Estoy casado} means "I'm married," but
some feel it implies "for the moment" or "at present," something akin
to "I am passing through a married phase at the moment." (\emph{Estoy de
casado} would say that unambiguously.) Nonetheless, a man wouldn't
say \emph{soy casado con} (wife's name) but \emph{estoy casado con} (wife's name).
In the past tense, the distinction becomes very clear: \emph{Fui casado con
Maria} means "I was (forcibly) married to Maria (and may still be)." \emph{Estuve casado con Maria} means "I was married to Maria (who is now my ex-wife)."

Look at the following examples and practice separating them
in your mind:

\bsk

\indu \emph{Fue cambiado}. = "It was changed (by someone)."

\indu \emph{Estaba cambiado}. = "It was (looked) changed." 

\indu \emph{Fue dormido}. = "It was put to sleep."

\indu \emph{Estaba dormido}. = "It was asleep."

\indu \emph{Fue rota}. = "It was broken (by someone)."

\indu \emph{Estaba rota}. = "It was (already) broken."

\section{The hard ones: descriptive adjectives}

Don't worry about mastering the gray areas between \emph{ser} and
\emph{estar} from the start. It's enough to know why the differences exist so
as to incorporate them intuitively as you go along. With practice, the
light gray areas will get progressively lighter and the pitch-black regions will soon turn a sort of dark gray. Examining examples and asking yourself why? is the best way to start shedding light on the matter.
Some dubious cases of \emph{ser} versus \emph{estar} follow. Absorb them at your
own pace.

Perhaps you've noticed in your dealings in Spanish that to compliment someone on, say, his beauty, you use \emph{estar}: \emph{Estás guapo}. But
aren't you in a sense suggesting that his beauty is just a temporary
state, that you're saying, "You are beautiful today (but not as a general
rule)?" Some compliment! There's a kernel of truth in your suspicion,
but perhaps because of human vanity, such a comment is generally
taken favorably to mean "You look especially beautiful today."

Compliments highlight one of the largest zones of overlap between \emph{ser} and \emph{estar}, the descriptive adjectives. For instance, if someone is tall, they are presumably tall all the time, and we would correctly expect \emph{es alto} to convey that. So what the devil are we to make
of \emph{está alto}, which you will undoubtedly come across sooner or later?
Certainly you can't temporarily be tall?

In general, using \emph{estar} with adjectives is a way of highlighting
the immediate and subjective nature of a perception---"This is my
impression" or "This seems especially that way to me now." \emph{Es alto}
means "He is tall." \emph{Está alto} means, more or less, "He's so tall," "He's
much taller than I thought," "Gosh, he's tall." To say "He's tall for his
age," for instance, you would use \emph{Está alto para su edad}. If he were
tall, period---in other words, a tall person---you would simply say
\emph{Es alto}.

Let's take another adjective. \emph{Es feo} would be "He is ugly"-no
debate permitted or even needed. Look up "ugly" in the dictionary and
you'll find his picture alongside. Follow him home and he'll have ugly
parents. So what's left for \emph{Está feo?} It could mean "He's temporarily
ugly"---because of a horrible haircut, for instance. Or it could suggest
"He sure looks ugly to me" or "He really is ugly."

Becoming adept at making this distinction is a matter of time
and exposure. What's the difference, for instance, between \emph{es difícil}
and \emph{está difícil?} Roughly, \emph{es difícil} describes something that is always
difficult and notoriously so: say, swimming the English Channel or
learning Chinese. \emph{Está difícil} suggests that something that has come
up is difficult or that something is harder than was expected. \emph{Es difícil aprender la diferencia entre ser y estar}
 = "It is difficult to learn the
 difference between ser and estar." \emph{Está difícil aprender la diferencia
 entre ser estar} = "I'm having real trouble with this \emph{ser} and \emph{estar}
business."

Now how about \emph{es viejo} versus \emph{está viejo?} The first example
means someone is old, period---a senior citizen. The second, with
\emph{estar}, is much more subjective and can cover a wide range of English
translations, including "He feels old," "He looks (seems) old," and
"He is too old (for some specific task)." A similar case is \emph{ser joven}
versus \emph{estar joven}. \emph{Soy joven} means "I am young (i.e., a member
of the group of young people)." \emph{Estoy joven} covers anything from "I
feel young" to "I'm young (for my position)" to "I'm still young" (a
washed-up pitcher to his coach), and so on. Note the difference in the
past: \emph{Cuando eras joven} = "When you were young (a youngster)."
\emph{Cuando estabas joven} = "When you still had some pep (weren't over-the-hill)."

How about \emph{es buena} versus \emph{está buena} in reference to, say, a
film? Again, the key is in the subjective appreciation. After seeing it
and liking it, you might say \emph{Está buena la película} to communicate
your personal approval. You could use \emph{Es buena}, too, but there you'd
be declaring "It is a good film"---well done, professionally made, with
good actors, and the recipient perhaps of several awards. If after seeing
the film you say \emph{Es una buena película}, you are subtly implying "It
was good, but\ldots{}" In my experience, people seem to say \emph{Es una
buena película} in reference to arty films---ones they presume to be
"good" but didn't understand or particularly enjoy.

Some final examples will call attention to an advanced aspect
of the distinction with descriptive adjectives. If a person has a permanent physical illness (polio, for instance), he or she can still be described using \emph{está enfermo} or \emph{está enferma}. \emph{Está enferma desde niña} =
"She's been ill since childhood." Likewise, the adjective \emph{loco} is generally used with \emph{estar}, even when referring to someone who has spent
fifty years in an insane asylum. You would say \emph{está loco} of this person,
less commonly \emph{es loco}. Why? Because illness and insanity, as in these
cases, are not an essential part of the person's character but an exceptional, uncharacteristic condition. That is, it is not in their very nature
as people to be sick; it is a condition, a state, an exception. When adjectives like \emph{enfermo} and \emph{loco} are made into nouns, though, they are
used with \emph{ser}---but usually only with a preceding indefinite article: \emph{Es
un enfermo} = "He's sick (a sick person)."

The extreme application of the "essential nature" principle is
\emph{está muerto} and \emph{está muerta}, which is the only way to say "He (or
she) is dead." Students, understandably, balk at this one, since for most
of us death is considered a fairly permanent state, worthy of \emph{ser}. Actually, the use of \emph{estar} makes sense if you take the perspective of the
individual involved: being dead may be a lasting experience, but it's
not an essential aspect of the individual's nature. When the person is
remembered and eulogized years later, people won't say "He (or she)
was a good person, a kind person, and a dead person." Besides, \emph{Es
muerto} means "He is killed"---the use of \emph{ser} and a past participle in a
passive construction. An illustrative if redundant example containing
both would be \emph{Fue muerto a tiros, y ahora está muerto}, literally "He
was killed by shots, and now he's dead."

\section{Sorting out \emph{ser} and \emph{estar} in the imperative}

Imperatives are a source of some added confusion with \emph{ser} and
\emph{estar}. I've never found a good rule governing their usage in the imperative, so I'll invent one: avoid using either of them in the imperative,
but if you must, always use \emph{ser}. It's a fairly drastic rule, and exceptions
can of course be found if you want to get picky. But it will do for the
most part.

Why avoid imperatives with the "to be" verbs? Because Spanish, unlike English, does not lend itself to them as a rule. In English
you can without hesitation say "Be good," "Be on time," "Be there,"
"Don't worry, be happy," and so on. To translate these constructions
into Spanish, you would almost always resort to a verb other than the
"to be" verbs: \emph{Pórtate bien, Llega a tiempo, Asiste, Anímate}. If you
insisted on using a "to be" verb in Spanish, you would almost always
use \emph{ser}, even when referring to a transitory state: you would say \emph{Sé
amable}, for instance, to express "Be friendly," even if you meant it
only for a short while. \emph{Sé amable con tu abuela, sólo nos visita de vez
en cuando} = "Be friendly to your grandmother, she only visits us from
time to time."

Encouraging you to avoid using "to be" in the imperative in
Spanish is not to say that you won't hear it. It's not an especially common construction, but neither is it rare. Here are some examples you
may run across:

\bsk

\indu \emph{Sé puntual}. = "Be on time."

\indu \emph{Sé buena gente}. = "Be a nice guy (or gal)."

\indu \emph{Estate quieto}. = "Be still." (said to children)

\indu \emph{Estate callado}. = "Be quiet." (said to children)

\bsk

The imperative with both ser and estar is much more frequent
in the negative in Spanish:

\bsk

\indu \emph{No seas malo}. = "Be a pal"

\indu \emph{No seas tonto}. = "Don't be a fool"

\indu \emph{No seas imbécil}. = "Don't be a jerk."

\bsk

With estar the negative imperative almost always is constructed with the present participle:

\bsk

\indu \emph{No estés molestando}. = "Quit bugging me."

\indu \emph{No estés gritando}. = "Quit shouting."

\bsk

Even in the negative, though, imperatives tend not use a "to
be" verb at all, as we have seen:

\bsk

\indu \emph{No te enojes}. = "Don't be mad."

\indu \emph{No llegues tarde}. = "Don't be late."

\indu \emph{No te aloques}. = "Don't be crazy."

\indu \emph{No hagas ruido}. = "Don't be noisy."

\bsk

For the student, the \emph{ser-estar} confrontation is a real and constant struggle. Penetrating these verbs' distinct natures can be time consuming and, frankly, a real pain in the backside. Effort is definitely
required, but it is also repaid, since the intuitive understanding of
Spanish you gain in separating \emph{ser} from \emph{estar} will prove indispensable
to true fluency. As that dictionary quoted above goes on to say about
the \emph{ser-estar} problem: "If [foreigners] feel irritated with Spanish for
this difficulty, they should consider that the differentiation between
the essence and the state of things in everyday speech is but one more
demonstration---perhaps the most brilliant one---of the logical sense
of this language."

\section{A matter of perspective}

Learning how to make your English turn into correct Spanish
is sometimes a matter of mastering the vocabulary and sometimes a
matter of mastering a concept. But sometimes---rarely---it's a matter of
mastering a whole new way of looking at things. It's a matter, in short,
of effecting a change of perspective. Long after basic fluency has been
achieved, many foreigners still have trouble remembering to make this
change. Most foreigners, it could honestly be said, never make the
change completely.

Nowhere does this perspective problem crop up with greater
frequency than with the indispensable verbs \emph{llevar} and \emph{traer}. The key
to getting this distinction right is to remember and implement a very
basic rule: in Spanish you can't "bring" something from where you are
to where you aren't. If you are going to a dinner at a friend's house, you
must ask if you should "take" (\emph{llevar}) something (a salad, a bottle of
wine, etc.). "Bringing" is only for cases when something away from the
speaker is being moved toward the speaker.

In English we tend to play loosely with what is essentially the
same rule. That is, we use "bring" regardless of whether the implied
movement is toward the speaker or away from the speaker. If we are
going to a party, we will offer to "bring" a salad; we will "bring" a
cooler with us when we go on a picnic. In Spanish you have to "take"
the salad and "take" the cooler.

\bsk

Imagine the following phone conversation:

\bsk

\inda JOSÉ:

\indu "Hey, Carlos, I think I left my wallet at your house last night.
Could you bring it over today?"

\inda CARLOS:

\indu "Sure. I'll bring it over in the afternoon."

\bsk

Now, in Spanish:

\bsk

\inda JOSÉ:

\indu Oye, Carlos, creo que deje mi cartera en tu casa anoche. ¿Me la
puedes \emph{traer} hoy?

\inda CARLOS:

\indu Claro, te la \emph{llevo} por la tarde.

\bsk

It's important to pay attention to the distinction not just to
sound better but to avoid sounding rude and demanding when you're
not. Imagine, in our phone conversation, that Carlos had no reason
to go by Jose's house that afternoon---in fact, imagine that it was an
hour out of his way. Imagine further that they have a class together
at five o'clock at the university. Now Jose, by asking Carlos to \emph{traer}
the wallet, is being dreadfully uncouth. Jose is saying, in effect, "Bring
it to me here at my house" when he may have meant "Take it to me
there at school"---in which case he would have to say \emph{¿Me la puedes
llevar hoy?}

The same problem of perspective comes into play with \emph{ir} and
\emph{venir}. In English we can call home and ask if the plumber "came" that
day; we can say we'll fix the sink when we "come" home; when we are
called to the phone, we say "I'm coming." In all of these cases in Spanish, however, you have to use \emph{ir} ("to go"), not \emph{venir} ("to come"). To
rephrase the \emph{llevar-traer rule}, you can't "come" (with \emph{venir}) to a location that is somewhere other than where you are at that moment.
\emph{Venir} can only refer to your present location---where you are sitting or
standing or, in a larger sense, to the city or country you are in. \emph{¿Fue el
plomero hoy?} = "Did the plumber come today?" \emph{Arreglaré el lavabo
cuando vaya} (or \emph{llegue}) \emph{a casa} = "I'll fix the sink when I come home."
\emph{Voy} = "I'm coming (to the phone)."

