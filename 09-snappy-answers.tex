\chapter{Snappy Answers}

Many native Spanish speakers will instinctively judge a foreigner's fluency in Spanish by the foreigner's ability to respond quickly
and conventionally to a wide range of stock situations and comments.
From such simple remarks as \emph{Mucho gusto} and \emph{¿Cómo estás?} to more
complex utterances, many statements seem to have a traditional, almost automatic response built in. The sooner you master some of
these answers, the sooner you will achieve an enviable degree of fluency in Spanish.

Mastering the comeback means having a handful of stock expressions on the tip of your tongue, ready to be activated by the right
comment. \emph{Mucho gusto}, someone says. A click goes off in your brain,
synapses slap together, and your tongue utters \emph{Igualmente}. The whole
episode should take less than a nanosecond. You needn't think at all,
and in fact the less you think, the better. That may not be an ideal rule
for all verbal communication, but for learning comebacks it is bliss.

\section{Affirmative}

For a perfectly adequate affirmative response you can always
say \emph{Sí}, of course, But for true fluency you should shop around for alternatives. How often, after all, do you answer with a simple yes in
English? If someone asks if you want to go swimming, you are more
likely to respond ``Sure" or ``All right" than ``Yes." If someone says ``I'll
call you tomorrow," you may utter an ``Okay" or ``Good." ``Can you
give me a hand?" elicits a ``Sure" or ``No problem," ``Nice weather
we're having." ``Really." ``See you for dinner?" ``Great." Likewise,
stock phrases exist in Spanish for all manner of situations. Get a
handle on a couple of them and you'll soon be saying yes in style.

\subsection{\emph{claro (que sí)}}

%CLARO (QUE SI)
Generally defined as ``of course," we would more colloquially
say ``sure" in English. ``Can you be here by four?" Claro. ``Can you
lend a hand with the dishes?" \emph{Claro}. It is also commonly used as a
sympathetic interjection for ``obviously" or ``naturally," and in this capacity it is often heard in one-sided phone conversations, with the listener repeating \emph{claro} every so often to show he or she is still listening.
Somewhat gaudier alternatives for claro include \emph{desde luego} and \emph{por
supuesto}. \emph{¿Cómo no?} is very natural-sounding, but you will have to
overcome your natural resistance to using a ``no" phrase to say ``yes."
\emph{Seguro} works fairly well as ``sure," but is also used ironically, as ``sure"
is in English. To avoid doubt, use \emph{claro} instead.

\subsection{\emph{está bien}}

%ESTÁ BIEN
This is the phrase of choice for avoiding ``okay," which is used
in Spanish and probably every other living language but is not native.
Furthermore, some uses of ``okay" are simply wrong if translated into
Spanish. You would never say \emph{Está O.K.} for ``It's okay," for instance. So
get used to saying \emph{Está bien} instead. ``All right" can also be rendered
\emph{Está bien}. In fact, \emph{está bien} should be one of the commonest expressions to come out of your mouth when you speak Spanish. ``I'm going
home now." \emph{Está bien}. ``I'll call you later." \emph{Está bien}. ``I'll pick you up
at eight." \emph{Está bien}. ``Let's sell everything and elope to Paraguay." \emph{Está
bien}. To sound slangier, you can say \emph{'tá bien}.

\subsection{\emph{conste}}

%CONSTE
This handy response comes from the verb \emph{constar}, which is
one of those untranslatable words that frustrate dictionary writers.
Basically, the idea behind \emph{conste} (from \emph{que conste}, or ``let the record
show") is roughly ``for the record"; as a comeback, it means ``That's a
confirmed fact" or ``I'll hold you to that." Usually it is used in somewhat jocular fashion, as when a friend promises to visit you soon at
your new home in Arizona. \emph{Conste}, you would respond, wagging a
friendly finger at your friend: ``You've promised." Often it's followed
by \emph{¿eh?} for emphasis. ``I'll be home by eleven, Mom." \emph{Conste, ¿eh?}
(``Remember later that you said so" or ``I'll be waiting up for you").

\subsection{\emph{de acuerdo}}

%DE ACUERDO
Another good choice for the ``all right" or ``okay" group. It
is usually reserved for cases where some definite accord has been
reached, even if it's only an agreement to ``see ya later." More formally,
you can use \emph{trato hecho} (``it's a deal") or just \emph{hecho} (``done") in the
same cases. ``I'll pick you up at eight." \emph{De acuerdo}. ``Pick me up at
eight." \emph{Hecho}.

\subsection{\emph{¿verdad?}}

%¿VERDAD?
A very simple remark but very common as well. It conveys the
idea of ``Ain't that the truth!" but works well for ``Really," especially
in response to questions that beg for an affirmative response. ``This
weather's lousy." \emph{¿Verdad?} ``Sure is nice to be here." \emph{¿Verdad?} ``That's
a great movie." \emph{¿Verdad?} It is often preceded by \emph{sí}: \emph{Sí, ¿verdad?}

\section{Negative}

Handy for requests made of you by friends, simple acquaintances, or even complete strangers, the negative phrase is everywhere
useful. The intensity of a rejection, even without straying over to the
side of obscenity, can range from the polite but firm to the impolite
but firm. As with \emph{sí} and the affirmative, you could just say no for your
negatives. But read on and see what fun you'd be missing!

\subsection{\emph{para nada}}

%PARA NADA
The correct translation for ``no way," as in ``No way, Jose."
``Wanna hear me play `Feelings' on my kazoo?" \emph{Para nada}.

\subsection{\emph{en absoluto}}

%EN ABSOLUTO
Watch out---the phrase means ``absolutely not," not ``absolutely." In fact, it means the same thing as \emph{para nada}, but is
more polite.

\subsection{\emph{¡qué esperanzas!}}

%¡QUE ESPERANZAS!
Another fairly polite way of saying ``Not on your life!" Literally, it means ``What hope!" and suggests ``You must be dreaming."

\subsection{\emph{en tus sueños}}

%EN TUS SUEÑOS
A more impolite way of saying ``You must be dreaming" or
``Dream on!" ``How about a kiss, sweetie pie?" \emph{¡En tus sueños!} Add the
appropriate qualifier at the end and you have a rude, power-packed rejection. \emph{¡En tus sueños, imbécil!} = ``Like hell, Dogbreath." Variants
include \emph{¡ni en sueños!} and \emph{¡ni soñarlo!}

\subsection{\emph{estás loco}}

%ESTÁS LOCO
This is the shortest way of saying ``You're nuts," ``You gotta be
kidding," or ``You must be out of your mind," but it's not always the
most natural-sounding. For that, use \emph{Estás como loco} or the slangier
\emph{Estás como operado}, which is like saying ``You ought to have your
head examined." Remember to make the phrase agree in gender and
number with the object of your derision. \emph{Están como locos} = ``You
guys must be crazy."

\subsection{\emph{ni loco}}

%NI LOCO
A very firm, very natural way of saying ``No way." Literally,
you're saying ``Not even if I were crazy (would I do such-and-such)."
``How about a quick game of Russian Roulette?" \emph{¡Ni loco!} Remember to make gender and number agree: \emph{ni loco, ni loca, ni locos}, or
\emph{ni locas}.

\subsection{\emph{ya basta}}

%YA BASTA
``Enough already!" or, in slangier situations, ``Cut it out!" If
someone is relating a series of dubious anecdotes and ignoring all of
your newly learned expressions of disbelief, you may be able to shut
them up with a \emph{¡Ya basta!} Said with a laugh, it means ``No, don't go
on," ``That's too much," or ``Get outta here!"

\subsection{\emph{eso sí que no}}

%ESO SÍ QUE NO
One you should learn just to impress your friends. It means---are you ready?---``no." But why just say ``no" when you can say \emph{¡Eso sí
que no!} Actually, it's a little stronger than just plain no and comes
closer to ``That's out of the question" or ``Forget it."

\subsection{\emph{fruits and other nonsense}}

%FRUITS AND OTHER NONSENSE
Just for fun, you can substitute lots of irrelevant words for an
emphatic \emph{¡No!} Two common ones are \emph{¡Naranjas!} (sometimes \emph{¡Naranjas de la China!}) and \emph{¡Mangos!} Also used are \emph{¡Narices!} and \emph{¡Cuernos!}
The latter expression refers to (and is sometimes accompanied by) the
two-finger ``hook-em-horns" gesture, which in turn can be traced back
to an indelicate reference to cuckoldry. Use it with caution.

\section{Surprise}

Surprises, by definition, catch you off guard. But that's no excuse to be tongue-tied as well. Here are some retorts that will enable
you to convey your surprise snappily.

\subsection{\emph{¿en serio?}}

%¿EN SERIO?
This is probably the best translation for the ubiquitous English
question ``Really?" and usually suggests that you genuinely believe
what you are hearing, though you might prefer not to. ``Mary's had an
accident." \emph{¿En serio?} It needn't be traumatic news. ``I think my watch
has stopped." \emph{¿En serio?}

\subsection{\emph{¡n'ombre!}}

%¡N'OMBRE!
This universal slang phrase is an abbreviated form of \emph{no hombre} but is pronounced very much like \emph{nombre}. It is like \emph{¿en serio?} but
is slangier and indicates more surprise: ``No way!" Said laconically, it's
a simple negative, equating perhaps with ``Nah."

\subsection{\emph{¡imagínese!}}

%¡IMAGÍNESE!
These exclamations mean ``Imagine!" or ``Imagine that!" but
suggest stronger surprise: ``Can you imagine that?" It's pretty formal
and is a good phrase to throw around when formality is expected of
you, as when your elders are describing shocking things that kids do
these days: ``And girls these days let the boys kiss them on the first
date!" \emph{¡Imagínese! abuela}. The familiar form is \emph{¡Imagínate!}

\subsection{\emph{¡qué barbaridad!}}

%¡QUÉ BARBARIDAD!
The phrase means ``Goodness gracious!" or the like but is
much more commonplace. ``Good Lord!" or some other religious imprecation is often heard in English. It is used to respond to surprising
and generally unpleasant news, such as a natural disaster or a problem
in the extended family. It is not to be used when the surprising news is
also good news. If you're told that your mother-in-law has just won a
beauty contest, saying \emph{¡Qué barbaridad!} might just earn you a fat lip.
\emph{¡Que locura!} (``What madness!") is used in much the same way when
the speaker is emphasizing the absurdity of the news as well.

\subsection{\emph{es el colmo}}

%ES EL COLMO
This can be a tough phrase to translate, but you'll recognize
its use right away. It can mean ``That's too much," ``If that doesn't beat
all," ``That's going too far," and so on. It is used for ``last straw" situations, as when you can't fathom why someone has finally done something utterly outrageous. ``Did you hear about the Van Gogh freak who
cut off his ear?" \emph{¡Na! ¡Es el colmo!} ``That rude son of mine forgot to
call me on my birthday!" \emph{¡Es el colmo!}

Other interjections for surprise and shock include one-word
responses that cover a fairly narrow range of meaning-from ``Gosh"
to ``Omigosh," roughly. Two safe ones are \emph{¡increíble!} and \emph{¡caray!} Both
of these can also be used with \emph{Que} to good effect: \emph{¡Qué increíble!} and
\emph{¡Qué caray!} The first generally is in response to good news, the second
to bad news. Other expressions are euphemistic but safe, including
\emph{¡Hijo de!} and \emph{¡Hijo!} as well as (in Mexico) \emph{¡Híjole!} They all mean ``Son
of a gun!" Common euphemisms for surprise in Mexico are \emph{¡Chin!} and
\emph{¡Chihuahua!} The \emph{Ca-} family of euphemisms is also a good source for
mild, universally understood expressions of shock: \emph{¡Caracoles! ¡Caramba! ¡Carachos! ¡Canarios! ¡Carape!} and so on.

\section{Disbelief}

If you are traveling through the Spanish-speaking world and
are not yet accustomed to its idiosyncrasies, disbelief may well be one
of your most common reactions. Oh, how you will long for some good
expressions when your reservations are not respected, when the bus
doesn't arrive after all, or when the town you want to visit apparently
no longer exists! A good rule for visiting this magical kingdom is to
suspend your disbelief, but when you don't succeed, you'll need to express it. Here's how.

\subsection{\emph{no puede ser}}

%NO PUEDE SER
Literally, ``It can't be," even though of course it often can be
and is. This is the ultimate expression of disbelief, the one that conveys ``I am not experiencing this!" or ``This is not really happening!"
Good for when your breakfast bill tallies up to several thousand
dollars.

\subsection{\emph{no me diga}}

%NO ME DIGA
This covers the laconic ``You don't say," as in English, but is
also widely used for ``Don't tell me that!" or ``It can't be!" Slightly less
desperate than \emph{No puede ser}, it indicates a less significant break with
reality. It can be used, for instance, when you arrive late to a concert
and are told that this performance is sold out. In the familiar, it's \emph{No me digas}.

\subsection{\emph{¡a poco!}}

%¡A POCO!
Slangy and not universal. It usually suggests that you don't believe what you are hearing and can even be a curt cutoff of what you
consider nonsense, like the English ``Bull!" but without the feistiness.
Maybe better English equivalents would be ``You gotta be kidding!" or
``Gimme a break!"

\subsection{\emph{déjese de cuentos}}

%DÉJESE DE CUENTOS
This means ``Cut the crap" but is somewhat more refined.
``Come off it" might be closer. It does mean that you think you are
being lied to---or at least being kidded with---so don't use it too loosely.
A similar phrase is \emph{No me venga con cuentos}. For a strong but
perfectly decent way of saying ``Like hell!" tell the neighborhood liar \emph{¡Eso
cuéntaselo a tu abuela!}---which means ``Go tell your grandma."

\subsection{\emph{¿cómo?}}

%¿CÓMO?
A simple but underrated form of expressing mild shock and
disbelief. If the cabbie says it will cost you seventy-five dollars to get
across town, you can very politely reply \emph{¿Como?}---as in, ``I'm going to
pretend I didn't hear that, so let's try a new answer." Robert de Niro's
character in \emph{Taxi Driver}, had he been speaking Spanish, might have
simply said \emph{¿Cómo?} for ``You talkin' to me?" To embellish this phrase,
you can add a form of the verb \emph{decir}: \emph{¿Cómo dijo?} or \emph{¿Cómo dice?}

\subsection{\emph{seguro}}

%SEGURO
This means ``sure," so it only works for disbelief if you ladle
irony all over it. Since ``sure" often gets this tone in English, that
shouldn't be too hard to manage. The effect is like the English ``Yeah,
right." Other expressions used ironically to express disbelief are \emph{claro,
¿en serio?} and \emph{cómo no}. ``Yes, Mr, Sanchez, your check is in the mail."
\emph{Cómo no}.

\section{Indifference}

A lively and specialized vocabulary for indifference exists in
Spanish, and it's well worth your while to learn a few of its forms. Regional differences are often pronounced, so do pay attention to what's
being used around you and how people react to your usages. Keep in
mind, too, that often the best way of expressing your total lack of interest is---again---with a phrase for surprise or shock and plenty of
irony. \emph{¿En serio?}---spoken as if you couldn't care less---works quite
well for ``I couldn't care less."

\subsection{\emph{da lo mismo, da igual, es lo mismo}, and \emph{¿qué más da?}}

%DA LO MISMO, DA IGUAL, ES LO MISMO, AND ¿QUÉ MÁS DA?
These four everyday phrases are used for polite indifference, as
when you don't have a strong preference. They all mean ``What's the
difference?" or ``It doesn't matter."

\subsection{\emph{me vale gorro, me vale sorbete}}

%ME VALE GORRO, ME VALE SORBETE
A considerably slangier and ruder form of indifference, especially in Mexico, where you will often hear \emph{Me vale} by itself. It
roughly means ``I couldn't give a damn." All of these \emph{me vale} expressions are strong and can be considered substitutes for a rather rude
expression discussed in Chapter 10. Don't worry about their literal meanings---as euphemisms, they are essentially meaningless.

\subsection{\emph{me importa un bledo/comino/pepino}}

%ME IMPORTA UN BLEDO/COMINO/PEPINO
More universal than the me vale constructions and slightly
tamer, it equals roughly ``1 couldn't care less." Literally, you are saying,
``It is as important to me as a goosefoot plant/cumin seed/cucumber,"
respectively. In fact, add the minor foodstuff of your choice and you
will probably be understood.

\subsection{\emph{no me importa en lo más mínimo}}

%NO ME IMPORTA EN LO MÁS MÍNIMO
Same idea using a less slangy construction. This is useful for
when you're worried that eloquence may detract from the directness of
your emotion.

\subsection{\emph{¿y qué?}}

%¿Y QUÉ?
Meaning ``So what?" this can sound quite rude, depending on
your tone. In more friendly circumstances, it works as a slightly impatient ``So what's the point?" \emph{¿Y eso qué?} is a variant.

\subsection{\emph{¿y a mí qué?}}

%¿Y A MÍ QUE?
This means ``What's it to me?" Said with a snarl, it means
``What the hell's that gotta do with me?" Said with a shrug, it's just a
simple ``I don't care."

\subsection{\emph{¿y?}}

%¿Y?
All by itself, this word means ``So?" Depending on the context,
it can be an interested ``So (what happened next)?" or an indifferent
``So what?"

\section{Assorted comebacks}

Here are several more comebacks that don't fit into any category but are useful in certain situations.

\subsection{\emph{¿no qué no?}}

%¿NO QUÉ NO?
A useful and cocky expression that means basically ``I told you
so." Coming back with tickets for the midnight train after being told
that they were sold out, you might flash your prize and say \emph{¿No qué
no?} In certain circumstances, the phrase would be construed as ``In
your face!"

\subsection{\emph{es un decir}}

%ES UN DECIR
This is a very handy expression that students of the language
should try to learn. It means something like ``It's just a way of speaking" but will help bail you out of any situation in which you think
you're being misunderstood. ``His Majesty would like to know if you
just called the queen a cow." \emph{No, es un decir nada más} (``Well, yes, but
that wasn't what I meant" or ``It's just a figure of speech").

\subsection{\emph{o sea}}

%O SEA
A very fine ``crutch word" (``you know" or ``um"), but also a
useful response in its own right. Again, the student of Spanish should
learn it early, since it is a formulaic way of saying ``In other words" or
``Sorry, I missed that." Someone explains to you a convoluted route
you must take to reach the museum, starting with a right at the stoplight. You listen, get mixed up, miss the rest of it, and then say, \emph{O sea,
a la derecha en este semáforo}\ldots{} This generally gets the person to
repeat the directions or clarify the point.

\subsection{\emph{no es para tanto}}

%NO ES PARA TANTO
Translated, this would mean something like ``Let's not get carried away here." Somehow, though, opportunities for it arise more often in speaking Spanish than the English equivalent would suggest.
``You look pretty hot. You want a Coke?" \emph{Sí}. ``Want me to pour it over
your head?" \emph{No es para tanto}. In English you might say ``I'm not \emph{that}
hot!" This is quite a useful phrase once you get the hang of it. ``I liked
that film." ``If you want, we can sit through another showing." \emph{No, no
es para tanto}. Sometimes a good translation is ``It's no big deal."

\subsection{\emph{ni hablar}}

%NI HABLAR
This common expression defies---or rather, embraces---categories. Depending on the context, it can convey ``Don't mention it,"
``No way," ``Oh well," and more. You're most likely to run across it
when confronting an insurmountable obstacle. ``The IRS has ruled
that you can't count your trip to Bermuda as a business expense."
\emph{Ni hablar}.

\subsection{\emph{ni modo}}

%NI MODO
This expression is heard primarily in Mexico instead of \emph{ni
hablar} when dealing with major obstacles or crisis situations. A single
translation is hard to imagine, but a resigned ``Oh well, what the hell"
conveys some of the sentiment. So does ``Tough luck!" ``A generator
blew and there'll be no electricity until July." \emph{¡Ni modo!} ``Mommy, I
don't want to eat the broccoli." \emph{Pues ni modo}. This is not a very sympathetic expression, especially if the problem is someone else's, so be
careful how you use it.

