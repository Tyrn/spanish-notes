\pagebreak
\thispagestyle{empty}
\vspace*{3in}

\begin{flushleft}

 Todo se lo debo a mi manager,

 Flavia,

 y con todo mi amor

\end{flushleft}

\chapter{Foreword and Forewarning}

This book is not a phrasebook and not a textbook, though
it can be used with either. It is more like a guidebook---not to the
Spanish-speaking countries but to the Spanish spoken in those places.
It shows you the dark alleyways, the bright meeting-places, the bohemian nooks, and the pulsing thoroughfares of the language. And it
shows you more than a few shortcuts, guiding you toward the Spanish
you want to learn. Like a guidebook, this book's goal is to help you get
around, whether you're in the boardroom or the barrio.

It is a helpful book, like a boy scout helping an elderly person
across the street, and it is an irreverent book, like an impish schoolchild making faces at the teacher. It is a serious book and it is a funny
book. It will tell you how to be polite to a grandmother and how to
shock a gangster. It preaches Spanish with a smile, a strut, and maybe
just a bit of an attitude. This book wants you to speak better Spanish, and it will stop at nothing, or almost nothing, to accomplish it.

Of course, no book can teach you how to speak Spanish. Only
by \emph{practicando}---and especially \emph{platicando}---can you learn that. So
why read it? Because, as you will soon see, this book makes learning
Spanish more fun. And if learning Spanish isn't going to be fun, why
bother?

\chapter{Acknowledgments}

The idea for this book was born several years ago in the hectic world of weekly journalism in Mexico City. In those busy days,
Walter Gaddis and Alan Robinson were central to polishing the idea
and molding the style and teaching technique that eventually came to
be used in this book. My thanks to them and to my other colleagues at
the late, lamented \emph{Mexico Journal}, among them Cindy Anders, Leon
Lazaroff, Talli Nauman, John Ross, Jim Weddell, and Mike Zellner.
\emph{Gracias} also to Pete Hamill for his encouragement throughout and to
Rodin Mendoza for his superb taste in music.

Many people revised parts or all of this text or made invaluable suggestions, additions, and corrections. Special thanks go to
Ted Bardacke; Bob Ciaffa; C. Bruce Fitch of Brenau University and
Academy in Gainesville, Georgia; William F. Harrison of Northern
Illinois University; Carol A. Klee at the University of Minnesota;
Dr. Marcia Rosenbusch of Iowa State University; and Nancy Rhodes
of the Center for Applied Linguistics in Washington, D.C.

\emph{Mil gracias} go to my sister, Fran Keenan, whose long chats on
the subject and contacts at the National Clearinghouse on Literacy
Education in Washington, D.C., were extremely useful; and to Dr. Uzi
Selzer, a \emph{todólogo} whose title could clearly have been earned in the
field of his choosing.

Naturally, the good folks at the University of Texas Press were
indispensable to making this book a reality. Their faith in the project
and hard work in the editing process were crucial to keeping reader
comprehension high and my errors to a minimum. Some of my errors
no doubt slipped by, and for the record I should state that they are in
fact mine. \emph{Ni modo}.

Finally, the one person whose contribution to the book must
be measured in gigatons instead of kilos is my wife, Flavia. It's a cliche
to say that without her this book would have been impossible, but I'm
going to say it anyhow: without her, this book would have been impossible. A better combination of knowledge and patience is hard even to
imagine. A zillion thanks also to little Flavia---the inimitable "Nena"---who serves as a constant and humbling reminder that learning a language is child's play.

\pagebreak
\thispagestyle{empty}
\vspace*{2in}

\epigraphfontsize{\normalsize\itshape}
\setlength\epigraphwidth{5in}
\setlength\epigraphrule{0pt}

\epigraph{Con mi caballo hablo en alemán,
con las damas de la Corte italiano,
para los asuntos de hombres en francés,
pero para hablar con Dios el español}
{--- \textup{attributed
to Charles V (1500\,--1558)}}

\epigraph{The United States now has the fourth-largest
number of Spanish native speakers in the world,
with 17.3 million speakers}
{--- \textup{CAL/NCLE Notes
(a publication of the National Clearinghouse
on Literacy Education)}}

\epigraph{`English Only'
Is History}
{--- \textup{headline},
Miami Herald,
\textup{May 19, 1993}}

\chapter{Introduction}

You're on a bus, heading south. You've crossed the U.S. border
and entered Latin America. English is behind you; a continent of Spanish lies ahead. Your pocket-size Spanish-English dictionary sits on your
lap within easy reach. For practice, you look up the Spanish words for
everything you see or think of: "bush," "barbed wire," "roadrunner,"
"driver." You made yourself understood at the ticket counter and
double-checked the bus's destination with a matronly passenger, but
you had some trouble telling the driver you wanted to keep your bag
with you instead of sticking it in the vehicle's luggage compartment.
You've held a brief conversation with the young man next to you,
who asked your name, your travel plans, and (you think) your favorite
major-league team. You sit back and close your eyes. Already you're a
little tired. How many weeks or months of speaking like a small, semiliterate child can I stand? you wonder.

You are at the beginning of more than a bus journey. You are
on your way to speaking and understanding a foreign language, a foreign culture, and a foreign people. For most of the world's inhabitants,
bilingualism and even trilingualism is nothing out of the ordinary. But
for most native English speakers, one language is the norm. Breaking
out of that mold will take work. But, as you are about to discover, it is
satisfying work, and its fruits---with a little practice---will last you a
lifetime.

People's reasons for learning Spanish are as varied as the approaches they take to it. You may be studying it for use in business,
school, travel, or the family. Unfortunately, there's no magic formula
or secret recipe to speed your way toward fluency. But there are a few
pointers that can help. Some fall under the heading of common sense;
others are more like folk wisdom. Keep them in mind as your voyage
progresses.

\section{Inhibitions}

%INHIBITIONS
The greatest enemy of learning a language, especially as an
adult, is a person's inhibitions. These vary with the individual, of
course. Some people seem to have been born without any, while others are so afraid of making a mistake that they never give themselves
the chance to. Methods of overcoming these inhibitions also vary with
the individual. Most people lose their fear of sounding silly after a few
weeks of speaking a foreign language; others lose all inhibitions entirely after a few \emph{cervezas} under the stars on the town plaza. One rule
applies universally: to learn a language you'll have to conquer your inhibitions eventually, so the sooner you get started, the better.

One way to get started is to remember that however silly you
might sound using your incorrect Spanish, you'll sound a lot worse trying to speak English to someone who speaks none. Then again, you
could simply choose to clam up altogether. After all, as they say, better
to keep quiet and be thought a fool than to open your mouth and remove all doubt (or, as it might be expressed in Spanish, \emph{en boca cerrada no entran moscas}---"flies don't enter a closed mouth"). If this
is your strategy, you'll neither improve your Spanish nor become acquainted with the new world---the Spanish-speaking one---that for
whatever reason you are making an effort to get to know. In fact, you're
probably better off staying home.

So relax. You'll definitely make mistakes. But you won't be the
first one to make them.

\section{How We Learn}

%HOW WE LEARN
Learning psychologists have covered this theme close to the
point of exhaustion (or beyond it, perhaps), but a few observations
might prove useful. One maxim says that you can chart your language-learning progress by three landmarks: speaking and understanding the
basics, then learning the language well enough to use it and understand it on the phone, and finally being able to understand the jokes.
Another common belief holds that language learning tends to be a
quantum experience. That is, you will progress by small leaps and
bounds, followed by long, frustrating plateaus. The plateaus, furthermore, always seem to hit when you think you should be progressing
the most---after an intensive course, for example. At times it will
seem that your brain is too busy absorbing new information to be
bothered with relaying it to your mouth. Fear not! The information is
oozing in and assuring itself a place, and one day it will suddenly be
available and act as if it had been there all along. So stick with it. The
day will come.

\section{Learning Tricks}

%LEARNING TRICKS
Are there shortcuts for getting around the long months, even
years, that are needed to reach a level of virtual fluency? In a word, no.
But there are some specific teaching tools that can help. One surefire
(and entertaining) way to boost comprehension and get a better "feel"
for Spanish is by listening intently to songs in Spanish and writing out
"the lyrics as well as you can. The catchier the song, the better. A friend
once learned an important usage of the subjunctive after prolonged,
late-night exposure to Rubén Blades's song "Pedro Navaja," about a
street tough who keeps \emph{las manos siempre en los bolsillos de su gabán, pa' que no sepan en cuál de ellos lleva el puñal} ("his hands at all
times in the pockets of his coat, so they don't know in which of them
he's carrying the knife"). Try singing along with radio songs and jotting
down the refrains. Learn to equate dance halls with lecture halls. Even
if you don't learn much more Spanish, you'll have a lot more fun!

Another shortcut, applicable of course only in certain cases, is
to turn your mind to a foreign-language romance. Just as a song can
stick in your head for hours at a time, so can Mr. or Ms. Right. Arrange
a date with a Spanish-speaking object-of-your-affections, and you'll be
amazed how your brain works overtime, for hours and days ahead,
thinking up cute and clever things to say at the appointed hour. It's
really just an advanced mnemonic device, but a far more pleasant one
than, say, word association.

In general---and you'll hear this repeatedly from your teachers
and coaches, formal or otherwise---try to speak to as many people in
Spanish as possible. While that sounds easy, the sad fact is that it's often awkward to speak to your fellow citizens in a foreign language, and
from there it's a short jump to seeking out your \emph{paisanos} wherever you
happen to be and speaking with them in English almost exclusively.

The intellectual energy that goes into starting a conversation
in a foreign language can be quite daunting, especially in the early
stages. Still, it's worth the effort. Concentrate at first on short "conversations" (or extended greetings) and gradually lengthen them as you
find people whom you feel comfortable speaking with (and are able to
get away from when your vocabulary expires). When the temptation
to chat with a fellow English speaker becomes too great, give in to
it---but try to steer the conversation toward anecdotes about the language you are both probably trying to learn.

A useful "trick" to improve your pronunciation, which is
handled in more detail in the next chapter, is to practice tongue-twisting words and phrases when you're off by yourself---in the
shower, walking down the street, waiting for a bus, or on walks in the
woods. Words like \emph{problema} and \emph{refrigerador} may require lengthy
repetition before they agree to come out sounding more or less like
they're supposed to. If in the process some people overhear you and
look at you funny, don't worry. In most of the Spanish-speaking world,
gringos are presumed to be a little daft almost by definition.

Finally, don't hesitate to ask others to speak slowly. No one
expects you to understand rapid-fire Spanish in your first few months
of learning it, yet many people speak that way out of habit and need to
be reminded that you're comprehending at about one-fifth the rate
they're speaking. A simple \emph{Más despacio, por favor} (or \emph{¿Puede hablar
un poco más despacio, por favor?}) will do wonders for your ability to
understand and respond.

\section{Slang and Cursing}

%SLANG AND CURSING
Most language books tell you not to use either slang or "four-letter words" in Spanish. The reasoning is somewhat foolish: by using
them in a way that will no doubt be incorrect at first, you are likely to
make a fool of yourself. This is true, of course, but if you were worried
about that, you would never have left the cozy confines of your native
tongue in the first place. Certainly it should be no deterrent to trying
your hand at this most lively and emotive aspect of Spanish.

In any case, most students ignore the advice and try to incorporate some slang phrases and even obscenities into their Spanish.
They do so for many of the same reasons they do so in English: (a) a
little invective can come in handy at times, if only to let off steam as
you travel through fascinating but sometimes frustrating new cultures;
(b) all your friends---in this case, your new Spanish-speaking acquaintances---are doing it; (c) talking tough can be fun; (d) talking like a
schoolmarm all the time can relay an inaccurate image of one's personality; and (e) it can also get insufferably boring.

What the language books should tell you is to slip in a little
slang, try out the occasional dirty word (if that's your inclination), and
make a go of it. In this book, at the risk perhaps of offending some
readers, I've tried to address the issues of using slang and cursing,
pointing out what to watch for and when not to use certain expressions (usually far more relevant than when to use them). As with the
language itself, there are definite traps to be avoided. The only difference is that in the world of \emph{groserías}, the penalty for falling into a trap
can be considerably more painful, up to and including lengthy hospital
stays. That would seem to be reason enough to include a little advice
on the subject.

\section{Regional Differences}

%REGIONAL DIFFERENCES
Wherever you go in the Spanish-speaking world, you will run
into speech idiosyncrasies that show up in dialect, word choice, slang,
and intonation. A Mexican in Chicago won't necessarily speak the
same as one in Guadalajara or Veracruz, and none of them will speak
the same as a Guatemalan or a Spaniard or an Argentine. No book can
cover all of these variations, and every book contains a built-in bias
toward one form or another. This book's bias is toward the Spanish of
the Americas, although an effort has been made to call attention to
expressions that are regional or that vary significantly from one place to
the next. Added emphasis is assigned Mexican Spanish, since that is
the Spanish most North Americans will hear---be it on vacation or in
their own countries. Aside from a handful of words that are common
in one place and considered risqué in another, most regional differences are more a matter of style than substance. In general, you'll be
understood regardless. And as an obvious foreigner wherever you go,
you'll almost always be humored---or forgiven, in the case of risqué
words---for your word choice.

\section{How I Came to Know the Territory}

%HOW I CAME TO KNOW THE TERRITORY
I chose the anecdote that begins this section for a very simple
reason: it was my introduction, give or take a few details, to the
Spanish-speaking world. When I said you were going to wonder how
many weeks or months of speaking like a small, semiliterate child
you could stand, I did so because that is exactly what I wondered on
my first brusque introduction to the Spanish-speaking world.

Which leads to a confession of sorts. I was not born speaking
Spanish, nor did I learn it in those absorptive first years of childhood. I
learned it in a way that is at once the hard way and the easy way: talking to people, struggling to listen to their answers, and grasping constantly for understanding. It took me ten years to reach a level that I
am still reluctant to call "fluency"---though friends, family, and colleagues who speak little or no Spanish are easily wowed by my prowess. Even kind-hearted native-speakers will tell me I have "no accent"
or ask what Spanish-speaking country my "slight" accent hails from.
But deep down I know better. I still make little mistakes---even frequently when tired or in a hurry---and my accent still seems to have
good and bad days. I'm fluent enough for most purposes, but I know
there's a next level, and I'm striving to get there.

This book is unusual because it was not written by a native
speaker, much less a world-class expert on Spanish. Books that are
written by these people are useful---indeed indispensable---because
they shed light on the nuances of Spanish that foreigners only rarely
manage to grasp. In fact, I used many of these books in preparing
this one.

Instead, this book was written by someone whose Spanish was
once halting and clumsy and, before that, simply nonexistent. I like to
think that's what makes this book special and useful. I know about the
frustrations of getting started in Spanish because I've been there. Many
of the mistakes I warn against in this book are ones I'm familiar with
because I made them. Many of the "trickster" words, discussed in
Chapter 3, tricked me. The advice I give on how to learn a language,
how to avoid pitfalls, and how to pronounce certain tongue twisters
is something that worked for me. It may not work for everyone, but
then again it may work for you. All I know is that if I wanted a guide
through a minefield, I'd be inclined to choose someone who has made
the trip before and not necessarily the person who drew the map. If
this someone can also make the trip more fun and less frightening,
then time's a-wasting! \emph{¡Vámonos!}
