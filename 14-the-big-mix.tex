\chapter{The big Mix}

Spanish, as we have seen, is in a way a melting pot of linguistic influences. All languages are. And few would deny that Spanish has
benefited from its contacts with other languages. Without its prehistoric Iberian, Celtic, and Basque influences, without the influx of Germanic terms during the reign of the Visigoths, without the thousands
of Moorish words imported from Arabic, without the Italianisms of
the Renaissance and the Gallicisms of recent centuries, Spanish would
simply not be Spanish as we know it and speak it today. It would be
Modern Latin.

In the twentieth century, English is the language that has most
influenced Spanish (and about every other world language, for that matter). U.S. commercial, political, and cultural dominance has in the field
of languages resulted in the steady infiltration of English words into
Spanish. What is thought up on Madison Avenue, put on screen in
Hollywood, or schemed in Washington almost inevitably turns up, in
one form or another, in spoken Spanish. And though the tendency to
adopt Anglicisms is often presented as an acutely Latin American phenomenon, any visitor to Spain or reader of Spanish magazines will see
that English has insinuated its way deep into the heart of Castile
as well.

As an English speaker learning Spanish, you are present on the
front line of this phenomenon. And the military metaphor is not out of
place. For there is a battle going on between perfectly adequate Spanish
words and aggressive English invaders. It is a battle that causes lexicographers to lament and purists' blood to boil.

Avoiding Anglicisms in your Spanish is generally synonymous
with being a careful and respectful speaker of the language. As a beginning speaker (or even a lazy advanced speaker!), you will find yourself
tempted to use English words as a shortcut or a crutch. At other times,
unknowingly, you may give Spanish words a meaning that their cognates have in English---but not in Spanish. Finally, even as a near-fluent speaker you will have to guard against using English constructions and translating English phrases ``incorrectly." Often these are
borderline judgments, as when choosing between \emph{peor es nada} and
\emph{mejor que nada} for ``better than nothing," or \emph{un día de estos} and
\emph{uno de estos días} for ``one of these days." It is the difference, in other
words, between being correct and being more correct. Proceed attentively, and let your ear be your guide.

These adjustments take time and attentiveness. And some English loanwords you may find useful or even necessary to your style
of speech, especially when everyone around you is using them, For
the most part, though, it is probably better for the ``contamination" of
Spanish to occur at its own rhythm, without your contribution.

Scholars have long debated the reasons for the adoption of
foreign words, but the only clear reasons are the obvious ones. When
something is invented or introduced to a culture, for instance, it often
comes with a word---or an entire nomenclature---attached. Computers
are a good example: software, hardware, and mouse are all ``Spanish"
words now. Similarly, when the word \emph{toronja} was adapted from the
Arabic, \emph{burócrata} from the French, and \emph{canoa} from the Arawakan, it
was because the products or concepts they described were new to the
Spanish-speaking world.

The adoption of foreign loanwords is not always so simple
a matter. In the fifteenth century, for instance, the Spaniards found
themselves in need of a word to describe ``mustache," it not having occurred to a Spaniard until that time that such a thing could exist without the rest of the beard. At first, the Germanism \emph{bigote} found favor
(itself a word that some claim came via Norman French from the English expression ``by God," presumably for what the English said when
they first saw a Norman wearing one). A century later, the Italianism
\emph{mostacho} (originally from the Medieval Greek \emph{moustaki}) became the
rage. Eventually, the Germanism won out, though \emph{mostacho} lives on
in Spanish as an infrequent archaism.

As the pace of inventions has accelerated in recent decades,
the need for new words to describe them has grown. But usually there
are numerous options for fulfilling this need. As the use of motorcars
spread, for example, the construction of special roads to accommodate
them began in earnest. In English, one of the words given to these roads
was ``highway." In Spanish, new words were needed as well, Which
word would be used? Had the sway of English been as powerful then,
Spanish speakers might have undertaken the direct importation of
``highway," as in the Spanishized \emph{jaígüey} or some similar monstrosity. A generation earlier ``highball" had been imported and rewritten
as \emph{jaibol}. Fortunately, perhaps, that didn't occur. Another possibility
might have been to translate the meaning of the English word into
Spanish, again presuming English was influencing the selection. This
would have produced a phrase like \emph{camino alto} for the English word.
But this didn't happen either. A third choice would have been to invest
an existing word or words with the new meaning. This in fact did happen with two words---\emph{carretera} and \emph{calzada}---both of which predate
highways but are now used in modern Spanish to describe them. Other
words came and went. As late as the 1950s, for instance, bilingual
dictionaries gave \emph{camino real} as a translation for ``highway," but this
phrase has since plummeted from common use. Lastly, an altogether
new word, or ``neologism," could be coined to handle the novel concept of a high-speed road. \emph{Autopista} was just such an invention.

Often the importation of a concept can result in a combination
of linguistic responses, or even different responses in different places.
One word may be coined in Mexico, for instance, while another is
coined in Puerto Rico, a third is invested with the new meaning in
Chile, and the English word imported lock, stock, and barrel in Spain.
A 1974 study of Latin American students in the United States explored
this process and found, among other things, that the concept ``sneakers" had led to seven different words in eighteen different countries:
\emph{tenis, zapatos de tenis, zapatillas, champion, zapatillas de goma, zapatos de caucho}, and \emph{zapatos de lona}. Other ``imported" concepts had
as many as nineteen different Spanish names! Dictionaries---and in
particular the dictionary of the Royal Spanish Academy---are reluctant
to include a word until it has shown some staying power. As a result,
speakers of the language are left with no guidance in the meantime and
must invent, invest, adopt, or adapt as they can.

Practical necessity, moreover, is but one of the many mothers
of the invention of new words and the importation of foreign ones.
More often than many like to admit, foreign words are imported because using them makes the speaker or writer seem more suave, chic,
and debonair. Foreign words have a certain je ne sais quoi that can
make them simply irresistible. We use them not because English or
Spanish doesn't have the words we need, but because we are a little
bored with the words we have. We feel the urge to break out of the
angst and the Sturm und Drang of our daily drudgery, so voila, we import ``new" words---ad nauseum. In short, we like to show off.

Sometimes, too, we use foreign words because we want specifically to convey a sense of the exotic. Advertisers are especially
guilty of this, but they are not the only ones---and they can't always
be blamed. Would you sooner choose Boca Raton or Mouse Mouth for
a Florida honeymoon? Need something dramatic to tell your daughter
and son-in-law as they head off for their week in Mouse Mouth? How
about ``Bon voyage!" instead of ``Have a good trip"? What is more explicit than ``blitz" to describe eleven grown men attacking a quarterback? And after a nice, refreshing shower, would you rather sprinkle
yourself with eau de toilette or toilet water?

In Spanish, too, words are imported left and right, and the
state of world superpowers dictates that many of those words are English ones, for the time being. Someday, to paraphrase Kurt Vonnegut,
Chinese linguists and historians might find the phenomenon quite
interesting.

\section{Anglicisms}

%ANGLICISMS
The Anglicisms that enter Spanish do so via different paths.
Some enter exactly as they are written and spoken in English and remain that way. Others are modified to meet Spanish rules of grammar,
pronunciation, and spelling. Some are misappropriated or misspelled.
Some are introduced and then acquire new meanings.

Though there is great variation, imported words tend to follow
a fairly predictable pattern. Imports in Spanish-speaking countries often start as the province of the educated middle and upper classes, who
generally speak a little (or a lot of) English and know how to spell it.
After a word is imported as a foreign term, it must withstand the test
of time. If there is a need for the word, real or perceived, it may survive
and spread into the general public. There it is twisted and shaped until
it slides more comfortably off the tongue and is written as it sounds to
each ear, half a dozen different ways, until a uniform spelling is settled
on or imposed. Then, much later, it may be approved by the gatekeepers of the Spanish language at the Royal Academy. And then again, it
may not.

In English-speaking or bilingual areas, the importation of
words seems to be almost a haphazard process: anyone can play, and
most do. It is these areas---the borderlands between Mexico and the
United States, the streets of New York, the barrios of Los Angeles---where hundreds if not thousands of Anglicisms are bred. The words
that emerge from these cultural and linguistic cauldrons are often the
ones that seem most grating and most unnecessary to language purists.
As the two languages mingle freely, terms like \emph{edible} (instead of \emph{comestible}) and \emph{marqueta} (for ``market") begin to appear as a result of
the, well, mélange.

These \emph{pochismos}, as the border loanwords are called in
Mexico, are hard to feel neutral about. On the one hand, it is difficult to
stand in the way of a language's inevitable and inscrutable evolution.
Linguists have found that the incorporation of loanwords follows a system, and argue that the invention or adaptation of English words by
immigrants is in many cases a defensible response to the trying conditions of immigration. Concepts that are common in the dominant culture but unfamiliar to the minority culture frequently require new
words to express them. The loanword-loaded Spanish of the border
areas, moreover, has become an integral part of the bilingual and bicultural identity of many of its speakers.

On the other hand, it is hard to justify pronouncing any English word that pops up with a Spanish accent and calling it Spanish.
Most people try to avoid the extremes and decide for themselves which
words are acceptable to them. A quick study of Appendix A---an annotated list of many of the more frequent imports---will help you draw
your own line.

An English word's transition into Spanish isn't always a
smooth one. In fact, there's many a slip twixt English and Anglicism.
The word \emph{crack}, for instance, appears on the financial pages in the
Spanish-speaking world to refer to a stock market ``crash." Similarly,
in Mexico a car's ``brake lights" are called \emph{luces de stop}, or ``stoplights," which of course are a different thing altogether. ``Hitchhiking,"
a practice almost certainly introduced to Spanish-speaking countries,
generally goes by the name of \emph{autostop}, though it is not known by that
term in any English-speaking country.

Other cases of the ``wrong" word being imported include \emph{lifting} (``facelift"); \emph{clip} (``paper clip"); \emph{smoking} (``tuxedo" but presumably
an abbreviated form of ``smoking jacket"); \emph{dancing} (usually a ``dancing club," not the act of dancing itself); and \emph{happy} (almost invariably
``drunk" or ``tipsy" in Spanish slang).

Nor does the importation process seem to pay too close attention to word morphology and gender. ``Punch" became \emph{ponche} but the
similar \emph{ponchar} comes from ``to puncture." Spelling presents problems
as well. Aware of the penchant of Spanish speakers to place an \emph{e} before English words beginning with \emph{st} (\emph{estrés, estéreo, estéik}, Rolling
Estones, etc.), some word-importers overcorrect and write ``establishment" as \emph{stablishment}.

What happens to the spelling once a word is in the public
domain is the stuff of orthographers' nightmares. Would you have
guessed, for instance, that \emph{estíur, bisté, friquearse}, and \emph{náilon} are your
old friends ``steward," ``beefsteak," ``to freak out" and ``nylon?" How
about \emph{uachimán} (``watchman") and \emph{yes} (``jazz")?

Ironically, using these foreign words is not at all frowned upon
in most Spanish-speaking circles, just as using ``chic," ``debonair," and
``suave" is considered perfectly acceptable if slightly snooty English.
Overall, different rules can apply for verbs, which require greater inventiveness and Spanish endings to adopt ``correctly." Few verbs, in
fact, overcome the initial years of sounding downright frightful.

\section{Barbarisms}

%BARBARISMS
Perhaps it just reflects an elitist attitude, but English words
are usually not stigmatized until they reach the level of popular
use---and misuse. When lunch becomes \emph{lonche} and every corner
boasts a \emph{lonchería}, then some of the charm of using a foreign word
wears off. \emph{Shopping} is used by Spanish speakers who can afford to take
shopping trips to the United States; pronounced \emph{chopping} or written
\emph{chópin}, it becomes a ``vulgar" usage. Thus you can sound cosmopolitan talking about tonight's \emph{speaker} at the club; call him or her an \emph{espíquer}, though, and you're liable to have your club membership revoked.

The distinction between cultured imports and vulgar imports is mostly a specious one. Is \emph{stand} somehow ``good Spanish" and
\emph{estand} ``bad Spanish"? Is \emph{hamburger} fair and \emph{hamburguesa} foul? You
can be your own judge. Is it acceptable to refer to your car's \emph{clutch} but
vulgar to call it a \emph{cloch}? Can you talk uprightly about \emph{containers} but
should you bow your head in shame when you utter \emph{contenedores}? Is
\emph{rin} worse than \emph{rim, sóquet} worse than socket, and \emph{zíper} worse than
\emph{zipper}? Is using \emph{nipple} crass only when you pronounce it ``knee-play"
and write it \emph{niple}? These, for the most part, are the ``barbarisms" that
you will be warned away from using by more careful speakers of the
language. And as a rule, these warnings are well taken. You won't impress anyone with your command of English---you will only disappoint
with your laziness in not learning the correct Spanish word.

Other barbarisms are a subtler prey, requiring far greater alertness than is needed to avoid saying \emph{breakecito}. As English contaminates Spanish in border areas and, indeed, around the world, odd phrasings and meanings begin to emerge from otherwise innocent-looking,
fully legitimate Spanish words. Phrases like \emph{vacunar la carpeta} appear,
trying to mean ``to vacuum the carpet" but in ``real" Spanish meaning
``to vaccinate the folder." ``Groceries" becomes \emph{groserías}, which in
fact means ``rude remarks" or ``offenses." The nice new neighborhood
on the outskirts of town is called a \emph{suburbio}, though that's what the
slummy part of town was called a generation ago. And on and on.

Most English-influenced barbarisms are considerably less astonishing than these, and Appendix B lists a number of them for your
perusal. Some are so subtle that it is really a judgment call whether
the English is influencing or not. For instance, \emph{admitir} has as one of
its accepted meanings ``to accept, to consider provisionally an explanation, thesis, etc. as good or true." But is it correct to say, as newspaper
headlines throughout the Spanish-speaking world do, that \emph{el criminal
admite su culpa} (``the criminal admits his guilt")? Technically, probably not, but you'll find it just the same.

If you think that this is but a semantic splitting of hairs, consider the word \emph{remover}, which has as its second accepted meaning ``to
take away an inconvenience or an obstacle." Still, its first and by far
most common usage in Spanish is ``to move about, to stir, to jiggle."
If the person using \emph{remover} is a doctor speaking to a nurse about the
catheter sticking out of your arm, you may not think it nitpicking at
all to explore the subtleties of this verb.

For many Spanish words, the culprit in their increasing confusion with English cognates may be the news media---specifically the
wire services, whose translated news reports are sent to newspapers,
radio stations, and television networks worldwide. When translating
vast amounts of text a night, as wire-service translators based in New
York or Miami must do, it is easy to accept an easy, ready-made translation instead of a more complicated one. Thus ``to remove" becomes
\emph{remover} instead of the more correct \emph{quitar}; ``admits his guilt" becomes \emph{admite su culpa} instead of the more precise \emph{reconoce su cu1pa};
and so on. These hairline infractions then travel the globe and reach
into people's living rooms at night and across their breakfast tables in
the morning. Their influence is incalculable.

Other scholars blame the movie industry---and especially the
movie-dubbing industry---for distorting Spanish grammar. When an
actor in a close-up says ``I am waiting for him," for instance, the temptation is to translate it \emph{Estoy esperando por él} to match the lip movements. This is bad Spanish, but that may be the price we pay for good
dubbing.

\section{Reverse \emph{pochismos}}

%REVERSE POCHISMOS
English has not always been the world's dominant language.
For most of the last three centuries, France held that position, and before that---for much of the sixteenth and seventeenth centuries---Spanish was the language that worldy men and women went out of their
way to learn. Those years of dominance have left their mark in the
Spanish we speak today. For just as border and other forms of Spanish
are being infiltrated by Anglicisms and English influences, so were Hispanisms and Spanish influences a ``problem" for English and French in
centuries past.

The centuries of Spanish sway span much of the European
exploration of the Americas, and it is no surprise to find that many of the
Europeans' discoveries bear Spanish names. Even when the original
name was Taino or Arawakan, it passed through a Spanish filter before
reaching French, English, and the rest of the Old World's languages.
Thus foods (``maize," ``tomato," ``chile"), animals (``peccary," ``llama,"
``jaguar," ``iguana"), and other concepts (``canoe," ``cacique," ``hammock," ``hurricane," ``cigarette," ``cannibal") reached English from native languages via Spanish. (The word ``potato," as a former U.S. vice
president would be happy to hear, has been misspelled from the start:
it originated as a confusion between the Quechua \emph{papa} and the Arawakan \emph{batata}, becoming thereby the Spanish \emph{patata} and the English
``potato.")

Spanish influence turns up unexpectedly in our everyday
speech. ``Tuna" comes from the Spanish, as do two of the three classic
ice cream flavors (``chocolate" and ``vanilla"). We have Spanish to
thank for ``mosquito" and ``cockroach," as well as for ``amigo" (of
course) and ``comrade." The Spaniards' dedication to settling the New
World gave us landholding terms like ``ranch," ``hacienda," and ``estancia." Early settlers' obsession with racial distinctions gave English
``mulatto," ``creole," ``mestizo," ``negro," and ``sambo." ``Coconuts" are
so called because the first Europeans to see them---Spanish or Portuguese sailors---saw scary faces in the stained husks, and thus called
them cocos, or ``bogeymen."

Many Spanish words entered English through the U.S. Southwest, which of course used to be the Mexican Northwest. Topographical terms such as ``canyon," ``barranca," ``arroyo," ``mesa," and ``cuesta"
reached English this way. So did ranching terms like ``chaps," ``stampede," ``buckaroo," and ``rodeo." The wildlife of the Southwest boasts
many Spanish names: ``coyote," ``armadillo," ``chuckwalla," ``chaparral," ``saguaro," ``yucca," ``pinyon," ``locoweed," and so on. The states
of the Southwest themselves point back to a Spanish-language heritage: Arizona, California, Colorado, and Nevada (plus Florida and Montana, of course).

From the Southwest and the Old West come some of English's
more colorful Spanish loanwords, a few of which lost more than a
little of their ``Spanishness" in the translation. Still, no western would
be complete without some ``desperado" ``lassoing" a ``bronco" (from
\emph{desesperado, lazo}, and \emph{bronco}). ``Lariat" is but the Spanish \emph{la reata},
just as ``alligator" is a lumping of \emph{el lagarto}. If you were a bad guy in
those parts, you could end up in the ``calaboose" or the ``hoosegow,"
which originated as \emph{calabozo} and \emph{juzgado}, respectively, When we
call something a ``cinch," furthermore, we are recalling the Spanish
\emph{cincha}, which was the sure and steady ``saddle-girth" used by southwestern horsemen. The bad guys, incidentally, wore black ``sombreros,"
whereas the good guys could be counted on to show up in white ``tengallon" (from \emph{tan galán}, or ``so gallant") hats. When they did show up,
they ``mosied" down Main Street, while the bad guys went ``vamoose"
(both from \emph{vamos}). After all, these were some tough hombres!

Some words are of more dubious Spanish origin. ``Hoosier" and
``hoodlum" are words whose beginnings are lost to history, but which
some scholars speculate had Spanish origins. ``Loafer" (and thus ``to
loaf") may come from \emph{gallofero}, an old word for ``vagabond," and ``palooka" may come from \emph{peluca}, equally archaic for ``severe reproof."
My own personal favorite folk etymology is for ``cocktail," perhaps in
part because it's a word that is now considered an Anglicism in Spanish. According to one published version, King Axolotl VIII of Aztec-era
Mexico had a daughter named Xochitl, who invented the cocktail and
had it named after her. Xochitl gradually became Coctel, and the rest is
history---or nearly so.

The process of linguistic infiltration continues in modern
times, of course. With North Americans' more recent ``discovery" of
Mexican food, words like ``taco," ``burrito," ``tortilla," and ``frijoles"
have entered everyday English. Spanish names play a prominent role in
recent automotive history; Mustang, Barracuda, Vega, Granada, Matador, Pinto, and Fiesta are some well-known examples. And ``macho" is
just as much English as Spanish these days, though the phrase ``macho
women," using the masculine form, must sound odd to Spanish speakers. Just as when words travel from English to Spanish, some that have
gone from Spanish to English have had a rough trip. You may hear ``mano
a mano" in English these days, for instance, but the usual meaning assigned to it is ``man to man," not ``hand to hand." Appendix C lists
other words that Spanish has contributed to English over the years.
Many other such words exist, and more still are being imported and
exported, across borders and within them, even as we speak.

As we speak, indeed.


%%% Local Variables:
%%% mode: latex
%%% TeX-master: "main-keenan-breaking-out"
%%% End:
