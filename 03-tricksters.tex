\chapter{Tricksters}

Every non-native speaker of a foreign language is afraid of
making mistakes. And good thing, too. A little fear forces us to concentrate harder and causes the memory of our mistakes to linger-long
enough to correct them the next time we open our mouths.

Most mistakes are grammatical and are eliminated only after
long periods of trial and error. Other mistakes, though, are almost the
fault of the languages themselves. Spanish and English, because of
their long history of coexistence and their many common origins, contain a lot of words that are similar on the surface but are used quite
differently. This makes instinctive translation at times as dangerous as
a cross-eyed knife-thrower.

Fortunately things aren't quite that bad. A lot of these false
cognates are so frequently tripped over by students of both languages
that they can be identified in advance. In the following list of ``tricksters," you'll find some of the Spanish words most commonly misused
by native English speakers. Review the words and keep them in the
back of your brain. You won't overcome all of your instincts overnight,
but knowing your enemies---in this case, the tricksters---is the first
step toward conquering them.

\section{\emph{acostar}}

%ACOSTAR
Not ``to accost," which is \emph{acosar}. \emph{Acostar} simply
means ``to lie down" and is usually used in the reflexive. \emph{Me voy a
acostar} = ``I'm going to lie down" or ``I'm going to bed." If by doing
this you are accosting someone, you may be in the wrong bed.

\section{\emph{actual}}

%ACTUAL
Not ``actual" but ``current" or ``present." \emph{La actual
administración} means ``the current administration" or the one in power
at the moment. To convey ``actual" in the sense of ``factual" or ``genuine," you would resort to \emph{verdadero, autentico, genuino, real}, etc. ``This
is an actual Picasso" is expressed by \emph{Este es un auténtico Picasso}.

\section{\emph{actualmente}}

%ACTUALMENTE
Worth a special mention because ``actually"
is so common as a sentence-starter in English: ``You must be starving."
``Actually, I just ate." \emph{Actualmente} won't work here, since it means
``at present" or ``currently." Confusing the two can lead to some
strange situations. Think of someone telling you, ``You and your
brother are invited for lunch"---to which you want to answer, ``Actually, he's my husband." If you use \emph{Actualmente, es mi esposo}, what
you are saying is ``At present, he's my husband." For ``actually," try the
phrase \emph{la verdad es que}: \emph{La verdad es que acabo de comer}. When ``actually" is used in mid-sentence for emphasis, you can use either
\emph{realmente} or some other construction altogether. For example, ``He actually ate it!" can be expressed as \emph{¡Realmente se lo comió!} or \emph{¡Sí se lo
comió!} or \emph{¡De verdad se lo comió!}

\section{\emph{afección}}

%AFECCIÓN
This doesn't mean ``affection" but usually a
medical condition. \emph{Tiene una afección cardiaca} = ``He (or she) has a
heart condition." Use \emph{afecto}---or better yet, \emph{cariño}---to convey your
affection.

\section{\emph{americano}}

%AMERICANO
Almost more of a political issue than a semantic one. Still, the use (and misuse) of \emph{americano} by ``Americans"---i.e.,
U.S. citizens---should be kept in mind by anyone eager to avoid unnecessary offense. In Spanish and Spanish-speaking countries an \emph{americano} is simply a person from ``the Americas," including South and
Central as well as North America. Thus if you tell a Chilean that you
are an \emph{americano}, you might get a smirk and the comment \emph{Yo también} (``Me too") in return. Incidentally, in the Spanish-speaking world
the correct way to refer to ``the Americas" is in the singular: \emph{América}.
Schoolchildren are taught that, from the Bering Strait to Tierra del
Fuego, it is but one continent.

Oddly enough, there really is no perfect way for U.S. citizens
to state their origin in Spanish. \emph{Norteamericano} is the most common
word, but technically it includes Canadians and Mexicans as well. \emph{Estadounidense} has caught on of late, but besides being a mouthful it
overlooks the fact that other countries (the United Mexican States, for
instance) are also technically ``United States." Even so, saying \emph{Soy de
los Estados Unidos} is probably the easiest way of explaining your
plight.

This confusion, incidentally, goes a long way toward explaining why \emph{gringo} is so common a word in countries like Mexico and why
you yourself will probably start to use it after a short time in these
places. The word can be used in an offensive way, but usually the
word's negative connotation is a result of the tone of voice. In Mexico,
especially, \emph{gringo} is very commonly used as an adjective as well; imports from the United States can easily (if colloquially) be referred to
as \emph{productos gringos}.

\section{\emph{argumento}}

%ARGUMENTO
A classic trickster. This word doesn't work
well as ``argument" in the usual sense of a ``heated discussion" or a
``quarrel." Instead, it should be reserved for prepared arguments (such
as lawyers' summations), debates, and carefully reasoned, often written
arguments. \emph{Argumento} describes a logical process, not a rather illogical throwing of pans and vases. For that, use \emph{pleito, disputa}, or \emph{disgusto} (see below), roughly in descending order of intensity. Perhaps the
best all-purpose translation of ``argument" is another trickster, \emph{discusión}, which refers to a far more heated exchange than what we consider a ``discussion" in English.

\section{\emph{asistir}}

%ASISTIR
Not ``to assist" but ``to attend" or ``to be present at."
\emph{Asistencia} is the proper word for ``attendance" (see \emph{atender} and \emph{audiencia} below), and an \emph{asistente} is technically just someone who is attending something. Nonetheless, you may come across \emph{gerente asistente} as a way of saying ``assistant manager," though you'll probably
come across it at a local branch of a US. company. A more natural
Spanish construction would be \emph{subgerente}. For ``to assist," use \emph{ayudar}.

\section{\emph{atender}}

%ATENDER
Not ``to attend" (see \emph{asistir} above) but ``to attend
to"---a significant difference. Remember, you can \emph{atender} a patient but you can't \emph{atender} a concert.

\section{\emph{audiencia}}

%AUDIENCIA
Another word in the ``attendance" versus ``assistance" imbroglio. Correctly, an \emph{audiencia} is usually a private interview granted to you by someone more important than you. In this
sense, it is usually used with \emph{conceder}: \emph{El ministro nos concedió audiencia} = ``The minister granted us an interview." This meaning is
still in use in English, but it is usually reserved for the pope. In Spanish the town sewage commissioner can grant you one. \emph{La audiencia}
can be used for ``the audience"---in a concert hall, for instance---but \emph{la
asistencia, el auditorio}, and \emph{el público} are all preferred.

\section{\emph{balde}}

%BALDE
This doesn't refer to hair loss but to a bucket. It's also
commonly seen in the expression \emph{en balde}, which means ``in vain."
For ``bald," you can use either \emph{calvo} (polite) or \emph{pelón} (irreverent).

\section{\emph{balón}}

%BALÓN
The common word for ``ball," from about grapefruit
size on up. Smaller balls are called \emph{bolas} or \emph{pelotas}. ``Balloon" is
\emph{globo}, just so you know.

\section{\emph{billón}}

%BILLÓN
A false cognate that many overlook. \emph{Un billón} is
1,000,000,000,000, or 10 to the twelfth power ($10^{12}$). It is equal to the
U.S. ``trillion." (In England, this quantity is a ``billion.") To convey the
US. ``billion," or 10 to the ninth power (UK. ``milliard"), you must say
\emph{mil millones}, or ``a thousand millions."

\section{\emph{bizarro}}

%BIZARRO
A fairly archaic term for ``chivalrous" or ``brave"
(not ``bizarre" in the sense of ``peculiar"). Both ``bizarre" and \emph{bizarro}
come from the Basque word for ``bearded," which apparently suggested
strangeness to some and gallantry to others. You don't need to know
\emph{bizarro} to speak Spanish, but you should be aware that it doesn't mean
``bizarre" if you tend to translate your thoughts fairly literally from English to Spanish. For ``bizarre," use \emph{raro} or \emph{extraño}.

\section{\emph{"capable"}}

%"CAPABLE"
This word doesn't even exist in Spanish, but if
you were to use it, it would be understood as ``castratable." Saying
someone is \emph{un hombre capable}, therefore, does not mean he is ``capable" but rather in imminent danger of being rendered ``incapable."
The right word for ``capable" is \emph{capaz}; ``incapable" is \emph{incapaz}, which
covers ``incompetent" as well.

\section{\emph{cargo}}

%CARGO
Almost works but not quite. \emph{Carga}, with the feminine ending, is the right word for ``cargo." \emph{Cargo} in Spanish generally
means ``job position" or ``post." \emph{El encargado} is ``the guy in charge."

\section{\emph{carpeta}}

%CARPETA
Usually a ``portfolio" of the sort for keeping and
carrying papers. It could also be a simple manila folder or a file. It is
never a ``carpet," which is covered by \emph{tapete} or \emph{alfombra}. Nonetheless,
thanks to the influence of English, it is said (perhaps apocryphally) that
Spanish-speaking residents of the United States say things like \emph{Voy a
vacunar la carpeta} for ``I'm going to vacuum the carpet." In fact, this
statement means ``I'm going to vaccinate the portfolio." (See Appendix
B for more English-influenced words and phrases.)

\section{\emph{chocar}}

%CHOCAR
A tricky trickster. This word has long existed in
Spanish to mean ``to crash," as in what happens to a car that is driven
recklessly. A few hundred years ago, though, the English word ``shock"
began to take on some trendy new scientific meanings, and a handful
of these were assigned to the old standby \emph{chocar} and its derivatives.
Thus while un choque has always meant ``a crash," in recent years it
has been expanded to cover ``a state of shock" (such as the driver's condition after the car's choque). It is also in widespread use for a powerful electrical shock, though \emph{chocar} as a verb is not used for ``to shock
(electrically)." For that, use \emph{dar choque} or, in many countries, \emph{dar toques}. (Generally, \emph{un choque} will kill you and \emph{un toque} won't.) \emph{La lámpara me dio toques} = ``The lamp gave me a shock." To confuse matters more, sometimes shock is used instead of \emph{choque}.

\section{\emph{chocante}}

%CHOCANTE
As an adjective, this works pretty well as
``shocking" but usually conveys a distinct disapproval that isn't implicit in the English word. In some contexts, especially in reference to
people, it comes close to meaning ``offensive" or ``rude." \emph{Me choca}, by
extension, is a common colloquial way of saying you strongly dislike
something: \emph{Me choca el chocolate} = ``I hate chocolate."

\section{\emph{complexión}}

%COMPLEXIÓN
Not your skin texture, as in English, but your
general physical structure and shape, or ``build"---in other words, fat,
skinny, pear-shaped, or just about right. Government forms in Spanish-speaking countries often ask about your \emph{complexión}, and as long as
you remember not to put ``cleared up years ago," you'll be fine. For
skin condition, stick to \emph{piel} (all skin) or \emph{cutis} (especially facial skin).
``You have a nice complexion" is expressed by \emph{Tienes buen cutis}.

\section{\emph{compromiso}}

%COMPROMISO
Yes, it works as ``compromise," but a far more
common usage is to mean ``commitment." The verb \emph{comprometer} is
similarly double-edged. \emph{Me comprometo con las mujeres} could be a
politician's way of saying he or she is committed to his or her female
constituents and connotes no ``compromising" situations. The same
sense of obligation is present in the common marketplace remark
\emph{sin compromiso}, which means you can try on a blouse, for instance,
``without committing" yourself to buy it. \emph{Compromiso} is also an
``appointment" or ``engagement." \emph{Tengo un compromiso después de la
comida} = ``I have an appointment after lunch." ``Commitment" can.
also be used this way, but it sounds a bit like power-breakfast English,
whereas in Spanish it is an everyday expression.

\section{\emph{copa}}

%COPA
If you ask for your ``cup of coffee" as a \emph{copa de café} in
the Spanish-speaking world, don't be surprised if you're served your
coffee ``Irish" or with a coffee liqueur. In general usage, a \emph{copa} is a
stemmed glass or goblet of the sort used for wine or champagne. Thus
\emph{copa}, in a restaurant setting, almost always suggests an alcoholic beverage of some sort. Just as you wouldn't think of asking for coffee in a
goblet, your waiter won't think of serving something in a \emph{copa} without
a little booze in it. \emph{Taza} is the correct word for a coffee-style ``cup."

\section{\emph{corriente}}

%CORRIENTE
This word is only partly tricky, but when it's
tricky, it's dangerously so. Basically you can use it safely as a noun to
mean ``current"---any sort of electrical, river, or political currents---but not as an adjective. As a modifier, \emph{corriente} suggests ``cheap" or
``trashy"; when used for people, it is a definite insult, equating with
``rude" or ``vulgar." To convey ``current" in the sense of ``now in progress," you should take care to use \emph{presente} or \emph{actual} (see above).

\section{\emph{decepción}}

%DECEPCIÓN
A classic trickster. \emph{Decepción} means ``disappointment" or ``disillusionment," often with no suggestion whatsoever
of deceit. Likewise, \emph{decepcionar} means ``to disappoint," and \emph{decepcionado} means ``disappointed." \emph{Me decepcionó su novio} means you
were unimpressed by someone's boyfriend, not that he talked you out
of your inheritance. For ``to deceive" and its derivatives you're better
off with \emph{engañar}. \emph{Defraudar} can work either way: ``to defraud (monetarily)" or ``to disillusion," ``to let (someone) down."

\section{\emph{disgustar}}

%DISGUSTAR
Not ``to disgust" but ``to cause displeasure"---a
subtle but important difference. Turned around, with the speaker as
the indirect object (as with gustar), it means simply ``to dislike." \emph{Me
disgustan los pepinos} means you don't like cucumbers, not necessarily
that they make you sick. For ``to disgust," \emph{asquear} or \emph{dar asco} is appropriate: \emph{Los pepinos me dan asco}. Similarly, ``disgusting" would be
\emph{asqueroso}. Used to describe a person, it conveys the idea of extreme
sleaziness. \emph{Es un tipo asqueroso} = ``He's a real slimeball."

\section{\emph{"dum dum da dum dum, tump tump"}}

%"DUM DUM DA DUM DUM, TUMP TUMP"
A non-verbal trickster of the most dangerous sort. It's hard to convey the sound pattern
in writing (one rendering, perhaps from vaudeville days, is ``shave and
a haircut---two bits"), but the pattern is familiar to everyone. Imagine
yourself knocking it on a door or tapping it out on your car horn. Got
it? Now consider that in certain Latin countries, Mexico especially,
what you've just said is, essentially, ``Fuck your mother." Knock that
on a door in Mexico and be prepared to see someone with a shotgun
open it; tap it on your car horn and you'll have twenty vehicles gunning for you. Tap it on your car horn when there's a police car in front
of you and you've got serious problems.

\section{\emph{embarazado}}

%EMBARAZADO
The most famous trickster of all, leading to
all sorts of colorful anecdotes from travelers. The word actually can
mean ``embarrassed" in certain contexts and in certain expressions,
but it also means ``pregnant." By using it, even correctly, you open
yourself to no end of silliness and smirks. Better to stick to the common ways of saying ``embarrassed": \emph{apenado} and \emph{penoso}, both from
the word \emph{pena} (see below). \emph{Dar pena} is good for ``to embarrass." If you
don't want to speak in public because it embarrasses you, bow out with
a \emph{Me da pena}. For ``How embarrassing!" (as in ``What a fool I've made
of myself!"), a simple \emph{¡Que pena!} will suffice. A stronger concept like
``shame," often with moralistic overtones, is covered by \emph{vergüenza}.
\emph{Pena} is more the embarrassment that comes of shyness or prudishness.

\section{\emph{en absoluto}}

%EN ABSOLUTO
Not ``absolutely" but the opposite---``absolutely not." If you want to avoid using it altogether, that's fine, too.
Use \emph{claro} for ``absolutely" and \emph{claro que no} for ``absolutely not." But
learn the correct meaning of \emph{en absoluto} for those times when it's
used on you.

\section{\emph{enfrente de}}

%ENFRENTE DE
A sneaky trickster and one that can cause
crossed signals with the best of them. It means ``in front of" in the
sense of ``across the way (or street) from" or ``facing." Thus, if you tell
your friends to pick you up \emph{enfrente del cine} and you're waiting in
front of the movie theater, expect to see your friends waiting across
the street. \emph{Frente a} is equally misleading, meaning the same as \emph{enfrente de}. For ``in front of" as we use it in English, try \emph{en la puerta de}
(``at the door of") to avoid any misinterpretations. \emph{Al frente de} can also
be used, but why risk the confusion? (See also Chapter 11 under
``Front.")

\section{\emph{excitado}}

%EXCITADO
An easy one to slip in unawares, this trickster
probably conveys more than you bargained for. The English translation
is ``aroused," sexual overtones included. For ``excited," use \emph{emocionado}; for ``exciting," \emph{emocionante}.

\section{\emph{éxito}}

%ÉXITO
Not ``exit" but ``success." In the music industry \emph{éxitos}
are ``hits," lest you think \emph{Los grandes éxitos de Frank Sinatra} refers to
his greatest stage exits. ``The exit" is \emph{la salida}.

\section{\emph{fábrica}}

%FÁBRICA
Not ``fabric" but ``factory." This is a good one to remember if you're going to get involved in business in Latin America, as
a lot of people are these days. If you're looking to buy fabric, on the
other hand, the word you want is \emph{tela}.

\section{\emph{gentil}}

%GENTIL
Not really ``gentle" but ``kind" or ``courteous." \emph{Que
gentil} is a somewhat stuffy (and often ironic) way of saying ``How
nice" or ``That's great." To say ``gentle," you'll want to use \emph{cuidadoso}
or even simply \emph{cuidado}: \emph{Mucho cuidado por favor con esa caja} =
``Please be gentle with that box." A ``gentle" wind might be \emph{suave},
while a ``gentle" person would be \emph{tierna}.

\section{\emph{informal}}

%INFORMAL
This word has come to have pretty much the
same meaning as its English cognate, with one important exception:
when used to refer to people, informal means ``unreliable." It's the
word you use to refer to the plumber who swore that he'd be by last
Thursday to unstop your drains, only to vanish from the face of the
planet instead. To say that someone is ``informal"---i.e., ``laid-back"---you could use \emph{relajado} (``relaxed") or \emph{despreocupado}. For uses other
than personal ones, \emph{informal} is widespread, though purists tend to dislike it. \emph{Fue una reunión informal} = ``It was an informal get-together."

\section{\emph{injuria}}

%INJURIA
An ``injury," yes, but a moral and psychological
one---better known in English as an ``insult" or ``offense," and usually
a real dinger of one at that. Don't use it for ``injury," which, if major, is
the noun \emph{herida}. If it's a small injury (a sprained ankle, for example),
the verb \emph{lastimarse} is more appropriate. Why is Pablo limping? \emph{Es que
se lastimó el pie jugando tenis} (``He hurt his foot playing tennis").

\section{\emph{intoxicado}}

%INTOXICADO
Unless it's a real binge you're talking about,
this word doesn't work for ``drunk." It means ``poisoned"---unintentionally, as a rule---and covers food poisoning, industrial toxins, overdoses, and the like. Another whole book could be written on how to
say ``intoxicated" in the sense of ``drunk" in Spanish. \emph{Ebrio} is the
best equivalent for ``intoxicated," while \emph{borracho} is closest in tone to ``drunk."

\section{\emph{introducir}}

%INTRODUCIR
You'll be tempted to use this for ``to introduce,"
but don't. It means ``to introduce" only in the sense of ``to insert" or
``to add something in." Often it is used as a fairly fancy substitute for
\emph{meter} or ``to stick in." For ``to introduce to" with people, always use
\emph{presentar}. \emph{Ven, quiero presentarte a un amigo} = ``Come on, I want to
introduce you to a friend of mine."

\section{\emph{largo}}

%LARGO
Not ``large" but ``long." Just a reminder.

\section{\emph{librería}}

%LIBRERÍA
Not ``library" but ``bookstore." Another reminder.
A ``library" is a \emph{biblioteca}.

\section{\emph{media}}

%MEDIA
In Spanish, this (presumably) has nothing to do with
your favorite newscaster. It means ``stocking," and in the plural,
``pantyhose." In math it means ``mean." ``Media," in the collective
sense of newspapers, magazines, radio, and television, is usually covered in Spanish by \emph{los medios} (de \emph{comunicación}).

\section{\emph{molestar}}

%MOLESTAR
In Spanish this word carries no sexual overtones
whatsoever. It's perfectly safe and extremely common for ``to bother,"
\emph{No me molestes} = ``Don't bother me." As an adjective, it works well
as ``upset," ``angry," or ``uncomfortable." \emph{¿Estás molesto por algo?} =
``Are you upset about something?"

\section{\emph{ordinario}}

%ORDINARIO
Be careful using this term in regard to people.
Far from meaning ``ordinary," it means ``vulgar," ``rude," and ``crass."
\emph{Es un tipo muy ordinario} = ``He's a slob." To describe an average, run-of-the-mill person just like you and me and a million other people, use
\emph{normal} or \emph{común:} \emph{Es un tipo normal}. \emph{El hombre común} is a good
translation of ``the man in the street." (See also Chapter 4.)

\section{\emph{parientes}}

%PARIENTES
Not ``parents," though they're \emph{parientes} too, but
``relatives"---all of them, from your grandparents and children to in-laws and cousins. What you share with these people is called \emph{parentesco}, or ``kinship."

\section{\emph{pena}}

%PENA
Not ``pain" (the physical kind) but ``sorrow" and ``embarrassment." \emph{Pena} is the word you want to learn so as to avoid using
\emph{embarazado} (see above). ``Pain" is usually handled by \emph{dolor}, as is
``ache." A ``headache" is a \emph{dolor de cabeza}. In the figurative sense of
a ``pain in the neck" (or even lower), a good equivalent is \emph{lata}. \emph{¡Qué
lata!} = ``What a pain!" \emph{Dar (la) lata} = ``to be a pain," ``to pester." To
people who are bothering you, you can say \emph{No des (la) lata} (``Stop being a pain").

\section{\emph{primer piso}}

%PRIMER PISO
If you're used to thinking of the ground floor as
the first floor, be prepared to think again in Spanish, where \emph{primer piso}
means ``one flight up." The ground floor is usually called \emph{1a planta
baja}, or PB in the elevators.

\section{\emph{quitar}}

%QUITAR
You may try to use this for ``to quit," especially
since Spanish at first glance doesn't seem to have a good word for ``to
quit." Unfortunately, quitar isn't that word either. It means ``to take
(something) off," ``to take away." The word you need for ``quit" depends on what you're quitting: if it's your job, the word is \emph{renunciar}, if
it's smoking or some other activity, \emph{dejar de}. To ``quit" a computer
program, most translated software programs simply use \emph{salir}.

\section{\emph{realizar}}

%REALIZAR
There is some overlap with this word and its English cognate. But for the most common use of ``to realize" in English,
\emph{realizar} does not work. The correct phrase is \emph{darse cuenta (de)}. ``I realized too late he had a gun" is expressed by \emph{Me di cuenta demasiado
tarde de que traía pistola}. ``Realize" in the common English sense can
often be handled perfectly competently by \emph{saber} (``to know") in Spanish. ``I realize you're busy" thus becomes \emph{Sé que estás ocupada}. ``I
didn't realize you were married" would be \emph{No sabía que estabas
casado}.

\section{\emph{receta}}

%RECETA
Almost always a medical prescription or a cooking
recipe in Spanish. It is never a ``receipt," which would be \emph{recibo} or
\emph{nota}.

\section{\emph{ropa}}

%ROPA
Not ``rope" but ``clothes." ``Rope" is usually either
\emph{soga} or \emph{cuerda}.

\section{\emph{sano}}

%SANO
This goes beyond mental health to cover all aspects of
health. In other words, it means ``healthy." The word for ``sane" is
\emph{cuerdo}. For ``insane," use \emph{loco}. It's a bit unscientific and insensitive, but then so is ``insane."

\section{\emph{sensible}}

%SENSIBLE
It means ``sensitive," not ``sensible." For ``sensible" you should use \emph{sensato}. \emph{Una persona sensible} is ``a sensitive
person." This is one of those confusing words you'll just have to
remember.

\section{\emph{sopa}}

%SOPA
Not ``soap" but ``soup." Creamy soups are simply
called \emph{cremas}, as in \emph{crema de champiñones} (``cream of mushroom
soup"). ``Soap" is \emph{jabón}.

\section{\emph{soportar}}

%SOPORTAR
A common word in Spanish for ``to tolerate,"
though it is more frequent in the negative in the sense of ``can't stand."
\emph{No soporto la televisión} = ``I can't stand television." Mixing it up will
get you some funny looks: \emph{Mi familia me soporta} doesn't mean your
family pays your bills but that your family tolerates you (barely). For
``to support" in the bill-paying sense use \emph{mantener}. In the sense of
physically supporting something (what a wall does for the ceiling, in
other words), \emph{sostener} is more accurate.

\section{\emph{sumar}}

%SUMAR
This verb may pop into your mind as a neat translation for ``to sum up," but you should pop it right back out of there if
you want to be understood. \emph{Sumar} is the word for ``to add" or ``to add
up." \emph{¿Cuánto suma?} = ``How much does that add up to?" For summaries and summations, stick to the verb \emph{resumir} (\emph{para resumir} = ``to
sum up," \emph{en resumen} = ``in sum"), although \emph{en suma} is safe for ``in sum" as well.

\section{\emph{suplir}}

%SUPLIR
Not ``supply," as you might think. Instead, it means
``to substitute" or ``to fill in for." \emph{Suplo en la oficina a mi hermana
cuando está de viaje} = ``I fill in for my sister at the office when she's
out of town." A \emph{suplente} is a ``substitute"---a substitute teacher, for
instance. \emph{Substituto} and \emph{substituir} also exist and mean about the
same thing, but \emph{suplir} and its derivatives are more frequent (and far
easier to pronounce).

\section{\emph{tremendo}}

%TREMENDO
Often a close fit for ``tremendous" but not always. Tremendo often comes closer to ``outrageous" and can mean
``outrageously bad," ``terrifying," or ``terrible" as well as ``outrageously
good" or ``tremendous." The word is especially tricky around children,
it seems. \emph{Es un niño tremendo} describes a monstrous child capable of
the worst mischief. In short, be aware of the negative connotations
that frequently surround this word.

\section{\emph{tuna}}

%TUNA
Here's a real menu trickster. Tuna is not the fish but
the fruit of the prickly pear cactus, or \emph{nopal}---that is, the ``prickly
pear" itself. ``Tuna" is \emph{atún}. Both are perfectly edible, but if you have
your heart set on a tunafish sandwich, a serving of prickly pears may
not quite fill the bill.

\section{\emph{últimamente}}

%ÚLTIMAMENTE
In correct usage not ``ultimately" but ``recently." \emph{Últimamente ha estado enferma} = ``She's been sick of late."
For ``ultimately," use \emph{al final} or \emph{a fin de cuentas}. \emph{A fin de cuentas, es
su decisión} = ``Ultimately it is his decision."

\section{\emph{vacunar}}

%VACUNAR
See carpeta above---and fast!

\section{\emph{vago}}

%VAGO
This word means ``vague" when applied to things, but
it means ``bum" or ``tramp" when used for people. ``Vagabond" is a
good cognate for this usage. To express ``vague," use \emph{vago}, carefully,
and use \emph{impreciso} when referring to people or when there's any chance
of being misinterpreted.

\section{\emph{voluble}}

%VOLUBLE
Those of you who studied your vocabulary lessons
in high school will remember that this means ``talkative." In Spanish,
though, it means ``unreliable," ``flighty," or ``fickle" and is a good deal
more common than its English cognate.

\section{\emph{zorra}}

%ZORRA
``Foxy lady," even before Jimi Hendrix's time, has
been a slangy way to refer to an attractive female. And ``a fox," by itself, means much the same thing for either men or women. Slang is a
dangerous thing to translate literally, however, and there is no better
example of this than the Spanish word \emph{zorra}, or ``fox" (i.e., a canine
of the genus Vulpes). When applied figuratively to human beings, it
means ``shrew," ``slut," or ``prostitute." What better way to learn your
tricksters than to get a slap in the face? If you must use corny come-ons, stick to \emph{guapo} and \emph{guapa}.

