\documentclass[14pt,a4paper,oneside]{memoir}

\setlrmarginsandblock{2.5cm}{2.2cm}{*}
\setulmarginsandblock{2.5cm}{2.5cm}{*}
\checkandfixthelayout

%\setlrmarginsandblock{3.5cm}{2.5cm}{*}
%\setulmarginsandblock{2.5cm}{*}{1}
%\checkandfixthelayout

% Keep indentations with ragged right.
\makeatletter
\newcommand\iraggedright{%
  \let\\\@centercr\@rightskip\@flushglue \rightskip\@rightskip
  \leftskip\z@skip}
\makeatother

\iraggedright

\usepackage{ifxetex}
\usepackage{amssymb} % Essential to be included before xunicode
\usepackage{xcolor}

%%%%%%%%%%%%%%%%%%%%%%%%%%%%%%%%%%%%%%%%%%

\ifxetex

\usepackage{fontspec}
\usepackage{xunicode}
\usepackage{xltxtra}
\usepackage{polyglossia}
\setromanfont[Mapping=tex-text]{Liberation Serif}
\setsansfont[Mapping=tex-text]{Liberation Sans}
\setmonofont[Mapping=tex-text]{Liberation Mono}
\setdefaultlanguage{russian}

\else % not XeTeX

\iffalse

\usepackage{type1ec}
\usepackage[T1]{fontenc}
\usepackage[utf8x]{inputenc}
\usepackage[english,russianb]{babel}

\else

\message{XeTeX only document}
\stop

\fi

\fi % not XeTeX

%%%%%%%%%%%%% Common preamble %%%%%%%%%%%%

%\usepackage[usenames,dvipsnames,svgnames,table]{xcolor}

\usepackage{graphicx}
\usepackage{textcomp}

\usepackage{indentfirst}
\frenchspacing
\clubpenalty=10000
\widowpenalty=10000

\sloppy

\newcommand{\bsk}{\vspace{20pt}}
\newcommand{\sk}{\vspace{5pt}}
\newcommand{\pp}{\textcolor{red}}   % Prepositional Pronoun.
\newcommand{\lp}{\textcolor{brown}} % Lexicalized Pronoun.
\newcommand{\bl}{\textcolor{blue}}  % Just blue.
\newcommand{\nb}[1]{{\small \emph{\textcolor{blue}{#1}}}}

\usepackage{indentfirst}

\chapterstyle{verville}
\usepackage[hidelinks]{hyperref}

\begin{document}

\frontmatter

\begin{titlingpage}
\title{Breaking Out of Beginner's Spanish}
\posttitle{\par\vskip1em{\normalfont\normalsize (A digest)\par}\end{center}}
\author{Joseph J. Keenan}
\date{~}
\maketitle
\end{titlingpage}

\tableofcontents*

\mainmatter

\chapter{Ten Ways to Avoid Being Taken for a Gringo}

\section{Pronunciation}
\section{The wrong word syndrome}
\section{\textit{Yo-ismo}}
\section{The stumbles}
\section{Formalities}
\section{The volume}
\section{Adjectives}
\section{Speed kills}
\section{Body language}
\section{Those crazy gringos}

\chapter{Minding Your Verbal Manners}

\section{Meetings and greetings}
\section{\textit{Usted} versus \textit{tú}}
\section{Magic words}
\section{Sugar versus saccharine}
\section{Asking and getting}
\section{Etc.}
\section{Sweet sorrow}

\chapter{Tricksters}

\chapter{Our Fellow Human Beings}

\section{The good}
\section{The \textit{amables}}
\section{The \textit{simpáticos}}
\section{The \textit{listos}}
\section{The bad}
\section{The \textit{pesados}}
\section{The \textit{imbéciles}}
\section{The \textit{malvados}}
\section{The \textit{cochinos}}
\section{The indifferent}
\section{Temporary states}

\chapter{The Secret Life of Verbs}

\section{The present}
\section{The future}
\section{The conditional}
\section{The preterit versus the imperfect}
\section{Special cases}
\section{Ser versus estar}
\section{The easy ones}
\section{Getting tricky: the past participles}
\section{The hard ones: descriptive adjectives}
\section{Sorting out ser and estar in the imperative}
\section{A matter of perspective}

\chapter{The Twilight Zone}

\section{Indirect commands (shallow twilight)}
\section{The eternal mystery (deep twilight)}
\section{\textit{Que} cues}
\section{Non-\textit{que} cues}
\section{The subjunctive with \textit{ser}: \textit{sea}}
\section{The traveler's subjunctive}

\chapter{Sixty-four Verbs, Up Close and Personal}

\chapter{Cranking Up Your Spanish}

Not long ago, I owned a somewhat antiquated motor vehicle
that in fits and starts, quite literally, became dear to me. It wasn't
much to look at-antique buffs generally looked away instead-but
it more often than not got me where I was going. The car's main flaw,
and what finally led me to get rid of it, was its temperament on chilly
mornings. Instead of responding to the turn of the key with a loud
growl and churning pistons, it tended to respond with a weak series of
hiccups and the grinding of metal. Once started, it sputtered, stuttered,
gurgled, and spit. Only after a few minutes of this litany did it begin to
growl a little roughly, then purr hesitantly, then hum.

Beginning Spanish speakers remind me of that car. Once
warmed up, they chatter along merrily, constructing elaborate clauses
and communicating their meaning. But getting warmed up is another
story. Frequent false starts, "ums," hiccups, and the grinding of gray
matter punctuate their speech, and English interjections pop in and
out. It is obvious that they are still thinking in English and trying to
translate, rather than thinking in Spanish.

What beginning students often lack are the appropriate words
and phrases to start a sentence smoothly. As in English, these words
don't always contribute a lot to the meaning of the sentence, but they
do give the brain a chance to warm up and kick into gear. Skipping
over them can make speech sound stilted and abrupt-sometimes
even impolite or hasty. Even when you are past the beginner stage, you
should pay attention to your "sentence starters." Learn a few well, get
accustomed to using them, and your Spanish will sound more natural,
flow more easily, and hum along to its desired destination.

\chapter{Snappy Answers}

\chapter{Invective and Obscenity}

\chapter{Which Is Which?}

\chapter{Say It Right}

\chapter{Spanish Roots}

\chapter{The big Mix}


\end{document}
