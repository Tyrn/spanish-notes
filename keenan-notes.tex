\documentclass[14pt,a4paper,oneside]{memoir}

\setlrmarginsandblock{2.5cm}{2.2cm}{*}
\setulmarginsandblock{2.5cm}{2.5cm}{*}
\checkandfixthelayout

%\setlrmarginsandblock{3.5cm}{2.5cm}{*}
%\setulmarginsandblock{2.5cm}{*}{1}
%\checkandfixthelayout

% Keep indentations with ragged right.
\makeatletter
\newcommand\iraggedright{%
  \let\\\@centercr\@rightskip\@flushglue \rightskip\@rightskip
  \leftskip\z@skip}
\makeatother

\iraggedright

\usepackage{ifxetex}
\usepackage{amssymb} % Essential to be included before xunicode
\usepackage{xcolor}

%%%%%%%%%%%%%%%%%%%%%%%%%%%%%%%%%%%%%%%%%%

\ifxetex

\usepackage{fontspec}
\usepackage{xunicode}
\usepackage{xltxtra}
\usepackage{polyglossia}
\setromanfont[Mapping=tex-text]{Liberation Serif}
\setsansfont[Mapping=tex-text]{Liberation Sans}
\setmonofont[Mapping=tex-text]{Liberation Mono}
\setdefaultlanguage{russian}

\else % not XeTeX

\iffalse

\usepackage{type1ec}
\usepackage[T1]{fontenc}
\usepackage[utf8x]{inputenc}
\usepackage[english,russianb]{babel}

\else

\message{XeTeX only document}
\stop

\fi

\fi % not XeTeX

%%%%%%%%%%%%% Common preamble %%%%%%%%%%%%

%\usepackage[usenames,dvipsnames,svgnames,table]{xcolor}

\usepackage{graphicx}
\usepackage{textcomp}

\usepackage{indentfirst}
\frenchspacing
\clubpenalty=10000
\widowpenalty=10000

\sloppy

\newcommand{\bsk}{\vspace{20pt}}
\newcommand{\sk}{\vspace{5pt}}
\newcommand{\pp}{\textcolor{red}}   % Prepositional Pronoun.
\newcommand{\lp}{\textcolor{brown}} % Lexicalized Pronoun.
\newcommand{\bl}{\textcolor{blue}}  % Just blue.
\newcommand{\nb}[1]{{\small \emph{\textcolor{blue}{#1}}}}

\usepackage{indentfirst}

\chapterstyle{verville}
\usepackage[hidelinks]{hyperref}
\setsecnumdepth{subsection}

\newcommand{\inda}{\hspace{40pt}}
\newcommand{\indu}{\hspace{20pt}}

\begin{document}

\frontmatter

\begin{titlingpage}
\title{Breaking Out of Beginner's Spanish}
\posttitle{\par\vskip1em{\normalfont\normalsize\maltese\par}\end{center}}
\author{Joseph J. Keenan}
\date{~}

\maketitle
\end{titlingpage}

\tableofcontents*

\mainmatter

\chapter{Ten Ways to Avoid Being Taken for a Gringo}

A gringo, in much of the Spanish-speaking world, is a person
who comes from abroad, speaks another language, and wears loud
shorts. In certain countries, such as Mexico, it refers specifically to
U.S. citizens, but even there the distinction is hazy. A Canadian or
German who acts like a gringo will be referred to as a gringo, birth certificates be damned. Act like a gringo and you will be called one; don't
act like one and you may be called one anyway. The word is descriptive
first-of a style, a cultural stance, a way of life-and derogatory only
later, if at all.

So how to avoid being taken for a gringo? The truth is, if you
were born outside the Spanish-speaking world, there is probably nothing you can do to hide the fact. You will never fully blend in nor
should you necessarily want to. But as you travel or mingle among
Spanish-speakers, you may wish to smooth over the most obvious differences that set you apart. We all want to be outstanding; standing out
is another matter altogether.

Since you're making an effort to speak and understand Spanish, you've already distinguished yourself from the stereotypical
gringo, that mythical beast of Latin American lore who wears obtrusive shirts, smacks gum, and tends to misplace his or her wallet. But
even if you're not one of them, you can still heed a few simple precautions that will help put you in tune with the local culture, be it in Patagonia or East Los Angeles. Dress codes and behavior are beyond the
scope of this book, but some general pointers may keep others from
pointing at you.

\section{Pronunciation}

Spanish pronunciation should be the easiest thing in the world
to master. Unlike English, where a letter can change its sound seemingly at will, Spanish letters have-with very few exceptions-the exact same sound word after word. To compare, think of the three different sounds of the letter a in the English pronunciation of the name
Abraham; in Spanish the letter a in Abraham has but a single sound,
repeated three times. Still, you'll need to practice to convince your
tongue to make the correct sound, to get your teeth to close or open on
cue, and to master the inflections. At first it will be a struggle, but
there is no reason why anyone can't learn to pronounce Spanish properly after enough use. Here are some tips on how to proceed:

\subsection{}

Spanish teachers always tell their students to practice repeating the vowel sounds: \emph{a, e, i, o, u}. Listen to these wise men and
women and practice, practice, practice. If it's more fun, follow your
litany with the phrase schoolchildren learn south of the border: \emph{El burro sabe más que tú} ("The burro knows more than you"). Move your
mouth as you repeat the vowels. Pretend someone fifty yards away is
trying to read your lips. Clip the vowel sounds short, as you might
imagine a Japanese colonel in a late-night, World War II movie would
do. \emph{A, E, I, O, U, A, E, I, O, U}\ldots{}

\subsection{}

Next come the sounds for the letters \emph{r} and \emph{rr}. The double
one trills, and so does the single one at the start of a word. Thus \emph{carro}
and \emph{rancho} have essentially the same \emph{r} sound. Your tongue won't want
to trill at first; it will make a scene about being made out of concrete
and will refuse to emit such a ridiculous sound. But you're not going to
let your tongue push you around, are you? Trill away! Pretend you're
Charo on "The Tonight Show": "R-r-r-really, Johnny, r-r-r-romance for
me is r-r-r-relaxing on a r-r-r-rug, listening to r-r-r-rock and r-r-r-roll." If
you've studied any French, this sound may be harder for you to get
used to at first. But if you try at all, you will learn it. Many gringos do
and, as far as anyone can tell, we're all born with the same kind of
tongue.

\subsection{}

The d between vowels or at the end of a word sounds more
like the \emph{th} in \emph{thus} than an English \emph{d}. \emph{Nada} is thus pronounced "natha," or close to it. So light is the mid-vowel \emph{d} that sometimes, in colloquial spoken Spanish, it's almost left out altogether; you may even
see \emph{nada} represented as \emph{na'a} or \emph{na'} in written dialogue. You shouldn't
take it that far, but do get the hang of the soft Spanish d by learning a
few words well: \emph{nada, limonada, edad, comida, ciudad, cansado}. All
regular past participles follow the same rule: \emph{hablado, conocido, bebido}, and so on. The \emph{d} at the beginning of a word in Spanish is also a
tad softer than its English counterpart, perhaps more like a \emph{dth} than a
solid \emph{d}. Shout the name David in English and you can almost feel
yourself spit; shouting it in Spanish is a much less moist affair.

\subsection{}

The Spanish letters \emph{c, z, j}, and \emph{ll} vary in pronunciation
from country to country. Don't let this bother you. Either adopt the
sound used in your country of choice or seek a middle ground. For
most purposes, you're safe pronouncing both the \emph{c} and \emph{z} as an English
\emph{s}, the \emph{j} as an \emph{h}, and the \emph{ll} as the \emph{y} of "yes." (Remember, of course, that
the hard \emph{c} in Spanish, as in \emph{ca-, co-}, and \emph{cu-}, sounds like a \emph{k}.) Other
noteworthy regional differences include the use of \emph{vos} instead of \emph{tú}
and the use of \emph{vosotros} instead of \emph{ustedes}. \emph{Vos} is used in much of
Central and South America and requires learning yet more verb endings (\emph{tú quieres = vos querés}). Still, if you are spending time in those
countries, you will probably want to use it. The same goes for \emph{vosotros}, which is used in Spain for the second-person plural and which
you will no doubt learn if you're picking up your Spanish there.

\subsection{}

Other regional differences are best left alone. In many
countries, for instance, there is a tendency to "swallow" the \emph{s} sound at
the end of a syllable, especially before consonants: estoy aquí becomes
e'toy aquí. (It reaches such extremes that there's even a joke about the
Cuban child who asks his mother how to form plurals. "Easy, chico,"
she says, "just add an \emph{s}: \emph{el coco, lo coco}.") There's really no good reason to learn to speak Spanish that way, or with any other regional dialect, unless you're keen on being identified as having studied in a specific country. Imagine an Asian immigrant speaking English like a
Boston cabbie, or a Uruguayan drawling like a Texan, and you'll understand why.

\subsection{}

In Spanish, the letters \emph{b} and \emph{v} sound the same: almost (but
not quite) like the English \emph{b}. Like its d sound, Spanish's \emph{b/v} sound is a
shade softer, especially between vowels. Thus \emph{ave} is not pronounced
either "ah-bay" or "ah-vay" but "ah-bvay." Say "ah-bay" fast, without
giving your lips time to spit out a hard \emph{b} sound, and you'll get the idea.

\section{The wrong word syndrome}

There are dozens of cases in Spanish where you will be
tempted to use a word that is patently wrong. Mostly this is a result of
misleading English cognates---words that look or sound the same in
English and in Spanish but harbor different meanings. Sometimes the
meaning will be close in Spanish, and the lazy language learner is content to use the cognate. But you aren't like that, are you? You want to
speak Spanish well and not be responsible for polluting it with English
usages. Swell. For people like you, Chapter 3 in this book is dedicated
to these "tricksters," and Appendix B covers more subtle nuances.
Don't worry about learning them all at once. Just try to remember
which words are tricky and then try to avoid stepping in them as you
go along.

\section{\textit{Yo-ismo}}

In English, a sentence is incomplete without a noun or pronoun for a subject. In Spanish, the subject of the sentence is often conveyed by the verb and is optional. Often, in fact, it is left out altogether, unless the speaker wants to emphasize the subject of the
statement. This quaint little grammatical fact affects you, the language
learner, in one important case: the first person. Since you are used to
including pronouns, you will tend to preface all of your first-person
comments with yo. But to a Spanish ear, this sounds like you are constantly calling attention to yourself: "\emph{I} want this" and "\emph{I} think that."
This affliction, dubbed "\emph{yo-ismo}," can in extreme cases make people
think you're a pretty snotty individual, when you and I know that's not
true. But why take chances? Try to say \emph{quiero} instead of \emph{yo quiero},
\emph{creo} instead of \emph{yo creo}, and so on. Later, when you've broken the habit,
you can go back to inserting the occasional \emph{yo} to emphasize a truly
personal opinion: \emph{El quiere casarse pero yo no quiero} ("He wants to
get married but \emph{I} don't").

\section{The stumbles}

The stumbles are what you get when you're asked a simple
question and your tongue runs off and hides behind your tonsils. It is a
common ailment of those who have studied some Spanish---and \emph{know}
what they want to say---but lack conversational practice. Remedying
this condition requires practice, of course, but also a careful study of
interjections, pert comebacks, snappy answers, and sentence starters.
These useful words and phrases will get you through almost every
situation requiring sudden tongue work, but they are often neglected
in textbooks. Lists of these gems are included in later chapters. Pick
your favorites (there are usually several acceptable ones for each situation) and store them close to your tongue.
Speaking of the stumbles, you should be especially cautious of
letting your English "crutch words" slip into your Spanish. It sounds
awful: \emph{Quiero\ldots{} um\ldots{} ir a\ldots{} you know\ldots{} the\ldots{} er\ldots{} cine,
okay?} If you must, learn some Spanish crutch words and lean on them
instead: \emph{Quiero\ldots{} este\ldots{} ir\ldots{} o sea\ldots{} al cine, ¿no?} You'll still
sound like a space cadet, but at least a fairly fluent one.

\section{Formalities}

The gringo who makes the effort to get off the beaten path and
find out-of-the-way shops and cafés is almost by definition a fairly gregarious soul. But fear of the language can make this same person seem
timid, uptight, or arrogant, browsing in a shop for half an hour without
saying a word to anyone and leaving without so much as a good-bye. In
general, you should be happy to inflict your fledgling Spanish on anyone who crosses your path. But you should be especially eager to unleash it in cases calling for common courtesy. There is probably no
faster way to separate yourself from the pack of tourists and gawkers
than to look someone in the eye and speak to them in their own language---even if it's only to say "hello" and "good-bye."

For starters, you should always greet people whose lives you
have invaded, if only briefly. This doesn't mean you should walk down
the streets of Buenos Aires saying \emph{hola} to everyone, but it does mean
you should immediately recognize the existence of shop clerks, waiters, secretaries, and guests. The formula is simple. From the time you
wake up until 11:59 a.m., you say \emph{buenos días}; from 12:00 noon until
dark, you say \emph{buenas tardes}; and from dark until bedtime, you say
\emph{buenas noches}. Say it loud, say it proud. You'll be amazed how service
improves and prices drop after a pleasant greeting.

When you leave a place, remember to say \emph{gracias} if it's a commercial establishment, \emph{adiós} or \emph{hasta luego} if it's not. Better yet,
when leaving a shop, say \emph{muchas gracias} or \emph{muy amable}, gracias or
even \emph{muchas gracias muy amable}, all run together. You'll feel much
more cheerful walking out of a place after a heartfelt farewell, even if
the clerk did nothing more than stare at your back the whole time you
were in the store.

In general, Spanish requires more spoken formalities than English (at least as it's spoken today), which is nice because it gives you a
lot of opportunities to practice those key words and phrases. Skipping
over the formalities, on the other hand, will tag you as a gringo from
the get-go, which is what we've decided we want to avoid. For a fuller
treatment of formalities and politeness in general, there's a whole
must-read chapter just ahead with bounteous tips and details.

\section{The volume}

In Baltimore or Toronto or Oxford our friends and neighbors
seem to speak in reasonable tones. So why is it that these same people
seem to start shouting as soon as they get through Customs? This is
one of the great mysteries of cultural intercourse, and I don't foresee
resolving it here. But it is worth mentioning that, as a native English
speaker, you are expected to shout instead of speak and that a whole
continent of Latin Americans will be grateful if you manage to do otherwise. Gringos (remember them?) tend to bunch together and speak
English at volumes appropriate for a rock concert. I've even seen a
Mexican head of family ask the maitre d' to be seated "away from the
gringos" at a restaurant. As when greeting people, speak loud and
proud-but not too loud. Contrary to gringo folk wisdom, comprehension does not increase with volume. Normal speaking tones are best,
and by the time you find yourself shouting, absolutely no one will admit to understanding you at all.

\section{Adjectives}

One of the most frustrating things about learning a new language is not having access to that grab bag of adjectives that we rely on
to express our opinions. And since adjectives are a relatively second-class part of speech (after the big shots like nouns and verbs), many
beginning students of language tend to put off learning them until
some later phase of their study. In practical terms this produces human
beings whose whole range of descriptions goes from "very, very" on
one extreme to "not very" on the other. "How was the film?" \emph{Muy,
muy, muy buena}. "How about the meal?" \emph{No muy buena}.

You should try to break out of this rut as quickly as possible,
learning alternative ways of expressing your likes and dislikes. Pay special attention to the use of prefixes and (especially) suffixes in modifying adjectives in Spanish by learning a few suffixes and attaching
them to words you already know, you can quickly multiply the coverage of your vocabulary. \emph{Grande}, for example, can go up in size to \emph{grandísimo} and even \emph{grandotote} or deflate to a semisardonic \emph{grandecito}.
Also learn some words for that vast middle ground between good and
bad where, sad to say, most of our experiences tend to fall. Chapter 4
will try to help you do just that in your encounters with people.

\section{Speed kills}

Just as loud doesn't equal intelligible, fast definitely doesn't
equal fluent. Take it easy. You wouldn't try to break speed records on a
Kawasaki when you're just learning to ride, so why try with your Spanish? Spoken English has raised slurring practically to an art form, and
it's considered normal to modify (i.e., mispronounce) consonants or
vowels that get in our way. The usual result is a series of phrases like
"I dunno" and "Waddaya-wamee-tudo?" In spoken Spanish, each vowel
and consonant retains its particular, unalterable sound, no matter how
fast you're speaking. True, if you speak fast enough, people may not
catch your errors. But they won't catch your drift either, and you'll end
up having to repeat everything. If that's your strategy for getting extra
practice, \emph{adelante}. Just say it slowly the second time.

\section{Body language}

If slurring in English is almost an art form, then downwardly
mobile dressing in English-speaking (and other) countries is long overdue for a major museum exhibition. Far be it from me to tell you how
to dress on vacation or when prowling the barrio, but do remember the
. Spanish \emph{dicho: Como te ven, te tratan} ("How they see you, they treat
you"). Dressing down has not yet caught on in most Latin cultures,
perhaps because millions of people dress that way for reasons not related to fashion. If you stay close to the tourist bus, what you wear
isn't so important---but who wants to do that? You don't have to dress
to the nines to go buy a Coke, but you should at least be in the low
sevens. Otherwise, your clothes will be saying things about you that
your mouth never would.

What gringos often do, since we seem to be talking about
them, is to convert their Sunday barbecue outfits back home into all-purpose wear for their introduction to Latin culture. What they would
never wear to church they don't think twice about wearing to a Colombian cathedral or Guatemalan village church. If you intend to show the
local people that you respect their culture, the best way to start is by
letting your clothes speak for you. And the clothes that speak best are
the ones that cover knees and shoulders---at least away from coastal
cities. You don't have to care about any of this, of course, but if you
do, remind yourself that sin and skin are still closely linked in many
minds.

\section{Those crazy gringos}

Every now and then, it won't matter how well you say something in Spanish if what you are saying is so patently absurd that it
transcends language altogether. And what is judged as absurd can vary
widely from place to place. I remember witnessing a frustrated tourist trying to request, in flawless Spanish, \emph{hielo hervido} for his soft
drink. Now, if you're familiar with tourists, you'll probably realize that
"boiled ice" is a sort of shorthand way of saying "ice made from boiled,
or purified, water." But if you're a waiter in a small-town bus-station
restaurant in Latin America, you may not make that conceptual leap at
all. Instead, you will try your darndest to make sense of your customer,
and then will probably go back to the kitchen and heat up a couple of
ice cubes. Many innocent requests like this can turn into Major Cultural Confusions if you're not careful. Asking for ice to put in a soda
that is already cold is considered downright silly in many places, for
instance. Asking for a "pizza with meat on it," in at least one place I've
been, can lead to a pizza with a slab of steak lying on top. And so on.

This, of course, is part of the beauty of getting to know foreign
cultures: learning that what you had considered a given all your life is
often not a given at all. So if you find that nothing you say in Spanish
seems to get your point across, consider changing tack. What you are
saying, not how you are saying it, may be the culprit. Then change
your order to \emph{pollo frito} and a beer and forget about it. Just pray that
they don't come in a bowl, and together.

\chapter{Minding Your Verbal Manners}

Being polite is something many people, especially the young,
associate with visits to Grandma. In daily life, only bootlickers and
dweebs make a special effort to be polite; the rest of us are as we
are---take us or leave us.

Actually much of what goes for politeness is implicit in our
behavior and requires no special effort. Society has carved on our
minds the notion that if we don't follow certain preestablished, communal norms, it will use harsh and unfriendly epithets to describe us
behind our backs. So we take it for granted that you should open the
door for the elderly, avoid using expletives in public places, and refrain
from cutting in front of people on the exit ramp. We don't think of it as
being polite; we just do it because society says so.

Spanish-speaking society has its own set of unspoken norms
that you, as an outsider, won't have had beaten into your head from
birth. This means that you will have to pay attention to them and actually work at being polite. In addition, you will want to master the
subtleties of verbal manners that you now unthinkingly control in English. Consider these examples. Do you say the same sweet-sounding
phrases to a mean-faced bureaucrat as you do to a pleasant cashier? Of
course not. Nor do you use the same words or tone with an elderly
person that you use with someone more your age. Getting a feel for
subtleties in Spanish requires getting a handle on the language that is
,used to express manners.

In Spanish, there is really no good translation for "polite."
\emph{Cortés, amable}, and \emph{pulido} all come close but are better translated by
their English cognates: "courteous,' "amiable" (or "nice"), and "polished." Instead, a Spanish speaker will talk about someone's \emph{educación}, which goes beyond a person's schooling to cover upbringing in
general and manners in particular. \emph{Es una persona educada} means that
so-and-so is a person who has good manners and is polite in dealing
with others. The person in question could be a grease monkey in the
neighborhood lube shop or a physics professor, a kindergarten dropout
or a triple Ph.D.; \emph{educado} simply means that the person has decent
manners. "Well bred,' though somewhat out of favor in modern egalitarian societies, conveys the right idea.

In passing, it's worth noting that \emph{rudo} is not the equivalent of
"rude," nor is it a good opposite for \emph{educado}. A Spanish-speaker would
probably use \emph{mal educado, sin educación}, or \emph{de poca educación}, or
would resort to \emph{grosero}. This last word, in context, refers to a foulmouthed individual, but in a more general sense it comes closer to
"rude." \emph{Se portó muy grosero con nosotros} = "He was very rude
to us."

Being polite, of course, is more than simply uttering elegant
phrases at key points in a conversation. To achieve the rank of \emph{educado} and skirt all that is associated with \emph{grosero}, you'll need one part
proper language and one part common sense. We've already addressed
the rudiments of good manners in our brief review of greetings and
good-byes (see Chapter 1). The phrases \emph{buenos días, buenas tardes},
and \emph{buenas noches}, in conjunction with \emph{gracias} and \emph{hasta luego},
will get you through 90 percent of your daily encounters. But that's
about all they'll do. If your goal is to go beyond the point of just getting by, you'll want a more in-depth look at the universe of Spanish
formalities.

\section{Meetings and greetings}

Politeness begins upon meeting a person. You meet 'em, you
greet 'em. The question is, how?

The answer depends on the person you're meeting. If there's
little or no chance of ever seeing the person again, you're safe and sufficient with the \emph{buenos días\ldots{} hasta luego} formula. If you think the
person may figure in your future life, or if the meeting is the result of
an introduction, you're expected to go beyond that. From a second encounter onward, except in the case of repeated encounters with employees (shop clerks, the doorman, waiters, the gardener), you should
usually employ a more personal greeting than just "good day" or "good
night."

Most students of Spanish have been drilled in the basic forms
of greeting, but they're worth a quick review. On being introduced to a
person, you have at your disposal a number of stock responses: \emph{mucho
gusto, tanto gusto}, and \emph{encantado} (or \emph{encantada}) in roughly descending order of frequency. For less formal introductions and situations
\emph{¿qué tal?} and \emph{hola} work well. Save \emph{muchísimo gusto} for someone
you've been dying to meet.

Once you've been introduced to a person, you'll naturally be
expected to greet this individual at all future encounters, be it on the
street or at a party. How you do this reflects (a) who the person is and
(b) who you are in relation to that person. Here are some of the common options, in more or less descending order of formality:

\bsk

1. \emph{¿Cómo está?} or \emph{¿Como está usted?}

2. \emph{¿Cómo le va?}

3. \emph{¿Qué tal?}

4. \emph{¿Cómo estamos?}

5. \emph{¿Cómo estás?}

6. \emph{¿Qué hay de nuevo?}

7. \emph{¿Qué pasó?} or \emph{¿Qué pasa?} (varies by country)

8. \emph{¿Qué me cuentas?} or \emph{¿Qué me dices?}

9. \emph{¿Qué onda?} (Mexico) or \emph{¿Quiúbole?} (mostly Mexico and
the Caribbean)

10. Slangy variations of the preceding, such as \emph{¿Qué pasotes?}
or \emph{¿Qué pasión?} (from \emph{¿Qué pasó?}), \emph{¿Qué hongos?} (from ¿Qué onda?),
and so on. Save these for the people whose street gang you're looking
to join.

\bsk

It bears noting that all these expressions---except \emph{¿Cómo está?}
and \emph{¿Cómo está usted?}---imply some level of friendliness. Said another way, if you're not on especially friendly terms with the person,
stick to the first expression listed above. It is the only appropriate
form for greeting a person whose social, familial, occupational, or political position warrants your respect, and thus is the safe choice for
those who aren't, strictly speaking, your buddies. \emph{¿Cómo estamos?} has
paternalistic overtones and is often used by older people to greet
younger ones---even if the younger ones are thirty or forty years old.
This greeting is also a safe one when you're on good terms with the
person but aren't sure whether to use tu or usted (more on that bugaboo in a bit).

A common way of sprucing up any greeting is to use the person's name, title, or both. The commonest titles are \emph{Don} and \emph{Doña}
(for older people) and professional titles like \emph{Doctor, Contador} (accountant or C.P.A.), \emph{Ingeniero, Profesor, Maestro} (any teacher or craftsperson and sometimes even mechanics and plumbers), and the ubiquitous \emph{Licenciado} (virtually anybody who wears a tie). These are used far
more often than their English equivalents, especially in the workplace.

As an example of how greetings work, let's take the case of
Juan Doe, assistant director in charge of flange production, arriving at
his office. For simplicity's sake, let's presume all of the males in his
workplace are named Alberto Alvarez and all the females Teresa Ruiz.
Juan parks his car on the street and walks toward the office building. In
order, he meets and greets the following:

\bsk

\inda The eighty-year-old doorman:

\indu \emph{Buenos días, Don Alberto.}

\inda The security guard:

\indu \emph{Buenos días.}

\inda The sixty-year-old elevator operator:

\indu \emph{¿Cómo le va, Doña Tere?}

\inda The receptionist:

\indu \emph{Hola, Tere. ¿Cómo estás?}

\inda A same-aged colleague in the hall:

\indu \emph{¿Qué tal, Alberto?}

\inda A younger colleague at her desk:

\indu \emph{Buenos días, Tere. ¿Qué hay de nuevo?}

\inda A visiting branch manager:

\indu \emph{Buenos días, Señora Ruiz. ¿Cómo le va?}

\inda A co-worker and best friend:

\indu \emph{¿Quiúbole, Beto?}

\inda The immediate boss:

\indu \emph{Hola, Alberto. ¿Cómo estás?}

\inda An older co-worker:

\indu \emph{¿Qué me cuenta, Don Alberto?}

\inda An employee:

\indu \emph{Buenos días, Alberto. ¿Cómo estamos?}

\inda The division director:

\indu \emph{Muy buenos días, Señor Alvarez. ¿Cómo le va?}

\inda The office boy:

\indu \emph{¿Qué pasó, Beto?}

\inda The factory owner and CEO:

\indu \emph{Buenos días, Don Alberto. ¿Cómo está usted?}

\inda The secretary:

\indu \emph{Buenos días, Tere. ¿Qué tal?}

\bsk

By now, as you might imagine, Juan is exhausted and it's time for his
coffee break.

A couple of general tips on greetings are in order. First, note
that when greeting so many people, you will naturally gravitate toward
new ways of saying the same thing. That's because saying \emph{buenos días}
to twenty-five consecutive people can be extremely boring.

Second, use nicknames only if the person is accustomed to being called that. In other words, pay attention to whether others call a
certain Jose "Pepe" before you call him that. Use generic nicknames---such as \emph{viejo, compadre}, and \emph{jovenazo}---only when you feel certain
that the person won't be offended by your informality.

Third, if you're a male, avoid affectionate pet names for female
friends, employees, and co-workers. In Latin America it is common to
hear men calling women co-workers and employees things like \emph{linda}
and \emph{cariño}. To most North Americans this treatment is patronizing at
best and at worst borders on sexual harassment. Men in Latin America
have been slow about concerning themselves with these matters, but
that's no reason for you to imitate them.

Fourth, if you're a woman, stick to more formal modes of address until you're sure that your friendliness won't be taken as encouragement by the wolfish male mind. It's unfortunate that you have to
consider this issue, but that doesn't make it any less real.

And fifth, greet everyone possible, especially when meeting
a group of people. If you've met the people before, you are expected
to take the trouble of greeting each of them individually. Not to do
so can be interpreted as an offense. The same goes for saying goodbye. If you're in too much of a hurry or there are simply too many
people involved, make sure you issue an all-encompassing \emph{¿Qué tal,
cómo están?} or \emph{Hasta luego}, and make sure it's interpreted as all-encompassing.

\section{\textit{Usted} versus \textit{tú}}

From the moment you greet a person, you will start to think
about whether so-and-so is an "\emph{usted} person" or a "\emph{tú} person." Native
Spanish speakers make this decision instinctively; you will have to
think it through, and repeatedly. It is a concept that doesn't have an
easy English equivalent, but it is usually not that hard to keep straight.
Perhaps the most functional system for converting the concept into
English is to use usted in Spanish with anyone you would address with
"Mr.," "Mrs.," "Ms.," or "Miss" in English. Mr. Brown is your neighbor, so you use \emph{usted} with him. Once you get to know him and call
him "Fred," you can switch to \emph{tú}. Your lawyer is Ms. Smith, so she's
an \emph{usted} person; if you call your lawyer "Betty," then you would also
probably use \emph{tú} with her. And so on.

This rule of thumb works even when you don't actually know
the person's name. Your waiter's name is Juan Perez, but you probably
wouldn't know that. If you did, though, would you call him "Juan" or
"Mr. Perez?" That will generally depend upon his age, your age, and so
on. If it would feel awkward calling a twenty-two-year-old "Mr. Perez,"
go ahead and use \emph{tú} with him. Then again, if the twenty-two-year-old
happens to be an undersecretary for tax policy in the finance ministry---or a traffic cop---you would probably call him "Mr. Perez" and
thus use \emph{usted}.

As a rule, people aren't bashful about telling you to use \emph{tú}
with them if they feel it's appropriate. Almost no one will tell you, to
use \emph{usted} when you're using tu---it's the equivalent of putting you in
your place. So when in doubt, you're far safer using \emph{usted} and waiting
until you're told to do otherwise.

The best way to choose the right form is to listen to the conversation around you. Let's say you're meeting a group of friends at a
restaurant. Upon arriving, someone you don't know is sitting with
your friends. You are introduced to this person as Betty (your name),
and she to you as Yolanda (her name). This is your first clue: almost
certainly you are expected to \emph{tutear} (use \emph{tú} with) this person. To play
it safe, though, you can return the greeting with a noncommittal \emph{mucho gusto} and keep your ears open. If Sam asks Yolanda, \emph{¿Dónde estás
trabajando ahora?} then you can feel safe using \emph{tú} yourself. Likewise, if
you're addressed with \emph{usted}, you should respond with \emph{usted}.

That said, there are a few cases in which the \emph{tú-usted} relationship is not reciprocal---that is, when you will use \emph{tú} with the person
and he or she will use \emph{usted} with you, or vice versa. Almost always
this is the result of a considerable age difference. You might use \emph{usted}
with your friends' parents, for instance, and they will likely use \emph{tú}
with you. (This, incidentally, fits under the "Mr. and Mrs." rule.) Turn
the formula around for your children's friends.

\section{Magic words}

Besides greeting people, expressing thanks is probably the
daily act that most requires a modicum of civility. \emph{Gracias} is the
obligatory comment, of course, but you can spice it up with a \emph{muchas}
before it or a \emph{muy amable} after it or both, as noted in Chapter 1. By
doing so, you'll sound both more polite and more fluent. \emph{Muy gentil} is
also used, but it sounds somewhat strained.

When being thanked, you can respond with \emph{por nada, de
nada, no hay por que}, or \emph{no hay de que}. They're all about the same
and mean "You're welcome." You will hear \emph{para servirle} a lot from
clerks and waiters, but don't use it yourself unless it is genuinely
your job or duty to serve the person, or unless you're feeling especially subservient.

Another common linguistic nicety is asking permission. In
Spanish, as in English, the most typical way of asking permission is
essentially to excuse oneself for having the audacity to ask. Thus we
ask a person's "permission" to squeeze by them in an aisle. Here are
some options for communicating your humility while asking someone
to move over or let you by, again in descending order of formality:

\bsk

1. \emph{Con permiso}

2. \emph{¿Me permite?}

3. \emph{Perdón}

4. \emph{¿Se puede?}

5. \emph{Comper'} (a slangy version of con permiso)

6. \emph{Hágase un poco para allá, por favor}

7. \emph{Abreme espacio} or \emph{Abreme cancha}

8. \emph{Hazte pa'llá}

\bsk

The first five expressions are formal enough for just about any occasion. \emph{¿Me permite?} is sometimes used ironically, as when someone is
clearly blocking the way; say it very innocently for greatest effect. \emph{¿Se
puede?} is also the common way to ask to see something in a store; it
presumes you will wait for an affirmative response before, say, taking a
painting down off the wall or an earring out of a glass case. Unless the
object you wish to see is obvious (you're pointing at it, for instance),
you should use the full phrase: \emph{¿Se puede ver?}

The last three expressions on the list convey informality or
rudeness, depending on the person you are speaking to and your
tone of voice. \emph{Hazte pa'llá}, for instance, can be used for either "Scoot
over a little" (with a friend) or "Get out of the way" (With a stranger).
\emph{Abreme cancha} is very slangy.

The phrase for "Coming through!"---as when carrying a two-hundred-pound sofa down the hall---is \emph{¡Golpe avisa! Excúsame}, incidentally, does exist as a Spanish expression, but it doesn't mean "Excuse me." Use something (anything!) else.

Here's a cultural tip. One nicety that many foreigners
have trouble learning is to say gracias when someone sends \emph{saludos}
through them to another person: \emph{Salúdame a tu esposa} ("My regards to your wife") is something you will hear constantly if you are
married, for instance. Your English-speaking reflex will be to answer
"Okay, I will," but in Spanish it is customary to thank the person for
this \emph{detalle} ("thoughtfulness" or "consideration").

As \emph{gracias} is the universal word for "thank you," \emph{por favor} is
all you will ever need for "please." Always remember to use it when
asking for something. If you're tired of it and want to flex some Spanish muscles, use instead \emph{si es tan amable}, generally placed after your
request: \emph{Un café, si es tan amable}. It means \emph{por favor}.

\section{Sugar versus saccharine}

Spanish speakers are famous for going a bit overboard with
their floridness and politeness, and it is a matter of opinion whether
foreigners learning the language should leap in after them. Use your
own judgment. For instance, Spanish speakers will often refer to their
house as \emph{su casa} or \emph{tu casa}---"your house"---meaning that now that
you know them, you should consider their abode to be yours. This can
get pretty confusing at times: someone will be giving you directions to
their home and at the end they will point to the map they've improvised and say, "And here is your house." You'll be tempted to respond,
"But my house is nowhere near there!"---until you recall this subtle
gesture of hospitality. It can get even more confusing when a nonnative speaker, from whom such a gesture is generally not expected,
tries to communicate it.

Slightly less silly-sounding in the mouths of foreigners is the
statement \emph{Está usted en su casa} when someone comes to visit. It's really nothing more than a way of saying "Make yourself at home" and
shouldn't be made to sound more grandiose than that. To use it really
correctly, save it for when a houseguest makes a simple request, such
as "Can I use the phone?" To this you reply, with a wave of the hand,
\emph{Estás en tu casa}.

A word on homes is in order at this point. In much of the
Spanish-speaking world, homes are considered private reserves, and it
is not especially common to receive an invitation to visit someone's
house. So first of all, don't expect to receive such an invitation. And
don't be surprised if your offers of hospitality---"Hey, Pedro, how about
popping by after work for a beer?"---are taken as a nice gesture instead
of as a genuine invitation. Furthermore, never drop in unannounced on
a friend in the Spanish-speaking world. Finally, you may be told while
visiting someone's house that some object you admire \emph{es tuya} ("it's
yours," "take it"). Not only shouldn't you accept such offers, you
should be careful about making them. Sooner or later a literal-minded
guest might take you up on it!

Other sweet-sounding phrases that border on the sickly sweet
include solemn declarations of humility (saying \emph{su servidor} instead of
"I"), exaggerated requests for cooperation (\emph{Tenga usted la bondad de
traerme un café}), and overly formal greetings (\emph{Me es grato tener la
oportunidad de conocerlo}). All sound like you're reading out of a
phrasebook---and a phrasebook written for royal weddings, at that.

\section{Asking and getting}

Politeness is particularly important when asking someone for
something, since otherwise you may not get it. Bicultural lore is full of
anecdotes about Party A getting something in twenty minutes that
took Party B three weeks to get, simply because Party A asked politely
and Party B was viewed as rude. Whether you plan to use your Spanish
to get government authorizations or good directions, your command of
these niceties is critical.

It's easy to sound rude or clumsy when making a special or
even routine request in Spanish, especially since most students of the
language are taught the imperative as the sole way of asking for things:
Thus many students of Spanish will tell their hostess, \emph{Tráigame un café} ("Bring me a coffee"), thinking that's the correct, formal, and polite way to petition one. A good hostess will bring you one anyhow,
but on some interior level she's thinking, "Sure, here's one in your lap,
schmuck." Fortunately, it's quite easy to sound natural when asking
for something, but most Spanish texts neglect to mention the most frequently used form of the imperative in daily life: the indicative. That
is, most people don't say \emph{Tráigame un café, por favor} but \emph{¿Me trae un
café, por favor?} An added advantage to this form is that you don't have
to worry about those strange imperative forms and can stick to the
tried-and-true indicative instead.

The \emph{¿me trae? + por favor} formula is all you'll ever need for
restaurants and the like. But this use of the indicative also works for
most other situations and can employ a number of different verbs in
addition to \emph{traer}, especially \emph{permitir, dar, prestar}, and \emph{regalar}. \emph{Permitir} is the most formal: \emph{¿Me permite un cigarro?} will get a cigarette off
your future father-in-law. \emph{Prestar} and \emph{regalar} are the least formal, and
the latter implies you're going to keep what you're given: \emph{¿Me prestas
tu pluma?} means "Lend me your pen." \emph{¿Me regalas tu pluma?} is "Can
I have your pen (forever)?" In colloquial use, \emph{pasar} is common and
equates with \emph{prestar}: \emph{¿Me pasas el cenicero?} means "Pass the ashtray
over here, would ya?" Another common formal way of asking for
things is equivalent to "May I" in English: \emph{¿Puedo tomar un cigarro?}
("May I have a cigarette?").

\section{Etc.}

Phone Spanish is generally even more polite than face-to-face
Spanish. Listening to it, you will hear a lot of phrases like \emph{si es tan
amable} and \emph{si no es mucha molestia} tacked on to simple requests.
Once you get the hang of it, you'll be doing the same thing. To ask for
someone on the phone, you can be formal---\emph{¿Me puede comunicar can
el Sr. So-and-so?}---or informal---\emph{¿Está por ahí So-and-so?}. It will depend on whom you're calling. To say "Speaking" when someone calls
and asks for you, just say \emph{Él} (or \emph{ella}) \emph{habla}.

Interrupting is considered bad form in any language, of course,
but some foreigners seem to do it more when the conversation they are
interrupting is in a foreign language---i.e., one they don't understand
that well. Be aware of this, and be conspicuously polite when you do
need to interrupt, directing yourself to the party whom you are momentarily cutting out of the picture. Act, in other words, as if you were
cutting in on a dance. \emph{¿Me permite un momento?} you can ask.

Certain situations call for specific graces from you. Some are
verbal graces and some aren't, but all come under the heading of \emph{buena educación}.

\bsk

1. When you pass by someone who is eating, and presuming
you are at least vaguely acquainted with the person, wish him or her
\emph{provecho} or \emph{buen provecho} ("bon appetit," as we'd say in English).
Don't say it to total strangers in restaurants, though.

2. North Americans like to toss things. "Here's the pencil you
asked for," they'll say as they wing a fine-pointed no. 2 lead pencil by
your ear. "This is no good," they'll say as they crumple a sheet of paper
and lob it at the wastebasket. In the Spanish-speaking world, this behavior is considered barely short of barbaric.

3. Ditto for pointing at people.

4. Be conscious of not "giving your back" to people. Many
people from non-Latin cultures do it without intending offense. But in
Spanish-speaking cultures it's common to see people realize there is
someone behind them listening, do a half-turn, say \emph{perdón}, and continue speaking in a way that includes the previously excluded person.

\section{Sweet sorrow}

Departures are easily handled in Spanish by any of a number
of words, but \emph{adiós} and \emph{hasta luego} are sufficient for almost every circumstance. \emph{Adiós}, as you may have been taught, is generally used for
more lasting farewells. \emph{Nos vemos} is also a common colloquial send-off, equating with "See you later." \emph{Ciao} (or \emph{chao}) and \emph{bye} are making
rapid headway into Spanish from their respective languages.

Slightly more formal leave-taking makes use of phrases like
\emph{Que le vaya bien}, used only when the person you say it to is doing the
leaving. More fancy is \emph{Vaya con Dios}, but unless you're a nun or a
priest, it comes close to the saccharine category. When taking leave of
a group or passing by people on your way out, it's considered nice
manners to say \emph{Con permiso} to the people who are staying on. Usually it is
used for people whom you didn't actually say hello to on your way in.
Leaving the dentist's office, for instance, you say \emph{Gracias} to the dentist
and the receptionist and \emph{Con permiso} to the people sitting in the waiting room.

For less formal departures and with the younger set, you can
say things like \emph{Cuídate} (roughly, "Take it easy") and \emph{Pórtate bien} ("Behave yourself") to people when they are leaving. If it's nighttime and
the person is presumably leaving to go home to sleep, you can say \emph{Que
descanses} ("Rest up"). As a rule, try to respond with a farewell that is
different from the one used by the other person. If someone says \emph{Hasta
luego}, you say \emph{Nos vemos}; if they say \emph{Nos vemos}, answer \emph{Hasta luego}.

\chapter{Tricksters}

Every non-native speaker of a foreign language is afraid of
making mistakes. And good thing, too. A little fear forces us to concentrate harder and causes the memory of our mistakes to linger-long
enough to correct them the next time we open our mouths.

Most mistakes are grammatical and are eliminated only after
long periods of trial and error. Other mistakes, though, are almost the
fault of the languages themselves. Spanish and English, because of
their long history of coexistence and their many common origins, contain a lot of words that are similar on the surface but are used quite
differently. This makes instinctive translation at times as dangerous as
a cross-eyed knife-thrower.

Fortunately things aren't quite that bad. A lot of these false
cognates are so frequently tripped over by students of both languages
that they can be identified in advance. In the following list of "tricksters," you'll find some of the Spanish words most commonly misused
by native English speakers. Review the words and keep them in the
back of your brain. You won't overcome all of your instincts overnight,
but knowing your enemies-in this case, the tricksters-is the first
step toward conquering them.

\section{\emph{acostar}}

%ACOSTAR
Not "to accost," which is acosar. Acostar simply
means "to lie down" and is usually used in the reflexive. \emph{Me voy a
acostar} = "I'm going to lie down" or "I'm going to bed." If by doing
this you are accosting someone, you may be in the wrong bed.

\section{\emph{actual}}

%ACTUAL
Not "actual" but "current" or "present." \emph{La actual
administración} means "the current administration" or the one in power
at the moment. To convey "actual" in the sense of "factual" or "genuine,' you would resort to \emph{verdadero, autentico, genuino, real}, etc. "This
is an actual Picasso" is expressed by \emph{Este es un auténtico Picasso}.

\section{\emph{actualmente}}

%ACTUALMENTE
Worth a special mention because "actually"
is so common as a sentence-starter in English: "You must be starving."
"Actually, I just ate." \emph{Actualmente} won't work here, since it means
"at present" or "currently." Confusing the two can lead to some
strange situations. Think of someone telling you, "You and your
brother are invited for lunch"---to which you want to answer, "Actually, he's my husband." If you use \emph{Actualmente, es mi esposo}, what
you are saying is "At present, he's my husband." For "actually," try the
phrase \emph{1a verdad es que}: \emph{La verdad es que acabo de comer}. When "actually" is used in mid-sentence for emphasis, you can use either
\emph{realmente} or some other construction altogether. For example, "He actually ate it!" can be expressed as \emph{¡Realmente se 1o comió!} or \emph{¡Sí se 1o
comió!} or \emph{¡De verdad se 1o comió!}

\section{\emph{afección}}

%AFECCIÓN
This doesn't mean "affection" but usually a
medical condition. \emph{Tiene una afección cardiaca} = "He (or she) has a
heart condition." Use \emph{afecto}---or better yet, \emph{cariño}---to convey your
affection.

\section{\emph{americano}}

%AMERICANO
Almost more of a political issue than a semantic one. Still, the use (and misuse) of \emph{americano} by "Americans"---i.e.,
U.S. citizens---should be kept in mind by anyone eager to avoid unnecessary offense. In Spanish and Spanish-speaking countries an \emph{americano} is simply a person from "the Americas," including South and
Central as well as North America. Thus if you tell a Chilean that you
are an \emph{americano}, you might get a smirk and the comment \emph{Yo también} ("Me too") in return. Incidentally, in the Spanish-speaking world
the correct way to refer to "the Americas" is in the singular: \emph{América}.
Schoolchildren are taught that, from the Bering Strait to Tierra del
Fuego, it is but one continent.

Oddly enough, there really is no perfect way for U.S. citizens
to state their origin in Spanish. \emph{Norteamericano} is the most common
word, but technically it includes Canadians and Mexicans as well. \emph{Estadounidense} has caught on of late, but besides being a mouthful it
overlooks the fact that other countries (the United Mexican States, for
instance) are also technically "United States." Even so, saying \emph{Soy de
los Estados Unidos} is probably the easiest way of explaining your
plight.

This confusion, incidentally, goes a long way toward explaining why \emph{gringo} is so common a word in countries like Mexico and why
you yourself will probably start to use it after a short time in these
places. The word can be used in an offensive way, but usually the
word's negative connotation is a result of the tone of voice. In Mexico,
especially, \emph{gringo} is very commonly used as an adjective as well; imports from the United States can easily (if colloquially) be referred to
as \emph{productos gringos}.

\section{\emph{argumento}}

%ARGUMENTO
A classic trickster. This word doesn't work
well as "argument" in the usual sense of a "heated discussion" or a
"quarrel." Instead, it should be reserved for prepared arguments (such
as lawyers' summations), debates, and carefully reasoned, often written
arguments. \emph{Argumento} describes a logical process, not a rather illogical throwing of pans and vases. For that, use \emph{pleito, disputa}, or \emph{disgusto} (see below), roughly in descending order of intensity. Perhaps the
best all-purpose translation of "argument" is another trickster, \emph{discusión}, which refers to a far more heated exchange than what we consider a "discussion" in English.

\section{\emph{asistir}}

%ASISTIR
Not "to assist" but "to attend" or "to be present at."
\emph{Asistencia} is the proper word for "attendance" (see \emph{atender} and \emph{audiencia} below), and an \emph{asistente} is technically just someone who is attending something. Nonetheless, you may come across \emph{gerente asistente} as a way of saying "assistant manager," though you'll probably
come across it at a local branch of a US. company. A more natural
Spanish construction would be \emph{subgerente}. For "to assist," use \emph{ayudar}.

\section{\emph{atender}}

%ATENDER
Not "to attend" (see \emph{asistir} above) but "to attend
to"---a significant difference. Remember, you can \emph{atender} a patient but you can't \emph{atender} a concert.

\section{\emph{audiencia}}

%AUDIENCIA
Another word in the "attendance" versus "assistance" imbroglio. Correctly, an \emph{audiencia} is usually a private interview granted to you by someone more important than you. In this
sense, it is usually used with \emph{conceder}: \emph{El ministro nos concedió audiencia} = "The minister granted us an interview." This meaning is
still in use in English, but it is usually reserved for the pope. In Spanish the town sewage commissioner can grant you one. \emph{La audiencia}
can be used for "the audience"---in a concert hall, for instance---but \emph{la
asistencia, el auditorio}, and \emph{el público} are all preferred.

\section{\emph{balde}}

%BALDE
This doesn't refer to hair loss but to a bucket. It's also
commonly seen in the expression \emph{en balde}, which means "in vain."
For "bald," you can use either \emph{calvo} (polite) or \emph{pelón} (irreverent).

\section{\emph{balón}}

%BALÓN
The common word for "ball," from about grapefruit
size on up. Smaller balls are called bolas or pelotas. "Balloon" is
\emph{globo}, just so you know.

\section{\emph{billón}}

%BILLÓN
A false cognate that many overlook. \emph{Un billón} is
1,000,000,000,000, or 10 to the twelfth power ($10^{12}$). It is equal to the
U.S. "trillion." (In England, this quantity is a "billion.") To convey the
US. "billion," or 10 to the ninth power (UK. "milliard"), you must say
\emph{mil millones}, or "a thousand millions."

\section{\emph{bizarro}}

%BIZARRO
A fairly archaic term for "chivalrous" or "brave"
(not "bizarre" in the sense of "peculiar"). Both "bizarre" and \emph{bizarro}
come from the Basque word for "bearded," which apparently suggested
strangeness to some and gallantry to others. You don't need to know
\emph{bizarro} to speak Spanish, but you should be aware that it doesn't mean
"bizarre" if you tend to translate your thoughts fairly literally from English to Spanish. For "bizarre," use \emph{raro} or \emph{extraño}.

\section{\emph{"capable"}}

\chapter{Our Fellow Human Beings}

\section{The good}
\section{The \textit{amables}}
\section{The \textit{simpáticos}}
\section{The \textit{listos}}
\section{The bad}
\section{The \textit{pesados}}
\section{The \textit{imbéciles}}
\section{The \textit{malvados}}
\section{The \textit{cochinos}}
\section{The indifferent}
\section{Temporary states}

\chapter{The Secret Life of Verbs}

\section{The present}
\section{The future}
\section{The conditional}
\section{The preterit versus the imperfect}
\section{Special cases}
\section{Ser versus estar}
\section{The easy ones}
\section{Getting tricky: the past participles}
\section{The hard ones: descriptive adjectives}
\section{Sorting out ser and estar in the imperative}
\section{A matter of perspective}

\chapter{The Twilight Zone}

\section{Indirect commands (shallow twilight)}
\section{The eternal mystery (deep twilight)}
\section{\textit{Que} cues}
\section{Non-\textit{que} cues}
\section{The subjunctive with \textit{ser}: \textit{sea}}
\section{The traveler's subjunctive}

\chapter{Sixty-four Verbs, Up Close and Personal}

\chapter{Cranking Up Your Spanish}

\chapter{Snappy Answers}

\chapter{Invective and Obscenity}

\chapter{Which Is Which?}

\chapter{Say It Right}

\chapter{Spanish Roots}

\chapter{The big Mix}


\end{document}
