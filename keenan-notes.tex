\documentclass[14pt,a4paper,oneside]{memoir}

\usepackage{ifxetex}
\usepackage{amssymb} % Essential to be included before xunicode
\usepackage{xcolor}

%%%%%%%%%%%%%%%%%%%%%%%%%%%%%%%%%%%%%%%%%%

\ifxetex

\usepackage{fontspec}
\usepackage{xunicode}
\usepackage{xltxtra}
\usepackage{polyglossia}
\setromanfont[Mapping=tex-text]{Liberation Serif}
\setsansfont[Mapping=tex-text]{Liberation Sans}
\setmonofont[Mapping=tex-text]{Liberation Mono}
\setdefaultlanguage{russian}

\else % not XeTeX

\iffalse

\usepackage{type1ec}
\usepackage[T1]{fontenc}
\usepackage[utf8x]{inputenc}
\usepackage[english,russianb]{babel}

\else

\message{XeTeX only document}
\stop

\fi

\fi % not XeTeX

%%%%%%%%%%%%% Common preamble %%%%%%%%%%%%

%\usepackage[usenames,dvipsnames,svgnames,table]{xcolor}

\usepackage{graphicx}
\usepackage{textcomp}

\usepackage{indentfirst}
\frenchspacing
\clubpenalty=10000
\widowpenalty=10000

\sloppy

\newcommand{\sk}{\smallskip}
\newcommand{\pp}{\textcolor{red}}   % Prepositional Pronoun.
\newcommand{\lp}{\textcolor{brown}} % Lexicalized Pronoun.
\newcommand{\bl}{\textcolor{blue}}  % Just blue.
\newcommand{\nb}[1]{{\small \emph{\textcolor{blue}{#1}}}}

\usepackage{indentfirst}

\chapterstyle{verville}
\usepackage[hidelinks]{hyperref}
\setsecnumdepth{subsection}

\begin{document}

\frontmatter

\begin{titlingpage}
\title{Breaking Out of Beginner's Spanish}
\posttitle{\par\vskip1em{\normalfont\normalsize (A digest)\par}\end{center}}
\author{Joseph J. Keenan}
\date{~}
\maketitle
\end{titlingpage}

\tableofcontents*

\mainmatter

\chapter{Ten Ways to Avoid Being Taken for a Gringo}

A gringo, in much of the Spanish-speaking world, is a person
who comes from abroad, speaks another language, and wears loud
shorts. In certain countries, such as Mexico, it refers specifically to
U.S. citizens, but even there the distinction is hazy. A Canadian or
German who acts like a gringo will be referred to as a gringo, birth certificates be damned. Act like a gringo and you will be called one; don't
act like one and you may be called one anyway. The word is descriptive
first-of a style, a cultural stance, a way of life-and derogatory only
later, if at all.

So how to avoid being taken for a gringo? The truth is, if you
were born outside the Spanish-speaking world, there is probably nothing you can do to hide the fact. You will never fully blend in nor
should you necessarily want to. But as you travel or mingle among
Spanish-speakers, you may wish to smooth over the most obvious differences that set you apart. We all want to be outstanding; standing out
is another matter altogether.

Since you're making an effort to speak and understand Spanish, you've already distinguished yourself from the stereotypical
gringo, that mythical beast of Latin American lore who wears obtrusive shirts, smacks gum, and tends to misplace his or her wallet. But
even if you're not one of them, you can still heed a few simple precautions that will help put you in tune with the local culture, be it in Patagonia or East Los Angeles. Dress codes and behavior are beyond the
scope of this book, but some general pointers may keep others from
pointing at you.

\section{Pronunciation}

Spanish pronunciation should be the easiest thing in the world
to master. Unlike English, where a letter can change its sound seemingly at will, Spanish letters have-with very few exceptions-the exact same sound word after word. To compare, think of the three different sounds of the letter a in the English pronunciation of the name
Abraham; in Spanish the letter a in Abraham has but a single sound,
repeated three times. Still, you'll need to practice to convince your
tongue to make the correct sound, to get your teeth to close or open on
cue, and to master the inflections. At first it will be a struggle, but
there is no reason why anyone can't learn to pronounce Spanish properly after enough use. Here are some tips on how to proceed:

\subsection{}

Spanish teachers always tell their students to practice repeating the vowel sounds: \emph{a, e, i, o, u}. Listen to these wise men and
women and practice, practice, practice. If it's more fun, follow your
litany with the phrase schoolchildren learn south of the border: \emph{El burro sabe más que tú} ("The burro knows more than you"). Move your
mouth as you repeat the vowels. Pretend someone fifty yards away is
trying to read your lips. Clip the vowel sounds short, as you might
imagine a Japanese colonel in a late-night, World War II movie would
do. \emph{A, E, I, O, U, A, E, I, O, U}\ldots{}

\subsection{}

Next come the sounds for the letters \emph{r} and \emph{rr}. The double
one trills, and so does the single one at the start of a word. Thus \emph{carro}
and \emph{rancho} have essentially the same \emph{r} sound. Your tongue won't want
to trill at first; it will make a scene about being made out of concrete
and will refuse to emit such a ridiculous sound. But you're not going to
let your tongue push you around, are you? Trill away! Pretend you're
Charo on "The Tonight Show": "R-r-r-really, Johnny, r-r-r-romance for
me is r-r-r-relaxing on a r-r-r-rug, listening to r-r-r-rock and r-r-r-roll." If
you've studied any French, this sound may be harder for you to get
used to at first. But if you try at all, you will learn it. Many gringos do
and, as far as anyone can tell, we're all born with the same kind of
tongue.

\subsection{}

The d between vowels or at the end of a word sounds more
like the \emph{th} in \emph{thus} than an English \emph{d}. \emph{Nada} is thus pronounced "natha," or close to it. So light is the mid-vowel \emph{d} that sometimes, in colloquial spoken Spanish, it's almost left out altogether; you may even
see \emph{nada} represented as \emph{na'a} or \emph{na'} in written dialogue. You shouldn't
take it that far, but do get the hang of the soft Spanish d by learning a
few words well: \emph{nada, limonada, edad, comida, ciudad, cansado}. All
regular past participles follow the same rule: \emph{hablado, conocido, bebido}, and so on. The \emph{d} at the beginning of a word in Spanish is also a
tad softer than its English counterpart, perhaps more like a \emph{dth} than a
solid \emph{d}. Shout the name David in English and you can almost feel
yourself spit; shouting it in Spanish is a much less moist affair.

\subsection{}

The Spanish letters \emph{c, z, j}, and \emph{ll} vary in pronunciation
from country to country. Don't let this bother you. Either adopt the
sound used in your country of choice or seek a middle ground. For
most purposes, you're safe pronouncing both the \emph{c} and \emph{z} as an English
\emph{s}, the \emph{j} as an \emph{h}, and the \emph{ll} as the \emph{y} of "yes." (Remember, of course, that
the hard \emph{c} in Spanish, as in \emph{ca-, co-}, and \emph{cu-}, sounds like a \emph{k}.) Other
noteworthy regional differences include the use of \emph{vos} instead of \emph{tú}
and the use of \emph{vosotros} instead of \emph{ustedes}. \emph{Vos} is used in much of
Central and South America and requires learning yet more verb endings (\emph{tú quieres = vos querés}). Still, if you are spending time in those
countries, you will probably want to use it. The same goes for \emph{vosotros}, which is used in Spain for the second-person plural and which
you will no doubt learn if you're picking up your Spanish there.

\subsection{}

Other regional differences are best left alone. In many
countries, for instance, there is a tendency to "swallow" the \emph{s} sound at
the end of a syllable, especially before consonants: estoy aquí becomes
e'toy aquí. (It reaches such extremes that there's even a joke about the
Cuban child who asks his mother how to form plurals. "Easy, chico,"
she says, "just add an \emph{s}: \emph{el coco, lo coco}.") There's really no good reason to learn to speak Spanish that way, or with any other regional dialect, unless you're keen on being identified as having studied in a specific country. Imagine an Asian immigrant speaking English like a
Boston cabbie, or a Uruguayan drawling like a Texan, and you'll understand why.

\subsection{}

In Spanish, the letters \emph{b} and \emph{v} sound the same: almost (but
not quite) like the English \emph{b}. Like its d sound, Spanish's \emph{b/v} sound is a
shade softer, especially between vowels. Thus \emph{ave} is not pronounced
either "ah-bay" or "ah-vay" but "ah-bvay." Say "ah-bay" fast, without
giving your lips time to spit out a hard \emph{b} sound, and you'll get the idea.

\section{The wrong word syndrome}

There are dozens of cases in Spanish where you will be
tempted to use a word that is patently wrong. Mostly this is a result of
misleading English cognates---words that look or sound the same in
English and in Spanish but harbor different meanings. Sometimes the
meaning will be close in Spanish, and the lazy language learner is content to use the cognate. But you aren't like that, are you? You want to
speak Spanish well and not be responsible for polluting it with English
usages. Swell. For people like you, Chapter 3 in this book is dedicated
to these "tricksters," and Appendix B covers more subtle nuances.
Don't worry about learning them all at once. Just try to remember
which words are tricky and then try to avoid stepping in them as you
go along.

\section{\textit{Yo-ismo}}

In English, a sentence is incomplete without a noun or pronoun for a subject. In Spanish, the subject of the sentence is often conveyed by the verb and is optional. Often, in fact, it is left out altogether, unless the speaker wants to emphasize the subject of the
statement. This quaint little grammatical fact affects you, the language
learner, in one important case: the first person. Since you are used to
including pronouns, you will tend to preface all of your first-person
comments with yo. But to a Spanish ear, this sounds like you are constantly calling attention to yourself: "\emph{I} want this" and "\emph{I} think that."
This affliction, dubbed "\emph{yo-ismo}," can in extreme cases make people
think you're a pretty snotty individual, when you and I know that's not
true. But why take chances? Try to say \emph{quiero} instead of \emph{yo quiero},
\emph{creo} instead of \emph{yo creo}, and so on. Later, when you've broken the habit,
you can go back to inserting the occasional \emph{yo} to emphasize a truly
personal opinion: \emph{El quiere casarse pero yo no quiero} ("He wants to
get married but \emph{I} don't").

\section{The stumbles}

The stumbles are what you get when you're asked a simple
question and your tongue runs off and hides behind your tonsils. It is a
common ailment of those who have studied some Spanish---and \emph{know}
what they want to say---but lack conversational practice. Remedying
this condition requires practice, of course, but also a careful study of
interjections, pert comebacks, snappy answers, and sentence starters.
These useful words and phrases will get you through almost every
situation requiring sudden tongue work, but they are often neglected
in textbooks. Lists of these gems are included in later chapters. Pick
your favorites (there are usually several acceptable ones for each situation) and store them close to your tongue.
Speaking of the stumbles, you should be especially cautious of
letting your English "crutch words" slip into your Spanish. It sounds
awful: \emph{Quiero\ldots{} um\ldots{} ir a\ldots{} you know\ldots{} the\ldots{} er\ldots{} cine,
okay?} If you must, learn some Spanish crutch words and lean on them
instead: \emph{Quiero\ldots{} este\ldots{} ir\ldots{} o sea\ldots{} al cine, ¿no?} You'll still
sound like a space cadet, but at least a fairly fluent one.

\section{Formalities}
\section{The volume}
\section{Adjectives}
\section{Speed kills}
\section{Body language}
\section{Those crazy gringos}

\chapter{Minding Your Verbal Manners}

\section{Meetings and greetings}
\section{\textit{Usted} versus \textit{tú}}
\section{Magic words}
\section{Sugar versus saccharine}
\section{Asking and getting}
\section{Etc.}
\section{Sweet sorrow}

\chapter{Tricksters}

\chapter{Our Fellow Human Beings}

\section{The good}
\section{The \textit{amables}}
\section{The \textit{simpáticos}}
\section{The \textit{listos}}
\section{The bad}
\section{The \textit{pesados}}
\section{The \textit{imbéciles}}
\section{The \textit{malvados}}
\section{The \textit{cochinos}}
\section{The indifferent}
\section{Temporary states}

\chapter{The Secret Life of Verbs}

\section{The present}
\section{The future}
\section{The conditional}
\section{The preterit versus the imperfect}
\section{Special cases}
\section{Ser versus estar}
\section{The easy ones}
\section{Getting tricky: the past participles}
\section{The hard ones: descriptive adjectives}
\section{Sorting out ser and estar in the imperative}
\section{A matter of perspective}

\chapter{The Twilight Zone}

\section{Indirect commands (shallow twilight)}
\section{The eternal mystery (deep twilight)}
\section{\textit{Que} cues}
\section{Non-\textit{que} cues}
\section{The subjunctive with \textit{ser}: \textit{sea}}
\section{The traveler's subjunctive}

\chapter{Sixty-four Verbs, Up Close and Personal}

\chapter{Cranking Up Your Spanish}

\chapter{Snappy Answers}

\chapter{Invective and Obscenity}

\chapter{Which Is Which?}

\chapter{Say It Right}

\chapter{Spanish Roots}

\chapter{The big Mix}


\end{document}
