\documentclass[14pt,a4paper,oneside]{memoir}

\setlrmarginsandblock{2.5cm}{2.2cm}{*}
\setulmarginsandblock{2.5cm}{2.5cm}{*}
\checkandfixthelayout

\usepackage{datetime2}

%\setlrmarginsandblock{3.5cm}{2.5cm}{*}
%\setulmarginsandblock{2.5cm}{*}{1}
%\checkandfixthelayout

% Keep indentations with ragged right.
\makeatletter
\newcommand\iraggedright{%
  \let\\\@centercr\@rightskip\@flushglue \rightskip\@rightskip
  \leftskip\z@skip}
\makeatother

\usepackage{ifxetex}
\usepackage{amssymb} % Essential to be included before xunicode
\usepackage{xcolor}

%%%%%%%%%%%%%%%%%%%%%%%%%%%%%%%%%%%%%%%%%%

\ifxetex

\usepackage{fontspec}
\usepackage{xunicode}
\usepackage{xltxtra}
\usepackage{polyglossia}
\setromanfont[Mapping=tex-text]{Liberation Serif}
\setsansfont[Mapping=tex-text]{Liberation Sans}
\setmonofont[Mapping=tex-text]{Liberation Mono}
\setdefaultlanguage{english}

\else % not XeTeX

\iffalse

\usepackage{type1ec}
\usepackage[T1]{fontenc}
\usepackage[utf8x]{inputenc}
\usepackage[english,russianb]{babel}

\else

\message{XeTeX only document}
\stop

\fi

\fi % not XeTeX

%%%%%%%%%%%%% Common preamble %%%%%%%%%%%%

%\usepackage[usenames,dvipsnames,svgnames,table]{xcolor}

\usepackage{graphicx}
\usepackage{textcomp}

\usepackage{indentfirst}
\frenchspacing
\clubpenalty=10000
\widowpenalty=10000

\sloppy

\newcommand{\bsk}{\vspace{20pt}}
\newcommand{\sk}{\vspace{5pt}}
\newcommand{\pp}{\textcolor{red}}   % Prepositional Pronoun.
\newcommand{\lp}{\textcolor{brown}} % Lexicalized Pronoun.
\newcommand{\bl}{\textcolor{blue}}  % Just blue.
\newcommand{\nb}[1]{{\small \emph{\textcolor{blue}{#1}}}}

\usepackage{indentfirst}

\usepackage{csquotes}        % Defines \enquote
\usepackage{microtype}       % For hanging indents (no config needed)

%\documentclass[14pt,a4paper,oneside]{memoir}

\setlrmarginsandblock{2.5cm}{2.2cm}{*}
\setulmarginsandblock{2.5cm}{2.5cm}{*}
\checkandfixthelayout

%\setlrmarginsandblock{3.5cm}{2.5cm}{*}
%\setulmarginsandblock{2.5cm}{*}{1}
%\checkandfixthelayout

% Keep indentations with ragged right.
\makeatletter
\newcommand\iraggedright{%
  \let\\\@centercr\@rightskip\@flushglue \rightskip\@rightskip
  \leftskip\z@skip}
\makeatother

\iraggedright

\usepackage{ifxetex}
\usepackage{amssymb} % Essential to be included before xunicode
\usepackage{xcolor}

%%%%%%%%%%%%%%%%%%%%%%%%%%%%%%%%%%%%%%%%%%

\ifxetex

\usepackage{fontspec}
\usepackage{xunicode}
\usepackage{xltxtra}
\usepackage{polyglossia}
\setromanfont[Mapping=tex-text]{Liberation Serif}
\setsansfont[Mapping=tex-text]{Liberation Sans}
\setmonofont[Mapping=tex-text]{Liberation Mono}
\setdefaultlanguage{russian}

\else % not XeTeX

\iffalse

\usepackage{type1ec}
\usepackage[T1]{fontenc}
\usepackage[utf8x]{inputenc}
\usepackage[english,russianb]{babel}

\else

\message{XeTeX only document}
\stop

\fi

\fi % not XeTeX

%%%%%%%%%%%%% Common preamble %%%%%%%%%%%%

%\usepackage[usenames,dvipsnames,svgnames,table]{xcolor}

\usepackage{graphicx}
\usepackage{textcomp}

\usepackage{indentfirst}
\frenchspacing
\clubpenalty=10000
\widowpenalty=10000

\sloppy

\newcommand{\bsk}{\vspace{20pt}}
\newcommand{\sk}{\vspace{5pt}}
\newcommand{\pp}{\textcolor{red}}   % Prepositional Pronoun.
\newcommand{\lp}{\textcolor{brown}} % Lexicalized Pronoun.
\newcommand{\bl}{\textcolor{blue}}  % Just blue.
\newcommand{\nb}[1]{{\small \emph{\textcolor{blue}{#1}}}}

\usepackage{indentfirst}


\chapterstyle{verville}
\usepackage[hidelinks]{hyperref}
\setsecnumdepth{subsection}

\newcommand{\inda}{\hspace{40pt}}
\newcommand{\indu}{\hspace{20pt}}

\begin{document}

\frontmatter

\begin{titlingpage}
\title{Breaking Out of Beginner's Spanish}
\posttitle{\par\vskip1em{\normalfont\normalsize\maltese\par}\end{center}}
\author{Joseph J. Keenan}
\date{~}

\maketitle
\end{titlingpage}

\tableofcontents*

\mainmatter

\chapter{Ten Ways to Avoid Being Taken for a Gringo}

A gringo, in much of the Spanish-speaking world, is a person
who comes from abroad, speaks another language, and wears loud
shorts. In certain countries, such as Mexico, it refers specifically to
U.S. citizens, but even there the distinction is hazy. A Canadian or
German who acts like a gringo will be referred to as a gringo, birth certificates be damned. Act like a gringo and you will be called one; don't
act like one and you may be called one anyway. The word is descriptive
first-of a style, a cultural stance, a way of life-and derogatory only
later, if at all.

So how to avoid being taken for a gringo? The truth is, if you
were born outside the Spanish-speaking world, there is probably nothing you can do to hide the fact. You will never fully blend in nor
should you necessarily want to. But as you travel or mingle among
Spanish-speakers, you may wish to smooth over the most obvious differences that set you apart. We all want to be outstanding; standing out
is another matter altogether.

Since you're making an effort to speak and understand Spanish, you've already distinguished yourself from the stereotypical
gringo, that mythical beast of Latin American lore who wears obtrusive shirts, smacks gum, and tends to misplace his or her wallet. But
even if you're not one of them, you can still heed a few simple precautions that will help put you in tune with the local culture, be it in Patagonia or East Los Angeles. Dress codes and behavior are beyond the
scope of this book, but some general pointers may keep others from
pointing at you.

\section{Pronunciation}

Spanish pronunciation should be the easiest thing in the world
to master. Unlike English, where a letter can change its sound seemingly at will, Spanish letters have-with very few exceptions-the exact same sound word after word. To compare, think of the three different sounds of the letter a in the English pronunciation of the name
Abraham; in Spanish the letter a in Abraham has but a single sound,
repeated three times. Still, you'll need to practice to convince your
tongue to make the correct sound, to get your teeth to close or open on
cue, and to master the inflections. At first it will be a struggle, but
there is no reason why anyone can't learn to pronounce Spanish properly after enough use. Here are some tips on how to proceed:

\subsection{}

Spanish teachers always tell their students to practice repeating the vowel sounds: \emph{a, e, i, o, u}. Listen to these wise men and
women and practice, practice, practice. If it's more fun, follow your
litany with the phrase schoolchildren learn south of the border: \emph{El burro sabe más que tú} ("The burro knows more than you"). Move your
mouth as you repeat the vowels. Pretend someone fifty yards away is
trying to read your lips. Clip the vowel sounds short, as you might
imagine a Japanese colonel in a late-night, World War II movie would
do. \emph{A, E, I, O, U, A, E, I, O, U}\ldots{}

\subsection{}

Next come the sounds for the letters \emph{r} and \emph{rr}. The double
one trills, and so does the single one at the start of a word. Thus \emph{carro}
and \emph{rancho} have essentially the same \emph{r} sound. Your tongue won't want
to trill at first; it will make a scene about being made out of concrete
and will refuse to emit such a ridiculous sound. But you're not going to
let your tongue push you around, are you? Trill away! Pretend you're
Charo on "The Tonight Show": "R-r-r-really, Johnny, r-r-r-romance for
me is r-r-r-relaxing on a r-r-r-rug, listening to r-r-r-rock and r-r-r-roll." If
you've studied any French, this sound may be harder for you to get
used to at first. But if you try at all, you will learn it. Many gringos do
and, as far as anyone can tell, we're all born with the same kind of
tongue.

\subsection{}

The d between vowels or at the end of a word sounds more
like the \emph{th} in \emph{thus} than an English \emph{d}. \emph{Nada} is thus pronounced "natha," or close to it. So light is the mid-vowel \emph{d} that sometimes, in colloquial spoken Spanish, it's almost left out altogether; you may even
see \emph{nada} represented as \emph{na'a} or \emph{na'} in written dialogue. You shouldn't
take it that far, but do get the hang of the soft Spanish d by learning a
few words well: \emph{nada, limonada, edad, comida, ciudad, cansado}. All
regular past participles follow the same rule: \emph{hablado, conocido, bebido}, and so on. The \emph{d} at the beginning of a word in Spanish is also a
tad softer than its English counterpart, perhaps more like a \emph{dth} than a
solid \emph{d}. Shout the name David in English and you can almost feel
yourself spit; shouting it in Spanish is a much less moist affair.

\subsection{}

The Spanish letters \emph{c, z, j}, and \emph{ll} vary in pronunciation
from country to country. Don't let this bother you. Either adopt the
sound used in your country of choice or seek a middle ground. For
most purposes, you're safe pronouncing both the \emph{c} and \emph{z} as an English
\emph{s}, the \emph{j} as an \emph{h}, and the \emph{ll} as the \emph{y} of "yes." (Remember, of course, that
the hard \emph{c} in Spanish, as in \emph{ca-, co-}, and \emph{cu-}, sounds like a \emph{k}.) Other
noteworthy regional differences include the use of \emph{vos} instead of \emph{tú}
and the use of \emph{vosotros} instead of \emph{ustedes}. \emph{Vos} is used in much of
Central and South America and requires learning yet more verb endings (\emph{tú quieres = vos querés}). Still, if you are spending time in those
countries, you will probably want to use it. The same goes for \emph{vosotros}, which is used in Spain for the second-person plural and which
you will no doubt learn if you're picking up your Spanish there.

\subsection{}

Other regional differences are best left alone. In many
countries, for instance, there is a tendency to "swallow" the \emph{s} sound at
the end of a syllable, especially before consonants: estoy aquí becomes
e'toy aquí. (It reaches such extremes that there's even a joke about the
Cuban child who asks his mother how to form plurals. "Easy, chico,"
she says, "just add an \emph{s}: \emph{el coco, lo coco}.") There's really no good reason to learn to speak Spanish that way, or with any other regional dialect, unless you're keen on being identified as having studied in a specific country. Imagine an Asian immigrant speaking English like a
Boston cabbie, or a Uruguayan drawling like a Texan, and you'll understand why.

\subsection{}

In Spanish, the letters \emph{b} and \emph{v} sound the same: almost (but
not quite) like the English \emph{b}. Like its d sound, Spanish's \emph{b/v} sound is a
shade softer, especially between vowels. Thus \emph{ave} is not pronounced
either "ah-bay" or "ah-vay" but "ah-bvay." Say "ah-bay" fast, without
giving your lips time to spit out a hard \emph{b} sound, and you'll get the idea.

\section{The wrong word syndrome}

There are dozens of cases in Spanish where you will be
tempted to use a word that is patently wrong. Mostly this is a result of
misleading English cognates---words that look or sound the same in
English and in Spanish but harbor different meanings. Sometimes the
meaning will be close in Spanish, and the lazy language learner is content to use the cognate. But you aren't like that, are you? You want to
speak Spanish well and not be responsible for polluting it with English
usages. Swell. For people like you, Chapter 3 in this book is dedicated
to these "tricksters," and Appendix B covers more subtle nuances.
Don't worry about learning them all at once. Just try to remember
which words are tricky and then try to avoid stepping in them as you
go along.

\section{\emph{Yo-ismo}}

In English, a sentence is incomplete without a noun or pronoun for a subject. In Spanish, the subject of the sentence is often conveyed by the verb and is optional. Often, in fact, it is left out altogether, unless the speaker wants to emphasize the subject of the
statement. This quaint little grammatical fact affects you, the language
learner, in one important case: the first person. Since you are used to
including pronouns, you will tend to preface all of your first-person
comments with yo. But to a Spanish ear, this sounds like you are constantly calling attention to yourself: "\emph{I} want this" and "\emph{I} think that."
This affliction, dubbed "\emph{yo-ismo}," can in extreme cases make people
think you're a pretty snotty individual, when you and I know that's not
true. But why take chances? Try to say \emph{quiero} instead of \emph{yo quiero},
\emph{creo} instead of \emph{yo creo}, and so on. Later, when you've broken the habit,
you can go back to inserting the occasional \emph{yo} to emphasize a truly
personal opinion: \emph{El quiere casarse pero yo no quiero} ("He wants to
get married but \emph{I} don't").

\section{The stumbles}

The stumbles are what you get when you're asked a simple
question and your tongue runs off and hides behind your tonsils. It is a
common ailment of those who have studied some Spanish---and \emph{know}
what they want to say---but lack conversational practice. Remedying
this condition requires practice, of course, but also a careful study of
interjections, pert comebacks, snappy answers, and sentence starters.
These useful words and phrases will get you through almost every
situation requiring sudden tongue work, but they are often neglected
in textbooks. Lists of these gems are included in later chapters. Pick
your favorites (there are usually several acceptable ones for each situation) and store them close to your tongue.
Speaking of the stumbles, you should be especially cautious of
letting your English "crutch words" slip into your Spanish. It sounds
awful: \emph{Quiero\ldots{} um\ldots{} ir a\ldots{} you know\ldots{} the\ldots{} er\ldots{} cine,
okay?} If you must, learn some Spanish crutch words and lean on them
instead: \emph{Quiero\ldots{} este\ldots{} ir\ldots{} o sea\ldots{} al cine, ¿no?} You'll still
sound like a space cadet, but at least a fairly fluent one.

\section{Formalities}

The gringo who makes the effort to get off the beaten path and
find out-of-the-way shops and cafés is almost by definition a fairly gregarious soul. But fear of the language can make this same person seem
timid, uptight, or arrogant, browsing in a shop for half an hour without
saying a word to anyone and leaving without so much as a good-bye. In
general, you should be happy to inflict your fledgling Spanish on anyone who crosses your path. But you should be especially eager to unleash it in cases calling for common courtesy. There is probably no
faster way to separate yourself from the pack of tourists and gawkers
than to look someone in the eye and speak to them in their own language---even if it's only to say "hello" and "good-bye."

For starters, you should always greet people whose lives you
have invaded, if only briefly. This doesn't mean you should walk down
the streets of Buenos Aires saying \emph{hola} to everyone, but it does mean
you should immediately recognize the existence of shop clerks, waiters, secretaries, and guests. The formula is simple. From the time you
wake up until 11:59 a.m., you say \emph{buenos días}; from 12:00 noon until
dark, you say \emph{buenas tardes}; and from dark until bedtime, you say
\emph{buenas noches}. Say it loud, say it proud. You'll be amazed how service
improves and prices drop after a pleasant greeting.

When you leave a place, remember to say \emph{gracias} if it's a commercial establishment, \emph{adiós} or \emph{hasta luego} if it's not. Better yet,
when leaving a shop, say \emph{muchas gracias} or \emph{muy amable}, gracias or
even \emph{muchas gracias muy amable}, all run together. You'll feel much
more cheerful walking out of a place after a heartfelt farewell, even if
the clerk did nothing more than stare at your back the whole time you
were in the store.

In general, Spanish requires more spoken formalities than English (at least as it's spoken today), which is nice because it gives you a
lot of opportunities to practice those key words and phrases. Skipping
over the formalities, on the other hand, will tag you as a gringo from
the get-go, which is what we've decided we want to avoid. For a fuller
treatment of formalities and politeness in general, there's a whole
must-read chapter just ahead with bounteous tips and details.

\section{The volume}

In Baltimore or Toronto or Oxford our friends and neighbors
seem to speak in reasonable tones. So why is it that these same people
seem to start shouting as soon as they get through Customs? This is
one of the great mysteries of cultural intercourse, and I don't foresee
resolving it here. But it is worth mentioning that, as a native English
speaker, you are expected to shout instead of speak and that a whole
continent of Latin Americans will be grateful if you manage to do otherwise. Gringos (remember them?) tend to bunch together and speak
English at volumes appropriate for a rock concert. I've even seen a
Mexican head of family ask the maitre d' to be seated "away from the
gringos" at a restaurant. As when greeting people, speak loud and
proud-but not too loud. Contrary to gringo folk wisdom, comprehension does not increase with volume. Normal speaking tones are best,
and by the time you find yourself shouting, absolutely no one will admit to understanding you at all.

\section{Adjectives}

One of the most frustrating things about learning a new language is not having access to that grab bag of adjectives that we rely on
to express our opinions. And since adjectives are a relatively second-class part of speech (after the big shots like nouns and verbs), many
beginning students of language tend to put off learning them until
some later phase of their study. In practical terms this produces human
beings whose whole range of descriptions goes from "very, very" on
one extreme to "not very" on the other. "How was the film?" \emph{Muy,
muy, muy buena}. "How about the meal?" \emph{No muy buena}.

You should try to break out of this rut as quickly as possible,
learning alternative ways of expressing your likes and dislikes. Pay special attention to the use of prefixes and (especially) suffixes in modifying adjectives in Spanish by learning a few suffixes and attaching
them to words you already know, you can quickly multiply the coverage of your vocabulary. \emph{Grande}, for example, can go up in size to \emph{grandísimo} and even \emph{grandotote} or deflate to a semisardonic \emph{grandecito}.
Also learn some words for that vast middle ground between good and
bad where, sad to say, most of our experiences tend to fall. Chapter 4
will try to help you do just that in your encounters with people.

\section{Speed kills}

Just as loud doesn't equal intelligible, fast definitely doesn't
equal fluent. Take it easy. You wouldn't try to break speed records on a
Kawasaki when you're just learning to ride, so why try with your Spanish? Spoken English has raised slurring practically to an art form, and
it's considered normal to modify (i.e., mispronounce) consonants or
vowels that get in our way. The usual result is a series of phrases like
"I dunno" and "Waddaya-wamee-tudo?" In spoken Spanish, each vowel
and consonant retains its particular, unalterable sound, no matter how
fast you're speaking. True, if you speak fast enough, people may not
catch your errors. But they won't catch your drift either, and you'll end
up having to repeat everything. If that's your strategy for getting extra
practice, \emph{adelante}. Just say it slowly the second time.

\section{Body language}

If slurring in English is almost an art form, then downwardly
mobile dressing in English-speaking (and other) countries is long overdue for a major museum exhibition. Far be it from me to tell you how
to dress on vacation or when prowling the barrio, but do remember the
. Spanish \emph{dicho: Como te ven, te tratan} ("How they see you, they treat
you"). Dressing down has not yet caught on in most Latin cultures,
perhaps because millions of people dress that way for reasons not related to fashion. If you stay close to the tourist bus, what you wear
isn't so important---but who wants to do that? You don't have to dress
to the nines to go buy a Coke, but you should at least be in the low
sevens. Otherwise, your clothes will be saying things about you that
your mouth never would.

What gringos often do, since we seem to be talking about
them, is to convert their Sunday barbecue outfits back home into all-purpose wear for their introduction to Latin culture. What they would
never wear to church they don't think twice about wearing to a Colombian cathedral or Guatemalan village church. If you intend to show the
local people that you respect their culture, the best way to start is by
letting your clothes speak for you. And the clothes that speak best are
the ones that cover knees and shoulders---at least away from coastal
cities. You don't have to care about any of this, of course, but if you
do, remind yourself that sin and skin are still closely linked in many
minds.

\section{Those crazy gringos}

Every now and then, it won't matter how well you say something in Spanish if what you are saying is so patently absurd that it
transcends language altogether. And what is judged as absurd can vary
widely from place to place. I remember witnessing a frustrated tourist trying to request, in flawless Spanish, \emph{hielo hervido} for his soft
drink. Now, if you're familiar with tourists, you'll probably realize that
"boiled ice" is a sort of shorthand way of saying "ice made from boiled,
or purified, water." But if you're a waiter in a small-town bus-station
restaurant in Latin America, you may not make that conceptual leap at
all. Instead, you will try your darndest to make sense of your customer,
and then will probably go back to the kitchen and heat up a couple of
ice cubes. Many innocent requests like this can turn into Major Cultural Confusions if you're not careful. Asking for ice to put in a soda
that is already cold is considered downright silly in many places, for
instance. Asking for a "pizza with meat on it," in at least one place I've
been, can lead to a pizza with a slab of steak lying on top. And so on.

This, of course, is part of the beauty of getting to know foreign
cultures: learning that what you had considered a given all your life is
often not a given at all. So if you find that nothing you say in Spanish
seems to get your point across, consider changing tack. What you are
saying, not how you are saying it, may be the culprit. Then change
your order to \emph{pollo frito} and a beer and forget about it. Just pray that
they don't come in a bowl, and together.

\chapter{Minding Your Verbal Manners}

Being polite is something many people, especially the young,
associate with visits to Grandma. In daily life, only bootlickers and
dweebs make a special effort to be polite; the rest of us are as we
are---take us or leave us.

Actually much of what goes for politeness is implicit in our
behavior and requires no special effort. Society has carved on our
minds the notion that if we don't follow certain preestablished, communal norms, it will use harsh and unfriendly epithets to describe us
behind our backs. So we take it for granted that you should open the
door for the elderly, avoid using expletives in public places, and refrain
from cutting in front of people on the exit ramp. We don't think of it as
being polite; we just do it because society says so.

Spanish-speaking society has its own set of unspoken norms
that you, as an outsider, won't have had beaten into your head from
birth. This means that you will have to pay attention to them and actually work at being polite. In addition, you will want to master the
subtleties of verbal manners that you now unthinkingly control in English. Consider these examples. Do you say the same sweet-sounding
phrases to a mean-faced bureaucrat as you do to a pleasant cashier? Of
course not. Nor do you use the same words or tone with an elderly
person that you use with someone more your age. Getting a feel for
subtleties in Spanish requires getting a handle on the language that is
,used to express manners.

In Spanish, there is really no good translation for "polite."
\emph{Cortés, amable}, and \emph{pulido} all come close but are better translated by
their English cognates: "courteous,' "amiable" (or "nice"), and "polished." Instead, a Spanish speaker will talk about someone's \emph{educación}, which goes beyond a person's schooling to cover upbringing in
general and manners in particular. \emph{Es una persona educada} means that
so-and-so is a person who has good manners and is polite in dealing
with others. The person in question could be a grease monkey in the
neighborhood lube shop or a physics professor, a kindergarten dropout
or a triple Ph.D.; \emph{educado} simply means that the person has decent
manners. "Well bred,' though somewhat out of favor in modern egalitarian societies, conveys the right idea.

In passing, it's worth noting that \emph{rudo} is not the equivalent of
"rude," nor is it a good opposite for \emph{educado}. A Spanish-speaker would
probably use \emph{mal educado, sin educación}, or \emph{de poca educación}, or
would resort to \emph{grosero}. This last word, in context, refers to a foulmouthed individual, but in a more general sense it comes closer to
"rude." \emph{Se portó muy grosero con nosotros} = "He was very rude
to us."

Being polite, of course, is more than simply uttering elegant
phrases at key points in a conversation. To achieve the rank of \emph{educado} and skirt all that is associated with \emph{grosero}, you'll need one part
proper language and one part common sense. We've already addressed
the rudiments of good manners in our brief review of greetings and
good-byes (see Chapter 1). The phrases \emph{buenos días, buenas tardes},
and \emph{buenas noches}, in conjunction with \emph{gracias} and \emph{hasta luego},
will get you through 90 percent of your daily encounters. But that's
about all they'll do. If your goal is to go beyond the point of just getting by, you'll want a more in-depth look at the universe of Spanish
formalities.

\section{Meetings and greetings}

Politeness begins upon meeting a person. You meet 'em, you
greet 'em. The question is, how?

The answer depends on the person you're meeting. If there's
little or no chance of ever seeing the person again, you're safe and sufficient with the \emph{buenos días\ldots{} hasta luego} formula. If you think the
person may figure in your future life, or if the meeting is the result of
an introduction, you're expected to go beyond that. From a second encounter onward, except in the case of repeated encounters with employees (shop clerks, the doorman, waiters, the gardener), you should
usually employ a more personal greeting than just "good day" or "good
night."

Most students of Spanish have been drilled in the basic forms
of greeting, but they're worth a quick review. On being introduced to a
person, you have at your disposal a number of stock responses: \emph{mucho
gusto, tanto gusto}, and \emph{encantado} (or \emph{encantada}) in roughly descending order of frequency. For less formal introductions and situations
\emph{¿qué tal?} and \emph{hola} work well. Save \emph{muchísimo gusto} for someone
you've been dying to meet.

Once you've been introduced to a person, you'll naturally be
expected to greet this individual at all future encounters, be it on the
street or at a party. How you do this reflects (a) who the person is and
(b) who you are in relation to that person. Here are some of the common options, in more or less descending order of formality:

\bsk

1. \emph{¿Cómo está?} or \emph{¿Como está usted?}

2. \emph{¿Cómo le va?}

3. \emph{¿Qué tal?}

4. \emph{¿Cómo estamos?}

5. \emph{¿Cómo estás?}

6. \emph{¿Qué hay de nuevo?}

7. \emph{¿Qué pasó?} or \emph{¿Qué pasa?} (varies by country)

8. \emph{¿Qué me cuentas?} or \emph{¿Qué me dices?}

9. \emph{¿Qué onda?} (Mexico) or \emph{¿Quiúbole?} (mostly Mexico and
the Caribbean)

10. Slangy variations of the preceding, such as \emph{¿Qué pasotes?}
or \emph{¿Qué pasión?} (from \emph{¿Qué pasó?}), \emph{¿Qué hongos?} (from ¿Qué onda?),
and so on. Save these for the people whose street gang you're looking
to join.

\bsk

It bears noting that all these expressions---except \emph{¿Cómo está?}
and \emph{¿Cómo está usted?}---imply some level of friendliness. Said another way, if you're not on especially friendly terms with the person,
stick to the first expression listed above. It is the only appropriate
form for greeting a person whose social, familial, occupational, or political position warrants your respect, and thus is the safe choice for
those who aren't, strictly speaking, your buddies. \emph{¿Cómo estamos?} has
paternalistic overtones and is often used by older people to greet
younger ones---even if the younger ones are thirty or forty years old.
This greeting is also a safe one when you're on good terms with the
person but aren't sure whether to use tu or usted (more on that bugaboo in a bit).

A common way of sprucing up any greeting is to use the person's name, title, or both. The commonest titles are \emph{Don} and \emph{Doña}
(for older people) and professional titles like \emph{Doctor, Contador} (accountant or C.P.A.), \emph{Ingeniero, Profesor, Maestro} (any teacher or craftsperson and sometimes even mechanics and plumbers), and the ubiquitous \emph{Licenciado} (virtually anybody who wears a tie). These are used far
more often than their English equivalents, especially in the workplace.

As an example of how greetings work, let's take the case of
Juan Doe, assistant director in charge of flange production, arriving at
his office. For simplicity's sake, let's presume all of the males in his
workplace are named Alberto Alvarez and all the females Teresa Ruiz.
Juan parks his car on the street and walks toward the office building. In
order, he meets and greets the following:

\bsk

\inda The eighty-year-old doorman:

\indu \emph{Buenos días, Don Alberto.}

\inda The security guard:

\indu \emph{Buenos días.}

\inda The sixty-year-old elevator operator:

\indu \emph{¿Cómo le va, Doña Tere?}

\inda The receptionist:

\indu \emph{Hola, Tere. ¿Cómo estás?}

\inda A same-aged colleague in the hall:

\indu \emph{¿Qué tal, Alberto?}

\inda A younger colleague at her desk:

\indu \emph{Buenos días, Tere. ¿Qué hay de nuevo?}

\inda A visiting branch manager:

\indu \emph{Buenos días, Señora Ruiz. ¿Cómo le va?}

\inda A co-worker and best friend:

\indu \emph{¿Quiúbole, Beto?}

\inda The immediate boss:

\indu \emph{Hola, Alberto. ¿Cómo estás?}

\inda An older co-worker:

\indu \emph{¿Qué me cuenta, Don Alberto?}

\inda An employee:

\indu \emph{Buenos días, Alberto. ¿Cómo estamos?}

\inda The division director:

\indu \emph{Muy buenos días, Señor Alvarez. ¿Cómo le va?}

\inda The office boy:

\indu \emph{¿Qué pasó, Beto?}

\inda The factory owner and CEO:

\indu \emph{Buenos días, Don Alberto. ¿Cómo está usted?}

\inda The secretary:

\indu \emph{Buenos días, Tere. ¿Qué tal?}

\bsk

By now, as you might imagine, Juan is exhausted and it's time for his
coffee break.

A couple of general tips on greetings are in order. First, note
that when greeting so many people, you will naturally gravitate toward
new ways of saying the same thing. That's because saying \emph{buenos días}
to twenty-five consecutive people can be extremely boring.

Second, use nicknames only if the person is accustomed to being called that. In other words, pay attention to whether others call a
certain Jose "Pepe" before you call him that. Use generic nicknames---such as \emph{viejo, compadre}, and \emph{jovenazo}---only when you feel certain
that the person won't be offended by your informality.

Third, if you're a male, avoid affectionate pet names for female
friends, employees, and co-workers. In Latin America it is common to
hear men calling women co-workers and employees things like \emph{linda}
and \emph{cariño}. To most North Americans this treatment is patronizing at
best and at worst borders on sexual harassment. Men in Latin America
have been slow about concerning themselves with these matters, but
that's no reason for you to imitate them.

Fourth, if you're a woman, stick to more formal modes of address until you're sure that your friendliness won't be taken as encouragement by the wolfish male mind. It's unfortunate that you have to
consider this issue, but that doesn't make it any less real.

And fifth, greet everyone possible, especially when meeting
a group of people. If you've met the people before, you are expected
to take the trouble of greeting each of them individually. Not to do
so can be interpreted as an offense. The same goes for saying goodbye. If you're in too much of a hurry or there are simply too many
people involved, make sure you issue an all-encompassing \emph{¿Qué tal,
cómo están?} or \emph{Hasta luego}, and make sure it's interpreted as all-encompassing.

\section{\emph{Usted} versus \emph{tú}}

From the moment you greet a person, you will start to think
about whether so-and-so is an "\emph{usted} person" or a "\emph{tú} person." Native
Spanish speakers make this decision instinctively; you will have to
think it through, and repeatedly. It is a concept that doesn't have an
easy English equivalent, but it is usually not that hard to keep straight.
Perhaps the most functional system for converting the concept into
English is to use usted in Spanish with anyone you would address with
"Mr.," "Mrs.," "Ms.," or "Miss" in English. Mr. Brown is your neighbor, so you use \emph{usted} with him. Once you get to know him and call
him "Fred," you can switch to \emph{tú}. Your lawyer is Ms. Smith, so she's
an \emph{usted} person; if you call your lawyer "Betty," then you would also
probably use \emph{tú} with her. And so on.

This rule of thumb works even when you don't actually know
the person's name. Your waiter's name is Juan Perez, but you probably
wouldn't know that. If you did, though, would you call him "Juan" or
"Mr. Perez?" That will generally depend upon his age, your age, and so
on. If it would feel awkward calling a twenty-two-year-old "Mr. Perez,"
go ahead and use \emph{tú} with him. Then again, if the twenty-two-year-old
happens to be an undersecretary for tax policy in the finance ministry---or a traffic cop---you would probably call him "Mr. Perez" and
thus use \emph{usted}.

As a rule, people aren't bashful about telling you to use \emph{tú}
with them if they feel it's appropriate. Almost no one will tell you, to
use \emph{usted} when you're using tu---it's the equivalent of putting you in
your place. So when in doubt, you're far safer using \emph{usted} and waiting
until you're told to do otherwise.

The best way to choose the right form is to listen to the conversation around you. Let's say you're meeting a group of friends at a
restaurant. Upon arriving, someone you don't know is sitting with
your friends. You are introduced to this person as Betty (your name),
and she to you as Yolanda (her name). This is your first clue: almost
certainly you are expected to \emph{tutear} (use \emph{tú} with) this person. To play
it safe, though, you can return the greeting with a noncommittal \emph{mucho gusto} and keep your ears open. If Sam asks Yolanda, \emph{¿Dónde estás
trabajando ahora?} then you can feel safe using \emph{tú} yourself. Likewise, if
you're addressed with \emph{usted}, you should respond with \emph{usted}.

That said, there are a few cases in which the \emph{tú-usted} relationship is not reciprocal---that is, when you will use \emph{tú} with the person
and he or she will use \emph{usted} with you, or vice versa. Almost always
this is the result of a considerable age difference. You might use \emph{usted}
with your friends' parents, for instance, and they will likely use \emph{tú}
with you. (This, incidentally, fits under the "Mr. and Mrs." rule.) Turn
the formula around for your children's friends.

\section{Magic words}

Besides greeting people, expressing thanks is probably the
daily act that most requires a modicum of civility. \emph{Gracias} is the
obligatory comment, of course, but you can spice it up with a \emph{muchas}
before it or a \emph{muy amable} after it or both, as noted in Chapter 1. By
doing so, you'll sound both more polite and more fluent. \emph{Muy gentil} is
also used, but it sounds somewhat strained.

When being thanked, you can respond with \emph{por nada, de
nada, no hay por que}, or \emph{no hay de que}. They're all about the same
and mean "You're welcome." You will hear \emph{para servirle} a lot from
clerks and waiters, but don't use it yourself unless it is genuinely
your job or duty to serve the person, or unless you're feeling especially subservient.

Another common linguistic nicety is asking permission. In
Spanish, as in English, the most typical way of asking permission is
essentially to excuse oneself for having the audacity to ask. Thus we
ask a person's "permission" to squeeze by them in an aisle. Here are
some options for communicating your humility while asking someone
to move over or let you by, again in descending order of formality:

\bsk

1. \emph{Con permiso}

2. \emph{¿Me permite?}

3. \emph{Perdón}

4. \emph{¿Se puede?}

5. \emph{Comper'} (a slangy version of con permiso)

6. \emph{Hágase un poco para allá, por favor}

7. \emph{Abreme espacio} or \emph{Abreme cancha}

8. \emph{Hazte pa'llá}

\bsk

The first five expressions are formal enough for just about any occasion. \emph{¿Me permite?} is sometimes used ironically, as when someone is
clearly blocking the way; say it very innocently for greatest effect. \emph{¿Se
puede?} is also the common way to ask to see something in a store; it
presumes you will wait for an affirmative response before, say, taking a
painting down off the wall or an earring out of a glass case. Unless the
object you wish to see is obvious (you're pointing at it, for instance),
you should use the full phrase: \emph{¿Se puede ver?}

The last three expressions on the list convey informality or
rudeness, depending on the person you are speaking to and your
tone of voice. \emph{Hazte pa'llá}, for instance, can be used for either "Scoot
over a little" (with a friend) or "Get out of the way" (With a stranger).
\emph{Abreme cancha} is very slangy.

The phrase for "Coming through!"---as when carrying a two-hundred-pound sofa down the hall---is \emph{¡Golpe avisa! Excúsame}, incidentally, does exist as a Spanish expression, but it doesn't mean "Excuse me." Use something (anything!) else.

Here's a cultural tip. One nicety that many foreigners
have trouble learning is to say gracias when someone sends \emph{saludos}
through them to another person: \emph{Salúdame a tu esposa} ("My regards to your wife") is something you will hear constantly if you are
married, for instance. Your English-speaking reflex will be to answer
"Okay, I will," but in Spanish it is customary to thank the person for
this \emph{detalle} ("thoughtfulness" or "consideration").

As \emph{gracias} is the universal word for "thank you," \emph{por favor} is
all you will ever need for "please." Always remember to use it when
asking for something. If you're tired of it and want to flex some Spanish muscles, use instead \emph{si es tan amable}, generally placed after your
request: \emph{Un café, si es tan amable}. It means \emph{por favor}.

\section{Sugar versus saccharine}

Spanish speakers are famous for going a bit overboard with
their floridness and politeness, and it is a matter of opinion whether
foreigners learning the language should leap in after them. Use your
own judgment. For instance, Spanish speakers will often refer to their
house as \emph{su casa} or \emph{tu casa}---"your house"---meaning that now that
you know them, you should consider their abode to be yours. This can
get pretty confusing at times: someone will be giving you directions to
their home and at the end they will point to the map they've improvised and say, "And here is your house." You'll be tempted to respond,
"But my house is nowhere near there!"---until you recall this subtle
gesture of hospitality. It can get even more confusing when a nonnative speaker, from whom such a gesture is generally not expected,
tries to communicate it.

Slightly less silly-sounding in the mouths of foreigners is the
statement \emph{Está usted en su casa} when someone comes to visit. It's really nothing more than a way of saying "Make yourself at home" and
shouldn't be made to sound more grandiose than that. To use it really
correctly, save it for when a houseguest makes a simple request, such
as "Can I use the phone?" To this you reply, with a wave of the hand,
\emph{Estás en tu casa}.

A word on homes is in order at this point. In much of the
Spanish-speaking world, homes are considered private reserves, and it
is not especially common to receive an invitation to visit someone's
house. So first of all, don't expect to receive such an invitation. And
don't be surprised if your offers of hospitality---"Hey, Pedro, how about
popping by after work for a beer?"---are taken as a nice gesture instead
of as a genuine invitation. Furthermore, never drop in unannounced on
a friend in the Spanish-speaking world. Finally, you may be told while
visiting someone's house that some object you admire \emph{es tuya} ("it's
yours," "take it"). Not only shouldn't you accept such offers, you
should be careful about making them. Sooner or later a literal-minded
guest might take you up on it!

Other sweet-sounding phrases that border on the sickly sweet
include solemn declarations of humility (saying \emph{su servidor} instead of
"I"), exaggerated requests for cooperation (\emph{Tenga usted la bondad de
traerme un café}), and overly formal greetings (\emph{Me es grato tener la
oportunidad de conocerlo}). All sound like you're reading out of a
phrasebook---and a phrasebook written for royal weddings, at that.

\section{Asking and getting}

Politeness is particularly important when asking someone for
something, since otherwise you may not get it. Bicultural lore is full of
anecdotes about Party A getting something in twenty minutes that
took Party B three weeks to get, simply because Party A asked politely
and Party B was viewed as rude. Whether you plan to use your Spanish
to get government authorizations or good directions, your command of
these niceties is critical.

It's easy to sound rude or clumsy when making a special or
even routine request in Spanish, especially since most students of the
language are taught the imperative as the sole way of asking for things:
Thus many students of Spanish will tell their hostess, \emph{Tráigame un café} ("Bring me a coffee"), thinking that's the correct, formal, and polite way to petition one. A good hostess will bring you one anyhow,
but on some interior level she's thinking, "Sure, here's one in your lap,
schmuck." Fortunately, it's quite easy to sound natural when asking
for something, but most Spanish texts neglect to mention the most frequently used form of the imperative in daily life: the indicative. That
is, most people don't say \emph{Tráigame un café, por favor} but \emph{¿Me trae un
café, por favor?} An added advantage to this form is that you don't have
to worry about those strange imperative forms and can stick to the
tried-and-true indicative instead.

The \emph{¿me trae? + por favor} formula is all you'll ever need for
restaurants and the like. But this use of the indicative also works for
most other situations and can employ a number of different verbs in
addition to \emph{traer}, especially \emph{permitir, dar, prestar}, and \emph{regalar}. \emph{Permitir} is the most formal: \emph{¿Me permite un cigarro?} will get a cigarette off
your future father-in-law. \emph{Prestar} and \emph{regalar} are the least formal, and
the latter implies you're going to keep what you're given: \emph{¿Me prestas
tu pluma?} means "Lend me your pen." \emph{¿Me regalas tu pluma?} is "Can
I have your pen (forever)?" In colloquial use, \emph{pasar} is common and
equates with \emph{prestar}: \emph{¿Me pasas el cenicero?} means "Pass the ashtray
over here, would ya?" Another common formal way of asking for
things is equivalent to "May I" in English: \emph{¿Puedo tomar un cigarro?}
("May I have a cigarette?").

\section{Etc.}

Phone Spanish is generally even more polite than face-to-face
Spanish. Listening to it, you will hear a lot of phrases like \emph{si es tan
amable} and \emph{si no es mucha molestia} tacked on to simple requests.
Once you get the hang of it, you'll be doing the same thing. To ask for
someone on the phone, you can be formal---\emph{¿Me puede comunicar can
el Sr. So-and-so?}---or informal---\emph{¿Está por ahí So-and-so?}. It will depend on whom you're calling. To say "Speaking" when someone calls
and asks for you, just say \emph{Él} (or \emph{ella}) \emph{habla}.

Interrupting is considered bad form in any language, of course,
but some foreigners seem to do it more when the conversation they are
interrupting is in a foreign language---i.e., one they don't understand
that well. Be aware of this, and be conspicuously polite when you do
need to interrupt, directing yourself to the party whom you are momentarily cutting out of the picture. Act, in other words, as if you were
cutting in on a dance. \emph{¿Me permite un momento?} you can ask.

Certain situations call for specific graces from you. Some are
verbal graces and some aren't, but all come under the heading of \emph{buena educación}.

\bsk

1. When you pass by someone who is eating, and presuming
you are at least vaguely acquainted with the person, wish him or her
\emph{provecho} or \emph{buen provecho} ("bon appetit," as we'd say in English).
Don't say it to total strangers in restaurants, though.

2. North Americans like to toss things. "Here's the pencil you
asked for," they'll say as they wing a fine-pointed no. 2 lead pencil by
your ear. "This is no good," they'll say as they crumple a sheet of paper
and lob it at the wastebasket. In the Spanish-speaking world, this behavior is considered barely short of barbaric.

3. Ditto for pointing at people.

4. Be conscious of not "giving your back" to people. Many
people from non-Latin cultures do it without intending offense. But in
Spanish-speaking cultures it's common to see people realize there is
someone behind them listening, do a half-turn, say \emph{perdón}, and continue speaking in a way that includes the previously excluded person.

\section{Sweet sorrow}

Departures are easily handled in Spanish by any of a number
of words, but \emph{adiós} and \emph{hasta luego} are sufficient for almost every circumstance. \emph{Adiós}, as you may have been taught, is generally used for
more lasting farewells. \emph{Nos vemos} is also a common colloquial send-off, equating with "See you later." \emph{Ciao} (or \emph{chao}) and \emph{bye} are making
rapid headway into Spanish from their respective languages.

Slightly more formal leave-taking makes use of phrases like
\emph{Que le vaya bien}, used only when the person you say it to is doing the
leaving. More fancy is \emph{Vaya con Dios}, but unless you're a nun or a
priest, it comes close to the saccharine category. When taking leave of
a group or passing by people on your way out, it's considered nice
manners to say \emph{Con permiso} to the people who are staying on. Usually it is
used for people whom you didn't actually say hello to on your way in.
Leaving the dentist's office, for instance, you say \emph{Gracias} to the dentist
and the receptionist and \emph{Con permiso} to the people sitting in the waiting room.

For less formal departures and with the younger set, you can
say things like \emph{Cuídate} (roughly, "Take it easy") and \emph{Pórtate bien} ("Behave yourself") to people when they are leaving. If it's nighttime and
the person is presumably leaving to go home to sleep, you can say \emph{Que
descanses} ("Rest up"). As a rule, try to respond with a farewell that is
different from the one used by the other person. If someone says \emph{Hasta
luego}, you say \emph{Nos vemos}; if they say \emph{Nos vemos}, answer \emph{Hasta luego}.

\chapter{Tricksters}

Every non-native speaker of a foreign language is afraid of
making mistakes. And good thing, too. A little fear forces us to concentrate harder and causes the memory of our mistakes to linger-long
enough to correct them the next time we open our mouths.

Most mistakes are grammatical and are eliminated only after
long periods of trial and error. Other mistakes, though, are almost the
fault of the languages themselves. Spanish and English, because of
their long history of coexistence and their many common origins, contain a lot of words that are similar on the surface but are used quite
differently. This makes instinctive translation at times as dangerous as
a cross-eyed knife-thrower.

Fortunately things aren't quite that bad. A lot of these false
cognates are so frequently tripped over by students of both languages
that they can be identified in advance. In the following list of "tricksters," you'll find some of the Spanish words most commonly misused
by native English speakers. Review the words and keep them in the
back of your brain. You won't overcome all of your instincts overnight,
but knowing your enemies-in this case, the tricksters-is the first
step toward conquering them.

\section{\emph{acostar}}

%ACOSTAR
Not "to accost," which is acosar. Acostar simply
means "to lie down" and is usually used in the reflexive. \emph{Me voy a
acostar} = "I'm going to lie down" or "I'm going to bed." If by doing
this you are accosting someone, you may be in the wrong bed.

\section{\emph{actual}}

%ACTUAL
Not "actual" but "current" or "present." \emph{La actual
administración} means "the current administration" or the one in power
at the moment. To convey "actual" in the sense of "factual" or "genuine," you would resort to \emph{verdadero, autentico, genuino, real}, etc. "This
is an actual Picasso" is expressed by \emph{Este es un auténtico Picasso}.

\section{\emph{actualmente}}

%ACTUALMENTE
Worth a special mention because "actually"
is so common as a sentence-starter in English: "You must be starving."
"Actually, I just ate." \emph{Actualmente} won't work here, since it means
"at present" or "currently." Confusing the two can lead to some
strange situations. Think of someone telling you, "You and your
brother are invited for lunch"---to which you want to answer, "Actually, he's my husband." If you use \emph{Actualmente, es mi esposo}, what
you are saying is "At present, he's my husband." For "actually," try the
phrase \emph{la verdad es que}: \emph{La verdad es que acabo de comer}. When "actually" is used in mid-sentence for emphasis, you can use either
\emph{realmente} or some other construction altogether. For example, "He actually ate it!" can be expressed as \emph{¡Realmente se lo comió!} or \emph{¡Sí se lo
comió!} or \emph{¡De verdad se lo comió!}

\section{\emph{afección}}

%AFECCIÓN
This doesn't mean "affection" but usually a
medical condition. \emph{Tiene una afección cardiaca} = "He (or she) has a
heart condition." Use \emph{afecto}---or better yet, \emph{cariño}---to convey your
affection.

\section{\emph{americano}}

%AMERICANO
Almost more of a political issue than a semantic one. Still, the use (and misuse) of \emph{americano} by "Americans"---i.e.,
U.S. citizens---should be kept in mind by anyone eager to avoid unnecessary offense. In Spanish and Spanish-speaking countries an \emph{americano} is simply a person from "the Americas," including South and
Central as well as North America. Thus if you tell a Chilean that you
are an \emph{americano}, you might get a smirk and the comment \emph{Yo también} ("Me too") in return. Incidentally, in the Spanish-speaking world
the correct way to refer to "the Americas" is in the singular: \emph{América}.
Schoolchildren are taught that, from the Bering Strait to Tierra del
Fuego, it is but one continent.

Oddly enough, there really is no perfect way for U.S. citizens
to state their origin in Spanish. \emph{Norteamericano} is the most common
word, but technically it includes Canadians and Mexicans as well. \emph{Estadounidense} has caught on of late, but besides being a mouthful it
overlooks the fact that other countries (the United Mexican States, for
instance) are also technically "United States." Even so, saying \emph{Soy de
los Estados Unidos} is probably the easiest way of explaining your
plight.

This confusion, incidentally, goes a long way toward explaining why \emph{gringo} is so common a word in countries like Mexico and why
you yourself will probably start to use it after a short time in these
places. The word can be used in an offensive way, but usually the
word's negative connotation is a result of the tone of voice. In Mexico,
especially, \emph{gringo} is very commonly used as an adjective as well; imports from the United States can easily (if colloquially) be referred to
as \emph{productos gringos}.

\section{\emph{argumento}}

%ARGUMENTO
A classic trickster. This word doesn't work
well as "argument" in the usual sense of a "heated discussion" or a
"quarrel." Instead, it should be reserved for prepared arguments (such
as lawyers' summations), debates, and carefully reasoned, often written
arguments. \emph{Argumento} describes a logical process, not a rather illogical throwing of pans and vases. For that, use \emph{pleito, disputa}, or \emph{disgusto} (see below), roughly in descending order of intensity. Perhaps the
best all-purpose translation of "argument" is another trickster, \emph{discusión}, which refers to a far more heated exchange than what we consider a "discussion" in English.

\section{\emph{asistir}}

%ASISTIR
Not "to assist" but "to attend" or "to be present at."
\emph{Asistencia} is the proper word for "attendance" (see \emph{atender} and \emph{audiencia} below), and an \emph{asistente} is technically just someone who is attending something. Nonetheless, you may come across \emph{gerente asistente} as a way of saying "assistant manager," though you'll probably
come across it at a local branch of a US. company. A more natural
Spanish construction would be \emph{subgerente}. For "to assist," use \emph{ayudar}.

\section{\emph{atender}}

%ATENDER
Not "to attend" (see \emph{asistir} above) but "to attend
to"---a significant difference. Remember, you can \emph{atender} a patient but you can't \emph{atender} a concert.

\section{\emph{audiencia}}

%AUDIENCIA
Another word in the "attendance" versus "assistance" imbroglio. Correctly, an \emph{audiencia} is usually a private interview granted to you by someone more important than you. In this
sense, it is usually used with \emph{conceder}: \emph{El ministro nos concedió audiencia} = "The minister granted us an interview." This meaning is
still in use in English, but it is usually reserved for the pope. In Spanish the town sewage commissioner can grant you one. \emph{La audiencia}
can be used for "the audience"---in a concert hall, for instance---but \emph{la
asistencia, el auditorio}, and \emph{el público} are all preferred.

\section{\emph{balde}}

%BALDE
This doesn't refer to hair loss but to a bucket. It's also
commonly seen in the expression \emph{en balde}, which means "in vain."
For "bald," you can use either \emph{calvo} (polite) or \emph{pelón} (irreverent).

\section{\emph{balón}}

%BALÓN
The common word for "ball," from about grapefruit
size on up. Smaller balls are called \emph{bolas} or \emph{pelotas}. "Balloon" is
\emph{globo}, just so you know.

\section{\emph{billón}}

%BILLÓN
A false cognate that many overlook. \emph{Un billón} is
1,000,000,000,000, or 10 to the twelfth power ($10^{12}$). It is equal to the
U.S. "trillion." (In England, this quantity is a "billion.") To convey the
US. "billion," or 10 to the ninth power (UK. "milliard"), you must say
\emph{mil millones}, or "a thousand millions."

\section{\emph{bizarro}}

%BIZARRO
A fairly archaic term for "chivalrous" or "brave"
(not "bizarre" in the sense of "peculiar"). Both "bizarre" and \emph{bizarro}
come from the Basque word for "bearded," which apparently suggested
strangeness to some and gallantry to others. You don't need to know
\emph{bizarro} to speak Spanish, but you should be aware that it doesn't mean
"bizarre" if you tend to translate your thoughts fairly literally from English to Spanish. For "bizarre," use \emph{raro} or \emph{extraño}.

\section{\emph{"capable"}}

%"CAPABLE"
This word doesn't even exist in Spanish, but if
you were to use it, it would be understood as "castratable." Saying
someone is \emph{un hombre capable}, therefore, does not mean he is "capable" but rather in imminent danger of being rendered "incapable."
The right word for "capable" is \emph{capaz}; "incapable" is \emph{incapaz}, which
covers "incompetent" as well.

\section{\emph{cargo}}

%CARGO
Almost works but not quite. \emph{Carga}, with the feminine ending, is the right word for "cargo." \emph{Cargo} in Spanish generally
means "job position" or "post." \emph{El encargado} is "the guy in charge."

\section{\emph{carpeta}}

%CARPETA
Usually a "portfolio" of the sort for keeping and
carrying papers. It could also be a simple manila folder or a file. It is
never a "carpet," which is covered by \emph{tapete} or \emph{alfombra}. Nonetheless,
thanks to the influence of English, it is said (perhaps apocryphally) that
Spanish-speaking residents of the United States say things like \emph{Voy a
vacunar la carpeta} for "I'm going to vacuum the carpet." In fact, this
statement means "I'm going to vaccinate the portfolio." (See Appendix
B for more English-influenced words and phrases.)

\section{\emph{chocar}}

%CHOCAR
A tricky trickster. This word has long existed in
Spanish to mean "to crash," as in what happens to a car that is driven
recklessly. A few hundred years ago, though, the English word "shock"
began to take on some trendy new scientific meanings, and a handful
of these were assigned to the old standby \emph{chocar} and its derivatives.
Thus while un choque has always meant "a crash," in recent years it
has been expanded to cover "a state of shock" (such as the driver's condition after the car's choque). It is also in widespread use for a powerful electrical shock, though \emph{chocar} as a verb is not used for "to shock
(electrically)." For that, use \emph{dar choque} or, in many countries, \emph{dar toques}. (Generally, \emph{un choque} will kill you and \emph{un toque} won't.) \emph{La lámpara me dio toques} = "The lamp gave me a shock." To confuse matters more, sometimes shock is used instead of \emph{choque}.

\section{\emph{chocante}}

%CHOCANTE
As an adjective, this works pretty well as
"shocking" but usually conveys a distinct disapproval that isn't implicit in the English word. In some contexts, especially in reference to
people, it comes close to meaning "offensive" or "rude." \emph{Me choca}, by
extension, is a common colloquial way of saying you strongly dislike
something: \emph{Me choca el chocolate} = "I hate chocolate."

\section{\emph{complexión}}

%COMPLEXIÓN
Not your skin texture, as in English, but your
general physical structure and shape, or "build"---in other words, fat,
skinny, pear-shaped, or just about right. Government forms in Spanish-speaking countries often ask about your \emph{complexión}, and as long as
you remember not to put "cleared up years ago," you'll be fine. For
skin condition, stick to \emph{piel} (all skin) or \emph{cutis} (especially facial skin).
"You have a nice complexion" is expressed by \emph{Tienes buen cutis}.

\section{\emph{compromiso}}

%COMPROMISO
Yes, it works as "compromise," but a far more
common usage is to mean "commitment." The verb \emph{comprometer} is
similarly double-edged. \emph{Me comprometo con las mujeres} could be a
politician's way of saying he or she is committed to his or her female
constituents and connotes no "compromising" situations. The same
sense of obligation is present in the common marketplace remark
\emph{sin compromiso}, which means you can try on a blouse, for instance,
"without committing" yourself to buy it. \emph{Compromiso} is also an
"appointment" or "engagement." \emph{Tengo un compromiso después de la
comida} = "I have an appointment after lunch." "Commitment" can.
also be used this way, but it sounds a bit like power-breakfast English,
whereas in Spanish it is an everyday expression.

\section{\emph{copa}}

%COPA
If you ask for your "cup of coffee" as a \emph{copa de café} in
the Spanish-speaking world, don't be surprised if you're served your
coffee "Irish" or with a coffee liqueur. In general usage, a \emph{copa} is a
stemmed glass or goblet of the sort used for wine or champagne. Thus
\emph{copa}, in a restaurant setting, almost always suggests an alcoholic beverage of some sort. Just as you wouldn't think of asking for coffee in a
goblet, your waiter won't think of serving something in a \emph{copa} without
a little booze in it. \emph{Taza} is the correct word for a coffee-style "cup."

\section{\emph{corriente}}

%CORRIENTE
This word is only partly tricky, but when it's
tricky, it's dangerously so. Basically you can use it safely as a noun to
mean "current"---any sort of electrical, river, or political currents---but not as an adjective. As a modifier, \emph{corriente} suggests "cheap" or
"trashy"; when used for people, it is a definite insult, equating with
"rude" or "vulgar." To convey "current" in the sense of "now in progress," you should take care to use \emph{presente} or \emph{actual} (see above).

\section{\emph{decepción}}

%DECEPCIÓN
A classic trickster. \emph{Decepción} means "disappointment" or "disillusionment," often with no suggestion whatsoever
of deceit. Likewise, \emph{decepcionar} means "to disappoint," and \emph{decepcionado} means "disappointed." \emph{Me decepcionó su novio} means you
were unimpressed by someone's boyfriend, not that he talked you out
of your inheritance. For "to deceive" and its derivatives you're better
off with \emph{engañar}. \emph{Defraudar} can work either way: "to defraud (monetarily)" or "to disillusion," "to let (someone) down."

\section{\emph{disgustar}}

%DISGUSTAR
Not "to disgust" but "to cause displeasure"---a
subtle but important difference. Turned around, with the speaker as
the indirect object (as with gustar), it means simply "to dislike." \emph{Me
disgustan los pepinos} means you don't like cucumbers, not necessarily
that they make you sick. For "to disgust," \emph{asquear} or \emph{dar asco} is appropriate: \emph{Los pepinos me dan asco}. Similarly, "disgusting" would be
\emph{asqueroso}. Used to describe a person, it conveys the idea of extreme
sleaziness. \emph{Es un tipo asqueroso} = "He's a real slimeball."

\section{\emph{"dum dum da dum dum, tump tump"}}

%"DUM DUM DA DUM DUM, TUMP TUMP"
A non-verbal trickster of the most dangerous sort. It's hard to convey the sound pattern
in writing (one rendering, perhaps from vaudeville days, is "shave and
a haircut---two bits"), but the pattern is familiar to everyone. Imagine
yourself knocking it on a door or tapping it out on your car horn. Got
it? Now consider that in certain Latin countries, Mexico especially,
what you've just said is, essentially, "Fuck your mother." Knock that
on a door in Mexico and be prepared to see someone with a shotgun
open it; tap it on your car horn and you'll have twenty vehicles gunning for you. Tap it on your car horn when there's a police car in front
of you and you've got serious problems.

\section{\emph{embarazado}}

%EMBARAZADO
The most famous trickster of all, leading to
all sorts of colorful anecdotes from travelers. The word actually can
mean "embarrassed" in certain contexts and in certain expressions,
but it also means "pregnant." By using it, even correctly, you open
yourself to no end of silliness and smirks. Better to stick to the common ways of saying "embarrassed": \emph{apenado} and \emph{penoso}, both from
the word \emph{pena} (see below). \emph{Dar pena} is good for "to embarrass." If you
don't want to speak in public because it embarrasses you, bow out with
a \emph{Me da pena}. For "How embarrassing!" (as in "What a fool I've made
of myself!"), a simple \emph{¡Que pena!} will suffice. A stronger concept like
"shame," often with moralistic overtones, is covered by \emph{vergüenza}.
\emph{Pena} is more the embarrassment that comes of shyness or prudishness.

\section{\emph{en absoluto}}

%EN ABSOLUTO
Not "absolutely" but the opposite---"absolutely not." If you want to avoid using it altogether, that's fine, too.
Use \emph{claro} for "absolutely" and \emph{claro que no} for "absolutely not." But
learn the correct meaning of \emph{en absoluto} for those times when it's
used on you.

\section{\emph{enfrente de}}

%ENFRENTE DE
A sneaky trickster and one that can cause
crossed signals with the best of them. It means "in front of" in the
sense of "across the way (or street) from" or "facing." Thus, if you tell
your friends to pick you up \emph{enfrente del cine} and you're waiting in
front of the movie theater, expect to see your friends waiting across
the street. \emph{Frente a} is equally misleading, meaning the same as \emph{enfrente de}. For "in front of" as we use it in English, try \emph{en la puerta de}
("at the door of") to avoid any misinterpretations. \emph{Al frente de} can also
be used, but why risk the confusion? (See also Chapter 11 under
"Front.")

\section{\emph{excitado}}

%EXCITADO
An easy one to slip in unawares, this trickster
probably conveys more than you bargained for. The English translation
is "aroused," sexual overtones included. For "excited," use \emph{emocionado}; for "exciting," \emph{emocionante}.

\section{\emph{éxito}}

%ÉXITO
Not "exit" but "success." In the music industry \emph{éxitos}
are "hits," lest you think \emph{Los grandes éxitos de Frank Sinatra} refers to
his greatest stage exits. "The exit" is \emph{la salida}.

\section{\emph{fábrica}}

%FÁBRICA
Not "fabric" but "factory." This is a good one to remember if you're going to get involved in business in Latin America, as
a lot of people are these days. If you're looking to buy fabric, on the
other hand, the word you want is \emph{tela}.

\section{\emph{gentil}}

%GENTIL
Not really "gentle" but "kind" or "courteous." \emph{Que
gentil} is a somewhat stuffy (and often ironic) way of saying "How
nice" or "That's great." To say "gentle," you'll want to use \emph{cuidadoso}
or even simply \emph{cuidado}: \emph{Mucho cuidado por favor con esa caja} =
"Please be gentle with that box." A "gentle" wind might be \emph{suave},
while a "gentle" person would be \emph{tierna}.

\section{\emph{informal}}

%INFORMAL
This word has come to have pretty much the
same meaning as its English cognate, with one important exception:
when used to refer to people, informal means "unreliable." It's the
word you use to refer to the plumber who swore that he'd be by last
Thursday to unstop your drains, only to vanish from the face of the
planet instead. To say that someone is "informal"---i.e., "laid-back"---you could use \emph{relajado} ("relaxed") or \emph{despreocupado}. For uses other
than personal ones, \emph{informal} is widespread, though purists tend to dislike it. \emph{Fue una reunión informal} = "It was an informal get-together."

\section{\emph{injuria}}

%INJURIA
An "injury," yes, but a moral and psychological
one---better known in English as an "insult" or "offense," and usually
a real dinger of one at that. Don't use it for "injury," which, if major, is
the noun \emph{herida}. If it's a small injury (a sprained ankle, for example),
the verb \emph{lastimarse} is more appropriate. Why is Pablo limping? \emph{Es que
se lastimó el pie jugando tenis} ("He hurt his foot playing tennis").

\section{\emph{intoxicado}}

%INTOXICADO
Unless it's a real binge you're talking about,
this word doesn't work for "drunk." It means "poisoned"---unintentionally, as a rule---and covers food poisoning, industrial toxins, overdoses, and the like. Another whole book could be written on how to
say "intoxicated" in the sense of "drunk" in Spanish. \emph{Ebrio} is the
best equivalent for "intoxicated," while \emph{borracho} is closest in tone to "drunk."

\section{\emph{introducir}}

%INTRODUCIR
You'll be tempted to use this for "to introduce,"
but don't. It means "to introduce" only in the sense of "to insert" or
"to add something in." Often it is used as a fairly fancy substitute for
\emph{meter} or "to stick in." For "to introduce to" with people, always use
\emph{presentar}. \emph{Ven, quiero presentarte a un amigo} = "Come on, I want to
introduce you to a friend of mine."

\section{\emph{largo}}

%LARGO
Not "large" but "long." Just a reminder.

\section{\emph{librería}}

%LIBRERÍA
Not "library" but "bookstore." Another reminder.
A "library" is a \emph{biblioteca}.

\section{\emph{media}}

%MEDIA
In Spanish, this (presumably) has nothing to do with
your favorite newscaster. It means "stocking," and in the plural,
"pantyhose." In math it means "mean." "Media," in the collective
sense of newspapers, magazines, radio, and television, is usually covered in Spanish by \emph{los medios} (de \emph{comunicación}).

\section{\emph{molestar}}

%MOLESTAR
In Spanish this word carries no sexual overtones
whatsoever. It's perfectly safe and extremely common for "to bother,"
\emph{No me molestes} = "Don't bother me." As an adjective, it works well
as "upset," "angry," or "uncomfortable." \emph{¿Estás molesto por algo?} =
"Are you upset about something?"

\section{\emph{ordinario}}

%ORDINARIO
Be careful using this term in regard to people.
Far from meaning "ordinary," it means "vulgar," "rude," and "crass."
\emph{Es un tipo muy ordinario} = "He's a slob." To describe an average, run-of-the-mill person just like you and me and a million other people, use
\emph{normal} or \emph{común:} \emph{Es un tipo normal}. \emph{El hombre común} is a good
translation of "the man in the street." (See also Chapter 4.)

\section{\emph{parientes}}

%PARIENTES
Not "parents," though they're \emph{parientes} too, but
"relatives"---all of them, from your grandparents and children to in-laws and cousins. What you share with these people is called \emph{parentesco}, or "kinship."

\section{\emph{pena}}

%PENA
Not "pain" (the physical kind) but "sorrow" and "embarrassment." \emph{Pena} is the word you want to learn so as to avoid using
\emph{embarazado} (see above). "Pain" is usually handled by \emph{dolor}, as is
"ache." A "headache" is a \emph{dolor de cabeza}. In the figurative sense of
a "pain in the neck" (or even lower), a good equivalent is \emph{lata}. \emph{¡Qué
lata!} = "What a pain!" \emph{Dar (la) lata} = "to be a pain," "to pester." To
people who are bothering you, you can say \emph{No des (la) lata} ("Stop being a pain").

\section{\emph{primer piso}}

%PRIMER PISO
If you're used to thinking of the ground floor as
the first floor, be prepared to think again in Spanish, where \emph{primer piso}
means "one flight up." The ground floor is usually called \emph{1a planta
baja}, or PB in the elevators.

\section{\emph{quitar}}

%QUITAR
You may try to use this for "to quit," especially
since Spanish at first glance doesn't seem to have a good word for "to
quit." Unfortunately, quitar isn't that word either. It means "to take
(something) off," "to take away." The word you need for "quit" depends on what you're quitting: if it's your job, the word is \emph{renunciar}, if
it's smoking or some other activity, \emph{dejar de}. To "quit" a computer
program, most translated software programs simply use \emph{salir}.

\section{\emph{realizar}}

%REALIZAR
There is some overlap with this word and its English cognate. But for the most common use of "to realize" in English,
\emph{realizar} does not work. The correct phrase is \emph{darse cuenta (de)}. "I realized too late he had a gun" is expressed by \emph{Me di cuenta demasiado
tarde de que traía pistola}. "Realize" in the common English sense can
often be handled perfectly competently by \emph{saber} ("to know") in Spanish. "I realize you're busy" thus becomes \emph{Sé que estás ocupada}. "I
didn't realize you were married" would be \emph{No sabía que estabas
casado}.

\section{\emph{receta}}

%RECETA
Almost always a medical prescription or a cooking
recipe in Spanish. It is never a "receipt," which would be \emph{recibo} or
\emph{nota}.

\section{\emph{ropa}}

%ROPA
Not "rope" but "clothes." "Rope" is usually either
\emph{soga} or \emph{cuerda}.

\section{\emph{sano}}

%SANO
This goes beyond mental health to cover all aspects of
health. In other words, it means "healthy." The word for "sane" is
\emph{cuerdo}. For "insane," use \emph{loco}. It's a bit unscientific and insensitive, but then so is "insane."

\section{\emph{sensible}}

%SENSIBLE
It means "sensitive," not "sensible." For "sensible" you should use \emph{sensato}. \emph{Una persona sensible} is "a sensitive
person." This is one of those confusing words you'll just have to
remember.

\section{\emph{sopa}}

%SOPA
Not "soap" but "soup." Creamy soups are simply
called \emph{cremas}, as in \emph{crema de champiñones} ("cream of mushroom
soup"). "Soap" is \emph{jabón}.

\section{\emph{soportar}}

%SOPORTAR
A common word in Spanish for "to tolerate,"
though it is more frequent in the negative in the sense of "can't stand."
\emph{No soporto la televisión} = "I can't stand television." Mixing it up will
get you some funny looks: \emph{Mi familia me soporta} doesn't mean your
family pays your bills but that your family tolerates you (barely). For
"to support" in the bill-paying sense use \emph{mantener}. In the sense of
physically supporting something (what a wall does for the ceiling, in
other words), \emph{sostener} is more accurate.

\section{\emph{sumar}}

%SUMAR
This verb may pop into your mind as a neat translation for "to sum up," but you should pop it right back out of there if
you want to be understood. \emph{Sumar} is the word for "to add" or "to add
up." \emph{¿Cuánto suma?} = "How much does that add up to?" For summaries and summations, stick to the verb \emph{resumir} (\emph{para resumir} = "to
sum up," \emph{en resumen} = "in sum"), although \emph{en suma} is safe for "in sum" as well.

\section{\emph{suplir}}

%SUPLIR
Not "supply," as you might think. Instead, it means
"to substitute" or "to fill in for." \emph{Suplo en la oficina a mi hermana
cuando está de viaje} = "I fill in for my sister at the office when she's
out of town." A \emph{suplente} is a "substitute"---a substitute teacher, for
instance. \emph{Substituto} and \emph{substituir} also exist and mean about the
same thing, but \emph{suplir} and its derivatives are more frequent (and far
easier to pronounce).

\section{\emph{tremendo}}

%TREMENDO
Often a close fit for "tremendous" but not always. Tremendo often comes closer to "outrageous" and can mean
"outrageously bad," "terrifying," or "terrible" as well as "outrageously
good" or "tremendous." The word is especially tricky around children,
it seems. \emph{Es un niño tremendo} describes a monstrous child capable of
the worst mischief. In short, be aware of the negative connotations
that frequently surround this word.

\section{\emph{tuna}}

%TUNA
Here's a real menu trickster. Tuna is not the fish but
the fruit of the prickly pear cactus, or \emph{nopal}---that is, the "prickly
pear" itself. "Tuna" is \emph{atún}. Both are perfectly edible, but if you have
your heart set on a tunafish sandwich, a serving of prickly pears may
not quite fill the bill.

\section{\emph{últimamente}}

%ÚLTIMAMENTE
In correct usage not "ultimately" but "recently." \emph{Últimamente ha estado enferma} = "She's been sick of late."
For "ultimately," use \emph{al final} or \emph{a fin de cuentas}. \emph{A fin de cuentas, es
su decisión} = "Ultimately it is his decision."

\section{\emph{vacunar}}

%VACUNAR
See carpeta above---and fast!

\section{\emph{vago}}

%VAGO
This word means "vague" when applied to things, but
it means "bum" or "tramp" when used for people. "Vagabond" is a
good cognate for this usage. To express "vague," use \emph{vago}, carefully,
and use \emph{impreciso} when referring to people or when there's any chance
of being misinterpreted.

\section{\emph{voluble}}

%VOLUBLE
Those of you who studied your vocabulary lessons
in high school will remember that this means "talkative." In Spanish,
though, it means "unreliable," "flighty," or "fickle" and is a good deal
more common than its English cognate.

\section{\emph{zorra}}

%ZORRA
"Foxy lady," even before Jimi Hendrix's time, has
been a slangy way to refer to an attractive female. And "a fox," by itself, means much the same thing for either men or women. Slang is a
dangerous thing to translate literally, however, and there is no better
example of this than the Spanish word \emph{zorra}, or "fox" (i.e., a canine
of the genus Vulpes). When applied figuratively to human beings, it
means "shrew," "slut," or "prostitute." What better way to learn your
tricksters than to get a slap in the face? If you must use corny come-ons, stick to \emph{guapo} and \emph{guapa}.

\chapter{Our Fellow Human Beings}

As we go about our daily lives, we are often put to the task of
describing our fellow human beings. And the vast range of personalities these humans represent requires an equally vast vocabulary of descriptive terms. Rare is it indeed to encounter a person whose behavior
can be wrapped up in so simple a concept as "good" or "evil." Instead,
human behavior runs the gamut from charming to wicked, honest to
malicious, overbearing to submissive, and happy to forlorn---sometimes even on the same day. For each of these many states and traits,
you will need words in Spanish.

Of course, learning all of the epithets employed to describe human personalities would be the intellectual equivalent of memorizing
a thesaurus. You could do that, naturally, but since you'll be wanting
to save a little brain space for such things as nouns and verbs, it's better to concentrate on a few common ways of referring to other individuals. Thus what follows is by no means an exhaustive list of all Spanish words describing people but rather a collection of some of the most
useful ones---ones you will probably hear, and may even be called, in
the company of Spanish speakers.

Where appropriate, rough English equivalents are provided.
The translations serve more to highlight the Spanish word's relative
strength and connotations than to provide a literal translation. As in
any language, you should choose your words carefully when referring
to other people. When in doubt, a good rule is to use them for the first
time when the person being described is not within hearing range.
Outside of a language-learning context, of course, this is called gossip.
But since you're still learning, it's allowed.

\section{The good}

Unless you are at a particularly low point in a mood swing,
you will probably concur that when all is said and done, there are still
a few good people out there. And with luck and perseverance, you may
even meet one of them. When you do, it's important to be ready with
the appropriate verbal match for your smile and that warm feeling in
your belly.

A good way to convey this sensation is by telling a person you
like them. Conveying this in Spanish, though, requires careful attention to differences in degree and intensity of the words. If you guide
yourself by the dictionary, for instance, you will think that \emph{querer}
means "to like." This is true---up to a point---but what a dictionary
often doesn't tell you is just how strong the "liking" reflected by
\emph{querer} really is. The distinction is important, since saying \emph{te quiero} to
someone whom you just "like" could provoke a considerably stronger
reaction than you bargained for. \emph{Te quiero} means "I love you"---in
some contexts even "I want you"---and is virtually synonymous with
the magic words \emph{Te amo}. Suppose someone asks, "What do you think
of my boyfriend?" If you try to say "I like him" with a response like \emph{Lo
quiero}, expect to be misunderstood: what you are really saying is "I
want him," as in "I want him for myself." Such slips of the tongue
could prove fatal.

So how do you say "Hey, you're not such a bad bloke" in Spanish? The simplest formula throughout the Spanish-speaking world is to
use \emph{caer bien}---literally "to fall well." \emph{Ese señor me cae bien} is the common, colloquial, and non-intimate way of saying "I like that fellow." It
is equally safe in direct speech. \emph{¿Sabes? me caes bien} = "You know, I
like you." This useful phrase will never fail to get your message across
loudly, clearly, and without confusion and misinterpretation.

But, you say, my dictionary also says to use gustar for "to
like," as in \emph{Me gustan las tortillas}. True---up to that point, again. Gustar with people is a slightly different matter. \emph{Me gusta Paco} does mean
"I like Paco," but it carries many of the connotations and commitments implied in \emph{Quiero a Paco}. "I fancy Paco" might be a good English equivalent. If you're male and you say \emph{Me gusta Paco}, expect to
receive funny looks. \emph{Paco me cae bien} is presumably what you want
to say.

As long as you're looking through the dictionary, you may also
discover \emph{agradar}. This verb does work for "I like you" (\emph{Me agradas})
without the romantic overtones, but it can sound a little forced or
formal.

Presumably your descriptions of others will occasionally go
beyond the fact that you like them or dislike them. You'll want to
describe, for the benefit of your listener, what kind of person Juan or
Juanita is. If Juan is a "nice guy," for instance, you could say \emph{es un
buen tipo, es buena persona}, or \emph{es buena gente} (closer to "he's good
people"). These work with Juanita as well, with one important exception: when referring to women, note that \emph{tipa} is derogatory. Thus \emph{Es
una buena tipa} sounds inherently contradictory---kind of like saying
"She's a nice bitch"---and should be avoided. Regional expressions
of general approval abound. In Mexican slang a likable person will often be described by saying simply \emph{Es buena onda}, roughly "He's (or
she's) cool."

Once you've mastered how to say you like a person, you'll
want to delve deeper and learn words for specific personality traits. For
the purposes of learning some of these, we'll divide positive personality traits into four entirely arbitrary groups: the \emph{amables}, or "nice"
people; the \emph{simpáticos}, or "cool" people; the \emph{listos}, or "sharp" people;
and the \emph{serios}, or "solid" people. Try to learn at least one description
from each, and you'll be well on your way to explaining why exactly a
given person "falls well to you."

Note as you go that all of these words, unless otherwise stated,
should be used with \emph{ser}. In fact, using some of these words with \emph{estar}
can lead to confusion and may change the meaning altogether. Where
this is the case, it is noted.

\section{The \emph{amables}}

Let's face it: some people are just downright nice. God knows
how they got that way or how they manage to stay that way, but the
fact is that they are kind, warm, and caring. And---why not admit
it?---we like them. On the other hand, they may never become our best
friends, and our dealings with them may never transcend the simply
social. In Spanish such a person could be called \emph{amable}, equating with
"kind." \emph{Amable} is a useful, generic, and somewhat bland word, perhaps most common in the stock phrase \emph{gracias, muy amable}. \emph{Gentil}
is used in much the same way, though it rings more formal. Note also
that \emph{gentil} is a trickster (see Chapter 3); it can mean "gentle," but it often covers much more.

Here are other commonly used favorable descriptions that fall
roughly in this category:

\subsection{\emph{generoso}}

%GENEROSO
Works safely as "generous."

\subsection{\emph{bondadoso}}

%BONDADOSO
Comes closer to "charitable" and "giving."

\subsection{\emph{desprendido}}

%DESPRENDIDO
A nice, multisyllabic word for "generous" in
the sense of "disinterested."

\subsection{\emph{atento}}

%ATENTO
Means "thoughtful."

\subsection{\emph{detallista}}

%DETALLISTA
Means "thoughtful," too, but goes further and is
used for the sort of people who send thank---you notes and tasteful, personalized gifts. It's the same in the masculine and feminine.

\subsection{\emph{cortés}}

%CORTÉS
Means "courteous" or "polite." A common maxim
in Spanish is \emph{Lo cortés no quita lo valiente}, which, very loosely translated, means something like "Real men can be polite, too."

\subsection{\emph{una persona considerada}}

%UNA PERSONA CONSIDERADA
A very safe cognate for "a considerate person."

\subsection{\emph{una persona comprensiva}}

%UNA PERSONA COMPRENSIVA
Similar to \emph{una persona considerada} but distinct, referring more to a very "understanding" or "compassionate" person.

\subsection{\emph{dulce}}

%DULCE
Means "sweet," a good word for some people.

\subsection{\emph{es un alma de Dios} and \emph{es un pan de Dios}}

%ES UN ALMA DE DIOS and ES UN PAN DE DIOS
Expressions
for extremely good-hearted people, of the sort who have never had a
harsh word for anyone.

\bsk

Also falling into the \emph{amable} category are a number of words
and phrases typical of polite society:

\subsection{\emph{una fina persona}}

%UNA FINA PERSONA
Suggests a kind, considerate person.
Turned around, \emph{una persona fina} is a well-bred, "fine" person.

\subsection{\emph{educado}}

%EDUCADO
Often used much like \emph{una persona fina}. Remember in passing that \emph{educado} has a far broader meaning in Spanish than
"educated" does in English (see Chapter 2). Sometimes \emph{educado} comes
closer to "classy" or "a class act" in English.

\subsection{\emph{una persona culta}}

%UNA PERSONA CULTA
Goes a step further, describing a dignified, tasteful individual.

\section{The \emph{simpáticos}}

This is the category that would include your best friends, and
the most descriptive word for them is \emph{simpático}. This word is also a
trickster, looking like "sympathetic" but meaning something quite different. A single English word doesn't really do it justice; probably you
would say something like "He (or she) is great!" The younger set might
find "cool" a close equivalent. Easier than translating the epithet is
imagining the sort of person who would be worthy of it: a happy,
friendly, attractive, charming, witty, altogether likable person.

About the only quirk to be noted about \emph{simpático} is its use
with the verb \emph{estar}, especially in reference to members of the opposite
sex and babies. In these contexts \emph{está simpático} can sound like a backhanded compliment. For babies, it would be like saying, "What an
interesting-looking child." For members of the opposite sex, especially
"eligible" ones about your own age, it's on a par with saying "Well, he
(or she) certainly has a good personality." Perhaps a parallel English
word would be "cute," in cases when it's understood that you're deliberately not using a more flattering word. If you're trying to be nice, use
simpatico with \emph{ser}.

If you want to pin down a \emph{simpático}'s personality further,
there is a wealth of words at your disposal:

\subsection{\emph{alegre}}

%ALEGRE
Covers "happy," "happy-go-lucky," "outgoing," and
the like.

\subsection{\emph{amigable}}

%AMIGABLE
A safe cognate for "friendly," but the Spanish
word is less common and more specific.

\subsection{\emph{atractivo}}

%ATRACTIVO
A straightforward cognate of "attractive." Like
the English word, it stresses the person's physical charms.

\subsection{\emph{encantador}}

%ENCANTADOR
A close fit for "charming."

\subsection{\emph{relajado}}

%RELAJADO
Describes a simpático who is "relaxed" or "laid-back."

\subsection{\emph{sociable}}

%SOCIABLE
Means "sociable" and is used to describe the outgoing, partying type.

\subsection{\emph{genial}}

%GENIAL
A useful word, though it almost ranks as a trickster.
It usually means "clever" or "great" when referring to things or ideas,
and "a character" or "a wild-and-crazy guy (or gal)" when referring to
people. It is usually a favorable assessment. For "genial," rely on \emph{amable} or \emph{amigable}.

\section{The \emph{listos}}

Some people rank high in our esteem because they are notably
"clever," "sharp," or "bright." Calling them that presupposes some intelligence on their part, but it's not the same thing as calling them
"smart" or "intelligent." "Witty" might come closer, in some cases. A
catch-all Spanish word for these people is \emph{listo}, used with \emph{ser} (with \emph{estar} it means "ready"). This is the word for people who appeal to you on
an intellectual level, people who enlighten, challenge, and entertain,
people you generally like to have around.
Not all bright and witty people are likable, of course, but the
ones who are can merit additional words:

\subsection{\emph{gracioso} and \emph{chistoso}}

%GRACIOSO and CHISTOSO
Stress the humorous aspect of the
person's personality.

\subsection{\emph{astuto} and \emph{ágil}}

%ASTUTO and ÁGIL
Emphasize sheer mental acumen.

\subsection{\emph{despierto}}

%DESPIERTO
A close fit for "bright," suggesting one part "brilliant" and one part "bright-eyed and bushy-tailed," "lively," "energetic." Used with \emph{estar}, of course, \emph{despierto} simply means "awake."

\section{The \emph{serios}}

Some people we like simply because they seem to be "good
people." They are honest, forthright, hard-working, and guileless. What
you see is what you get. These people tend to be our most esteemed
co-workers and our counselors in times of trouble, and the best word
for them in Spanish is \emph{serio}, used, as usual, with \emph{ser}. \emph{Es una persona
seria} = "He (or she) is a real straight-shooter" or "He (or she) has his
(or her) act together" or even is a "together person." \emph{Serio} does not
necessarily convey the idea of a droopy-faced soul or a rigid, unyielding
sort (although with \emph{estar} it can). Instead, it refers to a reliable, trustworthy,solid individual. \emph{Formal} means much the same, but is more, well, formal.

Since \emph{serio} is often used to describe the ideal qualities in a
professional colleague, many words in this category pop up frequently
in reference to the workplace:

\subsection{\emph{capaz}}

%CAPAZ
Means "capable." (Remember that \emph{capable} is a dangerous trickster; see Chapter 3.)

\subsection{\emph{trabajador}}

%TRABAJADOR
Means "hard-working."

\subsection{\emph{responsable} and \emph{dedicado}}

%RESPONSABLE and DEDICADO
These are our old friends "responsible" and "dedicated."

\subsection{\emph{cumplido}}

%CUMPLIDO
An extremely useful word for "reliable" or "competent." Saying \emph{Es muy cumplida} about someone means "She gets the job done."

\bsk

Some other qualifiers are slippery and should be used with care:

\subsection{\emph{muy vivo}}

%MUY VIVO
Sounds very complimentary---especially considering the alternative---but this somewhat slangy expression hints that
the person is "clever," "nobody's fool," and perhaps even a tad "shady,"
"underhanded" or "crooked." \emph{Despierto}, noted above, is a safer choice.

\subsection{\emph{movido}}

%MOVIDO
Suffers the same fate as \emph{muy vivo}. It refers to a very
active person---a real "go-getter"---but on the downside it can imply
that you're never really sure where this person is "moving," with
whom, or according to whose rules.

\bsk

Other "serious" words stress moral rectitude and solidness of
judgment:

\subsection{\emph{recto}}

%RECTO
Highlights the person's basic decency and honesty.

\subsection{\emph{honesto}}

%HONESTO
Includes "honest" but often covers a wider range
of moral traits; "decent" might be a better translation. For simply
"honest," \emph{honrado} is more accurate.

\subsection{\emph{una persona de confianza}}

%UNA PERSONA DE CONFIANZA
A common way to say a person can be trusted or confided in, either in the sense of personal honesty or competence.

\subsection{\emph{sensato}}

%SENSATO
Means "sensible" and is a good word to describe
someone's sound judgment and solid character.

\section{The bad}

All of us know that if you don't have anything nice to say
about a person, you shouldn't say anything at all. None of us practices
this belief, though, which is why this section is needed. For unless you
lead an extraordinarily blessed existence, you will probably find yourself in need of one or more of the following words sooner or later. After
all, even if you don't take pleasure in verbally thrashing your fellow
human being, you'll still need a few good words to describe those
who do.

Spanish, alas, is a rich language for belittling others. The range
of words, phrases, and expressions that can be employed to this end is
practically infinite and, furthermore, varies widely from region to region. Not all of these epithets fall within the range of dignified abuse,
of course, and some are so harsh and vulgar that they can reflect as
poorly on the person using them as on the person for whom they are
intended. For crude, harsh, and undignified abuse, flip ahead to
Chapter 10.

In this section, we'll concern ourselves with descriptions for
people whom you simply don't like. These people, in Spanish, can be
said to "fall badly to you"; \emph{te caen mal}. The converse of \emph{caer bien},
\emph{caer mal} is the safest and most universal way of expressing dislike in
Spanish. A watered-down version is \emph{no caer bien}. \emph{Pedro no me cae
bien} = "I not too fond of Pedro." A "watered-up" version is \emph{no caer
nada bien}. \emph{Pedro no me cae nada bien} = "I don't like Pedro at all." In
many countries, local and colloquial modifications of \emph{caer mal} have
been invented. In Mexico it is common to hear that so-and-so \emph{me cae
gordo}: "falls fat to me," literally. \emph{Me cae pesado}, or "he falls heavy to
me," is the same idea. Remember to use the correct gender depending
on who is "falling to you": \emph{él me cae pesado} but \emph{ella me cae pesada}.

As when expressing your likes, you are on shaky ground using
the verb \emph{querer} to convey dislike. \emph{No lo quiero} means "I don't love
him" or "I don't want him"---leaving a lot to the imagination of the
listener. \emph{Gustar} is also dubious. \emph{No me gusta} suggests "He (or she) is
not for me" or "is not my type."

To capture what it is about a person that "falls badly" to you,
you will need to enter the world of words for negative personality
traits. Again, we can divide them into four arbitrary groups: the \emph{pesados}, or "obnoxious" people; the \emph{imbéciles}, or "jerks"; the \emph{malvados},
or "mean" people; and the \emph{cochinos}, or "slobs." Remember that while
these words are less likely overall to provoke a brawl than their
four-lettered cousins in Chapter 10, you should still use them advisedly.
Pay attention to such intangibles as tone and situation. The same word
said lazily to a friend can offend or inflame a stranger. Unless otherwise noted, these epithets are best used with \emph{ser}.

\section{The \emph{pesados}}

This group covers the obnoxious boor, and \emph{pesado} is the word
we've chosen to carry the banner. \emph{Una persona pesada} can be anyone
from a nagger to an intolerable snob. What all \emph{pesados} have in common is that they get on our nerves after a while---in many cases after
a very short while. They are not really malicious; they're just underequipped in the personality department.

\emph{Pesados} tell crude jokes when no one wants to hear them,
make repeated passes at the same woman, use foreign words that nobody understands, drop names, sneer at your wardrobe, flaunt theirs,
gossip a lot, argue about everything, and never accept their mistakes.
Thank God we're not like them! In a lighter vein, you can use pesado
among friends or family to tell someone to behave themselves or to
"lighten up." \emph{No seas pesado}, said without much conviction, carries
the same message as "Don't be a pain."

Other words explore the \emph{pesado} phenomenon in more detail:

\subsection{\emph{pedante}}

%PEDANTE
For intellectually snobbish people (the ones who
use those foreign words). When we were kids, we called these people
"know-it-alls." Some people may refrain from correcting your Spanish
for fear of appearing \emph{pedante}. The word can be generalized to cover
anyone who has a higher opinion of himself or herself than others do.
"Stuck-up" comes to mind in English.

\subsection{\emph{presumido}}

%PRESUMIDO
The word you've been searching for to say
"show-off." Kids who stick their neat toys in your face but don't let
you play with them are \emph{presumidos}. Ditto for adults with fancy cars.

\subsection{\emph{prepotente}}

%PREPOTENTE
Implies a powerful person who abuses his or
her authority to the detriment of others. It's a good word for cops,
judges, politicians, and bosses, providing they don't overhear you. A
good English equivalent doesn't really exist.

\subsection{\emph{arrogante}}

%ARROGANTE
Safe for "arrogant," "excessively proud."

\subsection{\emph{vanidoso}}

%VANIDOSO
Means "vain" but is more commonly used than
its English equivalent.

\subsection{\emph{snob} or \emph{esnob}}

%SNOB or ESNOB
This term, for better or worse, is creeping
into Spanish. You may hear it, but try to use a Spanish equivalent to
avoid being taken for one yourself.

\subsection{\emph{creerse}}

%CREERSE
This verb, along with an appropriate adjective, will
deflate a pretentious person in a hurry. Someone who tells bad jokes
\emph{se cree chistoso}; someone who wears fancy clothes \emph{se cree elegante};
someone who uses foreign phrases \emph{se cree inteligente}; and so on. \emph{Se
cree mucho}, by itself, covers the lot of these people. In other words,
they believe themselves to be "great shakes," but this opinion is not
widely held.

\bsk

Regional words and slang expressions to cover snootiness are
frequent and often the most colorful. In Mexico \emph{sangrón} is widely if
somewhat slangily used to convey "obnoxious." In Mexico, too, you
can say of a snob, \emph{Le echa mucha crema a sus tacos}, or "He (or she)
puts a lot of cream on his (or her) tacos." Once I heard, in reference to
a particularly stuck-up young man, that \emph{se cree la última Pepsi del
desierto} ("he thinks he's the last Pepsi in the desert"). Keep your ears
open, wherever you are, and you should be able to add fun new descriptions to the stock you acquire here.

\section{The \emph{imbéciles}}

As a rule, none of us likes stupid people. It's not a matter of
their I.Q.---in fact, they may be quite intelligent---but we call them
"stupid" nonetheless. These are people who do stupid things and make
stupid comments. Sometimes they make us wonder how on earth anyone can be so downright, undeniably, irretrievably STUPID!
When you reach that point in your regard of a fellow human
being, the word you want is \emph{imbécil}. In truth, \emph{estúpido} exists and is
widely used as well, but it lacks the punch that \emph{imbécil} packs. Say it
with heavy, accentuated, spitting scorn on the middle syllable: \emph{im-BE-cil}. Now throw your hand in the air as you say it, as if casting this
person from the planet: \emph{¡im-BE-cil!} Isn't this fun?
\emph{Imbécil}, like "stupid," doesn't necessarily mean that the object of your scorn is unintelligent. This is a good thing, because it
means you can use it on an even wider range of people: college graduates, college professors, or even college presidents, if you like. An English equivalent might be "jerk" or perhaps "stupid jerk" or even "You
stupid jerk!" depending on the amount of spit and scorn you want to
add. It is a multipurpose, and strong, pejorative. I once watched as an
unseemly man pestered a young, well-dressed blonde woman on the
subway in a major Latin American city; taking her for a tourist, he
kept asking her "Whey a you from?" in barely pidgin English. Finally
the woman got fed up with this harassment, turned on the man, and
spit \emph{De aquí, ¡imbécil!} in his face. The man slinked off, humbled, and
the crowd was delighted. \emph{Estúpido} just wouldn't have worked.
The \emph{imbécil} category includes a number of other useful descriptions for those who make a distinctly bad impression on us:

\subsection{\emph{idiota}}

%IDIOTA
Similar to \emph{imbécil} but perhaps a shade weaker. Still,
native English speakers should be warned that this word is not used
nearly as casually in Spanish as in English---as in "Oh, don't be an
idiot" or "What a blunder! I feel like an idiot!" Same in masculine and
feminine: \emph{él es un idiota, ella es una idiota}.

\subsection{\emph{infeliz}}

%INFELIZ
Less a "jerk" and more a "klutz" or a "schmuck." It
is used to describe a sort of hapless clod whose luck is mostly bad, at
least in part because of a lack of willpower to make it better. An \emph{infeliz}
is not necessarily a scoundrel or even an entirely unpleasant person,
but you probably wouldn't want to have one as a friend. To soften the
blow, qualify it with \emph{pobre}. \emph{Un pobre infeliz} = "a poor sap," "a loser," a real "sad sack."

\subsection{\emph{tonto}}

%TONTO
Works as "silly" or a watered-down "fool."

\subsection{\emph{payaso}}

%PAYASO
Means "clown" and is a stronger word for "fool,"
particularly a clumsy, oafish one.

\subsection{\emph{baboso}}

%BABOSO
Less harsh than payaso but no less clear in its implications (it comes from \emph{babear}, "to slobber").

\subsection{\emph{tarado}}

%TARADO
A handy word suggesting that something crucial to
cerebral functioning may be missing; "moron" might fit.

\subsection{\emph{tarugo}}

%TARUGO
A term that means "a block of wood." Applied to
people, it equates nicely with "blockhead."

\subsection{\emph{burro}}

%BURRO
Meaning "donkey," the word calls to mind "dumbbell" or "dunce" and, like those words, is a favorite among schoolkids.

\subsection{\emph{bobo}}

%BOBO
In the same class as burro, suggesting "dummy."

\subsection{\emph{ignorante} and \emph{cretino}}

%IGNORANTE and CRETINO
Safe cognates that are heard on occasion.

\subsection{\emph{simple}}

%SIMPLE
Mostly a false cognate, especially when talking
about people; it means "simple-minded" or "simpleton" more than
"simple." If "simpleton" is the word you seek, \emph{simplón} is even more
expressive.

\subsection{\emph{torpe}}

%TORPE
Means "clumsy oaf" or "klutz."

\subsection{\emph{necio}}

%NECIO
A useful word that refers more to stubbornness than
stupidity, though the two often go hand in hand; "jackass" may be a
good equivalent.

\section{The \emph{malvados}}

In this group we lump people whom we consider mean, offensive, malicious, or simply "bad" people. \emph{Malvado} is a good catchall to
describe these people, especially if they have done us some harm. \emph{Maldito} is stronger, though without being crudely so, and is invariably the
word chosen in subtitled films to translate "bastard." (In Spanish \emph{bastardo} is usually reserved for persons born out of wedlock and thus
works poorly as a general-purpose insult.) Here are some other useful
descriptions of this type:

\subsection{\emph{mal parido} and \emph{mal nacido}}

%MAL PARIDO and MAL NACIDO
Meaning literally "born bad,"
these phrases convey the same ignominy as \emph{malvado} but sound a bit
snazzier.

\subsection{\emph{sinvergüenza}}

%SINVERGÜENZA
Literally meaning "without shame" or
"scoundrel," the word suggests that this person actually enjoys being
offensive. Very commonly used. Same form for masculine and feminine.

\subsection{\emph{canalla}}

%CANALLA
Refers generally to men, often a man who mistreats women. It works well for "lout." 1I Same form for masculine and
feminine.

\subsection{\emph{desgraciado}}

%DESGRACIADO
A common term of generic opprobrium, generally meaning an unpleasant person who has tried to do us some
wrong. It also implicitly tries to write the person off as trivial and unworthy of genuine scorn. It equates fairly well with "wretch" or "sap"
in English.

\subsection{\emph{infame}}

%INFAME
A good strong word used to describe someone who,
by his or her actions, has earned everlasting infamy---at least in our
eyes. Murderous tyrants are \emph{infames}; so, by extension (and slight exaggeration), is anyone who causes you genuine harm with malicious
intent.

\subsection{\emph{mentiroso}}

%MENTIROSO
A strong word, stronger even than "liar." It suggests the person is a habitual liar and uses lies to defame and gain advantage. Be careful using it.

\subsection{\emph{maléfico}}

%MALÉFICO
Describes a real evil sort, a pernicious "ne'er-do-well."

\subsection{\emph{malviviente}}

%MALVIVIENTE
Literally meaning "bad-liver" (one who lives
badly), the word suggests an incorrigible rogue with a long police
record.

\subsection{\emph{delincuente}}

%DELINCUENTE
Much the same as \emph{malviviente}, the term
means "delinquent" but stresses the criminal lifestyle that such a person has chosen. As a result of increasing drug use and the crime that
goes with it, words like \emph{mariguano} are coming into vogue to mean the
same thing. To an English speaker the word calls to mind a fading hippie, but in Spanish it describes a dangerous "druggie" and is becoming
the common word for scandalous and vandalous teenagers.

\section{The \emph{cochinos}}

For uncouth, classless slobs, \emph{cochino} is the word. It is one of
four common Spanish words for "pig" (the animal)---\emph{cerdo, puerco},
and \emph{marrano} are the others---and all work well for the human equivalent. If you dislike someone because he or she throws trash out of car
windows, takes up three train seats, or burps at the dinner table, this is
the category you should check. (You should be aware that the word
\emph{marrano} has a long anti-Semitic history, though it is rarely used that
way today.)

\subsection{\emph{grosero}}

%GROSERO
The catchall word for "rude." \emph{Se portó bastante
grosero} would be the common way of saying "He acted quite rude,"
"He treated us badly."

\subsection{\emph{vago}}

%VAGO
Describes an unkempt person or "bum." (It is also the
word for "vague"; see Chapter 3.)

\subsection{\emph{bajo}}

%BAJO
Widely used in the phrase \emph{un tipo de lo más bajo}, a
strong denunciation that translates well as "a low-life," "creep," or
even "scum." Note, however, that \emph{Es un tipo bajo} usually just means
"He's a short guy."

\bsk

A number of other common words refer to general "low-class"
behavior. Though most modern-day English speakers are not accustomed to think of behavioral traits in class terms, in much of the
Spanish-speaking world the class distinction is still a strong one.
Words like \emph{corriente, vulgar}, and \emph{ordinario} denote little more than
"common" or "characteristic of the masses," but all three connote
"uncouth," "slobbish," or "classless" (as in "he has no class") when
used to refer to people. \emph{Un hombre corriente} can be expected to meet
his dinner guests unshaven, in boxer shorts and scratching his paunch.
\emph{Inculto} falls into this group as well. It means "uncultured" but is
much broader and more frequently heard than its English cognate; an
\emph{inculto} not only doesn't appreciate Mozart but plays his radio too loud.

\section{The indifferent}

We have looked at the human race from both sides now. We
have met the lowest of the low, the snobs and scoundrels, and learned
what to call them. We have also been introduced to the creme de la
creme, the well-bred gentlefolk, and have had a few charmed words
with them. But what about the rest of the world's inhabitants---that
gray, faceless mass that hovers uncertainly between good and evil?
What do you say to describe a "nowhere type," a fair-to-middling sort,
a person who's nothing to write home about and just like the next guy,
only less so? What do you say about people when there isn't much to
say about them one way or the other?

A number of unspectacularly adequate Spanish words rise
humbly to the task. \emph{Regular, normal}, and \emph{común} all make the grade,
though you should be alert to their idiosyncrasies. \emph{Regular} with \emph{estar},
for example, is a notch or two below "regular" on the scale of descriptive tags; whereas in English it means "average," in Spanish it is closer
to "fair" or even "not so hot," especially when referring to objects.
\emph{Esta regular el camino} does not mean "It's an average road" but rather
"It's a pretty lousy road." "Regular" in the sense of "consistent" is
\emph{regular} with \emph{ser}. Be alert to the distinction: \emph{Es un cliente regular}
would be "He's a steady customer," while \emph{Ese cliente está regular} suggests the customer is a lousy tipper or otherwise "fair." Using \emph{asiduo}
instead of \emph{regular} to describe the steady customer can avoid a potential
misunderstanding and win you brownie points for using a good Spanish word.

\emph{Normal} is the best all-around word for the "nothing-special"
category. \emph{Mediocre} carries a clear connotation of insufficiency, as does
its cognate in English. Other words---specifically \emph{corriente} and \emph{ordinario}---are listed in dictionaries as synonyms for "normal" or "ordinary,"
but in fact can be charged with negative meaning when used to label
people, as noted above. \emph{Común} works in the expression \emph{un hombre común}, meaning "common folk" or "the man on the street." \emph{Común y
corriente} is a good translation for "run-of-the-mill," without the negative connotations of \emph{corriente} by itself. Still, your safest bet in almost
every circumstance involving people is normal.

In situations where physical appearance is being commented
on, a different set of words takes over. A person whose looks are "nothing special" could be described as \emph{pasadero} or \emph{pasable}, meaning "acceptable" or "tolerable" in the sense of "if nothing better comes
along." A useful, if regional, slang phrase for "not bad" (or "not great")
is \emph{dos tres}; it works equally well for people, places, and things and
sounds, to my ear, a lot better than \emph{así así}, which every textbook will
tell you means "so-so." \emph{Más o menos} (technically, "more or less") is
also commonly used in Spanish, even when there's no indication of
what is being compared: \emph{¿Qué tal la película? Más o menos}. = "How
was the film?" "Not bad." Sometimes in slang \emph{más o menos} gets
shortened to \emph{ma' o meno'} in this usage.

Some fun expressions for saying "nothing special" employ neither-nor constructions. A Peruvian, for instance, might say someone is
\emph{ni chicha ni limoná}, where \emph{chicha} is a local alcoholic drink and \emph{limoná} is "lemonade." The idea is that the person is neither alcoholic
nor nonalcoholic-that is, isn't anything well-defined. "Neither here
nor there" comes out as \emph{ni de aquí ni de allá}. \emph{Ni pinta ni da color}
means that someone (or something) "neither paints nor adds color,"
suggesting that there is no real reason for this person or thing to exist
at all.

Another common way of saying "no great shakes" is \emph{nada del
otro mundo} (literally, "nothing from the other world". Another handy
phrase is \emph{Es cosa de cada domingo} ("It's an every-Sunday thing"),
equating with "a dime a dozen" and showing that you are distinctly
unimpressed. \emph{No es nada fuera de lo común} is good but a bit stilted
for "It's nothing out of the ordinary." When you are completely unimpressed by something or someone, but still unwilling to turn it or
them down, resort to \emph{peor es nada}---the correct rendering for "better
than nothing."

\section{Temporary states}

Since we've spent so much of this chapter name-calling, it's
only decent of us to finish by noting that most bad traits are only
temporary and that most people, if we'd only give them time and get
to know them, would soon return to being their real sweet, lovable
selves. With this is mind, you'll need the proper vocabulary to describe
these warm and caring individuals who just happen to be acting like
miserable worms at the moment.

One way to get this across is by using \emph{estar}, the Verb of Temporary States, but not simply as a substitute for \emph{ser}. Instead, by inserting \emph{de} between estar and your adjective, you change "He's an S.O.B.
(or whatever)!" to "He's being an S.O.B.!" Thus \emph{Es un grosero} describes
a permanently rude, coarse, and unpleasant man; \emph{Está de grosero} describes a man who is acting rudely. \emph{Es un presumido} = "He's a showoff"; \emph{Está de presumido} = "He's showing off."

This construction is especially useful for capturing moods. For
instance, \emph{Está de pesado} = "He's in a lousy mood." Two other key
phrases for moods employ the generic terms \emph{buenas} and \emph{malas}. \emph{Está
de buenas} means someone is "in a good mood." \emph{Está de malas} is the
dreaded opposite.

\emph{Estar de} can also be used instead of \emph{ser} for ironic effect. For
instance, \emph{Está de generoso} suggests that someone who usually isn't
generous is for some reason suddenly giving away the store. Saying
that a child \emph{está de obediente} suggests that the child's obedient behavior is indeed most unusual. \emph{Estar de} can likewise be used to explain
away behavior as a temporary aberration: \emph{Estoy de tarado} = "I'm acting a little stupid." It should be noted that some adjectives by their
very meaning resist the \emph{estar de} construction; it is hard, for instance,
to imagine someone being \emph{infame} for only a few minutes or so.

\chapter{The Secret Life of Verbs}

Many students of Spanish can still recall their first encounter
with Spanish verbs. Often the reaction was something like, "What do
you mean it has different endings in every tense? What on earth for?
Aaaarrgh!"

Alas, for verb endings there is little alternative but to buckle
down and memorize them. But even when you've managed to separate
\emph{-aban} from \emph{-aron}, another unappetizing task awaits: using each ending
and each tense and mode in the right situation. To English speakers
the idea of an imperfect tense is unfamiliar, while the notion of a frequently used subjunctive mode can seem downright perverse. Can't
verbs be made just a little easier?

The answer is yes. There are several pointers on the use of the
tenses that will make learning them a less doleful task. And there are
a couple of obvious pitfalls that anyone wandering into the world of
tenses should be alert to. This chapter presumes you have some background knowledge on the use of tenses in Spanish; even so, I'll try to
summarize some of the basics of tense usage to refresh your memory.

\section{The present}

The present is the most straightforward of tenses in Spanish
and corresponds almost perfectly to the present tense in English: \emph{bailo}
is "I dance," \emph{estoy bailando} is "I am dancing," and so on. For the English compound present ("I do dance"), in Spanish you can slip a \emph{sí}
("yes") into the phrase. "He does eat a lot" = \emph{El sí come mucho}. "But,
honey, I do love you" = \emph{Pero cariño, sí te quiero}. Note that in English
the present progressive is used far more than in Spanish. Thus "She's
going" should almost always be rendered \emph{Se va} and only rarely \emph{Se esta
yendo}.

One tip on the present tense: it is used much more in Spanish
for future events than its English equivalent. For instance, the common way to ask "Are you coming tomorrow?" is simply \emph{¿Vienes mañana?} "We'll see you later" often gets rendered \emph{Nos vemos}---literally,
"We see each other." This trick works even when you're specifying a
future time or date. \emph{Te lo doy el martes} = "I'll give it to you Tuesday."
In Spanish the context makes it clear that the future is being referred
to. Compare these examples:

\bsk

\indu \emph{Te cuento}. = "I'll tell you (now)."

\indu \emph{Te contaré}. = "I'll tell you (someday)."

\indu \emph{Mañana te cuento}. = "I'll tell you tomorrow."

\indu \emph{Mañana te cantaré}. = "I shall tell you tomorrow."

\section{The future}

The future is not a particularly hard tense to learn, and when
to use it is pretty obvious. Still, it can be simplified considerably by
remembering the following rule: ignore it.

What? Just forget about the future? Well, maybe not altogether. There are a few uses for the future in common spoken Spanish.
But most of the time you can avoid it outright, and you'll even sound
more fluent by doing so.

The two most frequent substitutes for the future tense in
Spanish are the "present-as-future," discussed above, and the compound future using the verb \emph{ir} ("to go"), just as in English. \emph{Mañana voy
a llamar a mi hermano} = "Tomorrow I'm going to call my brother."
In common Spanish usage, one of these two constructions almost always replaces the future tense, although the future can be used and is
certainly understood: \emph{Mañana llamaré a mi hermano}. Sometimes,
though, the future sounds formal, stiff, and even awkward. \emph{Vuelvo en
seguida} = "I'll be right back." \emph{Volveré en seguida} = "I will return
promptly."

One special use of the future that anyone aspiring to fluency
must learn is what is called the "future of uncertainty." It is a very
common construction that has no real equivalent in English---which is
why many students shudder at the mere thought of it. The best way to
get a grip on it is by example. You often hear it, for instance, when
someone knocks unexpectedly on the door. "Who could that be?"
you'd blurt out in English. \emph{¿Quien será?} you'd say in Spanish. Other
examples: \emph{¿Habrá más?} = "Might there be more?" \emph{¿Estará en casa?} =
"Do you suppose he's at home?" And so on.

The future of uncertainty, applied to the past, uses the compound future perfect tense. That sounds difficult, but really it works
out to sticking \emph{habrá} or an equivalent in front of the past participle of
your choice. \emph{¿Quién habrá sido?} ("Who might that have been?"), you
might wonder on hearing a strange voice on your answering machine.
\emph{¿No habrán querido asustarme?} = "You don't suppose they wanted to
scare me, do you?"

\section{The conditional}

The conditional is a very predictable tense, conveying
thoughts in Spanish that in English rely on the auxiliary "would." It's
an easy tense to understand intuitively. "I'd like to eat now" is expressed as \emph{Me gustaría comer ahora}. "You would have liked it" = \emph{Te habría gustado}."

One special warning applies for the conditional: note that in
English we occasionally use "would" for repeated past actions, as in
"His father would eat every night at seven sharp." This construction
in Spanish calls for the imperfect---in fact, it's a textbook example of
when the imperfect is needed: \emph{Su padre cenaba todas las noches a las
siete en punto}. If you think about it long enough, it becomes clear why
this is not a true conditional, in either English or Spanish. But if you're
simply translating your "woulds," it's easy to make this mistake.

\section{The preterit versus the imperfect}

This face-off is one of the trickiest in Spanish, mostly because
in English we often gloss over the distinction. We do have the progressive past, of course, in such constructions as "He was flying a kite,"
but not all "ongoing" past activities use it. In many cases, you'll have
to slow your translating computer down a few megas and think about
exactly what kind of past action you are describing.

Basically the Spanish imperfect covers two constructions:

\subsection{}

\textbf{The progressive past}, including actions that are taking
place over a period of time in the past, usually in relation to some
other action that happened suddenly. For example, "He was sleeping
when the alarm went off." Or "She was overseas when the new president was elected." Keep your eye (and your mind's eye, if you're speaking)
out for these juxtapositions; they represent one of the most common uses of the imperfect. Because of the explicit juxtaposition, these
are also the easiest situations to recognize as an opportunity for the
imperfect.

\subsection{}

\textbf{Actions that took place over a period of time in the past},
often keyed to the English constructions "used to" and, on occasion,
"would" (see "The Conditional" above). This second common use of
the imperfect is indispensable in describing the way things were: "I
used to work nights" is expressed as \emph{Yo trabajaba de noche}. The key
words aren't always present, though. "When I was a kid, the teachers
beat the students" would be \emph{Cuando yo era niño los maestros golpeaban a los estudiantes}.

The problem of the disappearing key words can be seen in the
following sentences. Just as you could say "My father ate at seven every night," "My father used to eat at seven," or "My father would eat
at seven," you could just as easily state, pure and simple, "My father
ate at seven." Grammatically this last example is correct, but it is confusing without a clarifying context. Did he eat at seven o'clock just
once, or did he always eat at seven? Spanish lets you make the distinction in the verb itself: \emph{cenó a las siete} is clear in communicating that
he ate at seven on a certain occasion; \emph{cenaba a las siete}, using the
imperfect, indicates that it was his custom to eat at seven night after
night.

Understanding this distinction will clear up a lot of the conflicts between imperfect and preterit, but incorporating that knowledge
into your storytelling skills will take some time and practice. Often,
when relating a story, you'll have to jump nimbly back and forth from
imperfect to preterit, and this requires analyzing each action as it
pops up.

Let's say this father usually ate at nine, but on one particular
night he ate at seven; while he was eating, he found a fly in his soup
and fainted. What tenses will make this meaning clear in Spanish?
First, we have to explain that the father used to eat or usually ate (\emph{cenaba}) at nine; on the night in question, though, he ate (\emph{cenó}) at seven.
While he was eating (\emph{cenaba}---the imperfect again), he found a fly in
his soup (\emph{encontró una mosca en la sopa}---preterit) and fainted (\emph{se desmayó}---also preterit).

Sometimes, the differences between past and imperfect are
subtle to the point of near invisibility. In these cases, native Spanish
speakers will sense the distinction, but they have a hard time explaining it you. Let's say you answer the phone and your boyfriend is on the
other end. He could ask, \emph{¿Qué estabas haciendo?} Or he might ask,
\emph{¿Qué estuviste haciendo?} What's the difference? In the first example,
using the imperfect in the auxiliary, he is asking what you were doing
in relation to a sudden action---presumably, the ringing of the phone.
That is, "What are you doing right now (besides talking on the telephone)?" In the second example, he is asking what you were doing,
with the idea that you may have finished by now. "What have you
done all day?" or "What have you been up to?" is closer. Of course, if
he wanted to be completely clear, he could ask \emph{¿Qué hacías?} or \emph{¿Qué
hiciste hoy?} or \emph{¿Qué has estado haciendo?} and so on. But where's the
romance in that?

The important thing in these borderline cases is not to learn
the "right" form. (In fact, your boyfriend would probably just ask \emph{¿Qué
haces?}) The important thing is to search for the distinctions in these
close cases to get a feel for past tenses that will serve to guide you
when no key words or juxtapositions are there to help you along.

Let's take a final borderline example. You were at home last
night. But do you use the imperfect or the preterit to convey that to
your listener? Again, it depends. Both could be correct, but \emph{estaba en
casa} suggests you were there for the duration and something sudden or
specific happened during this time: \emph{Estaba en casa cuando se fue la
luz} ("I was at home when the lights went out"). The juxtaposition may
be implied, not stated, but you create the expectation of one by using
the imperfect. In other words, \emph{Estaba en casa anoche} in effect prompts
the question "And what happened?" \emph{Estuve en casa} is a much more
basic, self-contained, declarative sentence. It just says you were at
home, period. In this sense, it's a safer, more all-purpose choice than
\emph{estaba}.

\section{Special cases}

In Spanish the tense of certain verbs is almost as important as
word choice in getting your point across correctly, and students of the
language are generally unaware of the subtle twists they can give these
verbs by their choice of tense. Yet by manipulating the tenses well,
you can discover ways to express familiar English phrases in correct,
colloquial Spanish.

For instance, \emph{querer} ("to want") changes its meaning on the
trip from preterit to imperfect. Consider the case of \emph{quisieron} (preterit)
and \emph{querían} (imperfect). Both mean "they wanted," but a native Spanish speaker hears a difference. The former suggests that they wanted to
do something (and they did it). Thus their "wanting" came to an end,
at least for a while, so the verb goes into the preterit as a done deal.
\emph{Querían}, the imperfect, suggests they wanted to do something and,
evidently, they still want to. That is, they wanted but were unable to
do something. Thus, \emph{Quisieron ir al cine} and \emph{Querían ir al cine} both
mean "They wanted to go to the movies," but with a difference. The
preterit conveys the idea that they wanted to go the movies, so they
went. The imperfect suggests that they wanted to go the movies but
didn't---maybe after seeing the ticket prices.

The imperfect can also be used to say that they wanted to do
something and perhaps eventually did do it, but not before something
else intervened making that possible. For example, \emph{Querían ir al cine
y los mandé}, ("They wanted to go to the movies, so I sent them"). The
use of \emph{quisieron} implies that they would have gone without waiting
for someone to send them. As in the earlier example about being at
home,an imperfect on its own almost seems to raise the question \emph{¿Y
qué pasó?} It is, so to speak, the tense in which the other shoe is always
waiting to drop.

In the negative, the preterit-imperfect distinction is sharper
still. \emph{No quería ir a la fiesta} implies "I didn't want to go to the party
(but I went along anyway)," perhaps to avoid offending the host. \emph{No
quise ir a la fiesta} says bluntly "I didn't want to go to the party (so I
didn't)." Very often, \emph{querer} in the preterit and negative translates best
as "to refuse." \emph{No quisieron dar sus nombres} = "They refused to give
their names."

Similar changes occur to the verb \emph{poder} ("to be able") when it
makes the switch from preterit to imperfect. \emph{Mi hermano podía romperme la cabeza} suggests "My brother had the physical force necessary
to bust my head," while \emph{Mi hermano pudo romperme la cabeza} suggests that on the occasion in question he put this theoretical force to
the test. In the preterit, in other words, \emph{poder} means you not only
could do something but actually did do it. \emph{El podía nombrar todas las
capitales del mundo} ("He could name all of the world's capitals")
might be said of a smart lad. \emph{El pudo nombrar todas las capitales del
mundo} means the tyke was actually put to the test---and succeeded
("He was able to name all of the world's capitals").

As with \emph{querer}, the distinction is sharper in the negative. \emph{No
podía caminar} might mean that your feet hurt after a long hike and
you spent a day cuddled up on the couch watching sitcoms. \emph{No pude
caminar} suggests you actually tried to walk and fell on your face. In
general, the preterit with \emph{no poder} implies a fairly formidable obstacle
and suggests that an unsuccessful effort was at least made. The imperfect with \emph{no poder} is much less definite and does not hint that an effort was made. To take an extreme example, \emph{No podía volar del techo
de la casa} is stating the obvious: "I wasn't able to fly off the roof of my
house." \emph{No pude volar del techo de la casa} are words that, if uttered
at all, would most likely be uttered in an ICU ward by a sheepish lunatic---that is, after trying to fly and failing.

The last common verb whose meaning varies significantly
between the imperfect and the preterit is \emph{saber} ("to know"). In fact, this
variation causes students of Spanish no end of confusion. It seems unjust that such everyday statements as "I didn't know that!" and "Did
you know he was coming?" require a brain-racking choice of tense.

In fact, with \emph{saber} the difference is usually quite clear-cut.
Here's a rule of thumb that works well most of the time: use the imperfect. In the preterit \emph{saber} usually means "to find out." Some examples are in order. \emph{No sabia eso} = "I didn't know that." \emph{Sabia que
llegarías} = "I knew you'd show up." \emph{¿Sabías hablar español cuando
llegaste aquí} = "Did you know how to speak Spanish when you got
here?" In the preterit, in contrast, \emph{saber} generally refers to sudden
knowledge about a specific event. \emph{Supe que habías llegado} = "I heard
(I found out, word reached me) that you had arrived." \emph{¿Supiste\ldots{} ?} in
particular is almost always used for "Did you hear\ldots{} ?" and implies
some new gossip, revelation, or fast-breaking news. \emph{¿Supiste que gané
la lotería?} = "Did you know (hear) that I won the lottery?"

\section{\emph{ser} versus \emph{estar}}

In Spanish the question is not so much "to be or not to be?"
but "to be (\emph{ser}) or to be (\emph{estar})?" These two verbs are the source of
constant headaches and frequent errors for even intermediate and
advanced students of Spanish. Native Spanish speakers intuitively
choose the correct form without so much as a thought. You should
be so lucky.

One modern Spanish dictionary, in its introduction, makes
this very point (and rather smugly, I thought): "[Foreigners] should
know, so that they realize that the distinction between ser and estar is
clear and precise and that it is just a matter of managing to penetrate
the distinct nature of both verbs, that Spaniards, even the most uncultured ones, never use them wrong."

As a foreigner, of course, you will use them wrong, and about
10 percent of the cases will still seem mystifying to you even years
after you learn the common usages. But in at least 90 percent of the
cases the distinction between \emph{ser} and \emph{estar} is "clear and precise"---or
at least pretty easy to guess. As for that other 10 percent, well, you
gotta leave something to learn as you get older!

\section{The easy ones}

\emph{Ser} is the verb "to be" for things that are That Way, period.
They're not that way in relation to something else, or at certain times
of day, or in the spring or the fall, or only in election years. They are
that way because they were born that way and they will presumably
remain that way until the day they die. Ser is a solid, upstanding
verb---one that you can rely on to give you the same answer time
and time again.

\emph{Estar}, in comparison, is a flake. It is the variable, flighty, here-
today-gone-tomorrow verb "to be." \emph{Estar} covers personality traits that
are ephemeral and ethereal. It describes things that change from one
minute to the next. It's an all-over-the-place, outta-control kind of
verb. It's untrustworthy. It's slippery. You would never buy a used car
from a verb like \emph{estar}.

Let's take an example. Say your boss is a fool. \emph{Es una tonta},
you might say (though perhaps not to her face). But let's also say that
she spent all morning collecting mud samples and is now absolutely
filthy. \emph{Es una tonta and esta mugrienta}. What's more, in a moment of
inspired honesty, you told her that she looked like something that just
crawled out from under a rock, and now she's mad \emph{está enojada}. Just
like her to get so upset about a casual observation, you think. She's so
sensitive---\emph{es tan sensible}.

As you can see, we're getting a good picture of your boss: \emph{es
una tonta} and \emph{es muy sensible} (all the time), and \emph{esta mugrienta} and
\emph{esta enojada} (this afternoon). Probably not the best time to be kicking
back and reading a book, come to think of it.

Some words flat out change in meaning depending on whether
they are governed by \emph{ser} or \emph{estar}, and they can help us "penetrate"
those "distinct natures" we've been told so much about. Here's a rule
that can be applied in most cases: if you can add a "now" or "at the moment" to your description, you should be using \emph{estar}. If not, leave it to
\emph{ser}. \emph{Es un borracho}, for instance, means that someone is a "drunkard"---
a habitual drunk or a wino. \emph{Está borracho}, on the other hand, means
"He is drunk (at the moment)." \emph{Es callado} refers to a man who is
"quiet," not at any given moment but as a way of life---it's his nature;
he is a person who keeps to himself and speaks softly and rarely. To say
of another man \emph{está callado} means something quite different: he is
quiet---now. We are given no insight to his overall personality; we just
know that in this place and at this time, he's keeping his mouth shut.

Learning and reviewing examples is a good way to absorb the
essential difference between \emph{ser} and \emph{estar}. But a few other specific tips
may be helpful as well.

\subsection{}

Use \emph{ser} for general, permanent physical appearance: tall,
dark, handsome, short, light-skinned, ugly. (An exception will be dealt
with in a moment.) Use \emph{estar} for any temporary physical condition:
pale, flushed, disheveled, unshaven, and so on.

\subsection{}

For quantities, numerical or otherwise, always use \emph{ser}: \emph{somos veinte personas, es mucha, era poco, son dos}.

\subsection{}

For possession, use \emph{ser}: \emph{es mío, es de él, son de las señoras,
son suyos}.

\subsection{}

Location is always the province of \emph{estar}. This might throw
you if you think of the location of, say, a building as fairly unchanging.
But location is an implicit recognition of an object's relation to other
things---not a reflection of its indelible self---and thus is a job for
\emph{estar}.

\subsection{}

With all adverbs, adverbial expressions, and present participle forms or gerunds (the "-ing" form), use \emph{estar}: \emph{está bien, están en
buenas condiciones, está lloviendo, estoy nadando.}

\subsection{}

With all nouns, use \emph{ser}. If you have trouble recognizing
nouns, a good device is to key on the presence of the indefinite article
(\emph{un, una, unos, unas}). Thus \emph{es un doctor, eres una tonta, es un santo},
and so on. When the article is missing, as it often is in Spanish (that is,
\emph{es doctor}), you'll just have to remember that it could be used in that
situation and therefore it's a noun that requires \emph{ser}.

\section{Getting tricky: the past participles}

%GETTING TRICKY: THE PAST PARTICIPLES
With past participles (the "-ed" form in English, the \emph{-ado} and
\emph{-ido} forms in Spanish), things start to get tricky. Both \emph{estar} and \emph{ser}
can be used, but they mean different things. With \emph{estar} the participle
is generally being used as an adjective and to describe a passing state.
\emph{Estaba agotado} means "He was worn out (at that moment)."

With \emph{ser} the past participle is generally used to form a passive
construction or a predicate noun. In the case of the passive, you implicitly ask (and often must explicitly state) "whodunnit?"---that is,
who or what caused the action. \emph{Fue agotado}, for instance, means "He
was worn out," meaning something or someone wore him out. A few
past participles are used with \emph{ser} without any explicit causal agent, including \emph{conocido, sabido, tardado}, and \emph{parecido}.

Some examples may help clarify the distinction between \emph{ser}
and \emph{estar}. Say you went on an expedition to a remote patch of rainforest. When you got there, though, you found that it had recently been
bulldozed. On your return, someone may ask, "How was the forest?"
You could reply using either \emph{Estaba destruido}, referring to its destroyed state, or \emph{Fue destruido}, meaning essentially "It has been destroyed" and calling attention to the fact that someone or something
destroyed it. In English both senses can be covered by "It was destroyed." Spanish makes a finer distinction.

Another example: you can say both \emph{Las tiendas son cerradas a
las nueve} and \emph{Las tiendas están cerradas a las nueve}. What's the difference? With \emph{ser} you are saying that the stores are physically closed
by someone at nine 'o clock sharp. That is when the doors are shut and
the keys turn in the locks. With \emph{estar} you are saying that if you go to
the commercial district at nine you will find the stores closed. They
may have been closed at eight, or at six, or at five minutes to nine, but
in any case you will find that they are closed at nine. In other words,
\emph{Son cerradas a las nueve} = "They close at nine"; \emph{Están cerradas a las
nueve} = "They are closed by nine."

Finally, an example you will want to study assiduously if you
are of the married persuasion: \emph{soy casado} versus \emph{estoy casado}. Some
will argue that there's no big difference here. Others will say there's a
world of difference. Basically, \emph{soy casado} is "I'm a married man." It
describes a permanent state. \emph{Estoy casado} means "I'm married," but
some feel it implies "for the moment" or "at present," something akin
to "I am passing through a married phase at the moment." (\emph{Estoy de
casado} would say that unambiguously.) Nonetheless, a man wouldn't
say \emph{soy casado con} (wife's name) but \emph{estoy casado con} (wife's name).
In the past tense, the distinction becomes very clear: \emph{Fui casado con
Maria} means "I was (forcibly) married to Maria (and may still be)." \emph{Estuve casado con Maria} means "I was married to Maria (who is now my ex-wife)."

Look at the following examples and practice separating them
in your mind:

\bsk

\indu \emph{Fue cambiado}. = "It was changed (by someone)."

\indu \emph{Estaba cambiado}. = "It was (looked) changed." 

\indu \emph{Fue dormido}. = "It was put to sleep."

\indu \emph{Estaba dormido}. = "It was asleep."

\indu \emph{Fue rota}. = "It was broken (by someone)."

\indu \emph{Estaba rota}. = "It was (already) broken."

\section{The hard ones: descriptive adjectives}

Don't worry about mastering the gray areas between \emph{ser} and
\emph{estar} from the start. It's enough to know why the differences exist so
as to incorporate them intuitively as you go along. With practice, the
light gray areas will get progressively lighter and the pitch-black regions will soon turn a sort of dark gray. Examining examples and asking yourself why? is the best way to start shedding light on the matter.
Some dubious cases of \emph{ser} versus \emph{estar} follow. Absorb them at your
own pace.

Perhaps you've noticed in your dealings in Spanish that to compliment someone on, say, his beauty, you use \emph{estar}: \emph{Estás guapo}. But
aren't you in a sense suggesting that his beauty is just a temporary
state, that you're saying, "You are beautiful today (but not as a general
rule)?" Some compliment! There's a kernel of truth in your suspicion,
but perhaps because of human vanity, such a comment is generally
taken favorably to mean "You look especially beautiful today."

Compliments highlight one of the largest zones of overlap between \emph{ser} and \emph{estar}, the descriptive adjectives. For instance, if someone is tall, they are presumably tall all the time, and we would correctly expect \emph{es alto} to convey that. So what the devil are we to make
of \emph{está alto}, which you will undoubtedly come across sooner or later?
Certainly you can't temporarily be tall?

In general, using \emph{estar} with adjectives is a way of highlighting
the immediate and subjective nature of a perception---"This is my
impression" or "This seems especially that way to me now." \emph{Es alto}
means "He is tall." \emph{Está alto} means, more or less, "He's so tall," "He's
much taller than I thought," "Gosh, he's tall." To say "He's tall for his
age," for instance, you would use \emph{Está alto para su edad}. If he were
tall, period---in other words, a tall person---you would simply say
\emph{Es alto}.

Let's take another adjective. \emph{Es feo} would be "He is ugly"-no
debate permitted or even needed. Look up "ugly" in the dictionary and
you'll find his picture alongside. Follow him home and he'll have ugly
parents. So what's left for \emph{Está feo?} It could mean "He's temporarily
ugly"---because of a horrible haircut, for instance. Or it could suggest
"He sure looks ugly to me" or "He really is ugly."

Becoming adept at making this distinction is a matter of time
and exposure. What's the difference, for instance, between \emph{es difícil}
and \emph{está difícil?} Roughly, \emph{es difícil} describes something that is always
difficult and notoriously so: say, swimming the English Channel or
learning Chinese. \emph{Está difícil} suggests that something that has come
up is difficult or that something is harder than was expected. \emph{Es difícil aprender la diferencia entre ser y estar}
 = "It is difficult to learn the
 difference between ser and estar." \emph{Está difícil aprender la diferencia
 entre ser estar} = "I'm having real trouble with this \emph{ser} and \emph{estar}
business."

Now how about \emph{es viejo} versus \emph{está viejo?} The first example
means someone is old, period---a senior citizen. The second, with
\emph{estar}, is much more subjective and can cover a wide range of English
translations, including "He feels old," "He looks (seems) old," and
"He is too old (for some specific task)." A similar case is \emph{ser joven}
versus \emph{estar joven}. \emph{Soy joven} means "I am young (i.e., a member
of the group of young people)." \emph{Estoy joven} covers anything from "I
feel young" to "I'm young (for my position)" to "I'm still young" (a
washed-up pitcher to his coach), and so on. Note the difference in the
past: \emph{Cuando eras joven} = "When you were young (a youngster)."
\emph{Cuando estabas joven} = "When you still had some pep (weren't over-the-hill)."

How about \emph{es buena} versus \emph{está buena} in reference to, say, a
film? Again, the key is in the subjective appreciation. After seeing it
and liking it, you might say \emph{Está buena la película} to communicate
your personal approval. You could use \emph{Es buena}, too, but there you'd
be declaring "It is a good film"---well done, professionally made, with
good actors, and the recipient perhaps of several awards. If after seeing
the film you say \emph{Es una buena película}, you are subtly implying "It
was good, but\ldots{}" In my experience, people seem to say \emph{Es una
buena película} in reference to arty films---ones they presume to be
"good" but didn't understand or particularly enjoy.

Some final examples will call attention to an advanced aspect
of the distinction with descriptive adjectives. If a person has a permanent physical illness (polio, for instance), he or she can still be described using \emph{está enfermo} or \emph{está enferma}. \emph{Está enferma desde niña} =
"She's been ill since childhood." Likewise, the adjective \emph{loco} is generally used with \emph{estar}, even when referring to someone who has spent
fifty years in an insane asylum. You would say \emph{está loco} of this person,
less commonly \emph{es loco}. Why? Because illness and insanity, as in these
cases, are not an essential part of the person's character but an exceptional, uncharacteristic condition. That is, it is not in their very nature
as people to be sick; it is a condition, a state, an exception. When adjectives like \emph{enfermo} and \emph{loco} are made into nouns, though, they are
used with \emph{ser}---but usually only with a preceding indefinite article: \emph{Es
un enfermo} = "He's sick (a sick person)."

The extreme application of the "essential nature" principle is
\emph{está muerto} and \emph{está muerta}, which is the only way to say "He (or
she) is dead." Students, understandably, balk at this one, since for most
of us death is considered a fairly permanent state, worthy of \emph{ser}. Actually, the use of \emph{estar} makes sense if you take the perspective of the
individual involved: being dead may be a lasting experience, but it's
not an essential aspect of the individual's nature. When the person is
remembered and eulogized years later, people won't say "He (or she)
was a good person, a kind person, and a dead person." Besides, \emph{Es
muerto} means "He is killed"---the use of \emph{ser} and a past participle in a
passive construction. An illustrative if redundant example containing
both would be \emph{Fue muerto a tiros, y ahora está muerto}, literally "He
was killed by shots, and now he's dead."

\section{Sorting out \emph{ser} and \emph{estar} in the imperative}

Imperatives are a source of some added confusion with \emph{ser} and
\emph{estar}. I've never found a good rule governing their usage in the imperative, so I'll invent one: avoid using either of them in the imperative,
but if you must, always use \emph{ser}. It's a fairly drastic rule, and exceptions
can of course be found if you want to get picky. But it will do for the
most part.

Why avoid imperatives with the "to be" verbs? Because Spanish, unlike English, does not lend itself to them as a rule. In English
you can without hesitation say "Be good," "Be on time," "Be there,"
"Don't worry, be happy," and so on. To translate these constructions
into Spanish, you would almost always resort to a verb other than the
"to be" verbs: \emph{Pórtate bien, Llega a tiempo, Asiste, Anímate}. If you
insisted on using a "to be" verb in Spanish, you would almost always
use \emph{ser}, even when referring to a transitory state: you would say \emph{Sé
amable}, for instance, to express "Be friendly," even if you meant it
only for a short while. \emph{Sé amable con tu abuela, sólo nos visita de vez
en cuando} = "Be friendly to your grandmother, she only visits us from
time to time."

Encouraging you to avoid using "to be" in the imperative in
Spanish is not to say that you won't hear it. It's not an especially common construction, but neither is it rare. Here are some examples you
may run across:

\bsk

\indu \emph{Sé puntual}. = "Be on time."

\indu \emph{Sé buena gente}. = "Be a nice guy (or gal)."

\indu \emph{Estate quieto}. = "Be still." (said to children)

\indu \emph{Estate callado}. = "Be quiet." (said to children)

\bsk

The imperative with both ser and estar is much more frequent
in the negative in Spanish:

\bsk

\indu \emph{No seas malo}. = "Be a pal"

\indu \emph{No seas tonto}. = "Don't be a fool"

\indu \emph{No seas imbécil}. = "Don't be a jerk."

\bsk

With estar the negative imperative almost always is constructed with the present participle:

\bsk

\indu \emph{No estés molestando}. = "Quit bugging me."

\indu \emph{No estés gritando}. = "Quit shouting."

\bsk

Even in the negative, though, imperatives tend not use a "to
be" verb at all, as we have seen:

\bsk

\indu \emph{No te enojes}. = "Don't be mad."

\indu \emph{No llegues tarde}. = "Don't be late."

\indu \emph{No te aloques}. = "Don't be crazy."

\indu \emph{No hagas ruido}. = "Don't be noisy."

\bsk

For the student, the \emph{ser-estar} confrontation is a real and constant struggle. Penetrating these verbs' distinct natures can be time consuming and, frankly, a real pain in the backside. Effort is definitely
required, but it is also repaid, since the intuitive understanding of
Spanish you gain in separating \emph{ser} from \emph{estar} will prove indispensable
to true fluency. As that dictionary quoted above goes on to say about
the \emph{ser-estar} problem: "If [foreigners] feel irritated with Spanish for
this difficulty, they should consider that the differentiation between
the essence and the state of things in everyday speech is but one more
demonstration---perhaps the most brilliant one---of the logical sense
of this language."

\section{A matter of perspective}

Learning how to make your English turn into correct Spanish
is sometimes a matter of mastering the vocabulary and sometimes a
matter of mastering a concept. But sometimes---rarely---it's a matter of
mastering a whole new way of looking at things. It's a matter, in short,
of effecting a change of perspective. Long after basic fluency has been
achieved, many foreigners still have trouble remembering to make this
change. Most foreigners, it could honestly be said, never make the
change completely.

Nowhere does this perspective problem crop up with greater
frequency than with the indispensable verbs \emph{llevar} and \emph{traer}. The key
to getting this distinction right is to remember and implement a very
basic rule: in Spanish you can't "bring" something from where you are
to where you aren't. If you are going to a dinner at a friend's house, you
must ask if you should "take" (\emph{llevar}) something (a salad, a bottle of
wine, etc.). "Bringing" is only for cases when something away from the
speaker is being moved toward the speaker.

In English we tend to play loosely with what is essentially the
same rule. That is, we use "bring" regardless of whether the implied
movement is toward the speaker or away from the speaker. If we are
going to a party, we will offer to "bring" a salad; we will "bring" a
cooler with us when we go on a picnic. In Spanish you have to "take"
the salad and "take" the cooler.

\bsk

Imagine the following phone conversation:

\bsk

\inda JOSÉ:

\indu "Hey, Carlos, I think I left my wallet at your house last night.
Could you bring it over today?"

\inda CARLOS:

\indu "Sure. I'll bring it over in the afternoon."

\bsk

Now, in Spanish:

\bsk

\inda JOSÉ:

\indu Oye, Carlos, creo que deje mi cartera en tu casa anoche. ¿Me la
puedes \emph{traer} hoy?

\inda CARLOS:

\indu Claro, te la \emph{llevo} por la tarde.

\bsk

It's important to pay attention to the distinction not just to
sound better but to avoid sounding rude and demanding when you're
not. Imagine, in our phone conversation, that Carlos had no reason
to go by Jose's house that afternoon---in fact, imagine that it was an
hour out of his way. Imagine further that they have a class together
at five o'clock at the university. Now Jose, by asking Carlos to \emph{traer}
the wallet, is being dreadfully uncouth. Jose is saying, in effect, "Bring
it to me here at my house" when he may have meant "Take it to me
there at school"---in which case he would have to say \emph{¿Me la puedes
llevar hoy?}

The same problem of perspective comes into play with \emph{ir} and
\emph{venir}. In English we can call home and ask if the plumber "came" that
day; we can say we'll fix the sink when we "come" home; when we are
called to the phone, we say "I'm coming." In all of these cases in Spanish, however, you have to use \emph{ir} ("to go"), not \emph{venir} ("to come"). To
rephrase the \emph{llevar-traer rule}, you can't "come" (with \emph{venir}) to a location that is somewhere other than where you are at that moment.
\emph{Venir} can only refer to your present location---where you are sitting or
standing or, in a larger sense, to the city or country you are in. \emph{¿Fue el
plomero hoy?} = "Did the plumber come today?" \emph{Arreglaré el lavabo
cuando vaya} (or \emph{llegue}) \emph{a casa} = "I'll fix the sink when I come home."
\emph{Voy} = "I'm coming (to the phone)."

\chapter{The Twilight Zone}

First, an explanation. Why title this chapter, which is about
the subjunctive mode in Spanish, "The Twilight Zone?" The answer is
simple and twofold. First, the concept of a hazy, ephemeral Twilight
Zone accurately conveys the spirit of the subjunctive. And, second, if
it were called "The Subjunctive Mode," no one would read it.

What is it about the subjunctive that inspires such fear and
loathing in students of Spanish? Mostly, it is the task of retraining the
mind to recognize a concept that has no readily obvious equivalent in
English. After all, it's bad enough that Spanish puts different endings
on its verbs to denote mode, tense, and person. But to invent a whole
new mode outright---one that needs endings all its own---is nearly
criminal.

Spanish, of course, did not invent the subjunctive. In fact, the
subjunctive is widely used in English, though not nearly as frequently
as it is in Spanish. Take, for example, a sign hanging in the Sears restrooms in Waco, Texas: "It is important to us that our restrooms be
clean." A nicer, neater subjunctive was never seen.

But in Spanish it's often hard to get a grasp of why the subjunctive is needed and when. Thus "The Twilight Zone." For that, essentially, is what the subjunctive is: the Twilight Zone of the verb universe. The subjunctive gets the job of describing "could-have-beens,"
"might-bes," and "maybe-never-weres.'~Anything that has happened,
is happening, or may happen on the borders of our consciousness gets
handled by the subjunctive. Without the subjunctive, Garcia Marquez
would read like Hemingway. The subjunctive is more than a verb
mode; it is a complete separate reality.

Although beginning and intermediate students of Spanish find
it difficult to believe, many English speakers who have learned to live
with the Spanish subjunctive will tell you that it can actually be quite
enjoyable. With the mere flick of a verb ending you can cast doubt or
aspersions, relegating a simple occurrence to a different realm of understanding. It is a realm you can flirt with and explore, avoid when
you want (sometimes), and revel in at will. Almost certainly you will
find yourself wanting it more and more as you explore the Spanish-speaking world. Magical realism finds its home here, as does the seeming surrealism of much of daily life in the Spanish-speaking world.

So how do you go about learning it? Basically, there are two
equally important approaches. One approach is to learn the
rules-especially with the Twilight Zone concept in mind. A second
approach is to learn the common cues for the use of the subjunctive.
Practice both, and the subjunctive will appear to you one day-perhaps
even in a dream. From that day forward, it will be your constant companion, your escape hatch into the unreal. Signpost up ahead: Subjunctive Mode!

\section{Indirect commands (shallow twilight)}

This group covers giving orders, asking others to do things,
and engaging in other bossy behavior. Thus in a sentence like "Tell the
mariachis to go away," the English infinitive "to go away" must be rendered in the subjunctive in Spanish. Why? Because the action of the
mariachis going away doesn't become a reality until they actually go
away. Until then, it must be considered an entirely suspect notion,
lurking off in the unknown: Will the mariachis go away? Will they
stay? How about if we pay them to go away?

We can lump implicit indirect commands in this category, including "wishing" and "hoping" that the mariachis will go away. "I
hope the mariachis go away" is, after all, nothing more than a cowardly version of "Tell the mariachis to go away." And, as in the first
example, simply hoping that they will go away is no guarantee that
they will actually do so. They may want to sing "La Bamba" again.
You have no way of knowing, so you have to rely on the subjunctive. In
passing, note that you won't always have an obvious "telling" or "hoping" verb directing the action. Sometimes, the order is implicit or impersonal, and it often begins with que all by itself: \emph{Que se vayan los
mariachis} = "Have the mariachis go away." Similarly, \emph{ojalá} will often
initiate a wishing construction: \emph{Ojalá que se vayan los mariachis} =
"I sure hope that the mariachis will leave."

\section{The eternal mystery (deep twilight)}

For this category, we must venture even further into the Twilight Zone. This is the realm of doubt, uncertainty, suspicion, and
downright disbelief. For example, \emph{Es posible que los mariachis se
vayan} ("It's possible that the mariachis will go"), \emph{Dudo que se vayan}
1"1 doubt that they will go"), \emph{No creo que se vayan jamás} ("I don't
think they are ever going to go"), and so on. In each of these cases, we
return to the elemental problem of the mariachis' departure as a mystery, an eternal uncertainty, an action belonging to a separate realm.

Statements of negation also lurk in this shadowy world. At
first, their presence here confounds us: aren't they statements of fact
and thus perfect candidates for the indicative? On closer examination,
however, we can see why they are here. Negations are declarations of
something that never happened, actions that only exist in somebody's
mind. Here, with a little study, we can see the careful distinction that
turns a harmless indicative statement into an unruly, ethereal subjunctive. \emph{No asalté el banco} = "I didn't rob the bank" Straightforward and
indicative: I didn't do it. But when the sentence structure forces us to
make the action of bank robbery stand alone, it acquires its true character---that of an action that was not. \emph{Niego que haya robado el banco}
= "I deny that I robbed the bank." Here "I robbed the bank" is a nonevent, an untrue claim, a load of nonsense. It simply didn't happen---I swear! And since it didn't happen, it must be exiled to that world
where all the things that never happened---the "could-have-beens,"
"might-bes," and "maybe-never-weres"---reside. In short, it must go
to the Twilight Zone.

A similar treatment awaits things you "don't believe" or
"don't think." Since you don't believe them, you certainly don't have
to consider them real. \emph{No creo que esté aquí} = "1 don't think she's
here." Her presence here is something that for you, the speaker,
doesn't belong in your universe of hard facts. Thus into the subjunctive
it goes.

\section{\emph{Que} cues}

By now, being a sharp reader, you will have noticed that every
use of the subjunctive so far has been preceded by a certain word: \emph{que}.
And you're thinking, "Hey, maybe I'm on to something." In fact, you
are---sort of. Que is a good cue for using the subjunctive, though not
an entirely reliable one. That is, almost every time the subjunctive
appears, there will be a \emph{que} preceding it. But \emph{que} will not always be
followed by the subjunctive every time it appears. Still, if you pay
attention to when you use \emph{que}, you will be on your way to spotting
opportunities for showing off your subjunctive.

The context surrounding \emph{que} is the deciding factor in whether
the subjunctive should indeed follow. And often this context is little
more than the proper combination of words with \emph{que}. Thus certain
impersonal expressions followed by \emph{que} almost invariably take the
subjunctive:

\bsk

\indu \emph{Es posible (probable, factible, concebible) que}

\indu \emph{Es mejor (conveniente, preferible, oportuno) que}

\indu \emph{Es importante (necesario, preciso, urgente, obligatorio,
forzoso) que}

\bsk

As a matter of fact, only a handful of common adjectives can
be placed between es and que and still produce the indicative. Some of
these exceptions are \emph{claro, obvio, evidente}, and the like, which stress
that a fact is a fact is a fact. \emph{Es obvio que estoy aquí} = "It's obvious
that I'm here." So obvious a fact certainly has no business in the Twilight Zone.

Many of the adjectives in the \emph{es} + adjective constructions
used above also have verb forms. With our old friend \emph{que} these verbs
also take the subjunctive:

\bsk

\indu \emph{Urge que} (it's urgent that)

\indu \emph{Conviene que} (it suits/appropriate)

\indu \emph{Precisa que} (it states/specifies that)

\indu \emph{Prefiero que} (I'd rather)

\bsk

Along these lines are other impersonal expressions that take
the subjunctive:

\bsk

\indu \emph{Más vale que} ([you/we'd] better)

\indu \emph{Lástima que} (it's a pity/shame/too bad)

\bsk

So do certain "impersonalized" or reflexive constructions:

\bsk

\indu \emph{Se espera que} (always; it is expected)

\indu \emph{Se cree que} (sometimes; it is believed that)

\bsk

If you change these constructions from impersonal to personal, in
most cases you will still need the subjunctive. Some common verbs
that, when followed by \emph{que}, usually require the subjunctive include
\emph{esperar, sentir, querer, pedir, mandar, dejar}, and \emph{permitir}.

Note that when there is no change in subject, the infinitive
can be substituted for the subjunctive clause, as it is in English. These
are the constructions you won't have any trouble with:

\bsk

\indu \emph{Quiero ir}. = "I want to go."

\indu \emph{Esperan ganar}. = "They hope to win."

\bsk

Changing the subject of the second clause will require the subjunctive,
however:

\bsk

\indu \emph{Quiero que vayas}. = "I want you to go."

\indu \emph{Espero que gane ella}. = "I hope she wins."

\bsk

With some indirect command verbs, especially \emph{mandar, permitir}, and \emph{dejar}, the imperative can also be rigged together with the
infinitive to avoid the subjunctive altogether:

\bsk

\indu \emph{Manda traer el dinero}. = "Send for the money to be brought."

\indu \emph{Déjale traer el dinero}. = "Let him bring the money."

\bsk

A more natural-sounding construction in these and other cases of indirect commands is simply starting your sentence with \emph{que} and following it with the subjunctive, as in the earlier example \emph{Que se vayan los
mariachis}. This equates with the English "Have\ldots{}," which is one of
the most common indirect command forms in English.

\bsk

\indu \emph{Que traiga el dinero}. = "Have him bring the money."

\indu \emph{Que venga a las seis}. = "Have her come at six."

\bsk

\emph{Que} is also a reliable cue for the subjunctive when paired with
other words to form certain conjunctions. Most textbooks will give
you a laundry list of these conjunctions, half of which you will probably never need. Here are the important ones to remember:

\bsk

\indu \emph{para que} = "so that," "in order that"

\indu \emph{a menos que} = "unless"

\indu \emph{a pesar de que} = "despite," "even though"

\indu \emph{antes de que} = "before"

\section{Non-\emph{que} cues}

Another common use of the subjunctive is generally not introduced by que, so you'll have to be alert for it. Instead, it uses cuando,
donde, como, and other adverbs. The best guide in this case is the English translation. When you could substitute "-ever," as in "whenever," "wherever," or "however," follow the adverb with the subjunctive
in Spanish. If it helps you remember, memorize one of the classic
lines used to challenge someone to a fight in Spanish: \emph{Cuando quieras,
donde quieras, y como quieras} ("Whenever you want, wherever you
want, however you want"). (The expression is reputedly in use as well
as a "pick up" line, so make sure the person you use it on knows
whether you're a lover or a fighter!) Most likely, though, you will be
called upon to use this subjunctive construction in these more mundane situations:

\bsk

\indu \emph{¿Cuándo quieres ir? Cuando tú quieras}. = "When do you
want to go?" "Whenever you want."

\indu \emph{¿Adónde vamos? Donde quieras}. = "Where are we going?"
"Wherever you want."

\indu \emph{¿Cuándo me vas a dar el dinero? Cuando yo quiera}, = "When
are you going to give me the money?" "Whenever I
feel like it."

\section{The subjunctive with \emph{ser}: \emph{sea}}

\emph{Ser} is also commonly used in "-ever" constructions, and expressions with \emph{sea} are good to slip into your conversational Spanish.
\emph{Cuando sea, como sea, donde sea}, and \emph{quien sea} are equivalents for
"whenever," "however," "wherever," and "whoever" when used alone.
Often more common in English is to use an "any-" word---"anywhere," "anyhow," and so forth:

\bsk

\indu \emph{¿Con quién quieres ir al cine? Can quien sea}. = "Whom
do you want to go to the movies with?" "With
whomever."

\indu \emph{¿Dónde quieres comer? Donde sea}. = "Where do you want to
eat?" "Wherever (anywhere)."

\indu \emph{¿Cómo quieres la carne: con salsa, sin salsa, can papas, sin
papas, término medio, bien cocida? Como sea}. =
"How do you want your meat: with sauce, without
sauce, with potatoes, without potatoes, medium, well
done?" "Any ol' way will do."

\bsk

Often, the best English translation of expressions using \emph{sea}
would be a slangy expression like "It's up to you," "You name it," "I
don't care," or "It doesn't matter." All of these can be conveyed by the
Spanish subjunctive.

Although \emph{ser} and \emph{querer} are the two commonest verbs used in
"-ever" expressions, virtually any verb can be used:

\bsk

\indu \emph{¿Cuándo vas a llegar a la fiesta? Cuando pueda}. = "When are
you going to get to the party?" "Whenever (as soon as)
I can."

\indu \emph{Yo quiero salir ahora. Bueno, lo que tú digas}. = "I want to
leave now." "Okay, whatever you say."

\bsk

Once you get a feel for the \emph{cuando quiera- donde quiera- como quiera complex}, you'll be close to mastering one of the trickiest uses of the subjunctive: a clause containing the adverb plus the
subjunctive to refer to the future. Here you should keep the Twilight
Zone idea in mind. In Spanish, for instance, you would say \emph{Cuando
termine el libro, te llamaré} for "When I finish the book, I'll call you."
In this case, \emph{cuando} is followed by the subjunctive because it refers to
an event in the future that may never happen. A meteor could strike
the reader one page from the book's end, so the notion of "when I finish the book" must be considered uncertain.

Only when you are referring to a habitual action should you
use the indicative. In these cases, note that you are not so much referring to the future as to the past. \emph{Cuando termino de leer en las mañanas, voy a la tienda} = "When I finish reading in the morning, I go
to the store." Here the indicative is safe because presumably you have
done this sort of thing before and thus know that it can happen and has
happened.

\section{The traveler's subjunctive}

A final note on the subjunctive is especially useful for those
traveling in the Spanish-speaking world. A common question format
for lost, bewildered, or just-curious travelers goes as follows: "Is there
a such-and-such near here that does such-and-such?" For instance, you
might want to ask, "Is there a store near here that sells wine?" or "I'm
looking for place where I can leave my luggage." In all cases like these
you must use the subjunctive in Spanish, since the place you are seeking mayor may not exist. Or, put another way, it won't exist until your
question is answered, "Yes, there is such a place." Until such an answer is given, the place belongs in the never-never world of the Twilight Zone.
\emph{¿Hay una tienda por aquí que venda vino? Busco un lugar
donde pueda dejar mi equipaje}. The same logic applies when you ask
about people. ¿Hay alguien aquí que hable inglés? = "Is there anyone
here who speaks English?"

When the answer is in the negative, the place remains nonexistent and therefore must still be referred to in the subjunctive. \emph{No hay
una tienda cerca que venda vino} = "There is no store nearby that
sells wine." This is, after all, but a simple statement of negation, like
the ones we saw above. Ditto for nonexistent people: \emph{¡Na hay nadie en
esta ciudad que me entienda!} = "There's no one in this city who understands me!"

\chapter{Sixty-four Verbs, Up Close and Personal}

Pocket dictionaries will generally give you simple, one-word
equivalents for Spanish verbs. Better dictionaries will give you a list of
other possible meanings and maybe some examples. But rarely is the
dictionary reader given guidance on what usages are common---i.e.,
worth the bother of learning---and which are poetic or archaic and thus
irrelevant. And besides, who wants to read a dictionary?

What follows is a pared-down list of sixty-four Spanish verbs
whose basic meaning you probably already know, but whose inner secrets and common usage go far beyond that. The list cuts through the
clutter and highlights unexpected usages that a native English speaker
may not be on the lookout for or that are "out of character" for a given
verb. Each section also explains a chosen few idiomatic expressions,
selected for their frequency in everyday spoken Spanish. There's a lot
to absorb here, but the alphabetical listing will allow for hour after
hour of repeated consultation. Take heart---it could be worse. You
could be reading the dictionary!

\section{\emph{acabar}}

This is a synonym for \emph{terminar} and, like that word, is a good
equivalent for most uses of "to end" and "to finish" in English. Used
with \emph{con} as an intensifier, both verbs work well for "to finish off":
\emph{Acabé con la leche}. Often you will hear \emph{acabarse} in the reflexive to
mean "to run out of." \emph{Se nos acabó el dinero} = "We've run out of
money," \emph{Se acabo} by itself, meanwhile, means either "It's over" or
"I'm out of it," You'll hear it a lot in stores, at newsstands, and the like
when what you want to buy is no longer in stock. \emph{Terminar} is likewise
used this way; \emph{agotar} is also heard, especially in the phrase \emph{Está agotado} ("We're out of it"). \emph{Está acabado} usually isn't used in this sense,
since its meaning is closer to "It's finished" or, colloquially, "He's
washed up." \emph{Acabar} has one other very common use that \emph{terminar}
doesn't have: in the present tense, with \emph{de}, it means "to have just,"
as in Acabo de comer ("I've just eaten"). Used in the imperfect, it becomes "had just." \emph{Acababa de comer cuando llegaste} = "I had just
eaten when you arrived."

\section{\emph{amanecer}}

This somewhat uncommon word, which means "to dawn," is
included here because of one expression that has perplexed generations
of language students, especially those who live for a time with a
Spanish-speaking family when studying abroad. The expression is
\emph{¿Cómo amaneciste?}---which translates literally as "How did you
dawn?" but which means "How did you sleep?" The answer is (usually) \emph{Bien, gracias}. A fun Spanish expression for "to die in one's sleep"
uses this verb: \emph{Amaneció muerto}; if it translates at all, it would have
to be rendered "He woke up all dead."

\section{\emph{andar}}

Dictionaries say it means "to walk," which of course it does,
but that won't help you when you hear \emph{anda corriendo} or \emph{anda en
coche} for the first time. In English we would probably be inclined
to say "go around" for most uses of \emph{andar}. \emph{Pedro anda gritando tu
nombre} = "Pedro's going around shouting your name." \emph{Andar} also
covers slangy expressions like "to hang out" or "to hang around." \emph{Ya
no ando con ellos} = "I don't hang around with them anymore." \emph{¿Por
dónde andas?} works well for "Whereabouts are you?" or the colloquial
"Where are you at?" And in some countries \emph{anda} lends itself to the
common idiomatic expressions \emph{¡Ándale!} and \emph{¡Anda!} Said with vigor,
they mean "Let's get a move on!" or "Way to go!" Said in passing, they
mean the same as "okay" or "all right." In Mexico, for instance, you'll
hear \emph{ándale} all the time for "that's fine," "that's right," and even
"good-bye": \emph{Nos vemos mañana. Ándale}. Throw a \emph{pues} on the end
and you'll be saying nothing at all but will sound very fluent: \emph{Ándale
pues} ("Have a nice day"). Remember as well the use of \emph{andar} for "to
run" or "to work" in reference to objects. \emph{¿Qué tal anda tu coche?} =
"How's your car running?" (literally, "How's your car walking?").
Some wags have even argued that the different conception of time in
Spanish-speaking countries is due to the fact that in Spanish clocks
walk rather than run!

\section{\emph{antojar}}

Used generally as a reflexive (\emph{antojarse}), this is an exceptionally common verb and one that you should get to know well. To translate it, dictionaries offer "to long for" or "to desire earnestly," but its
use in Spanish covers a lot more ground than that. Closer to the mark
would be "to get a hankering for" or simply "to feel like." An \emph{antojo}
is a "craving" or an "urge" and covers both intense longings (like the
kind that make pregnant women eat pickles) and simpler pleasures.
You'll probably run across the verb frequently in these same situations. Some examples: \emph{Se me antoja una pizza} = "I'm dying for (could
go for) a pizza." \emph{¿Por qué no vas a ir al cine? Porque no se me antoja}.
= "Why aren't you going to the movies?" "Because I don't feel like it."
\emph{Déjame una dona; luego se me va a antojar} = "Leave me a doughnut;
I'll probably feel like having one later."

\section{\emph{bajar}}

This verb means "to go down," "to put down," "to get off,"
and so on. Most of its uses are predictable, but a few that may not be
include "to go downstairs," "to get out of a car (bus, train, etc.)," and
"to lose weight" (\emph{bajar de peso}). It also means "to get (something)
down," as when you ask someone to get your suitcase down off the
rack (\emph{¿Me baja la maleta, por favor?}). \emph{Bájale}, by itself, is usually "Turn
it down," referring to the volume or the general noise level; in the
right context, it can also mean "Slow down."

\section{\emph{caber}}

An irregular verb that you should learn. It means "to fit," but
only in the sense of "fit into" or "fit onto." It is not used for clothing.
In the first-person present, it's \emph{quepo}, and you're likely to hear it in
\emph{¿Quepo yo?}---meaning "Will I fit?" or "Is there room for one more?"
Otherwise, you may run into it in set expressions like \emph{cabe decir} ("it's
worth mentioning") and \emph{no cabe duda} ("there's no doubt").

\section{\emph{caer}}

It means "to fall," of course, and "to drop" when used reflexively (\emph{caerse}): \emph{Se me cayó el vaso} = "I dropped the glass," It's also
very frequently heard in the phrases \emph{caer bien} and \emph{caer mal} to express
likes and dislikes (see Chapter 4). You may also run across \emph{caer} for "to
visit unexpectedly" or "to drop in on." \emph{Te caigo en la tarde} is an informal way of saying "I'll drop in on you in the afternoon." Sometimes
it's used to suggest that someone's arrival was not only unexpected but
also unwelcome. "What are your in-laws doing here?" might be answered by \emph{Es que me cayeron} ("They just kind of showed up").

\section{\emph{cambiar}}

Meaning "to change" as well as "to make change" in the sense
of "Can you change a twenty?" (\emph{¿Me puede cambiar un billete de a
veinte?}), \emph{cambiar} also crops up in a number of common expressions.
These include \emph{cambiar de idea} or \emph{cambiar de opinión} ("to change
one's mind"), \emph{cambiar de ropa} ("to change one's clothes"), and \emph{cambiar de casa} ("to move").

\section{\emph{coger}}

This is one of those words that many dictionaries handle with
more discretion than clarity. The simple fact is that \emph{coger} is a vulgar
term for "to fornicate" in several countries (Argentina, Mexico, Uruguay, and others), where as a result it is rarely used in proper company
(see Chapter 10). That said, it is also one of the most commonly used
verbs in some other countries (especially Spain). What's a poor student
to do when faced with the choice? That will depend on where you are
learning the language and with whom you expect to be communicating. But if you want to use substitutes for coger right from the start---in the sense of "to get," "to take," "to grab"---it may not be such a bad
idea. The word that usually replaces it is \emph{tomar}, as in \emph{tomar el tren}. In
Mexico, particularly, \emph{agarrar} is often heard. Both substitutes are 
understood even where \emph{coger} is used, and both can save you considerable
embarrassment.

\section{\emph{conocer}}

This verb is often confused with \emph{saber} by students of Spanish;
both mean "to know," but \emph{conocer} is used in the sense of "to be familiar with." A limited rule of thumb: use \emph{conocer} for proper nouns and
all specific people, places, and things; use saber for everything else. Do
you know Paris? \emph{Conocer}. Do you know the French Quarter? \emph{Conocer}.
Do you know the old lady there who sells flowers on the street? \emph{Conocer}. Do you know her name? \emph{Saber}. Do you know what flowers she
sells? \emph{Saber}. Do you know what she's saying about you? \emph{Saber}.

\emph{Conocer} also means "to meet," but keep in mind that it only
works for the first time you meet someone. English speakers use "to
meet" to describe routine encounters, such as "I met my mother at
the train station." They also tend to say \emph{la primera vez que lo conocí}
to convey "the first time I met him" (instead of the less redundant
\emph{cuando lo conocí}). In Spanish you can only \emph{conocer} someone once, and
needless to say it would be difficult (not to mention dramatic) to \emph{conocer} your mother at a train station. Finding the right Spanish word for
"to meet" in these other situations will give you an idea of how overworked this poor verb is in English. When you are referring to a first
meeting, as noted, use \emph{conocer}. \emph{Lo conocí en París} = "I met him (for
the first time) in Paris." All chance encounters after that are handled
by \emph{encontrar}. \emph{La encontré en el cine} = "I met (ran into) her at the
movies." Meeting a plane or a train would be covered by \emph{recibir}. \emph{Me
recibieron en la estación} = "They met me at the station." For a
planned get-together you would use \emph{quedar en verse con} or \emph{quedar en
encontrarse con}: \emph{Quedé en verme con unas amigas en el centro} = "I
met (up with) some friends downtown."

\section{\emph{creer}}

It means "to believe," but it is also almost always the word
you want for "to think." Beginning students, incorrectly, often prefer
\emph{pensar} (see below). The distinction is subtle, but it works out roughly
as follows: if the emphasis is specifically on the thought process or the
act of thinking, use \emph{pensar}. If you're stating a personal belief or opinion, use \emph{creer}. Thus \emph{Creó que tienes la razón} = "I think you're right."
\emph{Creer} is also used in many interjections and phrases, such as \emph{¿Qué
crees?} ("Guess what?"), \emph{Créeme} ("Trust me"), \emph{¿Tú crees?} ("You really
think so?"), and so on. A good phrase to learn is \emph{ni creas}, which could
be translated as "don't expect" or "no way." \emph{Ni creas que te voy a ayudar} = "You're crazy if you think I'm going to help you" or "Don't expect me to help you," \emph{Creo que sí} and \emph{Creo que no}, finally, should be
on the tip of your tongue for "I think so" and "I don't think so."

\section{\emph{cuidar}}

Technically, this means "to care for," though in English we
would generally use some other word to translate it. \emph{Ana se quedó
para cuidar a los niños} = "Ana stayed home to watch (take care of)
the kids." \emph{Cuide su cartera en ese barrio} = "Watch your wallet in that
neighborhood." \emph{Cuídate} is "Take care of yourself" or, slangily, "Take it
easy"; it is sometimes used as a parting comment. \emph{Cuidar} in general is
the word you want for asking someone to "watch over" or "to keep an
eye on" something, as when you want to leave your luggage in the bus
station for a few minutes (not a recommended practice). \emph{¿Me puede
cuidar la maleta unos minutos?} = "Can you keep an eye on my bag
for a few minutes?" See also \emph{guardar}.

\section{\emph{dar}}

'To give" and much more. A common additional meaning of
dar is "to hit," giving rise to a number of expressions. \emph{Dar en el blanco}
is "to hit the bull's-eye" and is often heard for "to guess right," "to hit
the nail on the head." \emph{Dar en la torre} is a common idiomatic expression that covers physical beatings as well as more metaphorical thrashings. "How did the Cubs do against the Mets?" \emph{Le dieron en la torre}
("They beat the pants off 'em"). \emph{Dale}, by itself, is "Hit it" (or "Hit
him," "Hit her"); you can let off steam by shouting it at boxing matches.
More metaphorically, it means "Give it your best" or "Give 'em hell."
Often, in this sense, it's paired with \emph{duro} and said to give encouragement: \emph{¡Dale duro, Juan!} ("Give 'em hell, Juan!"). By the same token,
you can use \emph{dándole duro} to convey the intensity of an action or an
effort. "How are you coming with the term paper?" \emph{¡Dándole duro!}

Here are other dar expressions you'll want to know:

\bsk

\indu \emph{dar a la calle} (\emph{al patio, a la alberca}, etc.) = "to face the street
(the patio, the pool, etc.)"

\indu \emph{da igual} or \emph{da lo mismo} = "it's all the same to me" or "it
doesn't matter"

\indu \emph{dar coraje} = "to make mad"

\indu \emph{dar (la) lata} = "to pester" or "to be a pain"

\bsk

\emph{Dar} is also used in some parts of the Spanish-speaking world for
"what's on"---in the sense of showings on television or at the movie
theater. \emph{Están dando la segunda parte de} Arma Mortal = "They're
showing the second part of \emph{Lethal Weapon}." (See also pasar.) Finally,
though you won't find it in the dictionary, an increasingly common expression in American Spanish is \emph{dar chance} for "to give a break." \emph{Vamos, oficial, denos chance} = "Come on, officer, give us a break." Purists will tell you that this is a horrible barbarism and that you should
say \emph{denos una oportunidad} instead. But purists should consider cutting their losses, since more and more speakers of slang are already bypassing \emph{dar chance} in favor of \emph{dar un break}, which is more barbaric yet.

\emph{Darse}, the reflexive form, is also used for a few handy phrases,
such as \emph{darse cuenta de}, which means "to realize." \emph{Perdón, no me di
cuenta de que estaba estacionado en su pie} =: "Sorry, I didn't realize
I was parked on your foot." \emph{Darse por vencido} is the phrase you want
for "to give up," "to surrender." To say "I give up" in response to a
riddle, for instance, you can even use \emph{Me doy} (more formally, \emph{Me rindo}) all by itself.

\section{\emph{decir}}

"To tell" or "to say." You'll probably also be using it a lot in
the phrase \emph{querer decir}, or "to mean." \emph{¿Qué quiere decir esa palabra?}
= "What does that word mean?" \emph{Decir} also pops up in a lot of cute
little phrases, such as \emph{No me digas} ("You don't say), \emph{Dime} ("Tell me"
or simply "Yes?"), and \emph{¿Qué decías?} ("What were you saying?"---after
an interruption). You should also be alert to \emph{Dile} (and \emph{Dígale}) as a typical preface to an indirect command, and thus the subjunctive. \emph{Dile que
venga} = "Tell him to come here." If you're stating a fact instead of
issuing a command, it takes the indicative. \emph{Dile que estamos aquí} =
"Tell him (or her) we're here."

\section{\emph{dejar}}

Meaning "to leave" or "to let," this verb is often found in
other expressions and is not always a reliable vehicle for English "let"
phrases. For example, \emph{Déjenos entrar} means "Let us in," but \emph{Entremos}
(or \emph{Vamos a entrar}) means "Let's go in." \emph{Dejar} is heard in such phrases
as \emph{Déjame en paz} ("Leave me alone"), \emph{Déjalo} ("Leave it" or "Drop it"
or "Skip it," often referring to a sensitive topic or one best dealt with
later), and \emph{Deja ver} or \emph{Déjame ver} = "Let me see."

\section{\emph{disfrutar}}

"To enjoy." Only mentioned here to dissuade you from saying
\emph{Disfrútense} for "Enjoy yourselves." It does mean this, but probably not
in the way you intend. "Have fun with yourselves" or "Take pleasure
in yourselves" would probably be a more accurate translation. Instead,
you should say \emph{Que lo disfruten}, usually in reference to a specific
event. For the best translation of "Enjoy yourselves," forget about \emph{disfrutar} altogether and use \emph{divertirse}: \emph{Diviértanse} or \emph{Que se diviertan}.

\section{\emph{dormir}}

\emph{Dormir} is "to sleep" while \emph{dormirse} is "to fall asleep." Just a
reminder: "sleep," the noun, has to be expressed by \emph{sueño} (see \emph{soñar}).

\section{\emph{durar}}

This is one of the words you will need to ask how long things
take (movies, bus rides, flights, speeches, etc.). You can think of \emph{durar}
as a generally safe translation of "to last." \emph{La fiesta duró toda la noche}
= "The party lasted all night." Generally \emph{durar} is used for things that
have a specific duration---or \emph{dura}tion, if it helps you remember. (See
also \emph{tardar} and \emph{hacer}.)

\section{\emph{echar}}

\emph{Echar} is one of those words that take up about three pages in
the dictionary. Not all of those expressions are that common or useful,
though. Almost all of them have to do with a forcible casting out or
expulsion. File away this division of labor: \emph{meter} is for putting in, \emph{sacar} for taking out, and \emph{echar} for kicking out, more or less. Buses \emph{echan}
smoke, teachers \emph{echan} students, and so on. An unexpected use of
\emph{echar} is for "to pour," as with a liquid into a glass or from a pitcher.
Some of the idiomatic expressions using \emph{echar} are very handy. \emph{Echar
de menos a} is "to miss (someone)" (though in the Americas you are
more likely to hear extrañar). \emph{Echar a perder} is "to spoil," be it children or food in your refrigerator. \emph{Echar ganas} is a very common
expression for "to show enthusiasm" or "to give it a good effort: "Mom,
I can't understand my math homework." \emph{Vamos, échale ganas, hijo}.
\emph{Echar una mano} is the expression you will need for "to lend a hand,"
and \emph{Échame la mano} is "Gimme a hand." \emph{Echar un ojo} is "to have a
quick look," and \emph{echar la culpa} is "to blame."

\section{\emph{encargar}}

This verb, meaning "to entrust" or "to commission," is far
more common than these awkward English translations suggest. Almost always it conveys some notion of "charge"---to take charge, to be
in charge, to charge, and so on. With \emph{de}, for instance, it means "to take
charge of" or "to put someone in charge of" something: \emph{Yo me encargo
de la ensalada} = "I'll be in charge of (take care of) the salad." \emph{El encargado} is the all-purpose word for "the guy in charge." Without the
\emph{de}, \emph{encargar} means "to entrust" but often with more colloquial English equivalents. For example, if someone's going off to the store, you
might say \emph{¿Te encargo unas aspirinas?} ("Can you get me some aspirin?"). Many usages could almost be translated "to order." \emph{Le encargué
dos litros al lechero} = "I ordered two liters from the milkman." Sometimes, what you're ordering or entrusting is not spelled out, and "to
count on" would make for a better fit. If someone offers to fix your car
by nightfall and you plan to leave town that very night, you might say
to the mechanic, \emph{Se lo encargo mucho} ("I'm really counting on you").

\section{\emph{equivocar}}

Used as a reflexive (\emph{equivocarse}), this is the common verb for
"to make a mistake" (although \emph{estar equivocado} works as well). Usually used with the preposition \emph{de}, it means "to get something wrong."
For instance, when you've dialed the wrong number, you would say,
\emph{Perdón, me equivoque de teléfono} (or \emph{número}). If Pedro shows up on
the wrong day for a party, you would tell him, \emph{Te equivocaste de día}.
And so on. As a subtle distinction, \emph{estar equivocado} generally suggests someone is mistaken; \emph{equivocarse} means he or she has made a
mistake.

\section{\emph{esperar}}

Meaning both "to hope" and "to wait," the verb will usually
take the subjunctive in both senses. \emph{Espero que venga} = "I hope he (or
she) comes." \emph{Estoy esperando que regrese} = "I'm waiting for him (or
her) to get back." Unless the context makes it clear, you would use the
preposition \emph{a} with \emph{esperar} to say "to wait for." Thus, \emph{Espero a que
regrese} = "I'll wait for him (or her) to get back." (\emph{Esperar por} or \emph{esperar para} should never be used for "to wait for.") \emph{Esperar} is frequently
heard in the imperative for "Wait a minute" or "Hold on." \emph{Espérese,
por favor} is the form you're most likely to hear. In familiar usage it's
\emph{Espérate}, which often comes out sounding like \emph{'pérate}.

\section{\emph{estar}}

The other verb for "to be"---the one that covers transitory
states. The \emph{ser} versus \emph{estar} confrontation was covered in detail in
Chapter 5. So here it's enough to glance at one use of the verb you
might not have been expecting---about the only one that is out of character for it from the standpoint of an English speaker---\emph{estar} when
used to ask "What's today's date?": \emph{¿A qué estamos hoy?} or \emph{¿A cuántos
estamos?} The answer is phrased \emph{Estamos a} and the number: \emph{Estamos
a 25}, for example. Remember too that \emph{estar + de} is used for moods
and inclinations. \emph{El está de malas} = "He's in a bad mood." (This use
is covered in Chapter 4.)

\section{\emph{estrenar}}

This is not the commonest of verbs, nor is it one that you desperately need to learn. It means to use or display something for the
first time---to "debut" something, as it were. \emph{Juan está estrenando su
nuevo coche} = "Juan is trying out his new car." \emph{Voy a estrenar la camisa que me regalaste} = "I'm going to wear (for the first time) the
shirt that you gave me." \emph{Un estreno}, referring to films and theater, is a
"premiere" or "opening"; referring to an artist, it would be a "debut."

\section{\emph{guardar}}

Meaning "to guard" or "to save," this is the common verb to
use for telling someone to "hold on to" something or "put (something)
away." \emph{Guardar} is usually used when you give someone something
that you want them to put away until you need it again. If you're going
swimming and don't want to take your traveler's checks into the pool
with you, you might hand them to a friend and say \emph{¿Me los guardas?}
\emph{Guardar} is also commonly used by parents to tell their children, "Put
it away" (\emph{Guárdalo}), be it in their closet or in their pocket. \emph{Guárdame
un poco} is "Save a little for me," said of a favorite foodstuff, for instance. (See also "Save" in Chapter 11.) The word is a little tricky to
use in the sense of its English cognate. \emph{¿Me guarda la maleta?} could be
"Keep an eye on my suitcase" but also something like "Put my suitcase away"; so if you tell someone to "guard" it and they walk off with
it, you'll know why. Use \emph{cuidar} (see above) for "to keep an eye on,"
"to watch."

\section{\emph{haber}}

This is one of the megaverbs, with more uses than you might
ever be inclined to learn. Some are indispensable, though. \emph{Hay}, of
course, is the way to say "There is\ldots{}" and "There are\ldots{}" Say it
with a lilt and it becomes the questions "Are there\ldots{}?" and "Is
there\ldots{}?" \emph{No hay} is the typical curt response to questions about
availability: \emph{¿Hay café? No hay. ¿Hay cuartos? No hay}. \emph{Hay que} is the
impersonal way of saying "to have to"---that is, when it's not obvious
exactly who has to do something. \emph{Hay que ir a España para aprender
español} = "One has to go to Spain to learn Spanish." The imperfect
and preterit forms of \emph{haber} are \emph{había} and \emph{hubo}, respectively. \emph{Había}
is the more commonly used of the two. \emph{Había veinte personas en el
coche} = "There were twenty people in the car." \emph{Hubo} is for something that was there all at once or not for long. \emph{Hubo un choque en la
carretera} = "There was an accident on the highway." Remember never
to use \emph{habían} or \emph{hubieron} for "there were," regardless of the number
of persons or things involved: \emph{Había una monja en la lancha} and \emph{Había dos monjas en la lancha}.

\section{\emph{hablar}}

A straightforward word for "to talk" or "to speak," but keep
it in mind for use on the telephone, where its use is rampant. It's not
hard to imagine a phone conversation going something like this:
(R-i-i-i-ing.) \emph{Hola. ¿Quién habla? ¿Con quién quiere hablar? Habla
Juan. ¿Puedo hablar can Fred? El habla. ¡Ah! ¡Hablas español!} Of
these, the \emph{El habla} (or \emph{Ella habla}) response is the most important to
get straight. It will spare you countless episodes of saying \emph{Soy él} (or
\emph{ella}) or \emph{Hablando}, both of which are incorrect when you mean "This
is he (or she)" or "Speaking."

\section{\emph{hacer}}

Hacer means "to make" or "to do," but you know that already.
Where English speakers have to remember to use \emph{hacer} is in the many
weather-related expressions that in English are covered by "to be."
Some of the things that "make" in Spanish are \emph{frío, calor, viento}, and
\emph{sol} ("cold," "hot," "windy," "sunny"). \emph{Hacer} is also the way to say
"ago" in Spanish: \emph{Hace dos años nació mi hijo} ("My son was born two
years ago"). When did they leave? \emph{Hace un rato} ("A little while ago").
To say "We did it!" or "We made it!" in Spanish, you say \emph{¡Lo hicimos!}
With \emph{la} instead of \emph{lo} and intensified with \emph{ya}, it becomes a colloquial
\emph{¡Ya la hicimos!} ("We've got it made!"). Don't trouble your mind
searching for an antecedent to la---there is none. A very useful phrase
with \emph{hacer}, finally, is \emph{Hazte de cuenta} (or \emph{Haz de cuenta}), which introduces a thought with "Pretend\ldots{}" or "Let's say\ldots{}" (See also
Chapter 8 under "\emph{Haz de cuenta que}.")

\emph{Hacerse}, in the reflexive, also turns up in a lot of unexpected
expressions. It's a common way of translating "to become" and is also
common idiomatically for "to make like" or "to act like." \emph{El se hace el
payaso} = "He's acting like a clown." \emph{Se hace el loco para no ir a la
cárcel} = "He's pretending to be crazy to avoid going to prison." You'll
often encounter this expression in the negative as an exhortation.
\emph{No te hagas tonto} = "Don't play stupid." \emph{No te hagas la víctima} =
"Don't play the victim." In Mexico especially you'll encounter \emph{No te
hagas} all by itself, with the predicate understood. This is a good translation for common interjections like "Come off it!" or "Don't gimme
that!" or "Cut it out!"

\section{\emph{ir}}

The verb for "to go" comes into play in many situations that
parallel English usages, but you'll have to be on the lookout for them
to learn to use them well. It's widely used, for instance, for negative
imperatives, otherwise known as warnings, equating approximately
with the English "Don't go ... ": \emph{No te vayas a meter en líos} =
"Don't go getting yourself in trouble." \emph{No le vayas a decir} = "Now
don't go telling him" or just "Don't tell him." In many compound verb
forms, \emph{ir} is extremely useful to get your point across. It's used, as in
English, in the future (\emph{Voy a llamar} = "I'm going to call") and in the
imperfect (\emph{Iba a llamar} = "I was going to call"). \emph{}And \emph{Vamos a} means
"Let's\ldots{}" Other common expressions with \emph{ir} include \emph{irle a} ("to root
for," as in \emph{Yo le voy a los Orioles}), \emph{ir por} ("to go get" or "to go for," as
in \emph{Voy por el coche}), and \emph{Ahí te va} (meaning "Catch" or "Your turn").
\emph{Vaya} by itself is "All right" or "Omigosh," depending on tone and context. \emph{Vaya, vaya, vaya} is the common way to say "Well, well, well," as
in "What have we here?" \emph{Vaya} plus a noun is the equivalent of the
sarcastic comment "Some\ldots{}!" If you return home to find the plumber
has managed to flood the entire basement, you might say sarcastically
\emph{¡Vaya plomero!} ("Some plumber!").

\section{\emph{lograr}}

This is the word you want for "to manage" when used with
another verb in the infinitive. \emph{Logré reparar la tele} = "I managed to
fix the television." \emph{Si logro salir de esta reunión, estaré en casa en
media hora}, = "If I manage to get out of this meeting, I'll be home in
half an hour."

\section{\emph{llevar}}

\emph{Llevar} means "to carry" or "to take" and is often used for
"to bring," as we saw in Chapter 5. Here we'll concern ourselves with
\emph{llevar} in those expressions you'll want to have handy in your daily
doings. One common one is \emph{llevar} for expressions of time. A good way
to answer the inevitable question heard abroad, "How long have you
been here," is to say \emph{Llevo} + the number + \emph{meses} (\emph{años, días}) \emph{aquí}.
The question, in fact, will often be phrased \emph{¿Cuánto tiempo lleva
(usted) aquí?} Once you get a feel for this usage, you'll find yourself
needing it more and more. \emph{Llevo dos días en cama} is much smoother
and more colloquial than \emph{He estado en cama durante} (or \emph{desde hace})
\emph{dos días} for "I've spent two days in bed." \emph{Llevar} is also useful for "to
wear" or "to have on." \emph{Es el señor que lleva 1entes} = "He's the man
with the glasses on."

\emph{Llevarse}, the reflexive, is also handy for a couple of expressions. A common one is for "to get along." \emph{Ella y yo no nos llevamos}
= "She and I don't get along." \emph{No me llevo con ellos} = "I don't get
along with them." Another one you'll need, especially for shopping,
is \emph{llevarse} for "to take" something (after paying for it, naturally). It
means the same as llevar essentially, but the reflexive is added for emphasis. \emph{¿Dos mil pesos? Me lo llevo} = "Two thousand pesos? It's a
deal." \emph{Me llevo dos} = "I'll take two." \emph{Llévatelo} is "Take it," pure and
simple. Any usage that translates as "to take along" gets the reflexive
as well: \emph{Me llevo un suéter por si hace frío} = "I'll take a sweater in
case it gets cold."

\section{\emph{mandar}}

Meaning "to send" or "to order," this word can cause problems because of other words that can get the same message across.
Most uses of "to send," for instance, work better with \emph{enviar}, and "to
order" in a restaurant is \emph{pedir} or (increasingly) \emph{ordenar}. \emph{Mandar} is
used for real orders, of the sort that generals and bosses give, and you'll
come across it a lot in conjunction with a second verb. In these cases
it works as "to send out for" or "to have done." \emph{Manda hacer unas copias} = "Have some copies made." It usually implies an order to an underling, so don't use it freely unless you are in a position either to give
orders or to take orders. If you go to Mexico, your first encounter with
\emph{mandar} may be the ubiquitous expression \emph{¿Mande?} for "What?" or
"You called?" It's considered polite, but many foreigners (especially
from other Spanish-speaking countries) seem to think it servile and demeaning. If you feel that way, you can substitute \emph{¿Cómo?} or even the
brusque \emph{¿Qué?}---but you'll hear \emph{¿Mande?} just the same.

\section{\emph{meter}}

Meter, meaning "to put (something) in," is used in a far wider
range of circumstances. It's the common way of saying "to go inside"
(\emph{Vamos a meternos} = "Let's go inside") or "to go in" (\emph{Vamos a meternos al agua} = "Let's go in the water"). It can even be used for "to get
in" a car (as in \emph{Métete al coche}), but \emph{subirse} is preferred. The reflexive
form \emph{meterse} is also a good translation for "to get involved in" or "to
get mixed up with." \emph{No te metas} = "Don't get involved." \emph{No te metas
can mi hermana} = "Don't mess around with my sister." \emph{Meterse en
líos} takes the idea further and means "to get mixed up in problems,"
"to get into trouble." If you get caught in the middle of a family
squabble and find opposing sides of the squabble looking to you for
support, you might throw up your hands and say \emph{Yo no me meto} ("I'm
not getting involved in this").

\section{\emph{notar}}

This verb is worth learning in the reflexive form (\emph{notarse}) to
express "I can see that" or "It shows." It's a nice, dry comment that
says that you, too, can perceive the obvious. If someone in the midst of
a downpour reminds you that it's the rainy season, you might respond
\emph{Se nota} ("I figured that out" or "But of course"). Adding the personal
pronouns \emph{me, te}, or \emph{se} personalizes the phrase. Your friend, screaming,
tells you she's angry. You say, \emph{Se te nota} ("So I see," literally "One
notes that in you.") \emph{Notar}, incidentally, is not a good word for "to
note something down" or "to make a note of." For that, use \emph{anotar}
or \emph{apuntar}.

\section{\emph{parar}}

"To stop." \emph{Pare el mundo, quiero bajarme} = "Stop the world,
I want to get off." \emph{¿Dónde para el tren?} = "Where does the train stop?"
And so on. \emph{Párale} is sometimes employed to say "Stop it" or "Cut it
out," as when someone is talking too much or the kids are screaming.
In most of the Americas (and even parts of Spain), the reflexive \emph{pararse}
means "to stand up," and \emph{parado} is "standing up." Travelers will
sometimes ask if there is room on a train or bus and be told \emph{Si quiere
ir parado} ("If you want to travel standing up"). \emph{Párate que nos vamos}
would be a colloquial way of saying "Get up, we're going." Learn to
distinguish between \emph{parar} and \emph{pararse} for "to stop." The reflexive
form is appropriate for stopping unassisted, whereas \emph{parar} suggests
something stopping something else. \emph{Paré el coche} is "I stopped the
car." \emph{El coche se paró} is "The car stopped."

\section{\emph{parecer}}

Meaning "to seem," this verb is more frequently encountered
than its English equivalent. One of the most common ways of conveying likes and dislikes in Spanish, in fact, is with \emph{parecer}. Here are
some typical usages: \emph{¿Qué te parece?} = "What do you think?" or
"How does that strike you?" \emph{¿Te parece?} = "Is that okay with you?"
\emph{No me parece} = "I don't like it." \emph{Me parece bien} = "Fine with me."
The reflexive \emph{parecerse} is "to look like" or "to resemble." \emph{Me parezco a mi madre}
 = "I take after my mother." "To look like" in the sense of
 "to look as if" requires \emph{parece que}, not the reflexive. \emph{Parece que va a
llover} = "It looks like (as if) it's going to rain."

\section{\emph{pasar}}

One of several options for "to happen" (see Chapter 11), \emph{pasar}
is also usually safe for most uses of "to pass," although it's a bit slangy
at the dinner table (\emph{Pásame la sal}). One use you may not be expecting:
in some Latin American countries, \emph{pasar} is the verb to use when asking "what's on" television or at the local theater. \emph{¿Qué están pasando
en el 8?} = "What's on Channel 8?" \emph{Están pasando la nueva película
de de Niro en el Cine Colón} = "They're showing the new de Niro
movie at the Cine Colon." In South American countries the verb of
choice here would be \emph{dar}, usually in the phrase \emph{estar dando}.

As a colloquial greeting, both \emph{¿Qué pasa?} and \emph{¿Qué pasó?} are
used, though individual countries tend to have a preferred form. In
most situations, \emph{¿Qué pasa?} implies that something is wrong or abnormal; this is the question to ask when there are police cars parked in
front of your house. In Mexico, where \emph{¿Qué pasó?} is the preferred
greeting, saluting someone with \emph{¿Qué pasa?} may prompt raised eyebrows and the question \emph{¿Por qué?} So much for starting a pleasant
conversation.

Using \emph{pasar} can conveys "to happen to" in the sense of "to
become of." \emph{¿Qué pasó con Juan?} = "What's become of Juan?"---that
is, why is he late? In the present tense \emph{¿Qué pasa?} with an indirect
object pronoun (\emph{me, te, le, nos}, or \emph{les}) is like asking "What's (my, your,
his, her, our, their) problem?" In fact, \emph{¿Qué te pasa?} is roughly equivalent to "What's bugging you?" In the past tense, this same construction means "What happened to (me, you, him, her, us, them)?": \emph{¿Qué
le pasó?} = "What happened to him?"---for example, why are they carrying him off on a stretcher?

\emph{Pasarse}, the reflexive form, plus the preposition \emph{de} is very
handy in expressions meaning "to go too far," figuratively speaking. \emph{Se
pasó de listo} means someone "was too clever" or "was too sneaky,"
and implies that the person got caught at it. Sometimes it can be translated as "to get carried away." When a person \emph{se pasa de listo (lista)}
with another person, it can mean that he or she is making unwelcome
sexual advances. \emph{Ese señor se pasó de listo con María} = "That guy
made a (rude) pass at Maria." The formula \emph{pasarse de} can be used with
almost any adjective or quality, positive or negative. \emph{Usted se pasa de
generosa} = "You are being overly generous." \emph{Te pasaste de imbécil} =
"You were even stupider than usual."

\section{\emph{pedir}}

"To ask," "to ask for," and the correct verb for "to order" in
a restaurant (although \emph{ordenar} is gaining ground). \emph{Pedir} should make
you think of indirect commands and the subjunctive: \emph{Pídele que se
vaya} ("Ask him to leave"). It is also used in a number of stock phrases,
quite a few of which you are liable to need in the course of your dealings in Spanish. Some common ones include \emph{pedir permiso} ("to ask
permission"), \emph{pedir perdón} ("to apologize"), \emph{pedir informes} ("to ask for
information"), and \emph{pedir ayuda} ("to ask for help"). \emph{Pedir prestado} is
the correct phrase for "to borrow," but you'll often find \emph{prestar} (see below) handier. A rule of life for some in the Spanish-speaking world is
\emph{Es más fácil pedir perdón que pedir permiso}, or "It's easier to ask forgiveness than permission."

\section{\emph{pensar}}

The verb for "to think," though often \emph{creer} (see above) is preferred. To the extent that there is a rule for distinguishing them, use
\emph{pensar} when you might use "to have been thinking" or "to be thinking
about" in English. If it's just a simple statement of opinion, use \emph{creer}.
\emph{Pienso que debes irte} = "It is my feeling that you should go." \emph{Creo
que debes irte} = "You should probably go." Sometimes the two are
interchangeable. A usage of \emph{pensar} you should become very familiar
with is \emph{pensar} plus the infinitive to mean "to plan on" or "to intend."
\emph{Pienso irme mañana} = "I plan to go tomorrow." \emph{Pienso quedarme
unos días} = "I intend to stay a couple of days." \emph{Pensar + en} is "to
think of" or "to have in mind." \emph{Pensar + sobre} (or \emph{acerca de}) works as
"to think about" or "to consider" something. \emph{Pensar + de} is "to think
of," in the sense of an opinion, and is a lot like \emph{creer}. \emph{Estoy pensando en nuestras vacaciones}
= "I'm thinking of (remembering, daydreaming about) our vacation." \emph{Todavía estoy pensando sobre} (or \emph{acerca de})
\emph{nuestras vacaciones} = "I'm still thinking about (considering, analyzing) our vacation." \emph{¿Qué piensas de nuestras vacaciones?} = "What do
you think of our vacation (so far)? Stock phrases with \emph{pensar} include
\emph{¡ni pensarlo!} ("no way," "it's out of the question"), \emph{pensándolo bien}
("on second thought"), and \emph{sin pensar} ("without thinking,"
"unintentionally").

\section{\emph{poder}}

Meaning "to be able," this is in general a predictable word.
Aside from its quirks in the past tenses (see Chapter 5), it's just a matter of mastering a few stock phrases to get a hold on \emph{poder}. One such
phrase is \emph{poder + con}, which means something like "to handle" or "to
deal with." Examples will be useful here. A student who has trouble
learning biology might lament, \emph{No puedo con la biología}. In the sports
pages, you'll often come across headlines saying things like \emph{Los Leones
No Pudieron con Los Toros} ("The Lions Couldn't Handle the Bulls"---
that is, the Bulls beat the Lions, pure and simple). A Cuban postrevolutionary chant intones \emph{Fidel, Fidel, ¿qué tiene Fidel, que los americanos no pueden con él}, which means "Fidel, Fidel, what does Fidel
have, that the Americans can't handle (defeat) him?" Depending on the
context, \emph{poder + con} can also mean "to tolerate," and in this sense is
nearly synonymous with the verb \emph{aguantar}. \emph{¿Ay, no puedo con mi hermano!} = "Arrgh, I just can't stand my brother!" A fun way of describing an extremely irritating person is to say \emph{No puede ni consigo mismo}
("He can't even stand himself"). Other phrases using poder that you'll
want on the tip of your tongue: \emph{¿Se puede?} ("May I"), \emph{Puede ser}
("Could be" or "Maybe"), and \emph{Puede que} plus the subjunctive ("It
could be that\ldots{}" or "Maybe\ldots{}").

\section{\emph{prestar}}

\emph{Prestar}, "to lend," works for just about anything you might
want to borrow, just as "to lend" does in English. Thus \emph{Préstame tu
pluma} = "Lend me your pen." (For "lending a hand," though, you
would probably use dar or echar: \emph{Oye, échame una mano}.) What is
difficult for many English speakers is to switch between "borrow"
phrases and "lend" phrases in Spanish. This sometimes leads to convoluted constructions with \emph{pedir prestado}, like \emph{"Puedo pedir prestado tu
pluma?} Much more natural in Spanish is to turn it around (i.e., saying
"you lend" instead of "I borrow") and use \emph{prestar} by itself. If \emph{Préstame}
is too tactless for your tastes, say \emph{¿Me presta?} or \emph{¿Me prestas?}: \emph{¿Me
prestas tu pluma?} In very slangy speech you might hear \emph{Presta para
acá} or \emph{Presta pa'cá} for "Hand it over" or "Give it up." The latter example has sexual overtones, as it does in English.
\emph{Prestar} is also used for "to pay attention," which, if you think
about it, is much more realistic than the English concept (we don't really "pay" attention, we just lend it out). \emph{¿Niños, presten atención!} =
"Children, pay attention!" \emph{Prestarse}, the reflexive form, can be quite
useful for "to lend oneself," though it sounds much more idiomatic in
Spanish than in English. \emph{¿Tú crees que Juan nos deje copiar en el examen? No, él no se presta a eso}. = "Do you think Juan will let us copy
his exam?" "No, he doesn't lend himself to that."

\section{\emph{quedar}}

A megaverb that you'll want to have on your side as quickly as
possible. Its most straightforward uses revolve around "to stay" or "to
remain." \emph{Aquí me quedo} is "I'm staying here" and is sometimes used
as a name for cantinas. \emph{Quédate aquí} = "Stay here." Other uses
require "to have left" in English. \emph{Soló me quedan treinta dólares} = "I
only have thirty dollars left."

A host of other expressions with \emph{quedarse}, the reflexive, are
better covered in English by "to keep." \emph{Me quedé can treinta dólares}
= "I kept thirty dollars." \emph{Quédese con el cambio} would be "Keep the
change." \emph{Quédatelo} = "Keep it." For use in shopping, \emph{quedarse} is a
lot like \emph{llevarse} (see above). \emph{Me quedo can el azul} = "I'll take the blue
one." Often quedarse suggests a final or resultant state of affairs. \emph{Me
quedé helado} is, literally, "I was left frozen" and suggests you were frozen with fear. \emph{Me quedé en blanco} is to say "I ended up blank" or "I
didn't understand that at all." If someone asks you whether you understood an explanation of the theory of relativity, you could answer, \emph{Para
nada. Me quedé en blanco}. In English slang the equivalent might
even be"I spaced." A stock phrase you should remember for personal
dealings is \emph{¿En qué quedamos?} to mean something like "What's the
agreement, then?" or "So what's the deal?" Use it toward the end of
conversations to establish clearly the next step, be it the signing of a
multimillion-dollar merger agreement or a date to sip margaritas under
the stars.

Along these same lines of final or resultant states are the everyday expressions \emph{quedar bien} and \emph{quedar mal}. Like many of the expressions using \emph{quedar}, these seem to defy a simple English translation,
but the idea is "to end up well (or badly) with someone." Their use is
similar to \emph{caer bien} and \emph{caer mal}, and often they can be translated
with "impress," though that's a little strong. "To get on someone's
good side" might come closer for \emph{quedar bien}. \emph{Se puso corbata para
quedar bien can los suegros} = "He put on a tie to get on his in-laws'
good side." For \emph{quedar mal}, an example will be more helpful than an
English equivalent. \emph{Quedé mal con él porque no lo saludé} = "I've
gotten on his bad side (i.e., he's mad at me) because I didn't say hello
to him."

Finally, quedar has a couple of common uses that you should
be alert to since they don't fall into the "stay" or "remain" categories.
It is the common word for "to fit," for clothes and everything else. Remember to use it with the indirect object pronouns \emph{me, te, 1e}, etc. \emph{Este
saco no me queda} = "This coat doesn't fit." Also, and more natural to
a Spanish speaker, \emph{Este saco me queda grande} (or \emph{chico}) = "This coat
is too big (or small) for me." \emph{Quedar} also comes into play for describing locations. As a tourist, especially, you will hear it (and can even
use it!) a lot. \emph{Perdón, ¿donde queda la plaza? Adelante, a tres cuadras}.
= "Excuse me, where is the plaza?" "Three blocks up." \emph{Queda cerca}
and \emph{queda lejos} are both handy phrases for travelers. \emph{¿Queda cerca la
plaza? No, queda lejos}.

\section{\emph{querer}}

An absolutely vital word, and one this book gives a lot of space
to so you can get it right. It means "to like" or "to want" and, with
people, "to love" or "to want" (see Chapter 4). Like \emph{poder} and \emph{saber},
this verb acts a little strangely in the past tenses (see Chapter 5). Otherwise it's trustworthy, though it's worth rehashing a few tips for using
it well. \emph{Querer} performs many interesting tricks when paired with an
adverb and in the subjunctive. Before you faint, remember that we already went over that (Chapter 6) and you survived it quite nicely---\emph{cuando quieras, donde quieras, como quieras}, and so on, We've also
gone over how to say you "want" something (using \emph{traer}) without
dragging \emph{yo quiero} into your speech habits (Chapter 2). Another option
is to use \emph{quisiera} ("I'd like"). \emph{Querer} is also, in the phrases \emph{con querer}
and \emph{sin querer}, your best ticket for handling "on purpose" and "by
accident" (Chapter 12). \emph{Mamá, metí al gato en la piscina. ¿Fue con
querer o sin querer?} = "Mommy, I put the cat in the swimming pool."
"Was it on purpose or by accident?"

\section{\emph{repetir}}

"To repeat," of course, but also "to burp" or "to provoke
burps," The proper word for "to burp" is \emph{eructar}, which covers most
every burp, while \emph{repetir} is for those little, barely perceptible, good-eatin' burps. Just thought you'd want to know.

\section{\emph{romper}}

Remember that romper is "to break intentionally": \emph{Rompí
el vaso tirándolo contra la pared} = "I broke the glass by throwing it
against the wall." \emph{Romperse} is "to break" in the sense of an accidental
act: \emph{Se me rompió el vaso cuando lo estaba lavando} = "The glass
broke (on me) when I was washing it," In this construction the literal
meaning is "such-and-such broke itself to me (or you, him, her, us,
them)." The distinction is not dogma, but you should try to stick to
it. \emph{Romper con} is "to break up with" in the sense of lonely hearts and
whatnot.

\section{\emph{saber}}

A sometimes complicated verb, \emph{saber} bears watching for its
trickery in past tenses (Chapter 5) and in contrast to \emph{conocer} (see
above). An imperfect but useful rule of thumb: use \emph{conocer} with
proper and specific nouns and \emph{saber} or \emph{saber de} with the rest of them
and most clauses. \emph{¿Conoces París?} but \emph{¿Sabes dónde comen los parisinos?} and \emph{¿Sabes de su historia?} An exception to this rule are the
names of languages, which take \emph{saber}: \emph{¿Sabes inglés?} Another more
sweeping but also far-from-perfect rule: more often than not the word
you want for "to know" is \emph{saber}. \emph{Saber} is frequently followed by verb
infinitives; \emph{conocer} never is. \emph{Saber} carries with it the idea of "to know
how," so you don't have to say \emph{saber como}. \emph{¿Sabes esquiar?} \emph{No, pero
se caerme}. = "Do you know how to ski?" "No, but I know how to
fall down."
A few of the many stock expressions using \emph{saber} include
\emph{¿Sabes qué?} ("Know what?"), \emph{No sé} ("I don't know"), \emph{¿Quién sabe?}
("Who knows?"), \emph{¡De haberlo sabido!} ("If I had only known!"), \emph{¿Yo
qué sé?} ("What do I know?" or "Don't ask me!"). \emph{Un sabelotodo} is "a
know-it-all," and a useful phrase for "as far as I know" is \emph{que yo sepa}.
Finally, and just possibly to confuse you further, \emph{saber} is also
the word for "to taste," as in how something tastes to you. \emph{La sopa
sabe bien} is "The soup tastes good." In the first person (for use after
kissing or among cannibals), the correct form is \emph{sé}, but colloquially
you might hear \emph{sepo}. \emph{¿Que tal sé (sepo)? Sabes a pepinillos agrios}.
= "How do I taste?" "You taste like pickles." For the transitive "to
taste"---that is, to taste something---you need to use probar.

\section{\emph{seguir}}

"To follow," yes, but also frequently "to continue" or "to keep
(on)." The most common formula is \emph{seguir} plus the infinitive: \emph{Sigue
viniendo} ("He keeps coming"), \emph{Sigues comiendo} ("You keep eating")
\emph{Sigo llorando} ("1 keep crying"), and so on. \emph{Seguir} also works to translate a lot of the uses of "still" in English. In fact, using \emph{seguir} frequently sounds more natural in Spanish than using \emph{todavía}, which
is what native English speakers tend to resort to: \emph{Sigue creyendo en
Santa Claus}. = "She still believes in Santa Claus." \emph{Sigo enfermo} =
"I'm still sick." For the negative, you can use \emph{seguir + sin} and sound
very "Spanish" indeed: \emph{¿Sigues sin creer en Dios?} = "Do you still not
believe in God?" \emph{Seguir sin} and \emph{seguir con} are also useful constructions with a noun tacked on instead of the infinitive. \emph{Sigue sin trabajo}
= "He still doesn't have a job." A few odds and ends to take note of:
\emph{Síguele} is a handy phrase for "Keep it up" in both its genuine and
ironic senses; \emph{¿Quién sigue?} = "Who's next?"; \emph{¿Cómo sigues?} as a
greeting means "How are you getting along?" and implies that a person
has been sick or afflicted by some sort of trouble, even if it's only Spanish grammar exercises.

\section{\emph{sentir}}

Sentir means "to feel" and is a transitive verb used with direct
objects---that is, things you feel. \emph{Sentirse}, the reflexive, is used with
adjectives to express how you feel. Thus \emph{Siento frío} is "I feel cold" but
\emph{Me siento bien} is "I feel fine." \emph{Siento asco} is "I feel nauseous," and
\emph{Me siento mal del estomago} is "I feel sick to my stomach." \emph{Lo siento},
keep in mind, is "I'm sorry," and \emph{Lo siento mucho} is "I'm very sorry."
In case you need to explain further, you need only add \emph{haber} and a past
participle, dropping the \emph{lo}. Try to learn a couple of these by heart and
keep them at the ready: \emph{Siento haber llegado tarde} = "Sorry I'm
late." \emph{Siento mucho no haber podido ir} = "I'm very sorry I couldn't
come (or go)."

\section{\emph{ser}}

Except when you need \emph{estar} (Chapter 5), \emph{ser} is used for "to
be." It is the verb of permanent states, of the way things are, of telling
it like it is. There are a billion or so expressions using \emph{ser}, but most
of them are predictable and translate as "to be" in English. A few to
watch: \emph{¿De quién es?} = "Whose is it?" \emph{¿De qué es?} = "What is it
made of?" Like \emph{querer}, \emph{ser} in its subjunctive forms can be employed
to form "-ever" words or to say "any"; \emph{cuando sea, donde sea}, and so
forth. This is covered in Chapter 6, but a few examples here can't hurt.
Usually you employ \emph{que sea} after someone has already fed you the antecedent. It's better not to repeat the antecedent but think fast to get
the right gender. If someone asks, "What brand of beer do you want?"
you can answer \emph{La (marca) que sea}. "In what restaurant do you want
to eat?" \emph{En el (restaurante) que sea}. It's worth a little extra work to get
these expressions down pat, since otherwise you'll be inclined to say
ghastly things like \emph{Cualquier hora que quieres}, instead of the pure and
lilting \emph{Cuando sea}, for "Anytime."

\section{\emph{servir}}

"To serve," yes, but how often do you say "to serve" in an average day? "After I serve dinner, if it serves you, m'lord, I'll serve you
poisoned coffee and it'll serve you right!" In Spanish \emph{servir} is much
more commonly heard for "to work" in the sense of "to function."
\emph{No sirve mi teléfono} is "My phone doesn't work" \emph{¿Para qué sirve?} is
"What is it used for?" or even "What good is it?" When \emph{servir} is used
for "to serve," it is often dressed up in the stock phrase \emph{¿En qué le
puedo servir?} ("May I help you?"). \emph{Sírvase} (or \emph{Sírvete}), finally, can be
used for "Help yourself," but a Spanish speaker would probably say
\emph{Tome lo que quiera} instead.

\section{\emph{soler}}

There is no English equivalent for this word among the verbs,
which is odd considering how often you'll find yourself needing it. \emph{Soler} (meaning "to be in the habit of," "to be accustomed to") is useful to
describe something you usually do and is stuck before another verb
in the infinitive, making it a snap to use. \emph{Suelo comer a las dos} = "I
usually eat at two." \emph{Suele ir al cine saliendo del trabajo} = "He usually goes to the movies after work" See Chapter 12 (under "Usually")
for other ways to handle this concept.

\section{\emph{sonar}}

"To sound" or "to ring." \emph{Suena el timbre} = "The doorbell's
ringing," \emph{Eso suena dudoso} = "That sounds dubious." \emph{Sonar} is also
the verb to use to say "to ring a bell," as in what happens in your
memory when something sounds familiar. "Do you know Juan Pérez?"
\emph{No, pero el nombre me suena} ("No, but the name rings a bell"). \emph{Sonarse}, incidentally, is "to blow one's nose," and \emph{sonarse a alguien} is
"to smack someone."

\section{\emph{soñar}}

Meaning "to dream," the verb is used with \emph{con}. \emph{Sueño con serpientes} = "I dream about snakes." Here's something worth remembering: \emph{sueño}, the noun form, means both "dream" and "sleep." \emph{Una
falta de sueño} = "A lack of sleep." \emph{Soñado}, the adjective form, is a
fun word for "dreamy" or "ideal": \emph{la playa sonada} = "the beach of
your dreams."

\section{\emph{subir}}

The polar opposite of \emph{bajar} (see above), \emph{subir} works conversely
for gaining weight, getting on a bus, and so on.

\section{\emph{tardar}}

\emph{Tardar} works well for "to take" in time expressions. \emph{Durar}
(see above), on the other hand, generally works better for "to last."
\emph{Tarda el tiempo que quieras} = "Take as much time as you want."
\emph{¿Cuánto tardará en venir?} = "How long will he take in coming?"
Since we sometimes use "to take" and "to last" loosely in English,
there can be confusion in the correct Spanish choice. \emph{El avión tarda
media hora en venir de allá a acá}. = "The plane takes half an hour
to get here from there." \emph{El vuelo dura media hora} = "The flight lasts
half an hour." As a rule, use \emph{durar} whenever "to last" could work, and
use \emph{tardar} otherwise. \emph{Tardar} also means "to take time" with the suggestion of "to take too much time," "to dally," "to be late." \emph{Tardé en
llegar porque había mucho tráfico} = "I took a long time getting here
(I'm late) because there was a lot of traffic." \emph{El tren está tardando en
llegar} = "The train is taking too much time (is late) arriving."

\section{\emph{tener}}

A megaverb, \emph{tener} is almost worthy of a chapter of its own.
Many of its uses are predictable renditions of "to have" in English, including the indispensable \emph{tener que} for "to have to": \emph{Tengo que irme}
= "I have to go." It can also be used by itself in the negative to say "I
don't have any" when the antecedent is understood. \emph{Dame dinero. No
tengo}. = "Give me money." "I don't have any." \emph{¿Dónde está su boleto?
No tengo}. = "Where's your ticket?" "I don't have one." Another common use of tener with an implied complement is the question \emph{¿Qué
tienes?} (or \emph{¿Qué tiene?}), which can translate as "What's your (or his,
her) problem?" or "What's the matter with you (or him, her, it)?" Another example: "I don't like the house they've picked out for us." \emph{¿Qué
tiene?} ("What's wrong with it?")

Where you will have to pay special attention to \emph{tener} is in the
thousand and one expressions in Spanish that use \emph{tener} plus a noun for
what in English would be "to be" plus an adjective. In English, for instance, you "are cold," whereas in Spanish you "have cold." Common
examples of this construction include \emph{tener hambre} ("to be hungry"),
\emph{tener sed} ("to be thirsty"), \emph{tener frío} ("to be cold"), \emph{tener calor} ("to be
hot"), \emph{tener sueño} ("to be sleepy"), \emph{tener paciencia} ("to be patient"),
\emph{tener cuidado} ("to be careful"), \emph{tener razón} ("to be right"), \emph{tener prisa}
("to be in a hurry").
There are also another thousand and one expressions, many of
them quite colloquial, using \emph{no tener}. Here are some you should learn:

\bsk

\indu \emph{No tiene sentido}. = "It doesn't make sense."

\indu \emph{No tiene caso}. = "There's no point" or "What's the point?"
(indicating futility)

\indu \emph{No tiene chiste}. = "It's boring" or "What's the point?" (indicating insipidness)

\indu \emph{No tiene (nada) que ver}. = "That has nothing to do with it"
"That's irrelevant."

\indu \emph{No tiene vergüenza}. = "He (or she) is shameless."

\indu \emph{No tiene lógica}. = "It's illogical" or "It doesn't make sense."
(indicating incredulity)

\indu \emph{No tiene ni pies ni cabeza}. = "I can't make heads or tails of it."

\indu \emph{No tiene en donde caerse muerto}. = "He (or she) is flat
broke."
(literally, "doesn't even have anywhere to drop dead")

\indu \emph{No tiene remedio}. = "There's no way out" or "There's nothing to be done."

\indu \emph{No tiene} (\emph{ni la menor} or \emph{ni la más mínima}) idea. = "He (or
she) hasn't got the faintest idea" or "He (or she) hasn't
got a clue."

\section{\emph{tirar}}

The common verb for "to throw," though \emph{aventar} and \emph{arrojar}
are also used quite a lot regionally. \emph{Tirar} also has the implication of
"to throwaway" or "to throw out." \emph{Tira esa basura} = "Throw that
worthless thing out." It is also used for "to knock over," as in \emph{Tiré el
vaso} ("I knocked over the glass"). To convey "to toss," as in "Toss me
a pen," you could use \emph{aventar} (in the Americas) or \emph{echar} or even \emph{tirar},
but mostly you wouldn't use anything because it's considered rude
in the Spanish-speaking world to toss things (see Chapter 2). Unless
you're especially keen on seeing a particular object in flight, stick to
\emph{Préstame} (see \emph{prestar} above).

\section{\emph{tocar}}

"To touch," of course, but also "to play" (a musical instrument) and "to knock" or "to ring" (at someone's door). In Casablanca
Bogie would have told the pianist, \emph{Tócala, Sam}. An extremely common use of \emph{tocar} that is often glossed over in textbooks is for "to experience" or "to be one's turn." A simple and good translation is elusive,
but examples will get the point across. \emph{Me toca}. = "My turn." \emph{¿A
quién le toca?} = "Whose turn is it?" \emph{A mi no me toca decirle}. = "It's
not up to me to tell him." \emph{Al dueño le toca arreglar la casa}. = "It's
up to the owner to fix the house." Sometimes the best translation involves "to get": \emph{¿A quién le toca la última rebanada?} = "Who gets the
last slice?" \emph{A ti te tocó la más guapa de las hermanas}. = "You got the
prettiest of the sisters." \emph{A mi me tocó el más feo de los hermanos} =
"I got (stuck with) the ugliest brother." \emph{No me ha tocado verlo en concierto} = "1 haven't had (gotten) a chance to see him in concert." And
so on. A final note on \emph{tocar}: its past participle, \emph{tocado}, is a common
synonym for \emph{loco}---as in the English sense of being slightly "touched"
in the head. Children often adapt it to \emph{toca-toca}, translating perhaps as
"cuckoo."

\section{\emph{traer}}

\emph{Traer} is straightforward for "to bring," except when "take"
and "bring" (\emph{llevar} and \emph{traer}) get mixed up (see Chapter 5).

\section{\emph{tratar}}

"To treat," of course, but far more commonly encountered
with \emph{de} and meaning "to try." For some reason, though, many Spanish-English dictionaries refuse to acknowledge this fact, leaving the student to choose among \emph{ensayar, procurar, intentar}, and \emph{pretender}.
These are all worthy and acceptable verbs, of course, and someday you
might even want to learn them. But for now, remember: \emph{tratar de} =
"to try." \emph{Traté de dormir} = "I tried to sleep." \emph{Tratamos de llamarte}
= "We tried to call you." \emph{Trata de venir antes de las once} = "Try to
come before eleven." Only when you're using "to try" in the sense of
"to sample" or "to test" should you abandon \emph{tratar de}; the correct verb
here is \emph{probar}. An awkward but illustrative example: \emph{Trata de probar
el vino blanco 1985}. = "Try to try (i.e., make an effort to sample) the
1985 white wine."

\emph{Tratarse}, the reflexive form, is a useful verb for "to have to do
with" or "to be about." \emph{¿De qué se trata?} is the common way of asking
"What's it about?" (in reference to a film, a book, a scuffle, an argument, and the like). "To treat" in the sense of "to pay for someone
else" is usually handled by \emph{invitar}. \emph{Yo invito} = "I'm treating."

\section{\emph{valer}}

Meaning "to be worth," this verb is frequently encountered
in the stock phrases \emph{vale la pena} ("it's worth it") and \emph{no vale la pena}
("it's not worth it"). In addition, you may find \emph{valer} handy for asking
prices. \emph{¿Cuánto vale?} = "How much is it?" Another good use of valer,
preceded by \emph{más}, is to translate English phrases that use "better" or
"had better." \emph{Más te vale irte} = "You'd better get out of here." \emph{Más
vale preguntar} = "We'd better ask." And there's the old standby \emph{Más
vale tarde que nunca} ("Better late than never"). In Spain and less so
elsewhere, \emph{vale} by itself is a common interjection for "all right" or
"okay." In Mexico, especially, \emph{valer} with an indirect object pronoun
is a somewhat crude way of saying "couldn't care less." \emph{Me vale} = "I
couldn't care less." \emph{Le vale} = "He doesn't give a damn." Its crudeness
comes from the fact that it's a shortened and therefore euphemistic
form of another phrase, which you'll have to read about in Chapter 10.

\section{\emph{venir}}

Meaning "to come," naturally, and sometimes difficult to distinguish from "to go" (see Chapter 5). This verb has some common
but unexpected uses, as in \emph{No viene al caso} ("That's beside the point").
\emph{Que viene} is especially worth learning in stock phrases like \emph{la semana
que viene} ("next week") and \emph{el año que viene} ("next year"). \emph{Venirse},
the reflexive form, is usually an innocent intensifier for \emph{venir}, but it
has sexual overtones in some countries and should be used with care.

\section{\emph{ver}}

As "to see," this is a pretty straightforward verb. It can sometimes be confused with \emph{mirar}, since \emph{ver} also works as "to look at" in
many cases where you might be tempted to use \emph{mirar}: \emph{Ese señor se me
queda viendo} = "That man keeps looking at me." \emph{Estoy viendo tus
discos} = "I'm looking at your records." \emph{Mirar} would work fine in the
first example, but in the second would suggest you are gazing at the
records as if waiting for them to do something. An extremely common
expression is \emph{A ver}, which is simply "Let's see\ldots{}" but which is used
mainly to buy time while you think of a clever response. \emph{Vamos a
ver} means the same but is less common as an interjection or "crutch
word." \emph{Tener que ver con} is the easiest way to translate "to have to do
with." \emph{No tengo nada que ver con el asunto} = "I have nothing 'to do
with this business." \emph{No tiene que ver} = "That's irrelevant."

\emph{Ver} is sometimes used to express an opinion, in the sense of
how you "see" or "size up" a problem. \emph{La veo difícil} = "It looks difficult to me." \emph{Verse}, the reflexive, covers almost all uses of "to look"
that refer to the appearance of something or someone. You should etch
the phrase \emph{se ve} onto the end of your tongue and have it ready for such
common utterances as \emph{Se ve bien} ("It/he/she looks good") \emph{Se ve difícil}
("It looks difficult"), \emph{Se ve bonito} ("It looks nice"), and so on. \emph{Se ve que}
plus a clause is an easy way to communicate "You can tell that\ldots{}" or
"It's obvious that\ldots{}" \emph{Se ve que no han cambiado el agua en la piscina} = "You can tell they haven't changed the water in the pool." \emph{Se
ve que son grandes amigos} = "You can tell that they're great friends."

\section{\emph{volver}}

In the sense of "to return" or "to come back," \emph{volver} is interchangeable with \emph{regresar}. \emph{Volver} has another common use that you
will want to learn, though---one that \emph{regresar} does not share. \emph{Volver}
plus \emph{a} plus an infinitive is frequently found as a substitute for "to repeat" or "to do again." It will take some practice before you start to
use it properly---and as an alternative to \emph{otra vez}---but it's worth the
extra effort. Some typical examples of when to substitute: \emph{Gracias,
vuelvo a llamar más tarde} = "Thanks, I'll call back (again) later." \emph{Si
vuelves a pedírmelo, no te lo voy a dar} = "If you ask me for it again,
I'm not going to give it to you." \emph{Vuelve a intentar} = "Try again."
\emph{Volver} cannot be used transitively---to "return" a book to the library, for
instance, or "to give back" a borrowed item. Use either \emph{regresar} or \emph{devolver} in these situations. \emph{Devuélveme a mi chica} ("Give me back my
girl"), for instance, was the name of a pop song and movie of a few seasons ago. \emph{Volverse}, finally, is one of the common ways to handle "to
become" (or "to get"). Skip ahead to Chapter 11 for details.

\chapter{Cranking Up Your Spanish}

\chapter{Snappy Answers}

\chapter{Invective and Obscenity}

\chapter{Which Is Which?}

\chapter{Say It Right}

\chapter{Spanish Roots}

\chapter{The big Mix}


\begin{flushright}
    {\tiny{Nunc dimittis}, \DTMnow}
\end{flushright}

\end{document}
