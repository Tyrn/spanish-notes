\chapter{Appendix A. Spanish's Anglicisms and Barbarisms}
%SPANISH'S ANGLICISMS
%AND BARBARISMS
%A
What follows is an exhausting (but by no means exhaustive)
list of Anglicisms in use in modern Spanish. As a sample listing, this
list should enable the reader to get a feel for the process by which
some English words are incorporated into Spanish. Some of the examples given are quite rare in actual use, but are included for didactic
purposes or just for fun. Other examples are to be found in the text,
and another two to three thousand, or perhaps double or triple that, are
to be found on the streets of Lima, Los Angeles, Madrid, and Managua.

Many native Spanish speakers use English words to show their
worldliness and---why not say it?---to show off a bit. For much the
same reason, many native English speakers use French, Latin, German,
Greek, and even Spanish (see Appendix C) terms. But you, as a native
English speaker, will wow absolutely no one by using English terms in
your Spanish. To the contrary: you will impress by your command of
correct Spanish. Spanish is a living, ever-evolving language. But let's
allow the Spanish speakers be the ones to change it, \emph{¿no?}

Words are listed by their English source word, with accents
added when used. When the English spelling rarely or never appears
as such in Spanish, the common Spanish form follows in parentheses.
Thus "basketball" will sometimes be found written that way in Spanish, but "baseball" never appears with its English spelling.

\bsk

\appxe{ALIBI} This Latin term is an unnecessary import since \emph{coartada} (another Latin word) already exists in Spanish.

\appxe{ALLIGATOR} One of the most ridiculous barbarisms to enter
Spanish, since the word is itself a barbarism imported incorrectly from
Spanish into English: \emph{el lagarto}, or "the alligator," became "alligator."
Fortunately, it is a rare loanword.

\appxe{ANCESTOR} (\emph{ancestro}) Borrowed from either French or English.
It is not accepted yet by Spain's Royal Academy, although \emph{ancestral}
has been approved; \emph{antepasado} is the correct word for \emph{ancestro}.

\appxe{ANTIQUE SHOP} To attract the tourists?

\appxe{APPOINTMENT} (\emph{apointement}) Simply silly. Stick to \emph{cita}.

\appxe{ATTORNEY} I'm impressed. And you?

\appxe{AUDITÓRIUM} This barbarism is probably from English, but
since it's technically a Latin word, some scholars tolerate it in Spanish. Used for both "audience" (\emph{auditorio}) and for the building or hall
itself (\emph{sala}).

\appxe{AVERAGE} Unnecessary import; \emph{promedio} works fine.

\appxe{BABY} This word is starting to appear more frequently in the
Spanish-speaking world (perhaps because of car signs that say "Baby on
Board" and pop songs crooned to one's "baby"). You may be told that
\emph{bebé} is the correct Spanish word, but it is in fact a Gallicism of not
much greater vintage or approval. Try \emph{nene} instead.

\appxe{BACK} The verb \emph{backear} is a border loanword or \emph{pochismo}.
Especially common around cars, presumably since \emph{échate en reversa}
takes too long to say.

\appxe{BACON} In Spain they use this, though they don't have permission from the Academy. \emph{Tocino}, anyone?

\appxe{BANK NOTE} Banker talk considered too important to translate. Rare.

\appxe{BAR} Considered classier than \emph{cantina} and more mainstream.
An establishment that calls itself a \emph{ladies bar} is even open to that
other half of the population.

\appxe{BASEBALL} (\emph{béisbol}) Still awaiting approval by the Academy.
What else it might be called is beyond me.

\appxe{BASKETBALL} Usually \emph{basquetbol}, though the English form
still appears at times. The correct (i.e., invented) term for this sport is
\emph{baloncesto}. Note, however, that you must call the ball itself \emph{un balón
de baloncesto} or \emph{un balón de basquet}, never \emph{un basquetbol}. Note also
that one of the dangers of allowing \emph{basquetbol} is that you start to hear
\emph{basqueta} being used instead of \emph{canasta}!

\appxe{BAZOOKA} Usually written \emph{bazuca}; the shot from one is a
\emph{bazucazo}.

\appxe{BEEF} (\emph{bife, bifstec, bistec, biftec, bisté}) All ways of saying
\emph{carne de res} or \emph{filete de res}. \emph{Bife} is especially common in Argentina,
but one of the above forms will probably be on the menu in whatever
part of the Spanish-speaking world you visit. You may even hear \emph{un
bisté de puerco}, which is quite a mouthful. \emph{Bife} and \emph{bistec} have both
been declared kosher by the Academy. (See also \emph{steak}.)

\appxe{BIG BANG} Some scientists have suggested calling this theory
of the explosion of the universe \emph{el gran ¡pum!}, but for reasons I cannot
fathom it hasn't caught on.

\appxe{BIKINI} Also spelled \emph{biquini}. Neither is recognized by the
Academy.

\appxe{BINGO} Anglicism.

\appxe{BLOCK} Used for "pad" (of paper). Possibly a Gallicism, as
"block" doesn't have this meaning in English. \emph{Bloc} is the most commonly used form; \emph{bloque} is the linguists' attempt to "Spanishize" it.
(See also \emph{bloquear} in Appendix B.)

\appxe{BLUES} The same in any language.

\appxe{BLUFF} Whether as \emph{blof, blofeador, blofear, blofero}, or \emph{blufar},
this is a widespread import. \emph{Bluff} and \emph{blof} are most common, but none
meets with the approval of the Academy. They are used as in English,
except as nouns they can be applied to people as well: \emph{Ese tipo es puro
bluf} = "That guy's a fraud" (or "an imposter" or even "a disappointment"). \emph{Fanfarrón} (noun) and \emph{fanfarronear} (verb) are more traditional
Spanish words meaning much the same thing as "bluff."

\appxe{BOOM} In the business sense of "boom years" or a "financial
boom." \emph{Auge} is the correct Spanish word.

\appxe{BOWL} (\emph{bol}) Much used in most of the Spanish-speaking
world for "soup bowls" and larger receptacles.

\appxe{BOYCOTT} Variants are \emph{boicot, boicoteo, boicotear, boycotar}.
A word of Irish origin now accepted by the Academy and widely used.
(Consider it a fair trade for "embargo," which English imported from
Spanish.) The uses spelled \emph{boi-} are the official ones.

\appxe{BOX} A sports import that includes \emph{boxear, boxeador}, and
\emph{boxeo} ("to box," "boxer," and "boxing," respectively). The Spanishized
\emph{boxeo} is to be preferred to box. In a slightly related vein, \emph{box spring} is
widely used as well, as is \emph{boxers} ("boxer shorts").

\appxe{BOY SCOUT} Various attempts have been made to find a Spanish translation for this imported concept, but none has caught on.
Some of them are \emph{exploradores, escultistas}, and \emph{excursionistas}. "Girl
scouts," interestingly, has not been widely imported.

\appxe{BRAKE} Also appears as \emph{breik, brek, breke}, or \emph{breque}---border talk for what stops your car. Strangely, all of these words exist to
avoid saying \emph{freno}. (See also \emph{break}.)

\appxe{BRANDY} If this and \emph{whiskey} count as Anglicisms, then "tequila" and "mezcal" are "Hispanisms."

\appxe{BREAK} Popular in U.S. Spanish, especially in the sense of "give
me a break." \emph{Dame un break} or \emph{dame un breakecito} are the phrases
you are most likely to hear. The borrowing is not surprising, since
\emph{oportunidad} doesn't cut it and \emph{chance} (see below) is also an Anglicism.
In the sense of a "break" from work, though, \emph{descanso} is a better word.

\appxe{BROTHER} In street slang you may hear \emph{brodi, bróder, broíta}
and so on.

\appxe{BUDGET} Some claim it's a Gallicism, but it seems more
likely to have entered Spanish with a host of other financial Anglicisms. It needn't have, since \emph{presupuesto} was already here.

\appxe{BULLDOZER} Why not?

\appxe{BUNCH} (\emph{bonche}) A border \emph{pochismo} that is gradually
working its way south and into mainstream Spanish, Not there yet,
however.

\appxe{BUNGALOW} Actually comes from Hindi via English. Widespread wherever tourists are found. It will need an accent (\emph{búngalow})
at least if it stays in Spanish.

\appxe{BÚSHEL} Gradually disappearing, even in English. Never widespread in Spanish.

\appxe{BUSINESS} An inevitable import, since in much of the world
the business of English is business. Often used in street slang (spelled
\emph{bísnes}) in the sense of "to give someone the business."

\appxe{BYE} Widespread but not official. But then neither is \emph{ciao} (or
\emph{chao}), imported from Italian and also widespread.

\appxe{CADDY} All's fair in science and sports.

\appxe{CAKE} Pronounced as in English but often spelled \emph{queque},
particularly in South America. \emph{Torta, bizcocho}, or \emph{pastel}, depending
on the region, would be more traditional Spanish.

\appxe{CÁMEL} The color.

\appxe{CAMPING} A common Anglicism to describe what is essentially an imported concept. Some prefer the Spanishized \emph{campismo}.
The verb in all places is \emph{acampar}, and in many instances it can and
probably should be used instead of \emph{camping}.

\appxe{CASSETTE} The spelling may be Spanishized someday, but
otherwise this word is here to stay. \emph{Cinta} is the nearest Spanish term
for "tape" but usually refers to the reel-to-reel type.

\appxe{CATCH} (\emph{cachar}) This word is traveling south at a frightful
pace and replacing \emph{atrapar} and \emph{coger} (where it's not a dirty word).
This word sounds so Spanish (in fact, the word \emph{cachar}, with another
meaning, existed long ago in the language) that it is almost certain to
stick.

\appxe{CLIP} Means "paper clip." \emph{Clip para papel} would be understood but is redundant.

\appxe{CLOSE-UP} Filmmakers' talk. Not pretty, not approved, but
not surprising.

\appxe{CLOWN} Used in Spain especially (sometimes as \emph{clon}), where
\emph{payaso} apparently isn't precise enough.

\appxe{CLUB} Well-established Anglicism. Works for a garden club or
a country club (\emph{club de golf} or \emph{club campestre}).

\appxe{CLUTCH} The common word for the car part in some countries. Sometimes as \emph{cloch}.

\appxe{COACH} Usually spelled \emph{couch}, the term refers to the team
leader in sports. (See also \emph{manager}.)

\appxe{COCKTAIL} (\emph{cóctel}) Recently approved by the Academy. See
Chapter 14.

\appxe{COMPACT} Used for both cars and disks, though \emph{compacto} exists and is widely used as well.

\appxe{CONTAINER} (\emph{contenedor}) This Spanishized term is widespread. \emph{Recipiente} covers most of the same ground, but the use of containerized cargo in world commerce has brought about the birth of this
neologism based on the age-old Spanish word \emph{contener}.

\appxe{COOLER} This word is creeping into Spanish in the sense of
a container that keeps things cold---your Igloo or Thermos, in other
words. Others prefer \emph{hielera}, not least because \emph{cooler} sounds a lot like
a certain Spanish obscenity.

\appxe{COOLIE} Despite objections to \emph{cooler} (see above), the word
\emph{culi} has been approved by the Academy to refer to an "indigenous servant." Though it may be linguistically correct, it almost certainly isn't
politically so; in any case, it is but a rare or locally used word.

\appxe{CORNFLAKES} \emph{Hojuelas de maíz} just hasn't caught on.

\appxe{COTTAGE} This looks like a Gallicism but is in fact an Anglicism. (See also \emph{bungalow}.)

\appxe{COUNTRY CLUB} Hard to say whether this is better or worse
than \emph{club de golf}. (See also \emph{club}.)

\appxe{COVER} Used in the sense of "cover charge" in bars and nightclubs and for "cover versions" of songs.

\appxe{COWBOY} Part of U.S. folklore that has been exported, although a true cowboy would be called a \emph{vaquero} (or a \emph{gaucho}).

\appxe{CRACK} Also written \emph{crac, krac, krach, krack}, or \emph{craqueo}. A
strange import referring to a business or stock market "crash." This
word apparently merged with the onomatopoetic \emph{crac}, which probably
already existed in Spanish, to form these variants. Of them, \emph{crac} is the
most common. Also refers to the drug, just for the record.

\appxe{CRACKING} Refers specifically to a chemical decomposition
process employed in the petrochemical industry. This term and the
verb \emph{craquear} are approved by the Academy only for this technical use.

\appxe{CRAMP} (\emph{crampa}) Of either French or English origin. Use
\emph{calambre}.

\appxe{CRAWL} Approved as \emph{crol}, referring to the swimming style.

\appxe{CROONER} Transmogrified into \emph{crúner}, this word should be
an offensive \emph{pochismo} but for some reason it has appeal. \emph{Baladista} already existed, but somehow that just doesn't capture the essence of the
Latin lounge-singer.

\appxe{CHAMPION} Meaning \emph{campeón}, perhaps from the influence of
boxing. Not at all necessary, of course, but vintage slang.

\appxe{CHANCE} Written \emph{chance} or \emph{chanza} (as a result of confusion
with an existing word), this word is very common in regional slang for
what we might use "break" for in English. \emph{Dame chance} = "Gimme a
break." (See \emph{break}.) \emph{Chance} is also common slang for "maybe."

\appxe{CHECK} (\emph{cheque}) A ubiquitous Anglicism from the banking world (with U.K. English spelling but Spanish pronunciation). The
verbs \emph{checar} and \emph{chequear} are also probably Anglicisms; although unapproved, they are extremely common, especially in American Spanish. They mean much the same as in English and substitute for \emph{revisar,
averiguar, comprobar}, and so on. A \emph{chequeo} is generally a medical
"checkup."

\appxe{CHRISTMAS} Just to show off. \emph{Navidad} is adequate.

\appxe{DANCING} Usually refers to a "dancing club" more than to
the act itself.

\appxe{DANDY} An old import from a time when dandies populated
English and European society. Perhaps more common in Spanish than
in English these days.

\appxe{DERRICK} From the oil industry. (See \emph{cracking}.)

\appxe{DETECTIVE} Usually refers to a private (not police) detective,
It probably entered via Hollywood and has since been approved by the
Academy.

\appxe{DOCK} Port towns the world over pick up foreign words for local use. This one is unnecessary, however, as \emph{muelle} says the same
thing just as well.

\appxe{DRIVE} A computer term. Strangely, "hard drive" tends to be
translated as \emph{disco dura}, but "drive A" will be called \emph{drive A}. The
verb \emph{drive} is used only in borderspeak, where you may even see or hear
\emph{draivear}.

\appxe{DUMMY} Used but not approved in both journalism (a layout
imitation or guide) and libraries (to label stands). Sometimes it shows
up as \emph{domi} or \emph{domio}.

\appxe{DUMPING} The commercial practice of selling products under
cost in a foreign market. In widespread use in the business world.

\appxe{EXPRESS} Other forms are \emph{exprés} and \emph{expreso}. Considered an
Anglicism and used (but not approved) for trains, buses, rapid-mail services, and coffee.

\appxe{FAN} In English, we might call one of these an "aficionado,"
so I suppose there's no reason we can't let Spanish speakers call
theirs \emph{fans}.

\appxe{FASHIONABLE} What in English would be "tres chic."

\appxe{FEELING} A strange and regionally used import, usually referring to something that's missing. A painting or project that lacks that
"something special" could be said to \emph{faltar feeling}.

\appxe{FERRY} The proper word is \emph{transbordador}, but it is uncommon. I've never seen a Spanishized spelling, but it's probably on its
way. \emph{Ferryboat} is also sometimes used.

\appxe{FIFTY-FIFTY} Slang for splitting something "half and half."
Use \emph{mitad-y-mitad}.

\appxe{FILM} Not something for your camera (although that usage is
probably due to arrive any day) but in the sense of what in American
English is called a "movie." A snobby, unnecessary import, but the
Academy approved it anyhow, while changing its official spelling to
\emph{filme}.

\appxe{FLAMINGO} This is an English misappropriation of the Portuguese "flamengo" that has started slipping back into Spanish. The correct name of this bird is \emph{flamenco}, like the music.

\appxe{FLASH} Used widely for "camera flash," this word also has
spawned the verb \emph{flashear}, meaning "to flash," "to blink on and off."
A \emph{flashazo}, at least in Mexico, is a blinding flash of light.

\appxe{FLASHBACK} More filmmakers' talk.

\appxe{FLIRT} Meaning almost the same as in English but with more
sexual overtones. Maybe "tease," used in a fairly vulgar sense, would be
closer to the mark. Amazingly, it has been approved by the Academy as
\emph{flirteo} and \emph{flirtear}, the verb. \emph{Coquetear} and derivatives are the closest
Spanish equivalents, equating perfectly with "flirt" (and "coquette").

\appxe{FLOPPY} Computer talk. \emph{Disco flexible} may win out in the end.

\appxe{FOLDER} Widespread for the folder of the manila type.

\appxe{FOLKLORE} Also \emph{folklórico} and \emph{folklorísta}. The meaning is
the same as in English, with the added irony that the most traditional,
autochthonous, and indigenous groups often use this Anglicism to describe their style, beliefs, and way of life.

\appxe{FOOTBALL} (\emph{fútbol} or \emph{fut}) The neologism \emph{balompié} was invented to replace this Anglicism, but only bored sportswriters use it.
Remember that \emph{fútbol} means "soccer," and \emph{fútbol soccer} is sometimes
used when there's a doubt. For North American-style football, you
have to specify \emph{fútbol americano}.

\appxe{FOX-TROT} A musical term---one of dozens---imported into
Spanish (or exchanged for \emph{salsa, merengue, tango, rumba, cha-cha-cha}, etc.).

\appxe{FREAK} Youths both north and south of the border are starting to use words like \emph{friquearse} for "to freak out" and \emph{friqueado} for
"freaked-out." Academy beware.

\appxe{FREEZER} Used in some American Spanish instead of
\emph{congelador}.

\appxe{FRIGIDAIRE} Used in some parts for refrigerador.

\appxe{FUMBLE} As an example of what happens when you open the
door to sports Anglicisms, consider the following three verbs: \emph{fomblear, tacklear}, and \emph{driblear}. Repent now!

\appxe{GANGSTER} A term brought to the Spanish-speaking world by
Hollywood. It has stayed, though the Academy hasn't approved it.

\appxe{GARAGE} May well have entered from French, at least originally. The alternate spelling is \emph{garaje}.

\appxe{GAY} Now entering Spanish in its modern sense of "homosexual." Interestingly, the adjective \emph{gayo} already existed in Spanish,
meaning "happy" or "showy," and was probably an earlier import that
didn't prosper.

\appxe{GENTLEMAN} Actually, \emph{gentilhombre} exists in Spanish, but
\emph{caballero} is the common and correct word.

\appxe{GHETTO} Widely known and used, almost always in reference
to ghettoes in the United States. Other spellings: \emph{gueto, gheto, getto,
geto}. The word presumably entered Spanish originally from Italian, in
reference to Jewish sections of European cities.

\appxe{GIN AND TONIC} On second thought, make that a margarita.

\appxe{GOLF} Probably destined to stay, since \emph{golfo} is already the
Spanish word for "gulf."

\appxe{GONG} Like the kind on The Gong Show. The Academy
wants Spanish speakers to say and spell it \emph{gongo}, but no one does, yet.

\appxe{GRILL} Used widely and in much the same way as \emph{bar} (see
above) is. \emph{Parrilla} is the more appropriate word.

\appxe{GROG} A drink containing rum, sugar water, and lime. An old
(if not very widespread) Anglicism that comes originally from Admiral
Edward "Old Grog" Vernon (1684\,--1757), who fed the mixture to his
sailors.

\appxe{GROOM} Used for the person who grooms horses, not for
"bridegroom."

\appxe{HALL} The correct equivalent is \emph{pasillo}, but \emph{hall} is generally
used to refer to the vestibule or antechamber encountered immediately
upon entering a building.

\appxe{HAMBURGER} (\emph{hamburguesa}) On second thought, though,
we'll have tacos.

\appxe{HANDICAP} This term is spreading from horseracing lingo
into other enterprises; it is not used for disabled people, however.

\appxe{HAPPY} In some countries, the expression \emph{andar happy} means
"to he tipsy," "to be feeling no pain."

\appxe{HARDWARE} Used in the sense of computer equipment, but
not for your local hardware store, which is a \emph{ferretería} most places and
a \emph{tlalpalería} in Mexico. Terms like \emph{software} and \emph{mouse} are also appearing in Spanish.

\appxe{HEAVY METAL} Another musical term of recent importation.
Often just called \emph{heavy}.

\appxe{HIGH} This word creeps into Spanish in a number of usages.
In some places Spanish speakers refer to the \emph{high-life} and in others,
simply to \emph{la high}, meaning the same. A \emph{jaibol} is our old friend "highball," and in some countries \emph{el colegio high} is "high school."

\appxe{HIPPIE} A U.S. export from a bygone era; the type is known if
not always loved throughout the Spanish-speaking world.

\appxe{HIT} Also \emph{jit}, used in baseball. It never quite sounds as good as
\emph{un imparable} ("an unstoppable") or \emph{un indiscutible} ("an unarguable"),
though.

\appxe{HOBBY} Also written \emph{joby} or \emph{jobi}. This import is widespread
among Spanish speakers, although \emph{pasatiempo, afición}, and \emph{manía}
seem to cover the same idea pretty well.

\appxe{HOME RUN} Written as jonrón in some parts, this is but one
more example of baseballese. The correct term is \emph{un cuadrangular}.

\appxe{HOTCAKE} "Pancake" in most places.

\appxe{HOT DOG} Everywhere the same.

\appxe{ICEBERG} Pronounced roughly as in English, this is the only
commonly used Spanish word for this phenomenon. \emph{Témpano} (or \emph{témpano de hielo}) is sometimes substituted, but technically this means
"ice floe."

\appxe{INTERVIEW} Common especially in Spain, where it is often
written \emph{interviú}. An \emph{entrevista} is exactly the same thing.

\appxe{JAZZ} This word has even entered the common parlance of
some places as \emph{yes}!

\appxe{JEANS} Other regional words work, but people everywhere
know what this word refers to.

\appxe{JERSEY} Pronounced "hair-say," this import refers to a
sweater and is one of a handful of English words for clothes that
have been adopted into Spanish. The pronunciation "jair-see" is also
sometimes heard, usually in reference to New Jersey.

\appxe{JOCKEY} This is the only common Spanish word for the
small people who ride racehorses. Pronounced "yo-kay."

\appxe{JOGGING} Close to becoming a Spanish word if a substitute
isn't found fast.

\appxe{JOINT} In the sense of a marijuana cigarette.

\appxe{JUMPER} Girls' clothing (usually a sleeveless dress).

\appxe{JUNGLE} (\emph{jungla}) Originally a Hindi word meaning "wasteland." Spanish has borrowed this word from English and uses it (albeit
rarely) for "rainforest." \emph{Selva} remains the common word for this, though.

\appxe{JUNIOR} Probably an Anglicism; in any case, it is a foreign
way to refer to a son bearing the same name as his father. The typical
Spanish way would be to distinguish \emph{padre} and \emph{hijo}---i.e. Juan Pérez
padre and Juan Pérez hijo. As a formal title, it is not needed in Spanish-speaking countries where two last names are used, since a son would
presumably not share both of his father's \emph{apellidos}. In Mexico, \emph{un júnior} is "an obnoxious rich kid."

\appxe{KHAKI} Spelled many different ways; only \emph{caqui} is approved.

\appxe{KLEENEX} As in English, this brand name has done well
and become a generic term that is widely used in Spanish-speaking
countries.

\appxe{KNOCKOUT} Often spelled \emph{nocaut} on the sports pages, this
word is restricted to boxing.

\appxe{LEADER} (\emph{líder}) With its official Spanish spelling, many Spanish speakers don't realize this word is an import.

\appxe{LOBBY} What a more traditional Spanish speaker might call
\emph{un hall} (see above).

\appxe{LOCKOUT} An anti-union tactic that evidently knows no
borders.

\appxe{LONG-PLAY} A common word for that increasingly uncommon object, the phonograph record. Usually, as in English, just called
an LP, which is often written \emph{elepé}.

\appxe{LOOK} Used in the sense of "the fall look" or "the Paris
look," this word is probably imported from the fashion industry. An
individual can also have a "look"; if a friend shows up dressed in
firecracker-red leotards, she might ask \emph{¿Te gusta mi look?}

\appxe{LUNCH} Also written \emph{lonch} or \emph{lonche}. As often happens, this
word is grudgingly accepted as a noun but vehemently railed against as
a verb (\emph{lonchear} or \emph{lonchar}). It generally refers to a late-morning snack
or light meal, not the large midday meal (\emph{la comida}).

\appxe{LYNCH} (\emph{linchar}) This Anglicism has been around so long that
few think of it as a "foreign" word. (It owes its name to one Charles
Lynch, an eighteenth-century Virginia judge.)

\appxe{MANAGER} Another boxer's special, enshrined forever in the
typical post-bout declaration \emph{Todo se lo debo a mi manager} ("I owe it
all to my manager").

\appxe{MARATHON} (\emph{maratón}) It's a little hard to call this quintessentially Greek word an Anglicism, but most Spanish dictionaries
don't include it yet. It is widely used, however.

\appxe{MARKET} (\emph{marqueta}) Borderspeak. Some argue that a border
\emph{marqueta} isn't quite the same as a \emph{mercado} and thus warrants a
new word.

\appxe{MARKETING} An imported concept in its modern, quasi-scientific sense, this practice has produced a number of neologisms
in Spanish, including \emph{mercadeo} and \emph{mercadotecnía}. Some simply call
it \emph{marketing}.

\appxe{MASKING} As good a word as any for "masking tape." (See
also \emph{scotch}.)

\appxe{MASSACRE} (\emph{masacre}) Probably entered Spanish through the
French originally, though I suspect its persistence is due to the translated news services, "Massacre" can be easily handled by \emph{matanza}, but
the verb "to massacre" presents translators with a problem, which
some solve by resorting to \emph{masacrar}.

\appxe{MATCH} A tennis and soccer term, it is also written \emph{metcha}
and \emph{matche}.

\appxe{MEETING} (\emph{mitin} or \emph{mítin}) Like \emph{líder} (from "leader"), this is
so common a word in political circles that many don't realize it's an
import. It usually refers to a large-scale public rally rather than a private gathering.

\appxe{MOHAIR} This English word can trace its origins back to the
Arabic, but the Spanish word is simply the English one.

\appxe{MUFFLER} (mofles) Frequently seen on signs south of the
border.

\appxe{MUSIC HALL} Also spelled music6l, but otherwise there's no
difference between the English word and the "Spanish" one.

\appxe{NICE} A word that has made it into Spanish, sometimes as
nais. Often used snobbily to refer to the "beautiful people" or ironically to refer to the nouveaux riches who think of themselves as beautiful people. Strangely, the nearest Spanish word to "nice" etymologically is necio ("stupid," "stubborn"j, both originating from a Latin
word meaning "ignorant."

\appxe{NURSE} Used in Spain to refer to a "nanny," especially a foreign one.

\appxe{NYLON} Take your pick from nylon, ndilon, or nil6n-an example of what havoc can be wreaked while the Academy is making up
its mind.

\appxe{OFFSET} The printing technique.

\appxe{OVERALL} (overol) Another piece of clothing that had no
ready Spanish word to describe it.

\appxe{PAMPHLET} (panfletoj Usually refers to a political tract or manifesto issued in pamphlet form. One who does such things is known as
a panfletista.

\appxe{PANCAKE} (panque) This widespread Spanish word may come
from the English, despite its French appearance. It doesn't refer to a
"pancake," however, but to a muffinlike or loaflike bread. (See also
hotcake.)

\appxe{PANTS} Certain forms, such as pants, pantis, and pantimedias hint at an English origin, although panta16n is a perfectly legitimate Spanish word. The first generally refers to sweatpants; the second to women's underwear (or "panties"j; and the last to pantyhose.
In any case, both the English and Spanish words come from Italian via
French.

\appxe{PARK} (parkear or parquearj A pet peeve of almost every
Spanish scholar, these verbs are almost universally derided as one of
the world's worst pochismos. (Strangely, though, aparcar is considered
all right because it comes from the French!) In U.S. and border Spanish
you are likely to hear parking for "parking lot," whereas in Spain the
equally dubious aparcamiento is widely used. The correct words, according to those fussy prescriptive grammarians, are estacionar and
estacionamiento.

\appxe{PENTHOUSE} Many apartment buildings (as indicated on elevators) have a pentiaus, as it is sometimes written. No good Spanish
equivalent exists.

\appxe{PICKUP} This common Anglicism is widely used for what
would more correctly be called una camioneta. Note that the word is
feminine: la pickup.

\appxe{PICKLES} Some would have you call them pepinillos agrios,
while others will tell you that they are legumbres encurtidas. Increasingly, and not surprisingly, they are just called pickles.

\appxe{PICNIC} If the Academy adopts this word, will it be spelled piquenique? One can only hope. Dia de campo covers the same concept
as "picnic," but Spanish is not the only language that has felt the need
to adopt the English word.

\appxe{PIE} This means "foot" in Spanish, of course, but that doesn't
stop many Spanish speakers from using it, as in pie de manzana
("apple pie"). The spelling pay is also found,

\appxe{PLAYOFF} You may find this in the sports pages, often as
p1eiof·

\appxe{POKER} Also spelled paquer, it refers to the card game.

\appxe{PONY} (poney) This word is used everywhere Spanish is spoken to refer to a kid-size horse.

\appxe{POSTER} Of the sort that hangs on the wall. Perhaps there's a
proper word for this in Spanish, but I've never heard it used. Cartel
comes close, but without being as specific.

\appxe{PULLMAN} If this isn't foreign enough for you, call it a slipingcar. (See sleeping.)

\appxe{PULLOVER} One of several words for cold-weather wear lfersey, SUefer) that make one wonder what the Spaniards of generations
past wore for cold weather-and what they called it.

\appxe{PUNCH} (ponchel This alcoholic concoction has found a home
in Spanish. A ponchera is a "punchbowl." In beisbol a ponche is also
a "strikeout," and the verb ponchar can mean "to strike out" (in both
the transitive and intransitive sense). (See also puncture.)

\appxe{PUNCTURE} (ponchar) In some countries this verb means "to
puncture." (See also punch.)

\appxe{PUNK} Another English contribution to the world that refers
to this music and lifestyle (blue mohawk haircuts, etc.), a feature of
magazines and news shows worldwide.

\appxe{PUSH} This borrowing turns up in the expression pushele
along the border. Warning: do not try to use expressions like this
at home.

\appxe{PUZZLE} Pronounced "poose-Iay" and used in Spain, this is
called a rompecabezas ("headbuster") in most places.

\appxe{RAID} of the police-knocking-down-doors sort. (See also ride.)

\appxe{RECORD} Not approved by the Academy but everywhere in
the sports pages (and increasingly elsewhere) to refer to new world "records." Marca is a good word for this, too, but it's losing ground fast.

\appxe{RELAX} Used in slang as a noun and an adjective but virtually
never as a verb-its main use in English. As a noun relax refers to a
state of mind, almost a place: Estoy en el puro relax. As an adjective, it
means "relaxing" or "cool" (or "groovy," for that matter). iQue talla
flestat Estd muy relax. In Spain, the classified ads have a section entitled "Relax" that mostly advertises prostitution.

\appxe{RESORT} Tourism has given us this word, which is going to
be a toughie for the Academy to Spanishize. (Resorte means "spring,"
of the sort in your old-fashioned watch.)

\appxe{RIDE} This word is spreading fast in the sense of a ride in a
car. The correct words-vuelta and paseo, for instance-suggest someone deliberately taking you for a "ride," in the country perhaps. A rait
(slang) means you're just hopping in back to save bus fare. In Mexico
the word for this is avent6n, but the Academy and dictionaries don't
recognize it, either.

\appxe{RING} Where un match de box is held.

\appxe{ROASTBEEF} (rosbif) See beef, steak.

\appxe{ROCK, ROCK AND ROLL} The Academy is probably in no
hurry to approve this one, so I might as well tell you that it is already being Spanishized as rocanrol.

\appxe{ROUND} Boxing term.

\appxe{SANDWICH} Believe it or not, you can find this written sangiiichi or sanguche in some places (Argentina, for instance). Sandwich is used everywhere in the Spanish-speaking (and non-Spanishspeaking) world and is fast replacing torta and emparedado-and not
just linguistically.

\appxe{SCOTCH} A widespread word for Scotch tapej often written
and pronounced escotch, escoch.

\appxe{SELF-SERVICE} We tend to forget that a store where you serve
yourself off the shelves is a fairly new concept in much of the world
and thus needs a word to describe it. Self-service does that, as does
autoservicio.

\appxe{SETTER} The dog. Many breeds of dog have English names,
though others are translated (pastor aleman = "german shepherd"). A
cocker spaniel is called a cocker spaniel, which is a bit strange when
you consider that the word "spaniel" comes from the Old French word
espaignol, meaning "Spanish!"

\appxe{SEXY} No comment.

\appxe{SHAMPOO} English speakers may be surprised to find this
word written champu in Spanish, but then Hindi speakers are probably
a bit surprised to find their word champo written as "shampoo" in
English.

\appxe{SHOCK} Used either for the state of shock or for an electric
shock. (Some say that chocar and chocante were taken from the English "shock" and "shocking," but others see a common onomatopoetic origin.)

\appxe{SHOP} Sometimes Spanishized as chop, this refers to what
you do at the mall. Chopear, the verb, is one of borderspeak's uglier
contributions to Spanish. Shopping, lifted straight from English, is also
used, especially south of the border to refer to "shopping sprees" north
of it.

\appxe{SHORTS} SO un-Spanish is it traditionally for adults to wear
shorts that the word they have for this piece of clothing is lifted
from the English. You could call them pantalones cortos, but most
people will look at you quizzically and say, to sea, shortst ("You mean
shorts?").

\appxe{SHOW} This word appears on every marquee and television
variety program. Purists prefer espectdculo.

\appxe{SKETCH} Conserves an older meaning of "sketch," what we
would nowadays call a "skit."

\appxe{SKI} Rapidly adapted as esqui and esquiar.

\appxe{SLEEPING} In some places a train's "sleeping-car" (also
sliping-cart in others a "sleeping bag." (See also pullman.)

\appxe{SLOGAN} Commonly used for a political or publicity slogan.
Many Spanish words cover the same ground, among them lema, but
slogan and eslogan are winning out.

\appxe{SMOG} You can't read a newspaper in most Latin American
capital cities without seeing a story about the esmog. In fact, in certain capitals you may not even be able to see the newspaper because
of the esmog!

\appxe{SMOKING} From "smoking jacket"j in present usage it means
"tuxedo,"

\appxe{SNOB} Very widespread. If you are a true snob, you pronounce
it snob and not esnob, like the plebians do. Usually used as a predicate
adjective rather than as a noun: Es muy snob instead of Es un snob.

\appxe{SNORKEL} Not the sort of word that has a ready, Latin-based
equivalent. Expect to hear esnorklear on the tropical beaches of the
Spanish-speaking world.

\appxe{SPEAKER} Quite commonly heard, especially in Spain, for
the orator at a public gathering or on the radio. Also refers to a congressionalleader ("speaker of the house"), and is gradually coming
into use for stereo speakers as well. Spellings vary wildly: speaker,
espeaker, espiker, espiquer, and so on.

\appxe{SPEECH} What a speaker gives.

\appxe{SPLEEN} A Spanish word-bazo-exists, but this term, more
than most other medical ones, tends to appear in Spanish, as spleen,
splin, and even esplin. The common use in Spanish-for instance in
a famous Piazzola tango-is not medical but metaphorical or poetic,
meaning "tedium" or "despondency."

\appxe{SPORT} Now used almost exclusively for clothing, except in
the combined forms sportsman and sportswoman, which are generally
written sportman and sportwoman these days.

\appxe{SPRAY} Aerosols are on their way out, but they are still called
spray in many places. Usually pronounced to rhyme with "high."

\appxe{STAFF} Words exist in Spanish for this simple concept, but
many office workers and managers refer complacently to the staff, and
it's tempting to do the same.

\appxe{STAND} This word is common for the little booths set up at
trade shows and other events, probably because Spanish speakers visit
many trade shows in English-speaking countries. A more correct Spanish word for these is puesto.

\appxe{STANDARD} Used widely in the sense of "international
norm." Such phrases as tamano standard are typical of its use. It
also refers to stick-shift cars, which are called standard or estandar.

\appxe{STEAK} Spelled steak, steik, or esteik. (See also beef.)

\appxe{STEWARD} This word has been written stiuar and may end up
as estiuar, at which point its English origins will be lost to the average
observer.

\appxe{STOCK} Stores everywhere count up their stock or have
things in stock.

\appxe{STOP} In Spain the stopsigns say "STOP." Possibly got its
start in universal telegram language, in which a "stop" was needed to
sign off. In Mexico brake lights are called luces de stop, though "stoplights" means something else in English.

\appxe{STRESS} Too many foreign expressions at one sitting can
contribute mightily to estres, as the term is sometimes written.
Wouldn't you rather that English contributed words for "peace,"
"love," and "understanding"?

\appxe{STRIKE} Only as a baseball term. The English spelling has
held up so far, strangely.

\appxe{SWEATER} Now officially approved as sueter, you may still
see it with the English spelling. There may be a Spanish word for this
200 BREAKING OUT OF BEGINNER'S SPANISH
garment, but nobody I know uses it. My dictionary gives the following
definition: jersey.

\appxe{SWITCH} Usually written as swiche or suich; used in the electrical sense.

\appxe{SWING} The music, predictably, but also the baseball term
"swing." Used interchangeably in some places with abanicar ("to fan")
when it's "a swing and a miss."

\appxe{SYRUP} (sirop or siro) Sirop is the French spelling, and this
barbarism probably has French roots. [arabe is the proper word. Incidentally, maple syrup is generally called jarabe de maple, though a
maple is an Anglicism as well. Stranger still, the Canadian maple leaf
is known in Mexico as la flor de maple, though it is technically an
hoja, not a flor.

\appxe{TEST} Well-established, usually for the self-tests that popular
magazines offer their readers.
THINNER As in "paint thinner." Written tiner, among other
spellings.

\appxe{TICKET} Possibly a word adopted to help people deal with English tourists, it is now widespread even for laundry chits and the like.
There are many Spanish alternatives, including boleto, billete, papeleta, and resguardo.

\appxe{TIMER} No good Spanish word seems to exist for this concept,
and the English term is making inroads. A clothes dryer or microwave
oven without one is almost unheard of.

\appxe{TIP} Not used for a gratuity (that's a propina) but for a piece of
advice or helpful information: iGracias por el tip!

\appxe{TOAST} Used sometimes instead of brindis for the afterdinner, glasses-raised, bottoms-up procedure.

\appxe{TRAILER} Means "tractor-trailer" or "eighteen-wheeler." A
trailero is common borderspeak for "truckdriver."

\appxe{TROLLEYBUS} Where trolleys are still found, they are usually
called the trole or trolebUs.

\appxe{TRUCK} Borderspeak also gives us troca (or troc), though it
needn't have. Usually refers to a pickup. (See also pickup.)

\appxe{TRUST} TRUST, TRUSTE ,and trusti have limited use for
"trustee" arrangements with banks.

\appxe{VOUCHER} Sometimes spelled baucher, the word is largely
synonymous with ticket.

\appxe{w.c.} In addition to W. c., water, waterc1o, and waterc10s are
all vintage ways of saying "the john." Now used only locally. More
widespread terms are sanitario, retrete, inodoro, and baflO.

\appxe{WALKIE-TALKIE} Every politician worth his salt must have at
least three bodyguards carrying walkie-talkies.

\appxe{WALKMAN} Should a Sonyism count as an Anglicism?

\appxe{WATCH} (wacha) This is what a rela; is called on and near the
border. Even more widespread is wdchala for "keep an eye on it."

\appxe{WATCHMAN} Usually written guachimdn or uachimdn.

\appxe{WEEKEND} Properly, fin de semana. Not as widespread as it
might be, perhaps because the five-day work week is still a dream for
many employees in the Spanish-speaking world.

\appxe{WHISKEY} See brandy.

\appxe{YACHT} Now written yate in most places. Yachting is also
sometimes used. .

\appxe{YANKEE} The Anglicism of choice for those who rail against
the system and the American way of life, as it's known in Spanish.
Written (and approved) as yanqui, this word shows up in graffiti more
often than most countries'leaders like to acknowledge-and not in
reference to the baseball team.

\appxe{ZIPER} An Anglicism that was once widespread but that
seems to have since lost out to cierre, a fine, upstanding Spanish
word. In Spain this device is called a cremallera, from the French.

\appxe{ZOOM} Yet more filmmakers' talk.

\chapter{Appendix B. Nuances, Translation Tips, and Words and Phrases Influenced by English}

The following is a rather loose compendium pf words and
phrases that the careful speaker of Spanish will want to be aware of.
Few of them are clear-cut cases of "correct usage" versus "incorrect usage." Rather, they involve subtle differences in shades of meaning.

The words and phrases listed are similar to the "tricksters"
covered in Chapter 3, but-unlike the tricksters-many of these words
and phrases are "misused" by some Spanish speakers as well, usually
as a result of the influence of English on dubbed movies, cable news
or wire services, imported products, and personal contacts. And unlike the Anglicisms listed in Appendix A, most of the words here already
existed in Spanish but with a slightly-or markedly-different meaning.

Some of the English influences mentioned may be debated. In
several instances, what I am calling the "English-influenced meaning"
has existed in Spanish for a long time, though in various states of disuse and dormancy. The revival of these traditional meanings is thus a
result of the influence of English, although the meaning itself may not
strictly speaking be "English-influenced."

\bsk

\appxe{ACOMODAR} Has a more physical connotation than does "to
accommodate/' usually referring to the actual arrangement of objectson a bookshelf,for instance. Think of it as "to put away." For "accommodations/' acomodaciones is dubious; alo;amiento is much better.

\appxe{ADMIRAR} Conveys more emotion than its English equivalent.
Suggests more "to gaze at in awe" than "to look upon with approval."

\appxe{ADMITIR} Not really proper Spanish in the sense of "to acknowledge/' but widely used that way. See Chapter 14.

\appxe{AGONiA} In Spanish, it suggests "deathbed suffering" or imminent death,

\appxe{AGRADECER} A chiefly American Spanish construction for
"to thank" (probably with no English influence). In Spain you're more
likely to hear dar las gracias.

\appxe{AL MISMO TIEMPO} This phrase is gaining currency in the English sense of "at the same time," "however."

\appxe{APARENTE} The Spanish term suggests "that which seems
to be but is not," whereas in English "apparent" is only "that which
seems to be." The same distinction applies to aparentemente and "apparently." That which in Spanish is aparente is that which aparenta
something else-from the verb aparentar, "to make a false show" or
"to pretend to be."

\appxe{APLICACION} Thanks to the English, you may hear this for an
application form; solicitud is the proper word.

\appxe{APOLOGiA} An "intellectual defense"; has nothing to do with
saying you are sorry.

\appxe{APRECIAR} Not "appreciate" in the sense of conveying
thanks ("Thanks, I appreciate it"). Rather, it means "to value"-as
in "appreciate a good wine."

\appxe{ASUMIR} This works as "to assume" only in the sense of "to
assume a title" or "to assume a position." In the sense of "to suppose,"
asumir is the wrong word, though it seems to be inching its way into
the language. Use suponer instead.

\appxe{BLOQUEAR} This fine Spanish word means "to blockade."
The uses of it for "to block a pass," "to block a radio transmission,"
and "to block a path" are new, possibly English-influenced ones.

\appxe{BULBO} In the sense of "lightbulb," this is an Englishinfluenced term. Bulbo is correct Spanish only for bulbs of plants.

\appxe{CAMBIAR DE MENTE} Native Spanish speakers would rarely if
ever make a mistake with this one, despite its obvious resemblance to
"change one's mind." In Spanish it scans literally, suggesting surgical
intervention. Use cambiar de idea or cambiar de opinion instead.

\appxe{CANDIDO} The English word means "frank," "fair," or "unposed." Not so its Spanish cognate, which means "innocent," "naive,"
or "gullible." A sheltered young girl whose parents never let her go out
on dates might be called candida, suggesting purity and virtue.

\appxe{CASUAL} This means "by chance," though its English meaning of "relaxed" or "informal" is catching on-especially in reference
to clothes. Correctly, though, ropa casual would be "clothing put on at
random."

\appxe{CIVIL} Works well in the sense of "civilian," as in "civil authorities," but not well in the sense of "polite."

\appxe{COLEGIO} Not a college but any school, often a private one,
from kindergarten through high school. Also used for professional associations (as "college" is in England): El Colegio de Economistas.

\appxe{COLORADO} This word always means "reddish" or "redcolored," as in a blushing face or a silt-laden river (Colorado River).

\appxe{COMANDO} The Royal Academy has recently approved this in
the sense of a military "commando" unit, which is funny because "commando" originated as a Spanish word but picked up its new meaning
while passing through Dutch and Afrikaans. The next step is approving it for computer commands, which are widely known as comandos
as well.

\appxe{CONDUCTOR} Not the person who punches tickets on a train
but the person driving it.

\appxe{CONFIDENCIA} Doesn't work for "confidence" as in "I have
confidence in the pilot." For that, use confianza.

\appxe{CONGRATULACIONES} It exists in Spanish, but felicitaciones
or felicidades is better.

\appxe{CONSTIPADO} In some places this is used for "constipated,"
but technically it means "suffering from a head cold." The distinction
is rather important for getting the right medicine.

\appxe{CONTESTAR} It means "to answer," not "to contest." For the
latter, use contender, especially in legal situations.

\appxe{CONVENCION} This word is starting to take on the nonSpanish meaning of a planned gathering, as of dentists, dermatologists, or short-order cooks.

\appxe{COPIA} Refers to a photocopy or carbon copy. For a "copy" in
the sense of "issue" (of a magazine, for example), you should use un
ejemplar. If you ask the fellow at the newsstand for una copia of the
newspaper, he's likely to send someone down to the copy shop to
xerox the whole lot, all the while muttering about those gringos
locos.

\appxe{CRUCIAL} Some say that the use of this word in the sense of
"decisive" is the result of English's influence.

\appxe{CUESTION} Not a "question" in the interrogative sense,
which is pregunta. Still, cuestionar and cuestionable are getting a
lot of use these days, and in the process they seem to be inching closer
to the English words "to question" and "questionable." Cuesti6n properly means "matter (at hand)" or "problem."

\appxe{DEMANDAR} Terrorists who make demandas are watching
too much television. The right word is exigencias, and instead of
demandar they should be using exigir. Correctly used, demandar is
"to sue" and una demanda is "a lawsuit" or, occasionally, "a polite
petition."

\appxe{DESGRACIA} Unlike ((disgrace,(( this word has no moral overtones and no suggestion of shame. It is closer to ((tragedy(( or ((ruin,"
and desgraciar is a close fit for ((to ruin." Desgraciadamente likewise
means simply "unfortunately," even ((sadly."

\appxe{DESHONESTO} In reference to people, this means not so
much ((dishonest" as "lewd," ((lustful." It is an especially dangerous
word for women: una mujer deshonesta means, basically, ((a slut."
(See also honesto.)

\appxe{DISCUSION} Often closer to an ((argument/l thana"discussion."
EDITOR What in English would be the "publisher, /I but much
of the Spanish-speaking world has adopted the English meaning. You
may even find editor asistente, which technically means "attending
publisher," not ((assistant editor."

\appxe{EMERGENCIA} In very comprehensive Spanish dictionaries
you will find an old meaning for this word as /lcrisis" or ((serious and
unexpected action." Still, its usage for the past few centuries has been
as a derivate of emerger, ((to emerge," and emergencia has meant
((emergence." Nowadays you will see en caso de emergencia and salida de emergencia, but those are direct translations from the English.
In proper Spanish the word is urgencia.

\appxe{EN CONTACTO} Suggests physical touching. En comunicacion
is what you want to say for "in contact."

\appxe{ESCANDALOSO} Often means nothing more than ((noisy" or
"rowdy,/I though it also can mean "disgraceful./I

\appxe{ESPERAR POR} Supposedly used in some areas for "to wait
for." The por is superfluous, however, as esperar means "to wait for./I

\appxe{ESTIMAR} This shouldn't be used for ((estimate," in the sense
of "estimated time of arrival," but of course it is. Estimar means "to
hold dear," "to esteem,"

\appxe{ESTUDIO DE CASO} Taken from the English "case study./I
Some would have you use estudio monografico instead.

\appxe{EVENTO} In traditional Spanish this word is rarely used; it
means "chance happening./I In modern Spanish it is used as it is in
English: "a planned occurrence./I Eventualmente means ((occurring
by chance," not the more inevitable ((sooner or later/l as an English
speaker might expect.

\appxe{EVIDENCIA} Means "certainty" and has nothing to do (technically) with objects found by police and used in trials.

\appxe{EXTRACTAR} Not exactly "to extract"; more like "to excerpt."
"To extract" is covered by extraer.

\appxe{FACTORIA} Not "factory" and in fact not even a real word in
modern Spanish. (Centuries ago a factoria was a ((trading house set up
on foreign soil.") The word for ((factory" is fabrica.

\appxe{FASTIDIOSO} This means "boring," "bothersome," "a pain"-
and not "meticulous" or "picky" as in English. These concepts can
overlap, but Spanish may be starting to pick up the English meaning.

\appxe{FIRMA} Coming into use and recently approved by the Royal
Academy in the sense of a "business firm."

\appxe{FUTIL} This traditionally means "trivial," not "desperate" or
"useless."

\appxe{GANAR PESO} : This is being to used to mean "to gain
weight"; the concept should be expressed with subir de peso. (See
also perderpeso.)

\appxe{GROSERfAS} This word, which is misused for "groceries" but
which means "rude remarks" or "offenses," is one of the classic pochismos. In the Americas, the term for "groceries" is usually abarrotes. La compra can also be used virtually anywhere.

\appxe{HONESTO} In reference to people, this mostly means "decent"
or even "chaste." Or in the words of one dictionary, "careful not to excite the sexual instinct or offend the modesty of others," It can also
mean "honest," but wouldn't you feel safer with honrado? For many
uses, sincero is an even better translation: tu opinion sincera = "your
honest opinion." (See also deshonesto.)

\appxe{IGNORAR} This means "to know not," but its use in the sense
of "to ignore" is now firmly established in American Spanish.

\appxe{LECTURA} Today you can hear Voya atender la lectura on
university campuses, but fifty years ago you would have heard Voya
asistir a la conferencia.

\appxe{MEJOR QUE NADA} A translation of "better than nothing"
that in Spanish is more often phrased peor es nada, or "worse is
nothing."

\appxe{NiTIDO} This term is being appropriated in some bilingual
areas for the English "neat," which doesn't have a ready Spanish translation. There is a small zone of overlap in meanings {they share a Latin
root!, but nitido correctly means "well defined," "shiny," "sharp."

\appxe{NOTORIO} Generally doesn't have the negative connotations
of the English "notorious" but may be acquiring them. It usually just
means "well known" or "famous." Only in reference to women does it
have a negative tone. Una muter notoria is "a woman who has been
around," more or less.

\appxe{OPERAR UN NEGOCIO} Considered a translation of "to operate a business," though technically operar shouldn't be used this way.

\appxe{ORDEN} Used increasingly for an "order" in a restaurant, just
as ordenar is for "to order." The proper words are pedido and pedir.

\appxe{PERDER PESO} For "to lose weight"; the correct expression is
baiar de peso. (See also ganar peso.)

\appxe{PLAUSIBLE} The influence of the wire services seems to be
changing this word's meaning in Spanish. It means "praiseworthy/' but
it's starting to be used for "seemingly valid" or "apparently acceptable/' as in English.

\appxe{PLANTA} In the sense of "industrial establishment" or "factory" this is almost certainly a product of English influence.

\appxe{PRIMERO DE TODO} Possibly acceptable in proper Spanish,
but often it's just "first of all" dressed up as Spanish. A native speaker
would more likely say antes de nada or antes que nada.

\appxe{PROPAGANDA} This simply means "public relations" or "PR
materiaV' with no negative connotations, but one wonders how long
that wiIllast. It's hard to imagine a multinational corporation earmarking money for propaganda at its Mexican plant, though that's
a perfectly legitimate use of the word.

\appxe{POLiTICA} Fine for "politics/' but only recently in use for "policy." Very widespread as such, but not yet approved by the Academy.

\appxe{PRETENDER} This term should not be used for "to fake/' "to
claim falsely." Generally, it means "to try."

\appxe{RAPAR} I've never seen this used for "to rape/, but I won't be
surprised the first time I do. It means "to shave" or "to cut (one's hair)
short." The word rapista is rare in Spanish, but when used, it means
"barber." (See also violador.)

\appxe{RELUCTANTE} Apparently this word has existed in Spanish
for generations, but English influence has brought it out of its linguistic slumber.

\appxe{REMARCAR} This verb and the related adjective remarcable,
in the sense of "to remark" and "remarkable/' are listed by some as
English-influenced. More likely the influence is from the French, or
was originally. In Spanish remarcar means "to mark a second time"
or, figuratively, "to accentuate."

\appxe{REMOVER} One of its meanings is "to remove/' but usually
this word means "to jiggle about." The English-influenced meaning is
gaining ground. (See Chapter 14.)

\appxe{RESEMBLARSE} An archaism in Spanish, this word may be
making a comeback, thanks to the influence of "to resemble." Stick to
parecerse.

\appxe{RETRIBUCION} Usually suggests simply "payment/' not
"punishment."

\appxe{ROBAR UN BANCO} You should say asaltar un banco for "to
rob a bank/' since robar ("to steal") would suggest that the criminals
have made off with the entire bank building.

\appxe{ROMANCE} In proper Spanish, this noun has nothing whatsoever to do with wooing and cooing but refers instead to a type of
poem. In the real world, though, the battle is lost: un romance is "a
romance," plain and simple. Either way, romantico is approved and acceptable for "romantic."

\appxe{SALARIO} In traditional Spanish a sa1ario is generally a very
low "wage" paid to domestic help. In Roman times, it was even paid in
salt-sal. A "salary" as we know it today should be called un sue1do.

\appxe{SOFISTICADO} This word once existed in Spanish, but it
meant "adulterated" (wine) or "falsified" (logic). "Sophistic" was
the correct English cognate. Now, however, so/isticado has returned
with the English meanings: "refined," "worldly."

\appxe{SUBURBIO} Generally means "slum" or "shantytown" on the
outskirts of a city, though gradually (as they are built) it is coming to
mean "suburb."

\appxe{TEORETICO} An example of a morphological pochismo. In
Spanish the word is tearico.

\appxe{TODO LO QUE QUiERO} A syntactical pochismo, translated
from the English "all I want." Native Spanish speakers would presumably say 10 unico que quiero.

\appxe{TROPAS} Used in the plural to refer to "soldiers," this is
probably English-influenced, from "troops." The singular form is
more traditional.

\appxe{VIAJAR POR AVION} Should be viajar en avian. Again, English is blamed for the confusion.

\appxe{VIENE} Y VA For "comes and goes." Native speakers would say
va y viene.

\appxe{VICIOSO} Not exactly "vicious" in the sense of "unruly" and
"dangerous," but "vice-ridden," "morally debased." In Spanish you
wouldn't refer to animals or hurricanes as viciosos unless you would
consider sicking the vice squad on them.

\appxe{VIOLADOR} The whole vio1ar complex should be used with
care, as vio1ador is the common word for "rapist" and vio1ar means
"to rape." All vio1ar-related words also have the more mundane meaning of "to violate," as in English, but a concept like "Violators will be
prosecuted" would probably be expressed with transgresores or infractores in Spanish.

\appxe{VOLAR} For "to travel by plane," this is probably influenced
by English. Voya volar a Miami ("I'm going to fly to Miami") still
sounds to many as if you've got a lot of flapping ahead of you. Say Voy
en avian a Miami instead.

\chapter{Appendix C. English's "Hispanisms"}

The history of Spanish isn't a distressful one of bombardment
by other tongues but a proud one of influencing the languages with
which it has come into contact. For all languages, evolving is part taking, part giving, and Spanish has given far more than its share.

What follows is a list of Spanish words that have been adopted
into English, or loanwords from other languages (especially American
Indian tongues) that have entered English via Spanish. It is far from a
complete list, but rather, like the other appendices, serves as a sample.

A source of dozens if not hundreds more "Hispanisms" is the
southwestern United States, where writers like John Nichols have
brought many of these terms to a wider public. An asterisk (*) marks a
regional term of the Southwest, while \emph{AI} indicates a word originally
from an American Indian language.

\bsk

adios

adobe

aficionado

alamo*

alcove

alligator

alpaca (AI)

amigo

armada

armadillo

arroyo*

avocado (AI)

bandido/bandito

barbecue (of Haitian Creole origin)

barracuda

barranca*

barrio

bastinado

booby

bouyant

bozo

bravo (Spanish or Italian or both)

bronco

buckaroo

bunco

caballero

cabana/cabaña

cabotage (disputed)

cacao (AI)

cachucha

cacique (AI)

cacomistle (AI)

calabash

calaboose

camarilla

campesino

campo

camposanto*

cannibal (AI)

canoe (AI)

canyon

cantina*

carom

cassava (AI)

castanets

chaparejos*

chapparal

chaps

chile/chili (AI)

chile con carne

chinchilla (AI)

chocolate (AI)

chubasco*

cigar/cigarette (AI)

cinch

cochineal

cockroach

cocktail (disputed)

coconut

commando (or via Afrikaans)

compliment (or via French)

comrade

corral (of Hottentot origin)

coyote (AI)

creole

cuba libre

cuesta*

desperado*

doubloon

embargo

enchilada

encina*

estancia

fiesta

flotilla

frijol*

grandiose (Spanish or Italian or both)

gringo

guano (AI)

guava (AI)

guerrilla

guitar

gusto (Spanish or Italian or both)

hacienda

hammock (AI)

hazard

hombre*

hoodlum (disputed)

hoosegow

hoosier (disputed)

hurricane (AI)

iguana (AI)

intransigent (indirectly)

jaguar (AI)

junta

lariat

lasso

llama (AI)

loafer (disputed)

loco

macho

maguey (AI)

maize (AI)

mambo (of Haitian Creole origin)

mañana

manatee (AI)

margarita

marijuana

matador

maté/mate (AI)

mesa

mestizo

mosey (disputed)

mosquito

mulatto

mustang

negro

nopal (AI)

numero uno

padre (Spanish or Italian or both)

paisano (also Italian?)

palaver (Spanish or Portuguese or both)

palooka (disputed)

patio

peccadillo

peccary (AI)

peon

picaro

picaroon

pickaninny (Spanish or Portuguese or both)

pinto

plantain (in part)

platinum

poncho (AI)

potato (AI)

pueblo*

quixotic

ranch

rancho*/ranchero*/ranchería*

remuda*

retable/retablo*

rodeo

salsa

sambo

sarsaparilla

savanna/savannah (AI)

savvy

sherry

siesta

silo

sombrero

spade (Spanish or Italian or both)

stampede

tango (possibly of Niger-Congo origin)

tapioca (AI or Portuguese)

ten-gallon

tequila

tobacco

tomate (AI)

tornado

tuna

vamoose

vanilla

vicuña

villa

yucca (AI)

