\chapter{Invective and Obscenity}

Familiarity breeds contempt, as a sage once noted, but without
a little familiarity you can't breed anything. So it is, in a way, with
learning to curse in a foreign language. On the one hand, offering pointers may give the impression that I am encouraging students to blaspheme, excoriate, maledict, and heap billingsgate on their fellow humans. On the other hand, it is a simple fact that many students of
foreign languages will try their hand at four-lettered fricatives whether
I offer my counsel or not. And the danger of a little familiarity, one
fears, may be even greater than the danger of no familiarity at all. What
is a teacher to do?
My imperfect solution is to set forth a short checklist of Spanish swear words, making clear that it serves equally well as a short
course on how to sound rude, vulgar, and boorish. Most of these words
and phrases, used at the wrong moment or in the wrong company, are
definitely ``fighting words," and fights---verbal or physical---in a foreign language, foreign country, or even foreign neighborhood are fights
you are predestined to lose. So swear if you must, but swear intelligently. If in doubt about a word's strength or appropriateness, try it out
first in the third person and on a neutral audience. For several of these
words and phrases, in fact, your best bet is to forget that they can be
used in the second person at all!
Two other hazards besides potential physical injury come into
play when you indulge in the crasser forms of the King's Spanish. First,
you will defile the image that some may have of you as a fine, upstanding man or woman---and not one of those reckless, godless, drug-crazed
\emph{norteamericanos} that every Latin American and Spaniard knows so
well from U.S. television programs and movies. Foreign women are especially vulnerable to being ``downgraded" or ``depreciated" in this
way in Latin cultures; it may not be fair, but it happens to be true.
Second, you risk sounding like a complete fool, as anyone who
has heard a foreigner curse in English can imagine. I have seen U.S.
schoolkids maliciously teach foreign classmates the wrong way to
``cuss" only to reap pleasure from hearing Ivan or Pierre say, ``I am so
shit on myself I could piss dog farts!" So don't think you'll sound too
much better cursing in beginner's Spanish. That out of the way, read
on for some pointers in this most dangerous game. What follows is not
meant as a comprehensive treatment but as a bare-bones primer. That
said, I can assure you that if you follow the advice offered here, you
will at least never sound as bad as Ivan or Pierre.

\section{\emph{cabrón}}

%CABRÓN
A feisty fighting word. First, the etymology: a \emph{cabrón} is a ``he-goat" and, from there, ``one with horns," ``a cuckold," ``one whose wife
sleeps around." Its common use on the street, though, is usually to
refer to an aggressive, unlikable, mean-and-ornery kind of person---exactly the sort of person whom you wouldn't want to call a \emph{cabrón}
face to face. (A mean and ornery woman would be called a \emph{cabrona}---again, from a safe distance.) Among friends you may hear it, accompanied by a laugh, when one friend learns of a practical joke or dirty trick
(\emph{cabronada}) committed by another. If Juan is desperately waiting for
María to call and José sneakily turns off the ringer on the phone, Juan
might notice and say \emph{¡cabrón!} In some parts of the Spanish-speaking
world, \emph{cabrón} can also be heard among tough-talking friends in an almost casual fashion, not unlike the way ``motherfucker" is sometimes
heard on the streets of the United States. Be very, very careful about
imitating this quaint local custom: almost everywhere, saying \emph{Oye cabrón} is like asking, ``Hey, wanna knife me?" \emph{Cabrón} is also widely
used as an adjective for ``difficult," generally with \emph{estar}. \emph{Está cabrón}
would work nicely for ``It's a bitch" when referring to a homework
problem or a stubborn bolt. Like ``It's a bitch," \emph{cabrón} with \emph{estar} can
apply generally to any unpleasant extreme. ``How about this heat?"
\emph{Está cabrón}.

\section{\emph{cagar}}

%CAGAR
The rude way to say ``to shit" everywhere in the Spanish-speaking world. In some places it is ruder than in others, but nowhere
will you use it to explain to your grandmother where you're going. For
that, use simply \emph{Voy al baño} or, to be witty, \emph{Voy a donde nadie puede
ir por mí} (``I'm going where no one can go in my place").
You may find that \emph{cagar} is quite common in rude and figurative senses. \emph{Me caga} is a crude way of saying ``I hate" something. \emph{Me
caga el fútbol} = ``I hate soccer." \emph{Le caga ir al cine} = ``He hates going
to the movies." \emph{Cagarse en} means ``to shit on" and, figuratively, ``not
to give a shit for." \emph{Me cago en tu opinión} = ``I don't give a shit about
your opinion." This phrase reaches fullest expression in the classic
phrase of frustration and rage, \emph{¡Me cago en Dios!} Save that one for real
special occasions. \emph{Cagar} is sometimes used transitively in the sense of
``to screw (something) up." All of these usages should be considered
grade-A vulgar, considerably stronger than ``to shit" in English (see
also \emph{mierda}).

\section{\emph{carajo}}

%CARAJO
This is a useful expletive---universal, usually not too offensive, and plenty strong. In intensity, it is about on par with ``goddam"
in English. Thus \emph{No oigo (veo) un carajo} = ``I can't hear (see) a goddam thing." \emph{Carajo} (literally, but locally, ``prick") is often heard in certain fixed expressions, such as \emph{¡Me importa un carajo!} (``I don't give a
damn"). In this sense, it can also be used with \emph{valer}: \emph{¡Me vale carajos
(un carajo) tu opinion!} = ``I couldn't give a goddam about your
opinion!"
In Mexico, the \emph{me vale} format is especially common for expressing intense indifference. By following it with meaningless euphemisms (\emph{sorbete, gorra}, etc.), you approach ``I don't give a damn." Add
madre to it---i.e., \emph{Me vale madre(s)}---and you have a crude, common,
and colorful way in Mexico of saying you don't give a flying french-fry
for anything. It is definitely obscene, but on the other hand you'll find
it on T-shirts sold to tourists. By extension, in Mexico a \emph{valemadrista}
is someone who just couldn't care less about things and calls to mind
a person halfway through a week-long binge. These can be fun expressions, but they are also quite offensive. Use them with care.
Another set \emph{carajo} phrase is \emph{Váyase al carajo}, which equates
well with ``Go to hell." Also very common for ``Go to hell" is \emph{Váyase
al diablo}, literally ``Go to the devil." In all of these constructions, if
you know the person you are cursing, or if it's a young person, you
would use \emph{¡vete!} instead of \emph{¡váyase!} \emph{Mandar al carajo} is ``to tell (someone) to go to hell" or ``to tell (someone) to get lost." ``Where's your kid
brother?" \emph{Lo mandé al carajo}.

\emph{Carajo} or \emph{carajos} can also be inserted into almost any
question to give it a touch of rage and impatience. \emph{¿Cuándo me van a atender?} (``When are you going to wait on me?") is what you ask after waiting five minutes, \emph{¿Cuándo carajo(s) me van a atender?} is what you ask
after waiting forty-five minutes. Likewise, \emph{¿Qué carajo(s) estas haciendo?} = ``What the hell are you doing?" \emph{¿Qué carajo(s) esta pasando?}
= ``What the hell is going on?" To tone things down just a little bit,
you can substitute \emph{diablos} for \emph{carajo} in most questions.
Finally, \emph{¡Carajo!} or \emph{¡Carajos!} by itself makes for a robust expletive. Euphemistic equivalents are endless: \emph{caracoles, caramba,
caray, cáscaras}, and so on.

\section{\emph{chingar}}

%CHINGAR
As Mexican as the margarita, this word has been studied and
written about in considerable detail by some of Mexico's foremost
scholars, Octavio Paz and Carlos Fuentes, for instance, dedicate sizable
sections of \emph{The Labyrinth of Solitude} and \emph{The Death of Artemio Cruz},
respectively, to the psychosociological nature of the word in Mexican
life. Compare an English text with a Spanish one, and you'll get an excellent overview of how these words are used in Mexico.

In general, the word equates remarkably well with forms of ``to
fuck," and all of its uses should be considered just as potent and perilous, \emph{Chingue a su madre} is the classic Mexican curse: fists or bullets
answer it, not other curses. An \emph{hijo de la chingada} is about the nastiest thing you can call a person, and they're definitely fightin' words as
well. \emph{¡Vete a la chingada!} is the same as \emph{¡Vete al carajo!} (see above),
only more so. \emph{¡Chingada madre!} is as strong but can be used as a generic expletive (that is, a word for a hammer-smashes-finger situation).
If you must use it, make absolutely sure that no one around you even
suspects that you're referring to them or their maternal ancestor. \emph{¡Chingados!} (pronounced \emph{chingaos} to rhyme with ``house") is safer as an expletive, but just as strong. In fact, among the \emph{chin-} alternatives, only
\emph{¡Chin!} can be considered an acceptable euphemism for this word; the
others are too rude to count as euphemisms.

A rare few \emph{chingar} usages and derivatives can almost be rated
PG-13, including \emph{chingar} (``to pester"). I've heard a twelve-year-old tell
her father, \emph{Ay papi, no me estés chingando}. That phrase and \emph{No chingues} mean ``Layoff!" or even just ``Cut it out!" when used among good
friends (see \emph{joder}), A fun and useful word, meanwhile, is \emph{chingaquedito}, meaning someone who pesters (\emph{chinga}) in a sly, unobtrusive fashion (\emph{quedito}). A \emph{chingadera} is a rudish word for a ``dirty trick," ``nuisance," or ``major annoyance." A common euphemism for \emph{chingar} in
Mexico is \emph{fregar} (see \emph{joder}), which can be extended to substitute for
the naughty word in many of its various forms: \emph{una fregadera} = ``a
nuisance."

\section{\emph{coger}}

%COGER
\emph{Coger} is, of course, a perfectly ordinary, acceptable, and decent
verb meaning ``to take" or ``to grab." Unfortunately, it also has the
meaning of ``to fuck" in many areas, especially the Southern Cone
countries and Mexico. More precisely, \emph{coger} is a very crude way of saying ``to have sex," but supports none of the figurative uses of ``to fuck"
(See \emph{chingar} and \emph{joder}.) The more poetic (and polite) way to say this,
incidentally, is \emph{hacer el amor} (``to make love"). You can, by the way,
continue to use \emph{coger} for ``to take" in the countries mentioned above
(Spaniards, for instance, do it all the time), but expect a lot of snickering and strange looks.

\section{\emph{cojones}}

%COJONES
Mostly heard in Spain, though universally understood, this
word means ``balls" in the sense of ``testicles" and has many of the
same figurative uses. \emph{¡Qué cojones!} is ``What balls!" Used by itself---\emph{¡Cojones!}---it's just a generic expletive. In both usages this word ranks
as one of the crudest in Spanish. It's a good word to avoid, especially
given the existence of so many good euphemisms. Two common ones
for the first usage are \emph{¡Qué agallas!} (roughly ``What guts!") and \emph{¡Qué
pantalones!} (``What pants!"). The second makes the association between ``pants" and ``manliness." In fact, if the woman is perceived as
the one in charge in a relationship, it is said that \emph{ella lleva los pantalones} (``she wears the pants").
In the physical realm, ``balls" can be called---equally
crudely---\emph{huevos, pelotas}, or \emph{bolas} in addition to \emph{cojones}. The first
and second are mostly American (i.e., of the Americas), and the third is
quite rare. Polite alternatives are scarce, but then so are most polite
references to this part of the anatomy. The proper word is \emph{testículos},
but you can play cute and refer to the general zone as a man's \emph{hombría},
or ``manhood."
\emph{Huevos}, in Mexico especially, is so widely understood to have
a crude second meaning that people generally avoid using the word in
the plural. You too should learn to say \emph{huevo} instead of \emph{huevos} whenever possible and to use the neutral \emph{¿Hay huevo?} instead of the more
risqué \emph{¿Tiene huevos?} to ask a waiter or merchant if eggs are available.
The word \emph{blanquillos} (``little white ones") is also heard in Mexico, and
though it seems unlikely that it has arisen as a ``safe" alternative to
\emph{huevos}, you can use it that way if you wish. Also in Mexico you can
refer folklorically to the testicles as \emph{los aguacates} ["the avocados"),
especially since \emph{aguacate} comes from \emph{ahuacatl}, the Nahuatl word
for ``testicles." Note that ``Los Aguacates" is also the name of a traditional Mexican song, so be careful how you ask the band to play---i.e.,
\emph{tocar}---it.

\section{\emph{coño}}

%COÑO
A common interjection of the ``Jesus H. Christ!" variety, but
considerably cruder. It is sometimes used alone, but more often to introduce a comment. \emph{¡Coño! ¡Que jugada!} = ``Christ, what a play!"
This word is popular in Spain but only selectively so in the New
World, so by using it where it's not common you may sound a little
strange---like someone saying ``Gor Blimey!" in a Texas roadhouse.

\section{\emph{culo}}

%CULO
This is the crude but common word for ``ass" in Spanish. In
some countries it is cruder than in others. In Spain, \emph{culo} is used freely
and easily, extending even to the back end of a car, for instance. In
Mexico, on the other hand, it is downright rude. You will certainly
want to suss out the prevailing mood before unleashing a word like
this. In the meantime, you can safely use \emph{trasero} (roughly, ``rear end")
for most everyday anatomical references and \emph{la parte trasera} for the
rear end of cars. Other relatively polite words for the same corporal region include \emph{las nalgas} (``buttocks"), \emph{la popa} (``rear"), and \emph{el pompi} or
\emph{las pompis}. The last three are somewhat infantile. If you want to have
fun, refer to this region as \emph{donde la espalda pierde su digno nombre}
(``where the back loses its worthy name").
A related and quite offensive word, used mostly in the Americas, is \emph{culero}, loosely ascribing homosexual tendencies to its target
and most commonly heard in loud, lilting chants from the rowdies at
soccer matches: \emph{¡Cuuu-leeeeeeeroooooo!} The target here is usually the
referee or a dirty player on the opposing team.

\section{\emph{hijo de}}

%HIJO DE
Name-calling in Spanish often has less to do with insulting
your opponent---a true macho doesn't care what you say about him---than with insulting your opponent's mother. And the common way to
do that, syntactically speaking, is to refer to your adversary as the son
of something offensive. What with the expression ``S.O.B." in English,
the concept is not a hard one to grasp.

That said, it should be noted that the \emph{hijo de} complex of insults includes some of the strongest epithets in Spanish. English's
``sonnuva" selection is tame in comparison. In the most obvious case,
a beginner might equate \emph{hijo de puta} with ``son of a bitch," reasoning
that the structures are similar. Perhaps so, but the sentiments are a
world apart. ``Son of a bitch" hardly warrants a raised eyebrow these
days, but \emph{hijo de puta} is as strong as they come: ``motherfucker" is
probably the closest to it in intensity and offensiveness. \emph{Es un hijo de
puta} is quite simply the worst thing you can say of someone in most of
the Spanish-speaking world. (See also \emph{puta}.) Should you feel inclined
to use this phrase, note carefully the correct pronunciation: in most
countries, the \emph{d} gets swallowed altogether and the phrase ends up sounding like \emph{¡Hijo'e puta!} Incidentally, a fun word to describe the sort of
actions you would expect from a \emph{hijo de puta} is \emph{hijoeputez}, a word so
amusing-sounding that it almost avoids being obscene.

In Mexico, the form \emph{hijo de la chingada} is sometimes preferred, but \emph{hijo de puta} is universally used there as well. \emph{¡Híjole!} is
also common as an innocent and euphemistic interjection there. In
various parts of Latin America, \emph{hijo} is sometimes turned into \emph{jijo}---pronounced ``hee-ho"---for no particular reason. Either \emph{jijo} or \emph{hijo} can
be used alone or with nothing but \emph{de} to convey the anger of the full
expression without having to employ vulgarities. Also popular as euphemisms are the harmless \emph{hijo de su madre} and the regional, and fun,
\emph{hijo de su diez de mayo}. Keep in mind that phrases like \emph{¡Jijo de!} can
still be interpreted as an offense, even though it is quite general.

\section{\emph{joder}}

%JODER
Outside of Mexico, this is the common vulgarism for ``to
fuck." Even in Mexico, most expressions using \emph{joder} are readily understood, and some are used. Perhaps the commonest use of \emph{joder} is by
itself as an expletive: \emph{¡Joder!} Also heard is \emph{No me jodas}, meaning
roughly ``Don't fuck with me." \emph{¡No jodas!}---like \emph{¡No chingues!} (see
\emph{chingar})---is less forceful, about like ``Cut the shit!" or even ``Bullshit!" This can be quite strong, but said lazily, it carries no more force
than ``Stop bugging me." For this concept you'll also hear the present-participle form: \emph{No me estés jodiendo}. Again, strength depends largely
on the tone. A euphemism for this usage in much of the Americas is
\emph{fregar}, as in \emph{No me friegues} and \emph{No me estés fregando}.

\emph{joder} in the past-participle form conveys a slightly different
idea---closer to ``fucked over" than ``fucked with," though the difference is slight. \emph{Los jodidos} of any country are the poor, downtrodden,
miserable masses. A road in very bad shape might also be called \emph{jodido}. Likewise, you and your friends would say \emph{Estamos jodidos} if you
ran out of gas fifty miles from town on a dirt road at night. Variations
on this phrase will depend on the assessment of blame. If your group's
poor planning skills were at fault, you might say \emph{Nos jodimos} (``We
fucked up"). If someone stole your gas to put you in that predicament,
you could say \emph{Nos jodieron} (``They fucked us over good!"). In Mexico,
these forms of \emph{joder} are used as a somewhat weaker substitute for
\emph{chingar}, and in general in Latin America \emph{fregar} can again be substituted as a harmless euphemism: \emph{¡Estamos fregados!} (``We're screwed").

\section{\emph{lío}}

%LÍO
This word is neither rude nor obscene and, in a sense, doesn't
belong here at all. But it is a useful word to group together a number of
rude or rough-sounding expressions whose basic idea is ``It's a mess."
That would be \emph{Es un lío} in polite Spanish; it could be many other
things in less polite Spanish.

A lot of the ``messy" words and phrases are regional. In Spain
\emph{un jaleo} may heard; in Mexico, \emph{un relajo}. Neither of these is particularly strong, \emph{Bronca} is safe and universal for ``mess," ``altercation,"
or ``violent confrontation." Inversely, \emph{No hay bronca} works well as a
slangy ``No problem" or ``No sweat," and in slangy speech \emph{La bronca
es que} is a common sentence starter meaning ``The problem is\ldots{}"

Ruder words that you are likely to hear in Mexico include \emph{desmadre} (a ``bloody mess" or worse), for which euphemists can substitute \emph{despapaye}. In Mexico and some other countries, a very crude
word is \emph{pedo} (``fart"), meaning ``big trouble" or ``deep shit." (In other
countries, \emph{peo} is used instead of \emph{pedo}.) The equally vulgar \emph{No hay
pedo} (roughly, ``No problem, everything's cool") can be used---and
often is---to back out of an ugly scene. In Mexico \emph{pedo} also means
``blind drunk" or ``shit-faced."

\section{\emph{madre}}

%MADRE
The mother is the paragon of the Latin family and is supposedly held sacred by the Latin male (especially). To show their deep respect for motherhood, Latin males (especially) have taken to inventing an entire vocabulary based on insulting other people's mothers. In
Mexican street slang, for example, \emph{madre} shows up in just about every
phrase, and almost invariably and paradoxically refers to the worst of
everything. Some examples of what you might hear: \emph{hasta la madre} =
``wasted," ``fed up," ``drugged or drunk"; \emph{dar en la madre} = ``to beat
the shit out of"; \emph{romper la madre} or \emph{partir la madre} = ``to beat the
shit out of"; \emph{en la madre} = ``in deep shit." The only exception to this
linguistic paradox seems to be \emph{a toda madre,} which means ``great" (or,
more correctly, ``real fuckin' great"). Less vulgar alternatives include
\emph{a todo mecate} and \emph{a todas margaritas}; the original expression is \emph{a todo dar}.

Beyond \emph{madre} the Mexicans have invented a host of crude
words. A \emph{desmadre}, for example, is a ``total mess," and \emph{desmadrar}
means ``to make a total mess of," \emph{Madrazo}, besides being a common
last name, is a crude word for a physical ``blow" or even an ``ass-kicking." The situation is such that the word \emph{madre} is best avoided
even in polite conversation in Mexico---use \emph{mamá} instead. Finally, as
on certain U.S. streets where you may hear ``yo' mama" as a sort of all-purpose comeback, in Spanish \emph{tu madre} can be handy. It is plugged
into a sentence to rebuff a rude or inappropriate suggestion. For instance, \emph{¡Hágase a un lado!} (``Move over!") can be answered with \emph{¡Que
se haga a un lado tu madre!} (literally, ``Have your mother move over!").
\emph{¡Termina rápido!} (``Hurry up and finish!") \emph{¡Que termine rápido tu
madre!} For those reluctant to delve into the subjunctive, ¡Tu madre!
by itself generally gets the point across. So does a simple ¡La tuya!--- which packs quite a useful, all-purpose punch and translates roughly
as ``Up yours!" For a watered-down version, \emph{¡Tu abuela!} can be used
instead. Sassy kids (like Mafalda, for instance) might use this. ``Eat
your soup, Mafalda." \emph{¡Que la tome tu abuela!}

\section{\emph{maldición}}

%MALDICIÓN
As a mild expletive, \emph{¡Maldición!} is equal to ``Damn!" For
``Damn you!" use \emph{¡Maldito seas!} (or ¡Maldita seas! for a woman). This
works also for objects: \emph{¡Maldito sea tu cache!} = ``Damn your car!" For
``damn" as an adjective, \emph{maldito} is one of many words that fill the bill:
\emph{tu maldito coche} = ``your damn car." \emph{Mugre} is also in widespread use,
though it is much more polite---akin roughly to ``lousy" or ``no-good."
\emph{Condenado} and \emph{recondenado} are also heard, though less often. And in
a few choice expressions, even \emph{puta} is used as a modifier, chiefly in the
phrases \emph{¡Qué puta vida!} and \emph{¡Qué puta suerte (la mía)!} Its euphemistic substitute in these expressions is \emph{perra}. \emph{¡Qué perra vida!} = ``Life's
a bitch."

In Mexico the word \emph{pinche} serves this intermediary function
as well. Although it's a good bit more vulgar than \emph{maldito}, it is still
exceedingly common. When the Mexican bandido cop in \emph{Treasure of
the Sierra Madre} says to Bogart, ``Badges? We don't need no stinkin'
badges!" what he would have said in Spanish for ``stinkin'" is \emph{pinches}.
Literally, \emph{pinche} is a noun referring to a lowly kitchen employee; the
noun is still used, though now it can describe any low-level employee
or ``gopher." \emph{Pinche gringo}, incidentally, is one of the typical remarks
that sullen, gringo-hating Mexicans like to mutter under their breath
as you walk by. Ignore it, or, if you're feeling especially feisty, say
\emph{Gringo sí, pinche tu abuela}. Better yet, buy the guy a beer.

\section{\emph{maricón}}

%MARICÓN
In a macho culture like the Latin one, there is no getting away
from the epithets that attack an opponent's manhood. Most common,
most unequivocal, and strongest is \emph{maricón}, which is roughly equal to
``faggot." Used by a woman against a man, it is a truly debilitating insult; that is, it can be a good way in certain public circumstances to get
a man to leave you alone. Used by men, it is either a challenge equal to
``coward" (or ``pussy"), in which case it is a fighting word; or it is dismissive, as when the target of the insult really is effeminate---and thus
not a worthy opponent for a ``true" macho. Equally rude and offensive---though not as universal---are \emph{puto} and \emph{joto}. Somewhat less rude
choices include \emph{marica, mariquita, mariposón}, and \emph{barbilindo}. They
range in intensity roughly from ``queer" to ``queen." \emph{Poco hombre} is a
common expression for any ``unmanly" man. Both it and \emph{marica} mean
little more than ``wimp"---that is, without any allusion to sexual preference. In some parts of the macho's world, such epithets are appropriate for ``unmanly" men who help with the dishes, allow women to go
out by themselves, or show emotion when a woman leaves them.

\section{\emph{mear}}

%MEAR
A vulgar word for urination and the equivalent of ``to piss."
The cutesy, children's way of describing urination is \emph{hacer pis} or \emph{hacer
pipí}, which are essentially the same as ``to go wee-wee." \emph{Hacer aguas}
is also heard. The only widespread and correct way of saying ``to urinate" is \emph{orinar}; note that it's spelled and pronounced with an \emph{o}, not a
\emph{u}, at the start.

\section{\emph{mierda}}

%MIERDA
The universal vulgar word for ``excrement" in Spanish is
\emph{mierda}, and you'll find that it is usually an accurate substitute for
``shit." Some people consider \emph{mierda} very offensive, while others use it
quite freely. The kiddies' word is \emph{caca}, as in \emph{caca de perro} or \emph{caca de
vaca}. \emph{Excremento} is the safe way out.

Overall, \emph{mierda} is probably less frequently used in rude expressions than is ``shit." Neither \emph{caca de toro} nor \emph{mierda de toro}
would be understood as ``bullshit" in its figurative sense, for instance.
Nor is \emph{mierda} used commonly as an expletive. About the only phrase
in which it does take on a figurative meaning is \emph{de mierda}, tacked
on after a noun, to mean ``worthless" or ``piece of shit." \emph{¡Coche de
mierda!} = ``Piece-of-shit car!" Likewise, \emph{una mierda} by itself can
mean ``a piece of shit." \emph{¡Este reloj es una mierda!} = ``This watch is a
piece of shit!" The epithet \emph{comemierda} (literally, ``shit-eater") is regionally favored (in Cuba, for instance) to describe just about anything
or anyone disagreeable.

\section{\emph{pendejo}}

%PENDEJO
This strong, wide-ranging insult literally means ``pubic hair."
Figuratively, it is almost universally used to mean ``asshole" or ``shithead," generally highlighting the target's stupidity more than his or
her maliciousness. Thus a common phrase is \emph{verle la cara de pendejo
(a alguien)}, which means roughly ``to note someone's stupidity in their
face" or ``to size them up as an easy mark." If a sidewalk merchant
tells you a Coke costs four hundred dollars in local currency, you could
rightfully ask \emph{¿De qué me vio la cara?} (``What do you take me for?"). If
you pay the money and don't realize it until later, how you feel about
yourself is a pretty good summation of \emph{pendejo}. By extension, a \emph{pendejada} (``screwup") is any stupid behavior of the sort expected from a
\emph{pendejo}. Extended further, this helpful word can be stretched into \emph{pendejez}, describing someone's general stupidity. ``You paid four hundred
dollars for a Coke?" \emph{Fue un ataque de pendejez}. Polite substitutes for
\emph{pendejada} are \emph{disparate} and \emph{payasada}.

\section{\emph{puta}}

%PUTA
\emph{Puta} is a close fit for the English ``whore," and possibly the
most universal profane oath and expletive of all is \emph{¡Puta madre!} or
simply \emph{¡Puta!} Shortening it further, to \emph{¡Put!} or \emph{¡'uta!} or even to just
\emph{¡'ut!} has the effect of cleaning it up a tiny bit for polite company. And
as it can be trimmed down and thus sugarcoated, it can also be embellished for effect. Quality cursers can drag the curse on for umpteen syllables: \emph{¡La puta que te parió!} (``The whore that bore you!") is an example; \emph{La reputísima madre que te parió} a stronger, even longer one.

These phrases are fine as generic expletives, but as with other
curses, be careful to make it clear that you are not referring to someone else present. If in doubt in the preceding examples, change the \emph{te}
to \emph{me}, converting your oath against an unjust world to an oath against
your own stupidity or clumsiness---and your own mother, incidentally.
Note also that there is a thin line between a generic expletive and a
foul insult, so be forewarned. \emph{¡Puta madre!} is an unaimed expletive,
but \emph{¡Tu puta madre!} is a ballistic insult. You will want to be sure that
both you and your audience are aware of the difference.

