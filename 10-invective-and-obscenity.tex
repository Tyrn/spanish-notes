\chapter{Invective and Obscenity}

Familiarity breeds contempt, as a sage once noted, but without
a little familiarity you can't breed anything. So it is, in a way, with
learning to curse in a foreign language. On the one hand, offering pointers may give the impression that I am encouraging students to blaspheme, excoriate, maledict, and heap billingsgate on their fellow humans. On the other hand, it is a simple fact that many students of
foreign languages will try their hand at four-lettered fricatives whether
I offer my counsel or not. And the danger of a little familiarity, one
fears, may be even greater than the danger of no familiarity at all. What
is a teacher to do?
My imperfect solution is to set forth a short checklist of Spanish swear words, making clear that it serves equally well as a short
course on how to sound rude, vulgar, and boorish. Most of these words
and phrases, used at the wrong moment or in the wrong company, are
definitely "fighting words," and fights-verbal or physical-in a foreign language, foreign country, or even foreign neighborhood are fights
you are predestined to lose. So swear if you must, but swear intelligently. If in doubt about a word's strength or appropriateness, try it out
first in the third person and on a neutral audience. For several of these
words and phrases, in fact, your best bet is to forget that they can be
used in the second person at all!
Two other hazards besides potential physical injury come into
play when you indulge in the crasser forms of the King's Spanish. First,
you will defile the image that some may have of you as a fine, upstanding man or woman-and not one of those reckless, godless, drug-crazed
norteamericanos that every Latin American and Spaniard knows so
124 BREAKING OUT OF BEGINNER'S SPANISH
well from u.s. television programs and movies. Foreign women are especially vulnerable to being "downgraded" or "depreciated" in this
way in Latin cultures; it may not be fair, but it happens to be true.
Second, you risk sounding like a complete fool, as anyone who
has heard a foreigner curse in English can imagine. I have seen U.S.
schoolkids maliciously teach foreign classmates the wrong way to
"cuss" only to reap pleasure from hearing Ivan or Pierre saYJ "I am so
shit on myself I could piss dog farts!" So don't think you'll sound too
much better cursing in beginner's Spanish. That out of the way, read
on for some pointers in this most dangerous game. What follows is not
meant as a comprehensive treatment but as a bare-bones primer. That
said, I can assure you that if you follow the advice offered here, you
will at least never sound as bad as Ivan or Pierre.

\section{\emph{cabrón}}

%CABRÓN
A feisty fighting word. First, the etymology: a cabron is a "hegoat" and, from there, "one with horns," "a cuckold," "one whose wife
sleeps around." Its common use on the street, though, is usually to
refer to an aggressive, unlikable, mean-and-ornery kind of personexactly the sort of person whom you wouldn't want to call a cabron
face to face. (A mean and ornery woman would be called a cabronaagain, from a safe distance.) Among friends you may hear it, accompanied by a laugh, when one friend learns of a practical joke or dirty trick
(cabronada) committed by another. If Juan is desperately waiting for
Maria to call and Jose sneakily turns off the ringer on the phone, Juan
might notice and say icabron! In some parts of the Spanish-speaking
world, cabron can also be heard among tough-talking friends in an almost casual fashion, not unlike the way "motherfucker" is sometimes
heard on the streets of the United States. Be very, very careful about
imitating this quaint local custom: almost everywhere, saying Oye cabron is like asking, "Hey, wanna knife me?" Cabron is also widely
used as an adjective for "difficult," generally with estar. Esta cabron
would work nicely for "It's a bitch" when referring to a homework
problem or a stubborn bolt. Like "It's a bitch," cabron with estar can
apply generally to any unpleasant extreme. "How about this heat?"
Esta cabron.

\section{\emph{cagar}}

%CAGAR
The rude way to say "to shit" everywhere in the Spanishspeaking world. In some places it is ruder than in others, but nowhere
I ~
INVECTIVE AND OBSCENITY 125
will you use it to explain to your grandmother where you're going. For
that, use simply Voyal bano or, to be witty, Voya donde nadie puede
ir por mi ("I'm going where no one can go in my place").
You may find that cagar is quite common in rude and figurative senses. Me caga is a crude way of saying "I hate" something. Me
caga el futbol = "I hate soccer." Le caga ir al cine = "He hates going
to the movies." Gagarse en means "to shit on" and, figuratively, "not
to give a shit for." Me cago en tu opinion = "I don't give a shit about
your opinion." This phrase reaches fullest expression in the classic
phrase of frustration and rage, jMe cago en Dios! Save that one for real
special occasions. Gagar is sometimes used transitively in the sense of
"to screw (something) up." All of these usages should be considered
grade-A vulgar, considerably stronger than "to shit" in English (see
also mierda).

\section{\emph{carajo}}

%CARAJO
This is a useful expletive-universal, usually not too offensive, and plenty strong. In intensity, it is about on par with "goddam"
in English. Thus No oigo (veo) un carajo = "I can't hear (see) a goddam thing." Garajo (literally, but locally, "prick") is often heard in certain fixed expressions, such as jMe importa un carajo! ("I don't give a
damn"). In this sense, it can also be used with valer: jMe vale carajos
(un carajo) tu opinion! = "I couldn't give a goddam about your
opinion!"
In Mexico, the me vale format is especially common for expressing intense indifference. By following it with meaningless euphemisms (sorbete, gorra, etc.), you approach "I don't give a damn." Add
madre to it-i.e., Me vale madre(s)-and you have a crude, common,
and colorful way in Mexico of saying you don't give a flying french-fry
for anything. It is definitely obscene, but on the other hand you'll find
it on T-shirts sold to tourists. By extension, in Mexico a valemadrista
is someone who just couldn't care less about things and calls to mind
a person halfway through a week-long binge. These can be fun expressions, but they are also quite offensive. Use them with care.
Another set carajo phrase is Vayase al carajo, which equates
well with "Go to hell." Also very common for "Go to hell" is Vdyase
al diablo, literally "Go to the devil." In all of these constructions, if
you know the person you are cursing, or if it's a young person, you
would use jvete! instead of jvdyase! Mandar al carajo is "to tell (someone) to go to hell" or "to tell (someone) to get lost." "Where's your kid
brother?" Lo mande al carajo.
Garajo or carajos can also be inserted into almost any ques-
126 BREAKING OUT OF BEGINNER'S SPANISH
tion to give it a touch of rage and impatience. iCuando me van a atender/ ("When are you going to wait on me?") is what you ask after waiting five minutes, iCuando carajo(s) me van a atender/ is what you ask
after waiting forty-five minutes. Likewise, iQue carajo(s) estas hacienda = "What the hell are you doing?" iQue carajo(s) esta pasando/
= "What the hell is going on?" To tone things down just a little bit,
you can substitute diablos for carajo in most questions.
Finally, iCarajo! or iCarajos! by itself makes for a robust expletive. Euphemistic equivalents are endless: caracoles, caramba,
caray, cascaras, and so on.

\section{\emph{chingar}}

%CHINGAR
As Mexican as the margarita, this word has been studied and
written about in considerable detail by some of Mexico's foremost
scholars, Octavio Paz and Carlos Fuentes, for instance, dedicate sizable
sections of The Labyrinth of Solitude and The Death of Artemio Cruz,
respectively, to the psychosociological nature of the word in Mexican
life. Compare an English text with a Spanish one, and you'll get an excellent overview of how these words are used in Mexico.
In general, the word equates remarkably well with forms of "to
fuck," and all of its uses should be considered just as potent and perilous, Chingue a su madre is the classic Mexican curse: fists or bullets
answer it, not other curses. An hijo de la chingada is about the nastiest thing you can call a person, and they're definitely fightin' words as
well. iVete a la chingada! is the same as iVete al carajo! (see above),
only more so. iChingada madre! is as strong but can be used as a generic expletive (that is, a word for a hammer-smashes-finger situation).
If you must use it, make absolutely sure that no one around you even
suspects that you're referring to them or their maternal ancestor. iChingados! (pronounced chingaos to rhyme with "house") is safer as an expletive, but just as strong. In fact, among the chin- alternatives, only
iChin! can be considered an acceptable euphemism for this wordj the
others are too rude to count as euphemisms.
A rare few chingar usages and derivatives can almost be rated
PG-13, including chingar ("to pester"). I've heard a twelve-year-old tell
her father, Ay papi, no me estes chingando. That phrase and No chingues mean "Layoff!" or even just "Cut it out!" when used among good
friends (see joder), A fun and useful word, meanwhile, is cbingaquedito, meaning someone who pesters (cbinga) in a sly, unobtrusive fashion (quedito). A chingadera is a rudish word for a "dirty trick," "nuisance," or "major annoyance." A common euphemism for cbingar in
Mexico is fregar (see joderl, which can be extended to substitute for
INVECTIVE AND OBSCENITY
the naughty word in many of its various forms: una fregadera = "a
nuisance."

\section{\emph{coger}}

%COGER
Coger is, of course, a perfectly ordinary, acceptable, and decent
verb meaning "to take" or "to grab." Unfortunately, it also has the
meaning of "to fuck" in many areas, especially the Southern Cone
countries and Mexico. More precisely, coger is a very crude way of saying "to have sex/' but supports none of the figurative uses of "to fuck"
(See chingar and joder.) The more poetic (and polite) way to say this,
incidentally, is hacer el amor ("to make love"). You can, by the way,
continue to use coger for "to take" in the countries mentioned above
(Spaniards, for instance, do it all the time), but expect a lot of snickering and strange looks.

\section{\emph{cojones}}

%COJONES
Mostly heard in Spain, though universally understood, this
word means "balls" in the sense of "testicles" and has many of the
same figurative uses. IQUe cojones! is "What balls!" Used by itselfICojones!-it's just a generic expletive. In both usages this word ranks
as one of the crudest in Spanish. It's a good word to avoid, especially
given the existence of so many good euphemisms. Two common ones
for the first usage are iQUe agallas! (roughly "What guts!") and iQUe
pantalones! ("What pants!"). The second makes the association between "pants" and "manliness." In fact, if the woman is perceived as
the one in charge in a relationship, it is said that ella lleva los pantalanes ("she wears the pants").
In the physical realm, "balls" can be called-equally
crudely-huevos, pelotas, or bolas in addition to cojones. The first
and second are mostly American (i.e., of the Americas), and the third is
quite rare. Polite alternatives are scarce, but then so are most polite
references to this part of the anatomy. The proper word is testiculos,
but you can play cute and refer to the general zone as a man's hombria,
or "manhood."
Huevos, in Mexico especially, is so widely understood to have
a crude second meaning that people generally avoid using the word in
the plural. You too should learn to say huevo instead of huevos whenever possible and to use the neutral iRay huevo! instead of the more
risque iTiene huevos! to ask a waiter or merchant if eggs are available.
The word blanquillos ("little white ones") is also heard in Mexico, and
128 BREAKING OUT OF BEGINNER'S SPANISH
though it seems unlikely that it has arisen as a "safe" alternative to
huevos, you can use it that way if you wish. Also in Mexico you can
refer folklorically to the testicles as los aguacates ["the avocados"),
especially since aguacate comes from ahuacatl, the Nahuatl word
for "testicles." Note that "Los Aguacates" is also the name of a traditional Mexican song, so be careful how you ask the band to play-Le.,
tocar-it.

\section{\emph{coño}}

%COÑO
A common interjection of the "Jesus H. Christ!" variety, but
considerably cruder. It is sometimes used alone, but more often to introduce a comment. ;Coiio! ;Que iugada! = "Christ, what a play!"
This word is popular in Spain but only selectively so in the New
World, so by using it where it's not common you may sound a little
strange-like someone saying "Gor Blimey!" in a Texas roadhouse.

\section{\emph{culo}}

%CULO
This is the crude but common word for "ass" in Spanish. In
some countries it is cruder than in others. In Spain, culo is used freely
and easily, extending even to the back end of a car, for instance. In
Mexico, on the other hand, it is downright rude. You will certainly
want to suss out the prevailing mood before unleashing a word like
this. In the meantime, you can safely use trasero (roughly, "rear end")
for most everyday anatomical references and 1a parte trasera for the
rear end of cars. Other relatively polite words for the same corporal region include las na1gas ("buttocks"), 1a popa ("rear"), and e1 pompi or
las pompis. The last three are somewhat infantile. If you want to have
fun, refer to this region as donde 1a espalda pierde su digno nombre
("where the back loses its worthy name").
A related and quite offensive word, used mostly in the Americas, is cu1ero, loosely ascribing homosexual tendencies to its target
and most commonly heard in loud, lilting chants from the rowdies at
soccer matches: ;Cuuu-1eeeeeeeroooooo! The target here is usually the
referee or a dirty player on the opposing team.

\section{\emph{hijo de}}

%HIJO DE
Name-calling in Spanish often has less to do with insulting
your opponent-a true macho doesn't care what you say about him-
INVECTIVE AND OBSCENITY 129
than with insulting your opponent's mother. And the common way to
do that, syntactically speaking, is to refer to your adversary as the son
of something offensive. What with the expression "S.O.B." in English,
the concept is not a hard one to grasp.
That said, it should be noted that the hi;o de complex of insults includes some of the strongest epithets in Spanish. English's
"sonnuva" selection is tame in comparison. In the most obvious case,
a beginner might equate hijo de puta with "son of a bitch," reasoning
that the structures are similar. Perhaps so, but the sentiments are a
world apart. "Son of a bitch" hardly warrants a raised eyebrow these
days, but hi;o de puta is as strong as they come: "motherfucker" is
probably the closest to it in intensity and offensiveness. Es un hi;o de
puta is quite simply the worst thing you can say of someone in most of
the Spanish-speaking world. (See also puta.) Should you feel inclined
to use this phrase, note carefully the correct pronunciation: in most
countries, the d gets swallowed altogether and the phrase ends up sounding like iHi;o'e puta! Incidentally, a fun word to describe the sort of
actions you would expect from a hi;o de puta is hi;oeputez, a word so
amusing-sounding that it almost avoids being obscene.
In Mexico, the form hi;o de la chingada is sometimes preferred, but hi;o de puta is universally used there as well. iHi;ole! is
also common as an innocent and euphemistic interjection there. In
various parts of Latin America, hi;o is sometimes turned into ;i;opronounced "hee-ho"-for no particular reason. Either ;i;o or hi;o can
be used alone or with nothing but de to convey the anger of the full
expression without having to employ vulgarities. Also popular as euphemisms are the harmless hi;o de su madre and the regional, and fun,
hi;o de su diez de mayo. Keep in mind that phrases like iJi;o de! can
still be interpreted as an offense, even though it is quite general.

\section{\emph{joder}}

%JODER
Outside of Mexico, this is the common vulgarism for "to
fuck." Even in Mexico, most expressions using ;oder are readily understood, and some are used. Perhaps the commonest use of ;oder is by
itself as an expletive: iJoder! Also heard is No me ;odas, meaning
roughly "Don't fuck with me." iNa jadasl-like iNa chingues! (see
chingar)-is less forceful, about like "Cut the shit!" or even "Bullshit!" This can be quite strong, but said lazily, it carries no more force
than "Stop bugging me." For this concept you'll also hear the presentparticiple form: No me estes jadiendo. Again, strength depends largely
on the tone. A euphemism for this usage in much of the Americas is
fregar, as in No me friegues and No me estes fregando.
130 BREAKING OUT OF BEGINNER'S SPANISH
foder in the past-participle form conveys a slightly different
idea-closer to "fucked over" than "fucked with," though the difference is slight. Los jodidos of any country are the poor, downtrodden,
miserable masses. A road in very bad shape might also be called jodido. Likewise, you and your friends would say Estamos jodidos if you
ran out of gas fifty miles from town on a dirt road at night. Variations
on this phrase will depend on the assessment of blame. If your group's
poor planning skills were at fault, you might say Nos jodimos ("We
fucked up"). If someone stole your gas to put you in that predicament,
you could say Nos jodieron ("They fucked us over good!"). In Mexico,
these forms of joder are used as a somewhat weaker substitute for
chingar, and in general in Latin America fregar can again be substituted as a harmless euphemism: iEstamos fregados! ("We're screwed").

\section{\emph{lío}}

%LÍO
This word is neither rude nor obscene and, in a sense, doesn't
belong here at all. But it is a useful word to group together a number of
rude or rough-sounding expressions whose basic idea is "It's a mess."
That would be Es un lio in polite Spanish; it could be many other
things in less polite Spanish.
A lot of the "messy" words and phrases are regional. In Spain
un jaleo may heard; in Mexico, un relajo. Neither of these is particularly strong, Bronca is safe and universal for "mess," "altercation,"
or "violent confrontation." Inversely, No hay bronca works well as a
slangy "No problem" or "No sweat," and in slangy speech La bronca
es que is a common sentence starter meaning "The problem is ...."
Ruder words that you are likely to hear in Mexico include desmadre (a "bloody mess" or worse), for which euphemists can substitute despapaye. In Mexico and some other countries, a very crude
word is pedo ("fart"), meaning "big trouble" or "deep shit." (In other
countries, peo is used instead of pedo.) The equally vulgar No hay
pedo (roughly, "No problem, everything's cool") can be used-and
often is-to back out of an ugly scene. In Mexico pedo also means
"blind drunk" or "shit-faced."

\section{\emph{madre}}

%MADRE
The mother is the paragon of the Latin family and is supposedly held sacred by the Latin male (especially). To show their deep respect for motherhood, Latin males (especially) have taken to inventing an entire vocabulary based on insulting other people's mothers. In
INVECTIVE AND OBSCENITY 131
Mexican street slang, for example, madre shows up in just about every
phrase, and almost invariably and paradoxically refers to the worst of
everything. Some examples of what you might hear: hasta Ia madre =
"wasted," "fed up," "drugged or drunk"; dar en Ia madre = "to beat
the shit out of"; romper Ia madre or partir Ia madre = "to beat the
shit out of"; en Ia madre = "in deep shit." The only exception to this
linguistic paradox seems to be a toda madre, which means "great" (or,
more correctly, "real fuckin' great"). Less vulgar alternatives include
a todo mecate and a todas margaritas; the original expression is a
todo dar.
Beyond madre the Mexicans have invented a host of crude
words. A desmadre, for example, is a "total mess," and desmadrar
means "to make a total mess of," Madrazo, besides being a common
last name, is a crude word for a physical "blow" or even an "asskicking." The situation is such that the word madre is best avoided
even in polite conversation in Mexico-use mama instead. Finally, as
on certain U.S. streets where you may hear "yo' mama" as a sort of allpurpose comeback, in Spanish tu madre can be handy. It is plugged
into a sentence to rebuff a rude or inappropriate suggestion. For instance, jHagase a un Iado! ("Move over!") can be answered with jQue
se haga a un Iado tu madre! (literally, "Have your mother move over! ").
jTermina rapido! ("Hurry up and finish!") jQue termine rapido tu
madre! For those reluctant to delve into the subjunctive, jTu madre!
by itself generally gets the point across. So does a simple jLa tuya!-
which packs quite a useful, all-purpose punch and translates roughly
as "Up yours!" For a watered-down version, jTu abueIa! can be used
instead. Sassy kids (like Mafalda, for instance) might use this. "Eat
your soup, Mafalda," jQue Ia tome tu abueIa!

\section{\emph{maldición}}

%MALDICIÓN
As a mild expletive, jMaldicion! is equal to "Damn!" For
"Damn you!" use jMaldito seas! (or jMaldita seas! for a woman). This
works also for objects: jMaldito sea tu cache! = "Damn your car!" For
"damn" as an adjective, maldito is one of many words that fill the bill:
tu maldito cache = "your damn car." Mugre is also in widespread use,
though it is much more polite-akin roughly to "lousy" or "no-good."
Condenado and recondenado are also heard, though less often. And in
a few choice expressions, even puta is used as a modifier, chiefly in the
phrases jQue puta vida! and jQue puta suerte (Ia mia)! Its euphemistic substitute in these expressions is perra. jQue perra vida! = "Life's
a bitch."
In Mexico the word pinche serves this intermediary function
132 BREAKING OUT OF BEGINNER'S SPANISH
as well. Although it's a good bit more vulgar than maldito, it is still
exceedingly common. When the Mexican bandido cop in Treasure of
the Sierra Madre says to Bogart, IIBadges? We don't need no stinkin'
badges!" what he would have said in Spanish for IIstinkin'" is pinches.
Literally, pinche is a noun referring to a lowly kitchen employeej the
noun is still used, though now it can describe any low-level employee
or "gopher." Pinche gringo, incidentally, is one of the typical remarks
that sullen, gringo-hating Mexicans like to mutter under their breath
as you walk by. Ignore it, or, if you're feeling especially feisty, say
Gringo sf, pinche tu abuela. Better yet, buy the guy a beer.

\section{\emph{maricón}}

%MARICÓN
In a macho culture like the Latin one, there is no getting away
from the epithets that attack an opponent's manhood. Most common,
most unequivocal, and strongest is maricon, which is roughly equal to
"faggot." Used by a woman against a man, it is a truly debilitating insultj that is, it can be a good way in certain public circumstances to get
a man to leave you alone. Used by men, it is either a challenge equal to
"coward" (or "pussy"!, in which case it is a fighting wordj or it is dismissive, as when the target of the insult really is effeminate-and thus
not a worthy opponent for a "true" macho. Equally rude and offensive-though not as universal-are puto and jato. Somewhat less rude
choices include marica, mariquita, mariposon, and barbilindo. They
range in intensity roughly from "queer" to "queen." Poco hombre is a
common expression for any "unmanly" man. Both it and marica mean
little more than "wimp"-that is, without any allusion to sexual preference. In some parts of the macho's world, such epithets are appropriate for "unmanly" men who help with the dishes, allow women to go
out by themselves, or show emotion when a woman leaves them.

\section{\emph{mear}}

%MEAR
A vulgar word for urination and the equivalent of "to piss."
The cutesy, children's way of describing urination is hacer pis or hacer
pipf, which are essentially the same as "to go wee-wee." Hacer aguas
is also heard. The only widespread and correct way of saying "to urinate" is orinarj note that it's spelled and pronounced with an a, not a
u, at the start.

\section{\emph{mierda}}

%MIERDA
The universal vulgar word for "excrement" in Spanish is
mierda, and you'll find that it is usually an accurate substitute for
INVECTIVE AND OBSCENITY 133
"shit." Some people consider mierda very offensive, while others use it
quite freely. The kiddies' word is caca, as in caca de perro or caca de
vaca. Excrementa is the safe way out.
Overall, mierda is probably less frequently used in rude expressions than is "shit." Neither caca de taro nor mierda de taro
would be understood as "bullshit" in its figurative sense, for instance.
Nor is mierda used commonly as an expletive. About the only phrase
in which it does take on a figurative meaning is de mierda, tacked
on after a noun, to mean "worthless" or "piece of shit." iCoche de
mierda! = "Piece-of-shit car!" Likewise, una mierda by itself can
mean "a piece of shit." iEste reloj es una mierda! = "This watch is a
piece of shit!" The epithet comemierda (literally, "shit-eater") is regionally favored (in Cuba, for instance) to describe just about anything
or anyone disagreeable.

\section{\emph{pendejo}}

%PENDEJO
This strong, wide-ranging insult literally means "pubic hair."
Figuratively, it is almost universally used to mean "asshole" or "shithead," generally highlighting the target's stupidity more than his or
her maliciousness. Thus a common phrase is verle la cara de pendeja
(a alguien), which means roughly "to note someone's stupidity in their
face" or "to size them up as an easy mark." If a sidewalk merchant
tells you a Coke costs four hundred dollars in local currency, you could
rightfully ask iDe que me viola carat ("What do you take me for?"). If
you pay the money and don't realize it until later, how you feel about
yourself is a pretty good summation of pendeja. By extension, a pendejada ("screwup") is any stupid behavior of the sort expected from a
pendeja. Extended further, this helpful word can be stretched into pendejez, describing someone's general stupidity. "You paid four hundred
dollars for a Coke?" Fue un ataque de pendejez. Polite substitutes for
pendejada are disparate and payasada.

\section{\emph{puta}}

%PUTA
Puta is a close fit for the English "whore," and possibly the
most universal profane oath and expletive of all is iPuta madre! or
simply iPuta! Shortening it further, to iPut! or (uta! or even to just
(ut! has the effect of cleaning it up a tiny bit for polite company. And
as it can be trimmed down and thus sugarcoated, it can also be embellished for effect. Quality cursers can drag the curse on for umpteen syllables: iLa puta que te pario! ("The whore that bore you!") is an example; La reputisima madre que te pario a stronger, even IOllger one.
134 BREAKING OUT OF BEGINNER'S SPANISH
These phrases are fine as generic expletives, but as with other
curses, be careful to make it clear that you are not referring to someone else present. If in doubt in the preceding examples, change the te
to me, converting your oath against an unjust world to an oath against
your own stupidity or clumsiness-and your own mother, incidentally.
Note also that there is a thin line between a generic expletive and a
foul insult, so be forewarned. IPuta madre! is an unaimed expletive,
but iTu puta madre! is a ballistic insult. You will want to be sure that
both you and your audience are aware of the difference.

