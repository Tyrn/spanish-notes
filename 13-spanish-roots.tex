\chapter{Spanish Roots}

Until now, we have been treating "Spanish" as a single unit,
a monolith unchanging over time, an easily dissected collection of
words, phrases, syntax, and morphology. It makes sense to approach it
this way, of course. The book that presents the entire history of Spanish, while trying to make modern Spanish intelligible to foreign speakers, would be a much, much longer book than this. I'm glad I don't
have to write that book, and I hope you never have to read it.

But a little history is useful and, for some students, fascinating. For the history of a language is also the history of the people who
spoke it, the voyages they took, the castles and forts they built, the
wars they waged, and the ideas they attacked and defended. In the case
of Spanish, it is the family history of more than 300 million modern-day Spanish speakers. But it is also the history of Europe, of the New
World, of Western civilization, of the entire planet.

You may not have much need for an understanding of the linguistic differences between today's Spanish and what was spoken in
the kingdom of Castile a millennium ago. But it somehow adds to the
thrill of speaking Spanish to know that the language we speak today is
pulsing with the life of centuries. The language in its present form is
but a point on a continuum in the evolution of Spanish; by learning it,
and learning about the worlds that gave birth to it, we join the process
in a small but exciting way.

Spanish, of course, came from Latin, as did French, Italian,
Portuguese, Romanian, and several other less widespread European
languages. In fact, a good way to grasp the origins of Spanish is to examine where, why, and how it branched off from Latin. And for that,
we have to step back in time more than 3,000 years.

A thousand years before the birth of Christ, the Iberian Peninsula---what today is Spain and Portugal---was already home to distinct tribes and incipient civilizations. At the mouth of the Guadalquivir River, the Phoenicians founded the city of Tharsis, or Tartessus,
which became known for its wealth and commerce, according to Old
Testament reports. In the tenth century B.C., according to the Book
of Kings, King Solomon's navy went every three years to Tharsis
"and brought from thence gold and silver, and elephants' teeth, and
apes, and peacocks." Four centuries later the Hebrew prophet Ezechiel
talked of the "merchants of Tharsis," who apparently conducted a
busy trade with Phoenecia (modern-day Syria) at the other extreme of
the Mediterranean.

The Greek historian Herodotus, in the fifth century B.C., refers
to the inhabitants of "Iberia," located presumably near the mouth of
the Iber (Ebro) River in southern Spain. The Greeks also established
colonies in southern Spain, and by the third century the Carthaginians (from what today is Tunisia) began to conquer the entire Iberian
peninsula. Their capital was Cartagena---"New Carthage." Not until
201 B.C., with the conclusion of the Second Punic War and Rome's victory over Hannibal's Carthaginian forces, did Spain start to fall under
Roman rule. That date also marks the formal arrival of Latin in Iberia---the arrival, that is, of the antecedent of the Spanish language.

The Iberians, as we call them today, had a relatively advanced
culture that peaked in about the fourth century B.C., or right before
the Carthaginian invasion. A few examples of Iberian coins and sculpture survive to this day, and the remnants of their written language
still largely defy translation. Tharsis also boasted of a written language dating back, they said, some 6,000 years! The extant examples
of this language are totally unlike ancient Iberian but just as indecipherable. Their only common trait is that both seem to have originated in Africa.

Spain's early history (and, indeed, much of its later history) is a
function of its location between the European and African continents.
If the Iberians and others came from North Africa, then other tribes,
including the Celts and Ligurs, came from the European heartland. In
fact, the Celts, who began settling the northwest of the peninsula in
about the eighth century B.C., began mixing with the Iberians, creating
a "Celtiberian" culture in north-central Spain.

Of the pre-Roman groups, only the Basques have survived
as a separate ethnic group, and only the Basque language has reached
modern times more or less intact. The state of scholarship on the
Basque language reflects the mystery surrounding pre-Roman Iberia
as a whole. Scholars who struggle to locate the origins of a now extinct
Iberian or Turdetan language must also recognize that they are unable
even to trace the origins of Basque, which alone among
western European languages is from non-Indo-European roots. Where, then, did
Basque come from? Once it was thought the Iberians simply became
the modern-day Basques, but recent scholarship has poked holes in
that theory. Now scholars surmise that Basque may have come from
Caucasian, a language spoken in modern-day Georgia between the
Black and Caspian seas. Or, others say, perhaps it is from a Hamitic
source, like certain Sudanic languages, with which it has similarities.
Then again, it may be from Coptic, an Afro-Asiatic language of Egyptian stock still used today in the liturgy of the Coptic Church. One
modern scholar has even suggested that Basque is but another Romance (i.e., Latin-based) language!

\section{The fate of Latin in Iberia}

%THE FATE OF LATIN IN IBERIA
The combined effect of Iberia's pre-Roman cultures and languages on the birth of Spanish is small but significant. Words taken
from Celtic, from Basque, and even from Iberian still linger in the language. From Celtic, early forms of such common words as \emph{cerveza, caballo, carro, camisa}, and \emph{camino} were born. Celtic place-names on
the Iberian Peninsula include Segovia, Evora, Coimbra, and Coruña, although all of these names have undergone considerable changes from
the original Celtic words. Basque has contributed a handful of words,
most of them uncommon, to Spanish, with \emph{izquierda} being probably
the most used.

Other everyday Spanish words that can trace their lineage to
Iberia before the arrival of the Romans are \emph{barro, conejo, gordo, muñeca, perro}, and \emph{sapo}. Certain suffixes are also viewed by today's scholars as typically "Iberian" in origin (though not necessarily from the
Iberians themselves). These include \emph{-rro, -rra, -ago, -era, -iego}, and
\emph{-asco}. The ending \emph{-ez} also dates from pre-Roman times and can still
be found in names like Sánchez, López, Ramírez, González, Rodríguez,
Velázquez, Pérez, and so on.

Even some changes in "spelling" between Latin and Spanish
have been blamed on the residual impact of pre-Roman languages. The
conversion of the Latin \emph{f} to a now silent \emph{h} in Spanish has been explained as part of the legacy of these languages and can serve to begin
our study of Spanish's slow but steady departure from its Latin roots.
The Spanish words \emph{hijo, hacer, hoja, hundir}, and \emph{humo}, for example,
come from the Latin roots \emph{filius, facere, folia, fundere}, and \emph{fumus}. In
these cases, French and Italian remained truer to their Latin roots with
\emph{fils/figlio, faire/fare, feuille/foglia, enfoncer/affondare}, and \emph{fumeé/fumo}. Though now we look on these variants as "spelling changes," at
the time they were changes in the spoken language, which lexicographers
only later would get around to assigning a spelling to. Mostly, though,
the pre-Roman languages, written and otherwise, bowed to Latin and
had vanished by the year 100 A.D., only three centuries after the Romans' arrival. And Rome's relatively unmolested control of the Iberian
peninsula was to last another three centuries, giving Latin a firm foothold there that may never be shaken loose.

Latin, of course, was far from a "pure" language itself, and
Latin speakers borrowed freely from other languages, especially Greek,
to supplement their native tongue. And Vulgar Latin---spoken on the
streets of Rome and by the soldiers in the invading Roman legions---was a far cry from the literary Latin that has come down to us in Roman and other texts. Latin speakers' willingness to improvise probably
accounts for the ready reception of Iberian, Celtic, and Basque words
and morphology. The result was that by the third century the language
being spoken popularly was an even less pure, less "literary" Latin.

Except for a handful of appearances of "common folk" in Roman comedies, few written examples of Vulgar Latin have survived
to the present. Scholars, though, have been able to recreate it based
on how Latin looked a few hundred years later. The "sloppiness" of
its vocabulary and some of its forms also prompted linguists of the
time to write tracts condemning developments in spoken Latin. One
of these was written in the third century by a man named Probus,
whose work shed considerable light on how Latin was changing. Probus also became somewhat famous among later scholars because
nearly everyone of the "errors" and examples of bad grammar that he
railed against was subsequently incorporated---Probus be damned---into medieval Latin.

Students of languages amuse themselves by watching words
change. In this era of Latin, there was amusement enough to go
around. New words were absorbed and old words took on new meanings. \emph{Laborare}, a perfectly functional Latin word for "to work," was
gradually replaced by \emph{tripliare}, referring to a three-sticked instrument
of torture or punishment used on recalcitrant slaves. Thus as work
became more tortuous, a new word was needed to emphasize the unpleasantness of the experience. And so a word was "born" into popular
usage, eventually turning into the modern word \emph{trabajar}, "to work."

Other common words have similarly intricate histories and
show in passing how Spanish began to emerge from shared Latin roots.
The Latin word for "grandmother," for instance, was \emph{avia}; in France
the word was abandoned in favor of a Germanicized \emph{grand-mere}; in
Italian, \emph{nonna}, evidently a Late Latin derivative from child's talk (and
the source for the word "nun"), was adopted; only Spanish was to hold
on to the original Latin, and then only after "sweetening" the word
by attaching the diminutive suffix -ola. From there aviola became
abuela, the modern Spanish word for "grandmother." Knowing this
history makes it curious to hear the form abuelita, which has begun to
replace abuela in many parts of the Spanish-speaking world-and for
the same reasons that aviola replaced avia. The word, twice diminished, can hardly get any sweeter!

Some words, in use in the protected worlds of the church or
the law, survived this period of change intact. Sometimes these "original" Latin words continued to exist in these closed confines while
on the streets new words were being coined from the Classical Latin
roots. Sometimes, too, society's more cultured elements returned to
the Latin to express themselves, refusing thereby to "lower" themselves to the level of Vulgar Latin. This reintroduction of "cultisms"
would continue for centuries after Vulgar Latin had evolved into Spanish (or, strictly speaking, Castilian). As a result "Classical Latin" and
"Vulgar Latin" words survive side by side in Spanish today: frio and
frigido, oreta and auricula, delgado and delicado, entera and integra,
and so on. .

\section{Germanic Spanish}

%GERMANIC SPANISH
At this point, around A.D. 400, Spain and its language seemed
destined for a fairly uneventful development. Latin would continue to
be used, and abused, and the language that would evolve would be but
a regional, distorted, and "vulgar" version of the Classical Latin of
Cicero and Caesar.

It almost worked out this way, but not quite. In the year
409 A.D., the Vandals (the original ones, for whom all later vandals are
named) crossed the Pyrenees into Spain. The Vandals were a Germanic
people, and their invasion of Spain was to be the first of a wave of Germanic invasions. The Vandals themselves only lasted twenty years
on the peninsula. Hot on their trail-and fresh from sacking Rome in
41O-the Germanic Visigoths reached Spain only a few years later.
Their arrival, in fact, pushed the Vandals clean off the continent. The
Vandals crossed into North Africa in 429, conquered Carthage ten
years later, and sacked Rome themselves in 455. The Visigoths, for
their part, completed their conquest of Spain under their king, Euric
(466-484), and placed their capital in Toledo. Their stay was to last a
little longer than the Vandals'-some three hundred years, in fact.

The Visigoths' language was, of course, a Germanic one,
which meant that it shared an ancient Indo-European heritage-but
little else-with Latin. Still, the linguistic shock of the Germanic conquest was considerably less than might have been expected. First, the
Visigoths already spoke Latin and were themselves quite "Romanized"
as a result of being based for a full century in Toulouse before sacking
Rome and spreading into Spain. Second, the Visigoths kept to themselves ethnically, especially at first, swearing off mixed marriages. And
third, the Germanic languages had been filtering into Latin since at
least the first century A.D., which explains why many of the Romance
languages have parallel Germanic constructions.

Still, the Visigoths did leave their mark linguistically in proper
names, place-names, and in words associated with what they did best:
fighting. The word guerra ("war") is of Germanic origin, as are orgullo
("pride"), riqueza ("riches"), rabar ("to rob"), ganar ("to win"), and
bandido ("bandit"j. Of the place-names attributed to Germanic influence, Andalusia is probably the best known; it is but the Arabized
form of Vandalus, referring to the Vandals. And the proper names Alvaro, Rodrigo, and Fernando are "Hispanicized" versions of the Visigothic names Allwars, Hrothriks, and Frithnanth.

The lasting significance of the Visigoths on Spain, however,
had more to do with politics than with linguistics. Their conquest of
Hispania, for instance, effectively severed its ties with Rome for nearly
three centuries, leaving the Latin on the peninsula to develop in isolation and thus stray further from the Latin being spoken elsewhere. By
conquering most of the many disparate tribes on the peninsula, the
Visigothic lords also accomplished the first unification of Hispania,
giving birth to a larger political entity that later groups would struggle
to re-create.

\section{Arabic Spanish}

%ARABIC SPANISH
The three-century domination of Hispania by the Visigoths
ended in the year 711, when Arab forces (or Moors, as the Arab/Berber
Moslems on the peninsula were called) routed the Visigoth force.
Within seven years the Moslems, whose founder and prophet Mohammed had died only eighty years before, had conquered virtually the entire peninsula. The Visigoths stayed on under Moorish rule, enjoying
the tolerance of the new rulers. Science, art, and philosophy flourished
under the Moors, who promoted (or at least presided over) a bilingual
culture in which the Romanized groups became progressively more
Moorish and the Moors increasingly Romanized. The first group is
known as the Mozarabs (in Arabic, "would-be Arabs"j, the second, as
the Mudejars ("those who have been allowed to stay"-after the reconquest began, primarily). Hispania under Arab rule became a hotbed of
scientific activity, leading Europe in this sense and attracting some of
the best minds from around the continent.

The bicultural aspect of Moorish Spain loses none of its power
to astound more than a millennium later. Arab scholars wrote treatises
in Latinj Christians baptized their children with Arab names. Christian scholars, writing in Arabic, expounded on the fine points of their
religion. In Moslem Cordoba, churches, mosques, and synagogues
could be found in close proximity and operating openly.

The language, of course, opened its arms to embrace such
diversity. The legacy is a Vulgar Latin full of "Arabisms"-more
than 4,000 in all have survived in Spanish to our day. Many of these
words were to pass through Spanish (or more commonly, through
French) into English. A short list of the more indispensable ones
barely scratches the surface: azul ("blue"), escarlata ("scarlet"), limon ("lime"), naranja ("orange"), adobe ("adobe"L talco ("talcum
powder"L azar ("hazard/' "chance'l aceite ("oil"), cero ("zero"), cifra
("cipher/' "figure"L ajedrez ("chess"L cenit ("zenith"L nadir ("nadir"),
jazmin ("jasmine't azucar ("sugar't azafrcm ("saffran"L zanahoria
("carrot"), aduana ("customhouse"L tarifa ("tariff"), and arroz ("rice").

Ojala, which translates to English roughly as "I hope/' is but
a Hispanicized version of the Arabic expression wa-sa Allah (or in
sha'allahl, meaning "may Allah wish it."

Another vast group of Spanish words of Arab origin are those
beginning with al-, which is simply the definite article "the" in Arabic. This list also includes a number of words well known to English
speakers, such as algoritmo, algebra, alquimia ("alchemy"l, aleohol,
alcoba ("alcove"l, and almanaque ("almanac"). With a little imagination a few other al- Spanish words from Arab roots will look familiar
to us as well: aleanfor ("camphor"), algodon ("cotton"), and almirante
("admiral"). Some al- words you will need to learn in Spanish, if you
haven't already, include alllier ("pin"), almacen ("warehouse"L almohada ("pillow"), alcalde ("mayor'l alfombra ("rug't and almorzar
("to eat lunch").

Many of these"Arab" words are really just Arabized Latin and
Greek wordsj others are words that Arabic took from Persia, China, India, and Sumatra. But all of them entered Spanish (and some entered
French and Italian as well) through Arabic, and the vast majority of
them entered during the period of Moorish rule on the Iberian Peninsula. During this time the Moors also contributed mightily to Spanish
place-names, such that some of the places we consider most "Spanish"
are, in fact, Arab toponyms: Caceres, Gibraltar, Guadalquivir, and
Guadalajara, for instance. Many others are Arabized forms of Latin
names: Sevilla from Hispalia, for example, and Zaragoza from Caesaraugusta.

The kingdom of the Moors was to last almost 370 years in Toledo, some 500 years in Cordoba and Andalusia, and an astonishing
770 years or more in Granada. When the Moriscos, or "leftover"
Spanish Moors were forcibly expelled from Spain in the seventeenth century-more than 300,000 were "sent back" to Africa between 1609
and 1614-the "Arabs" had been in Spain three times as long as the
"English" have been in America.

\section{Castilian Spanish}

%CASTILIAN SPANISH
The stage is now set, in our lightning recapitulation of Iberian
history, for the "reconquest" of Spain by Romanized Christian forces.
But to understand well what was to happen to Spain and Spanish in
later centuries, we should pause here andJook at what else was going
on in Spain while the Moors held control of the bulk of the peninsula.

Beyond the mountains to the north and northeast, groups that
had managed to remain free of the Moorish yoke settled into small
kingdoms. To the northwest, the kingdom of Leon considered itself the
direct descendant of the Visigoth reignj to the northeast, the largely
Basque-speaking Navarre kingdom extended well into Francej south of
Navarre, a tiny kingdom called Aragon, speaking a Vulgar Latin much
like that of Leon, took hold under Navarre's protection and began an
independent expansionj Catalonia (Cataluna), to the east, was basically
part of France this whole timej Portugal set itself off as a kingdom and
eventually broke away altogether in the twelfth centurYj in Moorish
Spain itself, finally, the Christian Mozarabs continued to live in a variably bicultural and bilingual setting. Their language-the Mozarabic/
Visigothic/Vulgar Latin of southern Spain-evolved relatively little in
comparison with the tongues being spoken in the northern part of the
peninsula.

Lastly, a tiny kingdom carved out a niche for itself in the uninviting borderlands between Leon to the northwest, Moorish Spain to
the south, and Aragon to the northeast. This kingdom, called Castile,
was made up of a feisty, stubbornly independent people who fought
encroachment on all sides and with ever-increasing success. They resisted the Visigothic trappings and laws of Leon but refused to completely "Arabize" themselves either. Their language was coarse-unlike the relatively uniform dialects of Vulgar Latin being spoken on all
sides. In both Aragon and Leon, on either side of Castile, people said
mullerj in Castile they said mugerj in modern-day Spanish we say
muier. In both Aragon and Leon people said ferit, feito, and uelloj in
Castile these words were her-ir, hecho, and oio. In short, from the
standpoint of the rest of the peninsula's residents, the Castilians had
a strong and bizarre accent. And, thanks to political developments, in
a few short centuries almost all of Spain would have it as well.

From as early as about 1000 A.D., manuscripts from Castilian
monasteries show translations-in the form of margin notes or glossesfrom the Latin to the Vulgar Latin in use at that time. These translations are taken by many scholars to represent the first written record
of Castilian, and thus to mark the birth of the Spanish language some
1,000 years ago.

The Castilians were to become the "reconquerors" par excellence of the Iberian Peninsula. In 1029 the kingdom fell under Navarran control almost by default, and Ferdinand, the son of the Navarran
king, inherited it and declared himself Ferdinand I, king of Castile. His
neighbors in Leon didn't think much of Castile having a king and tried
to take the wind out of Ferdinand's sails, only to be defeated themselves in 1037. Ferdinand I thus became the king of Castile and Leon
and went on to annex a chunk of Navarre at the expense of his brother,
who had inherited it and whom Ferdinand murdered.

Castile thus became a regional power, and though the union
did not stay in place permanently, Castile was strong enough now
to begin expanding southward, into Moorish country. The advance
of Castile was like that of a wedge splitting the peninsula in two.
In 1085, Castile took Segovia, Avila, and Toledo and two years later
moved its capital to the last of these.

\section{The spread of Castilian}

%THE SPREAD OF CASTILIAN
The advance of the language was slower than the military and
political thrust. A century and a half after Castile's "reconquest" of Toledo, for example, Arabic was still being used for official documents
there. Castilian forms and pronunciations were being used in the "New
Castile" based in Toledo, but simultaneously many Mozarabic words
were infiltrating Castilian. Overall, Castilian was an open and innovative language, absorbing influences as its speakers advanced politically
and geographically.

Not until the reign of Ferdinand III (1217-1252) was the use
of Castilian imposed on New Castile. It was also Ferdinand III who renewed the spread of Castile's power, "liberating" Caceres to the west
(1227), Valencia to the east (12381, and Cordoba (1236) and Seville
(1248) to the south. By the end of his reign, Moorish Spain was backed
into one kingdom-Granada-and for his trouble and for his efficiency,
Ferdinand III was to be granted sainthood. Granada was defeated 250
years later (its liberation and the unification of Spain was the big news
on the peninsula in 14921 by Isabel. Isabel, queen of Castile, had a decade earlier married Ferdinand V, king of Aragon, thereby joining those
two kingdoms and setting the stage for Spain's final unification.

Castile's conquest of Spain, except for Granada, was essentially completed by the mid-thirteenth century. Consolidation came
next, and in the case especially of the language, the man for the task
was Ferdinand Ill's son, Alfonso X, a.k.a. "The Learned" (1252-1284).
Alfonso was a studious sort, not given to military escapades and not
very successful in those he did attempt. Instead, he surrounded himself
with sages, scientists, poets, and historians, and set up his court at Toledo as a patron of scholarship. Though Alfonso took pains to assure
the harmonic mingling of Spain's special "triple heritage," made up of
Christian, Hebrew, and Arabic learning, he adopted Castilian as the officiallanguage of the court and the kingdom, replacing Latin.

The importance of the decision also resided in Alfonso's work
(with the help of his hired scholars) in unifying criteria for his new language. Thus for Castilian to have any meaning as a language, someone
had to decide which form of a word-the Mozarabic, the Old Castilian,
the Aragon, the Leon-was to be adopted thereafter for incorporation
into New or "Toledan" Castilian. His choices naturally became the official ones, not so much by decree as by consent-and by the fact that
the court commissioned much of the literature and published most of
the peninsula's written texts. The growth of Castilian literature, promoted by Alfonso, marked the definitive emergence of the Spanish
language.

In certain essential respects, the Spanish written in 1200 and
1300 is much the same as the Spanish of today and can be read with
relatively little difficulty by modern Spanish speakers. Cantar del mio
Cid, dating back to about 1200 but not written down until 1350 or so,
is quite intelligible to any modern Spanish-speaking personj Chaucer's
Canterbury Tales, on the other hand, was written close to 1400 but
is usually "translated" for present-day English speakers. Certainly by
1500, except for the influx of foreign words, the shape and structure of
Spanish was pretty well carved in stone.

In fact, foreign influences had never stopped. From roughly
1100 to 1300, a vast number of French words entered through the
north of Spain, brought by religious "tourists" on pilgrimage to Santiago de Compostela in Galicia. The route they took across the north
of Spain, through Navarre, Castile, and Leon, was littered with Gallicisms, some of which survived the intervening centuries and persist in
modern Spanish. These include numerous -aje words (mensaje, homenaje, coraje, viaje, salvaje) and other terms related to the traveling life:
meson, jornada, jardin, and ligero. The 1400s, in contrast, were years
of great Italian influence, as Spanish scholars strove to catch up withor even keep up with-the advances of Renaissance Italy. The incorporation of Italianisms was an ongoing process, lasting several centuries,
that counted among its harvest such everyday words as marchar, millon, banca, balcon, diseiio, modelo, capricho, novela, cortejar, charlar,
manejar, pedante, grotesco, piloto, soldado, and alerta.

The close of the 1400s was typified by that dramatic year
1492, which saw the expulsion of Spain's Jewish population (the dispersed Sephardic Jews), the discovery of America, and the fall of Granada and reunification of Spain. In this year as well Nebrija published
his Gramatica de la lengua castellana, which was an attempt to standardize Spanish and the first published grammar in any modern language. In short, while Spain was extending its reach geographically, it
was slowly closing itself off to "outside," non-Christian intellectual influences. These years were to witness the start of the Spanish Inquisition as well, although its influence was not fully felt in Spanish culture until the late 1500s.

\section{International Spanish}

%INTERNATIONAL SPANISH
The 1500s began with Spain asserting itself internationally.
The conquest and colonization of America took place in these years,
and newly unified Spain was assured a prominent place in Europe.
Castilian was by the 1530s the language spoken throughout unified
Spain. Spain was ruled from 1516 to 1556 by Charles V (Carlos I of
Spain, known in the Spanish-speaking world as Carlos Quinto). Born
in Ghent, in present-day Belgium, Carlos didn't set foot on Spanish
soil until he was eighteen years old, but he is still considered one of
history's great defenders of the Spanish language. By 1550, Italian gentlemen took pains to learn Spanish, which for the next one hundred
years would reign as Europe's preeminent language. Around 1660, for
example, anybody who was anybody at the court of Brussels spoke
Spanish. .

The reign of Carlos's son, Felipe II, ran from 1556 to 1598 and
marked the high point for the Inquisition and Spain's xenophobia. It
also marked the low point in many other respects. Felipe II was a repressive, autocratic, and intolerant king even by sixteenth-century
standards, and the religious zeal that characterized his rule earned
Spain a reputation for backwardness and barbarism that it has never
overcome in some parts of Europe. Paradoxically, Spain's Siglo de Oro,
or Golden Age, began in earnest under Felipe. It was a century that saw
the emergence of such cultural greats as Cervantes, Gongora, Lope de
Vega, and Quevedo.

The exploration and conquest of America had a lasting effect
not just on Spain but on its language. Spanish became filled with
"Americanisms," many of which it then passed on to other European
languages. The names of fruits, animals, and American products came
into Spanish from the Carib, Nahuatl, Quechua, and other indigenous
tongues (see Chapter 14). From the ongoing exploration and slave trade
in Africa came words from that continent into Spanish, with banana,
conga, bongo, and samba among the ones that have survived to the
present.

Spain's importance geopolitically in these years translated into
what we now view as an excess of pride and arrogance. The same attitude that permitted the mass expulsion of 300,000 Arabs from Spain
in the period 1609-1614 appears in the language as well. It was in
the early years of the 1600s, for instance, when a new form of respect,
usted, was invented. Until that time, Spanish speakers had used vos,
the second-person plural, in respectful forms of address. This construction shares the logic of the "royal we" in English, by which royals refer to themselves in the plural. At some point their vassals must have
figured that if the king or queen considered himself or herself to be a
plural, it was not the vassal's duty to inform the monarch otherwise.
And thus plural forms of address like vos (vous in French, voi in Italian) were born. For certain Spanish lords of the seventeenth century,
though, vos had become too widely used, and a new form of address
was "needed" to distinguish those who genuinely merited adulation.
The form proposed and adopted, in a matter of decades, was vuestra
merced ("your grace"l, which in just twenty years of popular use was
chopped down in size to a more manageable usted.

By the 1700s, the effects of centuries of intellectual isolation
and repression were clearly showing. The Siglo de Oro was long gone,
and nothing even close to its output was accomplished through the
1700s. Instead, the eighteenth century was a good one for lexicographers and grammarians, as creative energy and talent was channeled
away from more risque endeavors. For example, in the years 1756,
1759, 1762, and 1764, respectively, the Inquisition banned in Spain and
its possessions the works of Montesquieu, Diderot, Voltaire, and Rousseau. The Royal Academy of the Language, meanwhile, was founded in
1713, and spent the century publishing reference works-the equivalent of decrees-on grammar, spelling, and literature. These had the
effect of establishing official spellings for Spanish words (much as
Noah Webster did for American English half a century later), and although Spanish has certainly changed since this period, its changes
have been minor ones.

French influence heightened in the 1700s, and in fact by century's end Spain found itself controlled outright by Napoleon. By far
the largest number of Gallicisms entered the language during this period, a collection of words that includes, for starters, asamblea, burocrata, controlar, jinanzas, pantalon, chaqueta, and equipaje (another
-aje wordl. French phrases and constructions were imported in this era
as well; for example, the expression hacer el amor in the sense of "to
court" or "to woo" became popular. (Its other, more modern sense is
attributed to the influence of American English.) One wonders how
many more French words would have entered in this period if Montesquieu, Diderot, Voltaire, and Rousseau had not been blacklisted!

Napoleon's misadventures in Europe gave Spain's colonies the
opportunity to slip away in the early 1800s, but the effect of this on
the language was minimal. American Spanish was already firmly established, a product and stepchild of the Andalusian Spanish spoken by
many early settlers. Thus the practice of seseo, or pronouncing c and z
as s (meses rhymes with veces), is universal in America and dominant
in Andalusia, but uncommon in the rest of Spain. In some parts of the
New World, expressions that died out at home (in Spain) were kept
alive. Other expressions were born in the colonies and made their
home there, never to infiltrate Castilian Spanish. The result, today, is
an American Spanish that is considered a distinct (though by no means
uniform) dialect, one easily understood by Spaniards if not "approved"
at all times by their official institutions-particularly the archconservative, arch-Castilian Royal Academy of the Language.

\section{Conserving Spanish}

%CONSERVING SPANISH
Conservativeness in language is a practice that tends to make
its practitioners look silly a few centuries or even a few decades later.
Resistance to change is almost invariably a losing battle. In the case of
Spanish, scholars and students of the language understandably seek to
preserve as much as possible the legacy of one thousand years of written tradition. They want speakers of the language one hundred and five
hundred years from now to be able to read and enjoy Cervantes, Garcia
Lorca, and Neruda in the original-not in "translations to modern
Spanish." And that, most would agree, is a goal worth looking a little
silly for.

The defense of Spanish in our time has largely become a stand
against the influx ("contamination," some would say) of English words,
a reflection of the fact that English, for the better part of the twentieth
century, has been the dominant language internationally (as Spanish
was from 1550 to 1670, more or less). That defense, and its trials and
tribulations, is the subject of Chapter 14.

This defense of Spanish, like all defenses that seek to protect a
language from its speakers, has little chance of final success. That said,
it is even less likely that an upstart language like English-useful in
commerce and science, trendy in popular entertainment-will ever
supplant Spanish when it is time to talk of the stars, of the gods, and
oflove. For these concepts, for words of passion and glory, power and
drama, Spanish need look no further than its own vibrant history.

