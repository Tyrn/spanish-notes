\chapter{Ten Ways to Avoid Being Taken for a Gringo}

A gringo, in much of the Spanish-speaking world, is a person
who comes from abroad, speaks another language, and wears loud
shorts. In certain countries, such as Mexico, it refers specifically to
U.S. citizens, but even there the distinction is hazy. A Canadian or
German who acts like a gringo will be referred to as a gringo, birth certificates be damned. Act like a gringo and you will be called one; don't
act like one and you may be called one anyway. The word is descriptive
first---of a style, a cultural stance, a way of life---and derogatory only
later, if at all.

So how to avoid being taken for a gringo? The truth is, if you
were born outside the Spanish-speaking world, there is probably nothing you can do to hide the fact. You will never fully blend in nor
should you necessarily want to. But as you travel or mingle among
Spanish-speakers, you may wish to smooth over the most obvious differences that set you apart. We all want to be outstanding; standing out
is another matter altogether.

Since you're making an effort to speak and understand Spanish, you've already distinguished yourself from the stereotypical
gringo, that mythical beast of Latin American lore who wears obtrusive shirts, smacks gum, and tends to misplace his or her wallet. But
even if you're not one of them, you can still heed a few simple precautions that will help put you in tune with the local culture, be it in Patagonia or East Los Angeles. Dress codes and behavior are beyond the
scope of this book, but some general pointers may keep others from
pointing at you.

\section{Pronunciation}

Spanish pronunciation should be the easiest thing in the world
to master. Unlike English, where a letter can change its sound seemingly at will, Spanish letters have---with very few exceptions---the exact same sound word after word. To compare, think of the three different sounds of the letter a in the English pronunciation of the name
Abraham; in Spanish the letter a in Abraham has but a single sound,
repeated three times. Still, you'll need to practice to convince your
tongue to make the correct sound, to get your teeth to close or open on
cue, and to master the inflections. At first it will be a struggle, but
there is no reason why anyone can't learn to pronounce Spanish properly after enough use. Here are some tips on how to proceed:

\subsection{}

Spanish teachers always tell their students to practice repeating the vowel sounds: \emph{a, e, i, o, u}. Listen to these wise men and
women and practice, practice, practice. If it's more fun, follow your
litany with the phrase schoolchildren learn south of the border: \emph{El burro sabe más que tú} (``The burro knows more than you"). Move your
mouth as you repeat the vowels. Pretend someone fifty yards away is
trying to read your lips. Clip the vowel sounds short, as you might
imagine a Japanese colonel in a late-night, World War II movie would
do. \emph{A, E, I, O, U, A, E, I, O, U}\ldots{}

\subsection{}

Next come the sounds for the letters \emph{r} and \emph{rr}. The double
one trills, and so does the single one at the start of a word. Thus \emph{carro}
and \emph{rancho} have essentially the same \emph{r} sound. Your tongue won't want
to trill at first; it will make a scene about being made out of concrete
and will refuse to emit such a ridiculous sound. But you're not going to
let your tongue push you around, are you? Trill away! Pretend you're
Charo on ``The Tonight Show": ``R-r-r-really, Johnny, r-r-r-romance for
me is r-r-r-relaxing on a r-r-r-rug, listening to r-r-r-rock and r-r-r-roll." If
you've studied any French, this sound may be harder for you to get
used to at first. But if you try at all, you will learn it. Many gringos do
and, as far as anyone can tell, we're all born with the same kind of
tongue.

\subsection{}

The d between vowels or at the end of a word sounds more
like the \emph{th} in \emph{thus} than an English \emph{d}. \emph{Nada} is thus pronounced ``natha," or close to it. So light is the mid-vowel \emph{d} that sometimes, in colloquial spoken Spanish, it's almost left out altogether; you may even
see \emph{nada} represented as \emph{na'a} or \emph{na'} in written dialogue. You shouldn't
take it that far, but do get the hang of the soft Spanish d by learning a
few words well: \emph{nada, limonada, edad, comida, ciudad, cansado}. All
regular past participles follow the same rule: \emph{hablado, conocido, bebido}, and so on. The \emph{d} at the beginning of a word in Spanish is also a
tad softer than its English counterpart, perhaps more like a \emph{dth} than a
solid \emph{d}. Shout the name David in English and you can almost feel
yourself spit; shouting it in Spanish is a much less moist affair.

\subsection{}

The Spanish letters \emph{c, z, j}, and \emph{ll} vary in pronunciation
from country to country. Don't let this bother you. Either adopt the
sound used in your country of choice or seek a middle ground. For
most purposes, you're safe pronouncing both the \emph{c} and \emph{z} as an English
\emph{s}, the \emph{j} as an \emph{h}, and the \emph{ll} as the \emph{y} of ``yes." (Remember, of course, that
the hard \emph{c} in Spanish, as in \emph{ca-, co-}, and \emph{cu-}, sounds like a \emph{k}.) Other
noteworthy regional differences include the use of \emph{vos} instead of \emph{tú}
and the use of \emph{vosotros} instead of \emph{ustedes}. \emph{Vos} is used in much of
Central and South America and requires learning yet more verb endings (\emph{tú quieres = vos querés}). Still, if you are spending time in those
countries, you will probably want to use it. The same goes for \emph{vosotros}, which is used in Spain for the second-person plural and which
you will no doubt learn if you're picking up your Spanish there.

\subsection{}

Other regional differences are best left alone. In many
countries, for instance, there is a tendency to ``swallow" the \emph{s} sound at
the end of a syllable, especially before consonants: estoy aquí becomes
e'toy aquí. (It reaches such extremes that there's even a joke about the
Cuban child who asks his mother how to form plurals. ``Easy, chico,"
she says, ``just add an \emph{s}: \emph{el coco, lo coco}.") There's really no good reason to learn to speak Spanish that way, or with any other regional dialect, unless you're keen on being identified as having studied in a specific country. Imagine an Asian immigrant speaking English like a
Boston cabbie, or a Uruguayan drawling like a Texan, and you'll understand why.

\subsection{}

In Spanish, the letters \emph{b} and \emph{v} sound the same: almost (but
not quite) like the English \emph{b}. Like its d sound, Spanish's \emph{b/v} sound is a
shade softer, especially between vowels. Thus \emph{ave} is not pronounced
either ``ah-bay" or ``ah-vay" but ``ah-bvay." Say ``ah-bay" fast, without
giving your lips time to spit out a hard \emph{b} sound, and you'll get the idea.

\section{The wrong word syndrome}

There are dozens of cases in Spanish where you will be
tempted to use a word that is patently wrong. Mostly this is a result of
misleading English cognates---words that look or sound the same in
English and in Spanish but harbor different meanings. Sometimes the
meaning will be close in Spanish, and the lazy language learner is content to use the cognate. But you aren't like that, are you? You want to
speak Spanish well and not be responsible for polluting it with English
usages. Swell. For people like you, Chapter 3 in this book is dedicated
to these ``tricksters," and Appendix B covers more subtle nuances.
Don't worry about learning them all at once. Just try to remember
which words are tricky and then try to avoid stepping in them as you
go along.

\section{\emph{Yo-ismo}}

In English, a sentence is incomplete without a noun or pronoun for a subject. In Spanish, the subject of the sentence is often conveyed by the verb and is optional. Often, in fact, it is left out altogether, unless the speaker wants to emphasize the subject of the
statement. This quaint little grammatical fact affects you, the language
learner, in one important case: the first person. Since you are used to
including pronouns, you will tend to preface all of your first-person
comments with yo. But to a Spanish ear, this sounds like you are constantly calling attention to yourself: ``\emph{I} want this" and ``\emph{I} think that."
This affliction, dubbed ``\emph{yo-ismo}," can in extreme cases make people
think you're a pretty snotty individual, when you and I know that's not
true. But why take chances? Try to say \emph{quiero} instead of \emph{yo quiero},
\emph{creo} instead of \emph{yo creo}, and so on. Later, when you've broken the habit,
you can go back to inserting the occasional \emph{yo} to emphasize a truly
personal opinion: \emph{El quiere casarse pero yo no quiero} (``He wants to
get married but \emph{I} don't").

\section{The stumbles}

The stumbles are what you get when you're asked a simple
question and your tongue runs off and hides behind your tonsils. It is a
common ailment of those who have studied some Spanish---and \emph{know}
what they want to say---but lack conversational practice. Remedying
this condition requires practice, of course, but also a careful study of
interjections, pert comebacks, snappy answers, and sentence starters.
These useful words and phrases will get you through almost every
situation requiring sudden tongue work, but they are often neglected
in textbooks. Lists of these gems are included in later chapters. Pick
your favorites (there are usually several acceptable ones for each situation) and store them close to your tongue.
Speaking of the stumbles, you should be especially cautious of
letting your English ``crutch words" slip into your Spanish. It sounds
awful: \emph{Quiero\ldots{} um\ldots{} ir a\ldots{} you know\ldots{} the\ldots{} er\ldots{} cine,
	okay?} If you must, learn some Spanish crutch words and lean on them
instead: \emph{Quiero\ldots{} este\ldots{} ir\ldots{} o sea\ldots{} al cine, ¿no?} You'll still
sound like a space cadet, but at least a fairly fluent one.

\section{Formalities}

The gringo who makes the effort to get off the beaten path and
find out-of-the-way shops and cafés is almost by definition a fairly gregarious soul. But fear of the language can make this same person seem
timid, uptight, or arrogant, browsing in a shop for half an hour without
saying a word to anyone and leaving without so much as a good-bye. In
general, you should be happy to inflict your fledgling Spanish on anyone who crosses your path. But you should be especially eager to unleash it in cases calling for common courtesy. There is probably no
faster way to separate yourself from the pack of tourists and gawkers
than to look someone in the eye and speak to them in their own language---even if it's only to say ``hello" and ``good-bye."

For starters, you should always greet people whose lives you
have invaded, if only briefly. This doesn't mean you should walk down
the streets of Buenos Aires saying \emph{hola} to everyone, but it does mean
you should immediately recognize the existence of shop clerks, waiters, secretaries, and guests. The formula is simple. From the time you
wake up until 11:59 a.m., you say \emph{buenos días}; from 12:00 noon until
dark, you say \emph{buenas tardes}; and from dark until bedtime, you say
\emph{buenas noches}. Say it loud, say it proud. You'll be amazed how service
improves and prices drop after a pleasant greeting.

When you leave a place, remember to say \emph{gracias} if it's a commercial establishment, \emph{adiós} or \emph{hasta luego} if it's not. Better yet,
when leaving a shop, say \emph{muchas gracias} or \emph{muy amable}, gracias or
even \emph{muchas gracias muy amable}, all run together. You'll feel much
more cheerful walking out of a place after a heartfelt farewell, even if
the clerk did nothing more than stare at your back the whole time you
were in the store.

In general, Spanish requires more spoken formalities than English (at least as it's spoken today), which is nice because it gives you a
lot of opportunities to practice those key words and phrases. Skipping
over the formalities, on the other hand, will tag you as a gringo from
the get-go, which is what we've decided we want to avoid. For a fuller
treatment of formalities and politeness in general, there's a whole
must-read chapter just ahead with bounteous tips and details.

\section{The volume}

In Baltimore or Toronto or Oxford our friends and neighbors
seem to speak in reasonable tones. So why is it that these same people
seem to start shouting as soon as they get through Customs? This is
one of the great mysteries of cultural intercourse, and I don't foresee
resolving it here. But it is worth mentioning that, as a native English
speaker, you are expected to shout instead of speak and that a whole
continent of Latin Americans will be grateful if you manage to do otherwise. Gringos (remember them?) tend to bunch together and speak
English at volumes appropriate for a rock concert. I've even seen a
Mexican head of family ask the maitre d' to be seated ``away from the
gringos" at a restaurant. As when greeting people, speak loud and
proud---but not too loud. Contrary to gringo folk wisdom, comprehension does not increase with volume. Normal speaking tones are best,
and by the time you find yourself shouting, absolutely no one will admit to understanding you at all.

\section{Adjectives}

One of the most frustrating things about learning a new language is not having access to that grab bag of adjectives that we rely on
to express our opinions. And since adjectives are a relatively second-class part of speech (after the big shots like nouns and verbs), many
beginning students of language tend to put off learning them until
some later phase of their study. In practical terms this produces human
beings whose whole range of descriptions goes from ``very, very" on
one extreme to ``not very" on the other. ``How was the film?" \emph{Muy,
	muy, muy buena}. ``How about the meal?" \emph{No muy buena}.

You should try to break out of this rut as quickly as possible,
learning alternative ways of expressing your likes and dislikes. Pay special attention to the use of prefixes and (especially) suffixes in modifying adjectives in Spanish by learning a few suffixes and attaching
them to words you already know, you can quickly multiply the coverage of your vocabulary. \emph{Grande}, for example, can go up in size to \emph{grandísimo} and even \emph{grandotote} or deflate to a semisardonic \emph{grandecito}.
Also learn some words for that vast middle ground between good and
bad where, sad to say, most of our experiences tend to fall. Chapter 4
will try to help you do just that in your encounters with people.

\section{Speed kills}

Just as loud doesn't equal intelligible, fast definitely doesn't
equal fluent. Take it easy. You wouldn't try to break speed records on a
Kawasaki when you're just learning to ride, so why try with your Spanish? Spoken English has raised slurring practically to an art form, and
it's considered normal to modify (i.e., mispronounce) consonants or
vowels that get in our way. The usual result is a series of phrases like
``I dunno" and ``Waddaya-wamee-tudo?" In spoken Spanish, each vowel
and consonant retains its particular, unalterable sound, no matter how
fast you're speaking. True, if you speak fast enough, people may not
catch your errors. But they won't catch your drift either, and you'll end
up having to repeat everything. If that's your strategy for getting extra
practice, \emph{adelante}. Just say it slowly the second time.

\section{Body language}

If slurring in English is almost an art form, then downwardly
mobile dressing in English-speaking (and other) countries is long overdue for a major museum exhibition. Far be it from me to tell you how
to dress on vacation or when prowling the barrio, but do remember the
Spanish \emph{dicho: Como te ven, te tratan} (``How they see you, they treat
you"). Dressing down has not yet caught on in most Latin cultures,
perhaps because millions of people dress that way for reasons not related to fashion. If you stay close to the tourist bus, what you wear
isn't so important---but who wants to do that? You don't have to dress
to the nines to go buy a Coke, but you should at least be in the low
sevens. Otherwise, your clothes will be saying things about you that
your mouth never would.

What gringos often do, since we seem to be talking about
them, is to convert their Sunday barbecue outfits back home into all-purpose wear for their introduction to Latin culture. What they would
never wear to church they don't think twice about wearing to a Colombian cathedral or Guatemalan village church. If you intend to show the
local people that you respect their culture, the best way to start is by
letting your clothes speak for you. And the clothes that speak best are
the ones that cover knees and shoulders---at least away from coastal
cities. You don't have to care about any of this, of course, but if you
do, remind yourself that sin and skin are still closely linked in many
minds.

\section{Those crazy gringos}

Every now and then, it won't matter how well you say something in Spanish if what you are saying is so patently absurd that it
transcends language altogether. And what is judged as absurd can vary
widely from place to place. I remember witnessing a frustrated tourist trying to request, in flawless Spanish, \emph{hielo hervido} for his soft
drink. Now, if you're familiar with tourists, you'll probably realize that
``boiled ice" is a sort of shorthand way of saying ``ice made from boiled,
or purified, water." But if you're a waiter in a small-town bus-station
restaurant in Latin America, you may not make that conceptual leap at
all. Instead, you will try your darndest to make sense of your customer,
and then will probably go back to the kitchen and heat up a couple of
ice cubes. Many innocent requests like this can turn into Major Cultural Confusions if you're not careful. Asking for ice to put in a soda
that is already cold is considered downright silly in many places, for
instance. Asking for a ``pizza with meat on it," in at least one place I've
been, can lead to a pizza with a slab of steak lying on top. And so on.

This, of course, is part of the beauty of getting to know foreign
cultures: learning that what you had considered a given all your life is
often not a given at all. So if you find that nothing you say in Spanish
seems to get your point across, consider changing tack. What you are
saying, not how you are saying it, may be the culprit. Then change
your order to \emph{pollo frito} and a beer and forget about it. Just pray that
they don't come in a bowl, and together.


%%% Local Variables:
%%% mode: latex
%%% TeX-master: "main-keenan-breaking-out"
%%% End:
