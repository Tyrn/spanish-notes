\chapter{Say It Right}

Some everyday concepts in Spanish just seem to resist translation. As a result, many students of Spanish never seem to learn to say
them right. Perhaps common usage doesn't conform to the bilingual
dictionary's definition, or maybe there's just not a good dictionary
translation to be found. Regardless, you'll need to express these concepts, and the sooner you learn how to say them, the better. Here is a
selection of the commonest hang-ups for students of Spanish.

\section{\emph{choice}}

%CHOICE
The problem here is really quite simple. In English, you
``choose" between ``choices," but in Spanish you can't escoger between
\emph{escogibles} or \emph{escogisiones} because these awkward, horrible-sounding
behemoths don't really exist. \emph{Escogimiento} does exist, but if you use
it, you're liable to be hit. Instead, use \emph{opciones} and \emph{selecciones} when
you want to translate, but keep your eyes and mind open for alternative constructions. ``What's your choice?" would be expressed as \emph{¿Qué
escoges?} ``There's no choice" would be \emph{No hay de otra} or \emph{No hay remedio}. ``Good choice" = \emph{Buena selección} and ``What are my choices?"
= \emph{¿Cuáles son mis opciones?}

While we're on the subject, ``choosy" is another concept that
confounds many students of Spanish---especially when you add in
``fussy" and ``picky." In Spanish you have to leave the universe of
``choice" altogether to express this concept. One of the most common
(if least expected) ways of expressing it is with \emph{especial} (``special").
\emph{Ella es muy especial para la comida} = ``She's a very picky eater."
More formally, \emph{exigente} (``demanding") works well, and in some parts
you may hear \emph{fastidioso} for someone who is annoyingly fussy. (Note
however that \emph{fastidioso} does not mean ``fastidious" so much as ``annoying" or ``bothersome," though it may be taking on the English
meaning more and more. See Appendix B.) In Mexico a slangy word
for ``annoyingly picky" is \emph{sangrón}. \emph{Es muy sangrona para la comida}.
Likewise, \emph{No seas tan sangrón} = ``Don't be so picky."

\section{\emph{enough}}

%ENOUGH
Most students of Spanish learn early on that ``enough" is bastante, pure and simple. So it will come as something of a surprise to
discover that ``enough" is not \emph{bastante}---or rarely. Some dictionaries
try to gloss over this problem, giving you a variety of forms---from
\emph{¡basta!} to even \emph{bastantemente}---to let you wriggle out of a jam. In
fact, \emph{bastante} in the sense of ``enough" almost never works as well as
\emph{suficiente}, which you should carve into your memory as the correct
word for ``enough."

This is not to say that \emph{bastante} is not a useful word; it is,
and immensely so. But it is much closer to ``plenty" than ``enough,"
as a few examples make clear. When the waiter is serving you brussel sprouts and you think that six is enough, you naturally would say,
``That's enough, thank you." In Spanish, this would be \emph{Es suficiente,
gracias}. \emph{Es bastante, gracias} sounds almost like a complaint---``Whew!
That's plenty." A waiter, on hearing it, might even offer to take some
of your sprouts back. \emph{Ya basta} will get your point across, but it comes
off sounding like a rude ``Enough already!" \emph{Tengo suficiente dinero}
means ``I have enough money" (to buy gas, or food, or whatever),
whereas \emph{Tengo bastante dinero} sounds like a bit of a boast: ``I've got
plenty of money."

A good verb to keep handy for situations involving money, especially, is \emph{alcanzar}, meaning ``to reach" but also ``to be enough." It is
a very common word, and you will probably hear \emph{¿Te alcanza para una
Coca?} more often than \emph{¿Tienes lo suficiente para una Coca?} Similarly,
``I don't have enough" is usually just \emph{No me alcanza}. Note that
\emph{alcanzar} requires inverting subject and object. In English, you have the
money; in Spanish, the money reaches for you. \emph{Alcanzar} is also probably the best way to translate ``to afford," a notorious bugaboo in translators' circles. ``Can you afford new shoes?" would be \emph{¿Te alcanza para
nuevos zapatos?}

Bastante is very common as a modifier of adjectives, much as
``plenty" is in colloquial English. \emph{Es bastante guapo} = ``He's plenty
handsome" (and not ``He's handsome enough," which sounds like a
backhanded compliment). \emph{Es bastante listo con las computadoras} =
``He's real clever with computers" (and not ``He's clever enough"). For
``clever enough" and ``handsome enough," you have to dig down deep
for \emph{suficientemente}, as in the phrase \emph{Es lo suficientemente listo}.

Of all the typical usages of ``enough," only ``Enough already!"
wouldn't be disposed of better by some form of \emph{suficiente} and would in
fact be rendered \emph{¡Basta!}

\section{\emph{fear}}

%FEAR
Here is a classic minefield for the foreign speaker of Spanish,
Not only are there many alternatives to choose from, but most of them
depart from our customary perspective in English. \emph{Asustar, dar susto,
temer, temor, pavor, espantar, dar miedo, tener miedo}---all of these
are used, and it's quite a trick to learn exactly how.

To start with, keep in mind that ``fear" in Spanish is something that you ``have" or ``give," Thus, you aren't ``afraid" but ``have
fear." And you don't ``scare," you ``give fear." With this in mind, lie
down on the couch over there and proceed to analyze your fears. Fear,
in Spanish, can be broken down roughly as follows. A \emph{temor} is a very
specific fear, almost more like a deep-seated, haunting feeling; \emph{temer}
is the verb that goes with this type of fear. Thus \emph{Temo que algo haya
pasado} = ``I'm afraid something has happened." \emph{Mi temor es que escapen los animales} = ``My fear is that the animals may escape." \emph{Temer}
is also used in polite expressions, just as ``fear" is in English: ``I fear we
may have arrived at a bad moment" would be expressed as \emph{Temo haber
llegado en un mal momento} or \emph{Temo que hayamos llegado en un mal
momento}. Note that all subordinate clauses built off of \emph{temer} and \emph{temor} require the subjunctive, because fear of what may happen or what
may be happening is naturally the province of the subjunctive.

\emph{Miedo}, meanwhile, is your run-of-the-mill sort of fear. You
``have this fear" of the dark, of scorpions, of large fellows who drool,
and of public speaking. Of these fears you would say, \emph{Me dan miedo}
or \emph{Tengo miedo de ellos}. Perfectly natural. These things always ``give
you" this fear, regardless of the circumstances.

A \emph{susto} is more sudden than a ``fear." Properly speaking, it is a
``scare," even a ``shock." \emph{¡Que susto me diste!} (``What a scare you gave
me!") you tell your roommate, whom you found hiding in the closet
with a wolf mask on. This type of fear even gets its own verb: \emph{asustar}
(``to scare"), which means exactly the same as \emph{dar susto}. Here, though,
you have to be careful about your English, since ``to scare" is commonly used for both sudden fears and for fears which you always have,
regardless of the circumstances. ``Bears scare me" would normally
be \emph{Tengo miedo de los osos} in Spanish. You might hear \emph{Los osos me
asustan}, especially from kids, but it suggests that certain bears are in
the habit of sneaking up behind you and yelling ``Boo!" in your ear.

When ``to scare" means ``to scare away," \emph{espantar} is used instead of \emph{asustar}. Thus an \emph{espantapájaros}---literally a ``scare-away-birds"---is a ``scarecrow." \emph{Espanto} in Latin America is also a common
word for ``spooks" or ``ghosts"; the ``haunted house" at the amusement
park is in many countries called \emph{La Casa de los Espantos}.

Another problem is the word ``scary." Some authors cite ``scary"
as one of those words that simply doesn't have a Spanish translation.
(And two Spanish-English dictionaries I have at hand simply mistranslate ``scary" as \emph{miedoso} and \emph{pusilánime}---both of which mean ``cowardly.") I remember one student of Spanish describing a film he had
seen as \emph{espantosa}, thinking it meant ``scary" or ``frightening." Instead,
it usually means ``frightfully bad" or ``dreadful," far from the speaker's
intent. Possibly the safest way to convey ``scary" is with \emph{miedo}. ``It's a
scary movie" = \emph{Es una película de miedo} or \emph{Es una película que da
miedo} (or \emph{una película de horror} in the specific case of Friday the 13th-style horror flicks). You might also hear \emph{tétrico} for ``scary," though
it tends more to ``spooky" or even ``eerie." Still, it can be the perfect
word to describe a lonely alley late at night. \emph{Aterrador} is like ``scary"
but considerably scarier; ``terrifying" hits the mark.

Two other words round out our study of fears. They are \emph{pavor}
and \emph{pánico}, the second being a slangy addition to the list. \emph{Pavor} is an
intense, almost phobic fear, mingled with equal parts dread and loathing. \emph{Me dan pavor las tarántulas} = ``I'm terrified of tarantulas." Like
\emph{miedo}, \emph{pavor} is a long-term, constant fear, not a sudden shock like
\emph{susto}. Colloquially, you may hear \emph{me da pánico} to mean much the
same as \emph{me da pavor}, and \emph{me da terror} fits this mold as well. \emph{Me da
horror} is also used to mean ``It disgusts me" or ``It grosses me out."
Along these lines, you'll sometimes hear \emph{Me da cosa}, a slangy phrase
that could be rendered ``It gives me the creeps." \emph{Ese señor me da cosa}
= ``That man gives me the creeps."

\section{\emph{guess}}

%GUESS
This word crops up a lot more in spoken English than you
might at first think. And until you break the habit of translating your
thoughts, you'll need a good selection of phrases for it. ``To guess" often can be handled by \emph{adivinar}. \emph{Adivina quién es} = ``Guess who it
is." \emph{¡Adivina que!} = ``Guess what!"

But in English ``guess" is stretched well past its original intent
in many colloquial instances. Obviously, we aren't actually guessing at
our own actions when we say, ``I guess I'll be going now." For ``guess"
in this sense, you would need to resort in Spanish to some similar
qualifier. \emph{Suponer} works fine: \emph{Supongo que ya me voy}. In some other
examples, \emph{imaginarse} might be called for. \emph{¿Vas a ir esta noche? Me
imagino que sí}. = ``Are you going tonight?" ``I guess so."

Two other verbs---\emph{atinar} and \emph{acertar}---also come into play
here. Both mean, more or less, ``to guess right." \emph{Le atinó a mi nombre}
= ``He guessed my name." \emph{A ver si le aciertas al ganador} = ``Let's see
if you can guess the winner." Both of these words are commonly heard
in Spanish, and you would do well to learn to recognize them. \emph{¿De qué
país eres? Adivina. Pues, de Canadá. ¡Atinaste!} = ``What country are
you from?" ``Guess." ``Canada?" ``You got it!" Many times, when we
would be tempted to use \emph{adivinar}, we should probably use \emph{atinar} or
\emph{acertar}. ``Let me guess" could be rendered either literally, \emph{Déjame adivinar}, or more naturally, \emph{A ver si le atino} (``Let's see if I can guess it").

\section{\emph{half}}

%HALF
Learning about ``half" and ``middle" in Spanish is essentially a
matter of learning one word---\emph{medio}---and how to use it. Only when
``half" is a noun does another word---\emph{mitad}---come into play. At all
other times, \emph{medio} or a related form covers ``half" and ``middle." Examples: \emph{la mitad de la pizza} = ``half of the pizza" but \emph{media pizza} =
``half a pizza"; \emph{una pizza media cocida} (or \emph{una pizza cocida a medias})
= ``a half-cooked pizza."

Students of Spanish mostly err in this case when they overuse
\emph{mitad}. Note how much easier and smoother it is to say \emph{media botella
de vino} instead of \emph{la mitad de una botella de vino}. Both mean ``half a
bottle of wine." \emph{Mitad} is much used but fairly specifically, and especially where no noun is present. \emph{Dame la mitad} = ``Give me half."
\emph{Irse a mitades} is a slangy phrase for ``to split the cost of something" or
``half-and-half." \emph{Si compramos una pizza, nos vamos a mitades} = ``If
we buy a pizza, we each pay half."

The adverb \emph{medio} is extremely common as a modifier of adjectives. It equates with ``kind of" in English in the sense of ``kind of
ugly," ``kind of happy," ``kind of drunk": \emph{medio feo, medio feliz, medio
borracho}. Since it's an adverb, it never changes genders. For example,
\emph{Ella está medio borracha} and \emph{La pizza está medio cocida}.

A couple of \emph{medio}-related phrases are \emph{a mediados de, en medio de}, and \emph{a medias}. The first one means ``around the middle of"
and is almost always used with time constructions: \emph{a mediados de diciembre, a mediados del mes}, and the like. \emph{En medio de} is ``in the
middle of" in the physical or figurative sense: \emph{en medio de la calle}
= ``in the middle of the street"; \emph{en medio de un gran lío} = ``in the
middle of a big mess." \emph{A medias} is an adverb describing how something is done, and its best translation is ``half-way" or just ``half." \emph{¿Hiciste la tarea? La hice a medias}. = ``Did you do the homework?" ``I
kinda half did it." Note that this is not the same as \emph{Hice la mitad},
which means ``I did half of it." You often hear \emph{a medias} to describe a
poorly done job; in this sense, it equates with ``half-assed." \emph{¡Esos albañiles lo hicieron a medias!} = ``Those construction workers did a
half-assed job!"

\section{\emph{how}}

%HOW
This innocent-looking word earns special mention for the
widespread but incorrect use of \emph{cómo} to translate common questions
like ``How do you like it?" Even almost-fluent foreigners, especially
those whose native tongue is English, make this mistake. \emph{Cómo}
should never be used this way. \emph{¿Qué tal?} can often be used instead.
\emph{¿Que tal estuvo la película?} = ``How was the movie?" Shorter and
sweeter: \emph{¿Qué tal la sopa?} = ``How's the soup?" You can also construct ``How did you like" questions with \emph{parecer} (¿Qué te pareció\ldots{}?) and \emph{gustar} (¿Te gustó\ldots{}?) But \emph{¿Cómo te gustó\ldots{}?}
is always an absolute no-no.

\section{\emph{maybe}}

%MAYBE
At least once in each of his films, the Mexican comic Capulín would answer someone's urgent request for information with a
thoughtful, unhelpful \emph{No lo sé, puede ser, a lo mejor, tal vez, ¿quién
sabe?}---in other words, ``Maybe." There's one other common translation that Capulín left out, and that's \emph{quizás} or \emph{quizá}. Now you too
are equipped to be noncommittal and unhelpful. Use anyone of these
phrases to say ``maybe" in good Spanish.

But wait, you say breathlessly, isn't there any way to differentiate among these many alternatives? \emph{No lo sé, puede ser, a lo mejor, tal vez, ¿quién
sabe\ldots{}?}
But seriously, you should be careful about
overusing \emph{quizás} and neglecting \emph{a lo mejor}, which in many countries
is the common conversational way of saying ``maybe." \emph{¿Vienes esta
noche? A lo mejor}. = ``Are you coming tonight?" ``Maybe." \emph{Tal vez} is
also extremely common and all-purpose. \emph{Quizás}, on the other hand,
somehow sounds loftier and more grandiose. \emph{Puede ser que, puede
que, a lo mejor, tal vez}, and \emph{quizás} can all be used to start sentences.
All but \emph{a lo mejor} are usually followed by the subjunctive to highlight
the uncertainty being expressed. \emph{Tal vez esté enojado} = ``Maybe he's
mad." \emph{A lo mejor no quiere} = ``Maybe he doesn't want to." On the
streets, finally, it is increasingly common to hear \emph{chance} for ``maybe,"
but don't tell your teacher I told you. \emph{¿Vienes esta noche? Chance}. The
words you want to avoid are \emph{probablemente} and \emph{posiblemente}, especially the former. Both are perfectly legitimate Spanish words, but
they're not as common as their English cognates. Were they as common, Capulín would have thrown them in for good measure.

\section{\emph{motive}}

%MOTIVE
Life being what it is, there will come times when you slip up,
trip up, or flub up. It can't be helped. You're walking down the street
in, say, Santiago, happy as a lark, when your hand knocks into the ice
cream cone of a six-year-old girl and sends the cone scattering on the
pavement. The passersby eye you with suspicion. The six-year-old eyes
you with disbelief, then with unbridled hatred, and then starts to bawl.
You look around at the crowd assembling on all sides. Think quick:
what can you say to them?

In English, you might breezily announce ``Sorry, I didn't mean
to" or ``It was an accident." You might even add, ``It wasn't on purpose." In Spanish, your simplest, most universal excuse---and one you
should have at the tip of your tongue at all times---is \emph{Fue sin querer}
(literally, ``It was without wanting"). \emph{Sin querer} can also be added on
to any verb to mean ``accidentally" or ``not on purpose." \emph{Lo pateé sin
querer} = ``I kicked it by accident." \emph{Sin querer borré tu nombre de la
lista} = ``I accidentally erased your name from the list." To vary your
vocabulary (if you flub up a lot, for instance), you can use \emph{sin pensar},
\emph{par distracción}, or \emph{involuntariamente}. Of course, if you can pronounce
that last one, you probably don't have to worry about flubs.

When you want to say something was ``on purpose,"
\emph{a propósito} (or in some places \emph{de propósito}) works well. But don't overlook
\emph{con querer}, the opposite of \emph{sin querer} and every bit as common and
as handy. \emph{Lo hice con querer} = ``I did it on purpose." \emph{Adrede}, which
means the same thing, is less common, but you will hear it. Save \emph{deliberadamente} for effect.

Be careful with \emph{accidente} and its derivatives. Although both
\emph{por accidente} and \emph{accidentalmente} are used widely, they sound a bit
forced, and neither gets the heavy work assigned to its English cognate.
\emph{Fue un accidente} will get you away from the maddened crowd and the
six-year-old. But \emph{Fue sin querer} will do the job more smoothly, without overdramatizing the incident.

\section{\emph{obligation}}

%OBLIGATION
Here's one you simply must learn. You have to. Really, you
should. You'd better, anyhow. That is, you ought to.
The difference in degrees of obligation are a subtle issue that
you will master only after many years of listening to Spanish being
spoken. As in English, there are many ways to convey the notion, but
there is no simple way to explain which is the correct choice for any
given sentence.

Perhaps the most useful way to classify the ``musts" is by the
strength of the obligation they imply. Intuitively, you can sense the
different strengths in English: ``You must go" is greater than ``You have
to go," which is stronger than ``You'd better go," which is a tad more
forceful than ``You ought to go," which is up a notch from ``You should
go." In Spanish, a parallel ranking, in order of decreasing obligation,
might go something like this: \emph{Tienes que ir, Más te vale ir, Has de ir,
Hay que ir, Mejor vas, Debes/deberías ir}. Of these, you can safely discard \emph{Has de ir}, which has a scolding schoolmarmish quality about it.
\emph{Hay que} plus a verb, meanwhile, is very common but is limited to impersonal situations. You cannot say \emph{Hay que ir tú}, for instance; you
would have to say \emph{Tienes que ir tú}. \emph{Mejor vas} and \emph{Más te vale ir} are
both good translations of ``You'd better go," with the second implying
``or else\ldots{}"

\emph{Deber} with an infinitive is ``should," and whether you use the
present indicative (\emph{debes}), the conditional (\emph{deberías}), or even the past
subjunctive (\emph{debieras}) seems to be largely a matter of personal preference. For all intents and purposes, \emph{Debes ir mañana, Deberías ir mañana}, and \emph{Debieras ir mañana} are equivalents.

Adding a \emph{de} to \emph{deber} creates another construction that is
hard to find a consistent English equivalent for. It is used when the
speaker has some sign or evidence that something ``is supposed to"
be or ``should" be. An example will help here. \emph{No debe de haber llegado} means ``He (or she) must not have arrived yet" and would be used
when the speaker is looking around an untouched room or notices that
all the lights are turned off, for instance. That said, many Spanish speakers will use \emph{No debe haber llegado}, without the \emph{de}, to say the
same thing, and in some countries \emph{debe de} seems to replace a simple
\emph{debe} in many cases. Note though that by changing the verb to the
past, you change the meaning significantly. \emph{No debió haber llegado}
would be saying ``He shouldn't have arrived"---that is, the person did
in fact arrive but against the speaker's better judgment.

For events in the past, stick to a past form of \emph{deber} with a past
participle for most constructions. \emph{Debías} (or \emph{debiste}) \emph{haber ido} =
``You should have gone." If you instead use the preterit plus the infinitive---\emph{debiste ir}---you are stressing more the obligation than the
missed opportunity. \emph{Debías haber ido a la fiesta} suggests you missed a
good time at the party; \emph{Debiste haber ido a la fiesta} suggests the same,
only a fraction stronger, and hinting maybe at ``I told you so"; \emph{Debiste
ir a la fiesta} means it was your duty to go---the party was in your
honor, for instance. The fourth combination---imperfect plus infinitive, or \emph{debías ir}---is not used, so forget about it. Perhaps the best construction for past obligations---that is, your ``should haves"---is also
the simplest: \emph{hubiera} plus the past participle. \emph{(Te) hubieras ido} = You
should have gone. \emph{Hubiéramos llamado} = We should have called.

A final note on words of obligation deals with the word \emph{obligar} itself. In Spanish, this word is much more widely used than ``to
obligate" is in English. In fact, in many cases where in English we use
``to make" or ``to force," in Spanish you should be thinking \emph{obligar}. \emph{La
policía nos obligó a salir} = ``The police forced us to leave." \emph{El diablo
me obligó a hacerlo} = ``The devil made me do it." One way to remember to use \emph{obligar} is to remind yourself that \emph{forzar} is relatively rarely
used. It's a perfectly legitimate word, of course, but until you're accustomed to using \emph{obligar}, you're better off forgetting that \emph{forzar} exists.

\section{\emph{usually}}

%USUALLY
The basis of the confusion here is the word \emph{usualmente},
which does exist but which you should treat as if it didn't. Dictionaries tend to give it as the translation for ``usually," but in actual practice you could go a month or a year without hearing a native Spanish
speaker use it. It is easy to remember, and technically it's accurate, but
it sounds stiff and strange. You decide.

If you decide to avoid \emph{usualmente} (smart choice!), you will be
left with the slight problem of substitutes. Never fear; they abound.
\emph{Generalmente} is safe, but a mouthful; \emph{normalmente} is easier and just
as good. \emph{Por lo general} is also a good choice. A verbal construction that
works very well involves the irregular verb \emph{soler}. \emph{Suele pasar a las seis}
= ``It usually comes by at six."

While we're discussing the frequency of things, note that if
you're still translating from the English, you'll find your brain feeding
you lines like ``most of the time" and expecting you to translate them.
This leads to hideous constructions like \emph{la mayoría del tiempo}, as in
\emph{La mayoría del tiempo sólo bebo agua} (``Most of the time I just drink
water"). In Spanish, this sounds as if you have made a scientific study
rather than an off-the-cuff remark. In general, forget about translating
``most of the time" and all other ``most of" expressions---``most of the
people," ``most of my life," ``most of the class"---and look for ways
around them. The use of \emph{casi} (``almost") often provides a way out: \emph{casi
siempre} = ``most of the time"; \emph{casi todos} = ``most of the people";
\emph{casi toda la clase} = ``most of the class."

\section{\emph{waste}}

%WASTE
The tendency here, once again, is to translate our English
thoughts too literally into Spanish. The proper word for ``to waste" in
Spanish is \emph{desperdiciar}---a real mouthful that, once you learn it, you
are justified in wanting to use as often as possible. Unfortunately, for
most expressions requiring ``waste" in English, \emph{desperdiciar} is not the
word. Generally speaking, simple old \emph{perder} (``to lose") is. Thus ``to
waste time" is \emph{perder el tiempo} and ``to waste an opportunity" is \emph{perder una oportunidad}. The use of \emph{perder} carries over to the nouns: \emph{una
perdida de tiempo} = ``a waste of time." \emph{Malgastar} can also be ``to
waste," but perhaps translates better as ``to misspend." Some dictionaries list \emph{echar a perder} as a translation for ``to waste," but correctly
this phrase means ``to ruin," as in \emph{Echaste a perder el juego} (``You ruined the game"). Its reflexive form, \emph{echarse a perder}, means ``to go to
waste" or ``to spoil," generally in reference to perishable foodstuffs.
With people, it also means ``to spoil." \emph{Estás echando a perder al niño}
= ``You're spoiling the boy."


%%% Local Variables:
%%% mode: latex
%%% TeX-master: "main-keenan-breaking-out"
%%% End:
