\chapter{Say It Right}

Some everyday concepts in Spanish just seem to resist translation. As a result, many students of Spanish never seem to learn to say
them right. Perhaps common usage doesn't conform to the bilingual
dictionary's definition, or maybe there's just not a good dictionary
translation to be found. Regardless, you'll need to express these concepts, and the sooner you learn how to say them, the better. Here is a
selection of the commonest hang-ups for students of Spanish.

\section{\emph{choice}}

%CHOICE
The problem here is really quite simple. In English, you
"choose" between "choices," but in Spanish you can't escoger between
escogibles or escogisiones because these awkward, horrible-sounding
behemoths don't really exist. Escogimiento does exist, but if you use
it, you're liable to be hit. Instead, use opciones and selecciones when
you want to translate, but keep your eyes and mind open for alternative constructions. "What's your choice?" would be expressed as iQue
escoges~ "There's no choice" would be No hay de otra or No hay remedio. "Good choice" = Buena selecci6n and "What are my choices?"
= iCuciles son mis opciones~
While we're on the subject, "choosy" is another concept that
confounds many students of Spanish-especially when you add in
"fussy" and "picky." In Spanish you have to leave the universe of
"choice" altogether to express this concept. One of the most common
(if least expected) ways of expressing it is with especial ("special").
SAY IT RIGHT 153
Ella es muy especial para 1a comida = IIShe's a very picky eater.1I
More formally, exigente (lldemandingll) works well, and in some parts
you may hear fastidioso for someone who is annoyingly fussy. (Note
however that fastidioso does not mean IIfastidiousll so much as lIannoyingll or IIbothersome,lI though it may be taking on the English
meaning more and more. See Appendix B.) In Mexico a slangy word
for lIannoyingly pickyll issangr6n. Es muy sangrona para 1a comida.
Likewise, No seas tan sangr6n = IIOon't be so picky. II

\section{\emph{enough}}

%ENOUGH
Most students of Spanish learn early on that lIenoughll is bastante, pure and simple. So it will come as something of a surprise to
discover that lIenoughll is not bastante-or rarely. Some dictionaries
try to gloss over this problem, giving you a variety of forms-from
jbasta! to even bastantemente-to let you wriggle out of a jam. In
fact, bastante in the sense of lIenoughll almost never works as well as
su{iciente, which you should carve into your memory as the correct
word for lIenough.1I
This is not to say that bastante is not a useful word; it is,
and immensely so. But it is much closer to IIplentyll than lIenough,lI
as a few examples make clear. When the waiter is serving you brussel sprouts and you think that six is enough, you naturally would say,
IIThat's enough, thank you.1I In Spanish, this would be Es su{iciente,
gracias. Es bastante, gracias sounds almost like a complaint-IIWhew!
That's plenty. II A waiter, on hearing it, might even offer to take some
of your sprouts back. Ya basta will get your point across, but it comes
off sounding like a rude IIEnough already! II Tengo su{iciente dinero
means III have enough moneyll (to buy gas, or food, or whateverl,
whereas Tengo bastante dinero sounds like a bit of a boast: IIl've got
plenty of money."
A good verb to keep handy for situations involving money, especially, is a1canzar, meaning lito reachII but also lito be enough.1I It is
a very common word, and you will probably hear iTe a1canza para una
Coca? more often than iTienes 10 su{iciente para una Coca? Similarly,
III don't have enoughll is usually just No me a1canza. Note that a1-
canzar requires inverting subject and object. In English, you have the
money; in Spanish, the money reaches for you. A1canzar is also probably the best way to translate lito afford,lI a notorious bugaboo in translators' circles. lIean you afford new shoes? II would be iTe a1canza para
nuevos zapatos?
Bastante is very common as a modifier of adjectives, much as
IIplenty" is in colloquial English. Es bastante guapo = IIHe's plenty
154 BREAKING OUT OF BEGINNER'S SPANISH
handsome" (and not "He's handsome enough," which sounds like a
backhanded compliment). Es bastante listo can las computadoras =
"He's real clever with computers" (and not "He's clever enough"). For
"clever enough" and "handsome enough," you have to dig down deep
for suficientemente, as in the phrase Es 10 sufiCientemente listo.
Of all the typical usages of "enough," only "Enough already!"
wouldn't be disposed of better by some form ofsuficiente and would in
fact be rendered iBasta!

\section{\emph{fear}}

%FEAR
Here is a classic minefield for the foreign speaker of Spanish,
Not only are there many alternatives to choose from, but most of them
depart from our customary perspective in English. Asustar, dar susto,
temer, temor, pavor, espantar, dar miedo, tener miedo-all of these
are used, and it's quite a trick to learn exactly how.
To start with, keep in mind that "fear" in Spanish is something that you "have" or "give," Thus, you aren't "afraid" but "have
fear." And you don't "scare," you "give fear." With this in mind, lie
down on the couch over there and proceed to analyze your fears. Fear,
in Spanish, can be broken down roughly as follows. A temor is a very
specific fear, almost more like a deep-seated, haunting feeling; temer
is the verb that goes with this type of fear. Thus Temo que alga haya
pasado = "I'm afraid something has happened." Mi temor es que escapen los animales = "My fear is that the animals may escape." Temer
is also used in polite expressions, just as "fear" is in English: "I fear we
may have arrived at a bad moment" would be expressed as Temo haber
llegado en un mal momenta or Temo que hayamos llegado en un mal
momenta. Note that all subordinate clauses built off of temer and temar require the subjunctive, because fear of what may happen or what
may be happening is naturally the province of the subjunctive.
Miedo, meanwhile, is your run-of-the-mill sort of fear. You
"have this fear" of the dark, of scorpions, of large fellows who drool,
and of public speaking. Of these fears you would say, Me dan miedo
or Tengo miedo de ellos. Perfectly natural. These things always "give
you" this fear, regardless of the circumstances.
A susta is more sudden than a "fear." Properly speaking, it is a
"scare," even a "shock." iQue susto me diste! ("What a scare you gave
me!") you tell your roommate, whom you found hiding in the closet
with a wolf mask on. This type of fear even gets its own verb: asustar
("to scare"), which means exactly the same as dar susto. Here, though,
you have to be careful about your English, since "to scare" is commonly used for both sudden fears and for fears which you always have,
SAY IT RIGHT 155
regardless of the circumstances. "Bears scare me" would normally
be Tengo miedo de los osos in Spanish. You might hear Los osos me
asustan, especially from kids, but it suggests that certain bears are in
the habit of sneaking up behind you and yelling "Boo!" in your ear.
When "to scare" means "to scare away," espantar is used instead of asustar. Thus an espantapajaros-literally a "scare-awaybirds"-is a "scarecrow." Espanto in Latin America is also a common
word for "spooks" or "ghosts"; the "haunted house" at the amusement
park is in many countries called La Casa de los Espantos.
Another problem is the word "scary." Some authors cite "scary"
as one of those words that simply doesn't have a Spanish translation.
(And two Spanish-English dictionaries I have at hand simply mistranslate "scary" as miedoso and pusildnime-both of which mean "cowardly.") I remember one student of Spanish describing a film he had
seen as espantosa, thinking it meant "scary" or "frightening." Instead,
it usually means "frightfully bad" or "dreadful," far from the speaker's
intent. Possibly the safest way to convey "scary" is with miedo. "It's a
scary movie" = Es una pelicu1a de miedo or Es una pelicu1a que da
miedo (or una pelicu1a de horror in the specific case of Friday the 13thstyle horror flicks). You might also hear tetrico for "scary," though
it tends more to "spooky" or even "eerie." Still, it can be the perfect
word to describe a lonely alley late at night. Aterrador is like "scary"
but considerably scarier; "terrifying" hits the mark.
Two other words round out our study of fears. They are pavor
and panico, the second being a slangy addition to the list. Pavor is an
intense, almost phobic fear, mingled with equal parts dread and loathing. Me dan pavor las tarantulas = "I'm terrified of tarantulas." Like
miedo, pavor is a long-term, constant fear, not a sudden shock like
susto. Colloquially, you may hear me da panico to mean much the
same as me da pavor, and me da terror fits this mold as well. Me da
horror is also used to mean "It disgusts me" or "It grosses me out."
Along these lines, you'll sometimes hear Me da cosa, a slangy phrase
that could be rendered "It gives me the creeps." Ese senor me da cosa
= "That man gives me the creeps."

\section{\emph{guess}}

%GUESS
This word crops up a lot more in spoken English than you
might at first think. And until you break the habit of translating your
thoughts, you'll need a good selection of phrases for it. "To guess" often can be handled by adivinar. Adivina quien es = "Guess who it
is." iAdivina que! = "Guess what!"
But in English "guess" is stretched well past its original intent
156 BREAKING OUT OF BEGINNER'S SPANISH
in many colloquial instances. Obviously, we aren't actually guessing at
our own actions when we say, "I guess I'll be going now." For "guess"
in this sense, you would need to resort in Spanish to some similar
qualifier. Supaner works fine: Supanga que ya-me vay. In some other
examples, imaginarse might be called for. iVas air esta naehe~ Me
imagina que sf. = "Are you going tonight?" "I guess so."
Two other verbs-atinar and aeertar-also come into play
here. Both mean, more or less, "to guess right." Le atin6 a mi nambre
= "He guessed my name." Aver si le aeiertas al ganadar = "Let's see
if you can guess the winner." Both of these words are commonly heard
in Spanish, and you would do well to learn to recognize them. iDe que
pais eres~ Adivina. Pues, de Canada. iAtinaste! = "What country are
you from?" "Guess." "Canada?" "You got it!" Many times, when we
would be tempted to use adivinar, we should probably use atinar or
aeertar. "Let me guess" could be rendered either literally, Dejame adivinar, or more naturally, Aver si le atina ("Let's see if I can guess it"l.

\section{\emph{half}}

%HALF
Learning about "half" and "middle" in Spanish is essentially a
matter of learning one word-media-and how to use it. Only when
"half" is a noun does another word-mitad-come into play. At all
other times, media or a related form covers "half" and "middle." Examples: la mitad de la pizza = "half of the pizza" but media pizza =
"half a pizza"; una pizza media coeida (or una pizza coeida a medias)
= "a half-cooked pizza."
Students of Spanish mostly err in this case when they overuse
mitad. Note how much easier and smoother it is to say media batella
de vina instead of la mitad de una batella de vina. Both mean "half a
bottle of wine." Mitad is much used but fairly specifically, and especially where no noun is present. Dame la mitad = "Give me half."
lrse a mitades is a slangy phrase for "to split the cost of something" or
"half-and-half." Si eampramas una pizza, nas vamas a mitades = "If
we buy a pizza, we each pay half."
The adverb media is extremely common as a modifier of adjectives. It equates with "kind of" in English in the sense of "kind of
ugly," "kind of happy," "kind of drunk": media fea, media feliz, media
barraeha. Since it's an adverb, it never changes genders. For example,
Ella esta media barraeha and La pizza esta media eaeida.
A couple of media-related phrases are a mediadas de, en media de, and a medias. The first one means "around the middle of"
and is almost always used with time constructions: a mediadas de dieiembre, a mediadas del mes, and the like. En media de is "in the
SAY IT RIGHT 157
middle of" in the physical or figurative sense: en media de la calle
= "in the middle of the street"; en media de un gran lio = "in the
middle of a big mess." A medias is an adverb describing how something is done, and its best translation is "half-way" or just "half." iHiciste 1a tarea? La hice a medias. = "Did you do the homework?" "I
kinda half did it." Note that this is not the same as Hice 1a mitad,
which means "I did half of it." You often hear a medias to describe a
poorly done job; in this sense, it equates with "half-assed." iEsos albaiiiles 10 hicieron a medias! = "Those construction workers did a
half-assed job!"

\section{\emph{how}}

%HOW
This innocent-looking word earns special mention for the
widespread but incorrect use of como to translate common questions
like "How do you like it?" Even almost-fluent foreigners, especially
those whose native tongue is English, make this mistake. Como
should never be used this way. iQue taU can often be used instead.
iQue tal estuvo 1a pelicula? = "How was the movie?" Shorter and
sweeter: iQue talla sopa? = "How's the soup?" You can also construct "How did you like" questions with parecer (iQue te pareClO 0' . . ..l) and ( gustar i 'T' L e gusto. ' . . .. l) But iC' omo te gusto'l . ... IS 0
always an absolute no-no.

\section{\emph{maybe}}

%MAYBE
At least once in each of his films, the Mexican comic Capulin would answer someone's urgent request for information with a
thoughtful, unhelpful No 10 se, puede ser, a 10 me;or, tal vez, iquien
sabe?-in other words, "Maybe." There's one other common translation that Capulin left out, and that's quizQS or quizQ. Now you too
are equipped to be noncommittal and unhelpful. Use anyone of these
phrases to say "maybe" in good Spanish.
But wait, you say breathlessly, isn't there any way to differentiate among these many alternatives? No 10 se, puede ser, a 10 me-
;or, tal vez, iquien sabe ... ?But seriously, you should be careful about
overusing quizQS and neglecting a 10 me;or, which in many countries
is the common conversational way of saying "maybe." iVienes esta
noche? A 10 me;or. = "Are you coming tonight?" "Maybe." Tal vez is
also extremely common and all-purpose. QuizQs, on the other hand,
somehow sounds loftier and more grandiose. Puede ser que, puede
que, a 10 me;or, tal vez, and quizQS can all be used to start sentences.
158 BREAKING OUT OF BEGINNER'S SPANISH
All but ala mejor are usually followed by the subjunctive to highlight
the uncertainty being expressed. Tal vez este enojado =ItMaybe he's
mad. 1t Ala mejor no quiere = ItMaybe he doesn't want to. 1t On the
streets, finally, it is increasingly common to hear chance for Itmaybe, It
but don't tell your teacher I told you. iVienes esta nochet Chance. The
words you want to avoid are probab1emente and posib1emente, especially the former. Both are perfectly legitimate Spanish words, but
they're not as common as their English cognates. Were they as common, Capulin would have thrown them in for good measure.

\section{\emph{motive}}

%MOTIVE
Life being what it is, there will come times when you slip up,
trip up, or flub up. It can't be helped. You're walking down the street
in, say, Santiago, happy as a lark, when your hand knocks into the ice
cream cone of a six-year-old girl and sends the cone scattering on the
pavement. The passersby eye you with suspicion. The six-year-old eyes
you with disbelief, then with unbridled hatred, and then starts to bawl.
You look around at the crowd assembling on all sides. Think quick:
what can you say to them?
In English, you might breezily announce ItSorry, I didn't mean
to lt or ItIt was an accident. 1t You might even add, ItIt wasn't on purpose. 1t In Spanish, your simplest, most universal excuse-and one you
should have at the tip of your tongue at all times-is Fue sin querer
(literally, ItIt was without wantinglt). Sin querer can also be added on
to any verb to mean Itaccidentallylt or Itnot on purpose. It La patee sin
querer = ItI kicked it by accident. It Sin querer barre tu nombre de 1a
1ista = ItI accidentally erased your name from the list. 1t To vary your
vocabulary (if you flub up a lot, for instance), you can use sin pensar,
par distracci6n, or invo1untariamente. Of course, if you can pronounce
that last one, you probably don't have to worry about flubs.
When you want to say something was Iton purpose,lt a prop6-
sito (or in some places de prop6sito) works well. But don't overlook
can querer, the opposite of sin querer and every bit as common and
as handy. La hice can querer = ItI did it on purpose. It Adrede, which
means the same thing, is less common, but you will hear it. Save deliberadamente for effect.
Be careful with accidente and its derivatives. Although both
par accidente and accidenta1mente are used widely, they sound a bit
forced, and neither gets the heavy work assigned to its English cognate.
Fue un accidente will get you away from the maddened crowd and the
SAY IT RIGHT 159
six-year-old. But Fue sin querer will do the job more smoothly, without overdramatizing the incident.

\section{\emph{obligation}}

%OBLIGATION
Here's one you simply must learn. You have to. Really, you
should. You'd better, anyhow. That is, you ought to.
The difference in degrees of obligation are a subtle issue that
you will master only after many years of listening to Spanish being
spoken. As in English, there are many ways to convey the notion, but
there is no simple way to explain which is the correct choice for any
given sentence.
Perhaps the most useful way to classify the Itmustsll is by the
strength of the obligation they imply. Intuitively, you can sense the
different strengths in English: IIYou must go" is greater than IIYou have
to go/' which is stronger than IIYou'd better go,o which is a tad more
forceful than ItYou ought to go/' which is up a notch from IIYou should
go. II In Spanish, a parallel ranking, in order of decreasing obligation,
might go something like this: Tienes que ir, Mas te vale ir, Has de ir,
Hay que ir, Mejor vas, Debes/deberias ir. Of these, you can safely discard Has de ir, which has a scolding schoolmarmish quality about it.
Hay que plus a verb, meanwhile, is very common but is limited to impersonal situations. You cannot say Hay que ir tu, for instance; you
would have to say Tienes que if tu. Mejor vas and Mas te vale ir are
both good translations of IIYou'd better go/' with the second implying
IIor else . . . .II
Deber with an infinitive is IIshould, 1I and whether you use the
present indicative (debes), the conditional (deberias), or even the past
subjunctive (debieras) seems to be largely a matter of personal preference. For all intents and purposes, Debes if manana, Deberias ir manana, and Debieras ir manana are equivalents.
Adding a de to deber creates another construction that is
hard to find a consistent English equivalent for. It is used when the
speaker has some sign or evidence that something "is supposed toll
be or IIshouldll be. An example will help here. No debe de haber llegada means ItHe (or she) must not have arrived yetlt and would be used
when the speaker is looking around an untouched room or notices that
all the lights are turned off, for instance. That said, many Spanishspeakers will use No debe haber llegado, without the de, to say the
same thing, and in some countries debe de seems to replace a simple
debe in many cases. Note though that by changing the verb to the
past, you change the meaning significantly. No debi6 haber llegado
160 BREAKING OUT OF BEGINNER'S SPANISH
would be saying "He shouldn't have arrived"-that is, the person did
in fact arrive but against the speaker's better judgment.
For events in the past, stick to a past form of deber with a past
participle for most constructions. Debias (or debiste) haber ido =
"You should have gone." If you instead use the preterit plus the infinitive-debiste ir-you are stressing more the obligation tllan the
missed opportunity. Debias haber ido a 1a fiesta sugges-ts you missed a
good time at the partYj Debiste haber ido a 1a fiesta suggests the same,
only a fraction stronger, and hinting maybe at "I told you SO"j Debiste
ir a 1a fiesta means it was your duty to go-the party was in your
honor, for instance. The fourth combination-imperfect plus infinitive, or debias ir-is not used, so forget about it. Perhaps the best construction for past obligations-that is, your "should haves"-is also
the simplest: hubiera plus the past participle. (Te) hubieras ido = You
should have gone. Hubieramos llamado = We should have called.
A final note on words of obligation deals with the word ob1igar itself. In Spanish, this word is much more widely used than "to
obligate" is in English. In fact, in many cases where in English we use
"to make" or "to force/' in Spanish you should be thinking obligar. La
po1icia nos oblig6 a sa1ir = "The police forced us to leave." E1 diablo
me oblig6 a hacer10 = "The devil made me do it." One way to remember to use obligar is to remind yourself that forzar is relatively rarely
used. It's a perfectly legitimate word, of course, but until you're accustomed to using obligar, you're better off forgetting that forzar exists.

\section{\emph{usually}}

%USUALLY
The basis of the confusion here is the word usua1mente,
which does exist but which you should treat as if it didn't. Dictionaries tend to give it as the translation for "usually;" but in actual practice you could go a month or a year without hearing a native Spanish
speaker use it. It is easy to remember, and technically it's accurate, but
it sounds stiff and strange. You decide.
If you decide to avoid usua1mente (smart choice!), you will be
left with the slight problem of substitutes. Never fearj they abound.
Genera1mente is safe, but a mouthfulj norma1mente is easier and just
as good. Por 10 general is also a good choice. A verbal construction that
works very well involves the irregular verb soler. Sue1e pasar a las seis
= "It usually comes by at six."
While we're discussing the frequency of things, note that if
you're still translating from the English, you'll find your brain feeding
you lines like "most of the time" and expecting you to translate them.
This leads to hideous constructions like 1a mayoria del tiempo, as in
SAY IT RIGHT 161
La mayoria del tiempo s610 bebo agua ("Most of the time I just drink
water"). In Spanish, this sounds as if you have made a scientific study
rather than an off-the-cuff remark. In general, forget about translating
"most of the time" and all other "most of" expressions-"most of the
people," "most of my life," "most of the class"-and look for ways
around them. The use of casi ("almost") often provides a way out: casi
siempre = "most of the time"; casi todos = "most of the people";
casi toda 1a c1ase = "most of the class."

\section{\emph{waste}}

%WASTE
The tendency here, once again, is to translate our English
thoughts too literally into Spanish. The proper word for "to waste" in
Spanish is desperdiciar-a real mouthful that, once you learn it, you
are justified in wanting to use as often as possible. Unfortunately, for
most expressions requiring "waste" in English, desperdiciar is not the
word. Generally speaking, simple old perder ("to lose") is. Thus "to
waste time" is perder e1 tiempo and "to waste an opportunity" is perder una oportunidad. The use of perder carries over to the nouns: una
perdida de tiempo = "a waste of time." Ma1gastar can also be "to
waste/' but perhaps translates better as "to misspend." Some dictionaries list echar a perder as a translation for "to waste," but correctly
this phrase means "to ruin," as in Echaste a perder e1 fuego ("You ruined the game"). Its reflexive form, echarse a perder, means "to go to
waste" or "to spoil," generally in reference to perishable foodstuffs.
With people, it also means "to spoil." Esttis echando a perder a1 nino
= "You're spoiling the boy."

