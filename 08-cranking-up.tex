\chapter{Cranking Up Your Spanish}

Not long ago, I owned a somewhat antiquated motor vehicle
that in fits and starts, quite literally, became dear to me. It wasn't
much to look at---antique buffs generally looked away instead---but
it more often than not got me where I was going. The car's main flaw,
and what finally led me to get rid of it, was its temperament on chilly
mornings. Instead of responding to the turn of the key with a loud
growl and churning pistons, it tended to respond with a weak series of
hiccups and the grinding of metal. Once started, it sputtered, stuttered,
gurgled, and spit. Only after a few minutes of this litany did it begin to
growl a little roughly, then purr hesitantly, then hum.

Beginning Spanish speakers remind me of that car. Once
warmed up, they chatter along merrily, constructing elaborate clauses
and communicating their meaning. But getting warmed up is another
story. Frequent false starts, ``ums," hiccups, and the grinding of gray
matter punctuate their speech, and English interjections pop in and
out. It is obvious that they are still thinking in English and trying to
translate, rather than thinking in Spanish.

What beginning students often lack are the appropriate words
and phrases to start a sentence smoothly. As in English, these words
don't always contribute a lot to the meaning of the sentence, but they
do give the brain a chance to warm up and kick into gear. Skipping
over them can make speech sound stilted and abrupt-sometimes
even impolite or hasty. Even when you are past the beginner stage, you
should pay attention to your ``sentence starters." Learn a few well, get
accustomed to using them, and your Spanish will sound more natural,
flow more easily, and hum along to its desired destination.

\section{\emph{bueno}}

%BUENO
Bueno means nothing more than ``well," but it is nearly indispensable to good spoken Spanish. It simply prefaces comments and tends
to tone down their hastiness or prepare the listener for a message. The
difference between \emph{Ya me voy} and \emph{Bueno, ya me voy} is subtle but noticeable. The first is ``I'm going now" and comes across as a sort of dare
to anyone who might try to stop you, while the second suggests ``Well,
I think I'll be going now" and sounds more laid-back and more polite,

\emph{Bueno} often insinuates a ``stop" into an ongoing conversation
or activity. You might sit down to chat with friends for five or ten minutes and then simply say \emph{bueno} to announce your imminent departure. Or you could sit down to talk business with colleagues and, after
a few minutes of small talk and niceties, signal the beginning of the
business talk with a \emph{bueno}. If the person speaking before you at a conference decides to launch a vitriolic attack against everything the audience holds dear, you might begin your presentation with a sighed
\emph{bueno}; here it would convey ``anyhow" or ``as we were saying" or
even ``whew!" Usually it's much simpler: a pause while you get your
thoughts together, a signal that you have the floor, an acceptance of
what has gone before but a statement that you intend to begin something different.

\section{\emph{pues}}

%PUES
Similar to buena but simpler still. \emph{Pues} means ``well, then"
but often replaces an English ``ummm" or ``let's see" It means, in
short, that you are thinking about your response and would like your
listener to hold on for a few seconds while you try to bring order to the
mild chaos reigning in your head. Often, \emph{pues} is accompanied by eyes
rolling about, looking up, or darting off to some distant object and
other telltale and foolproof signs of intense mental activity. ``Wanna
go to a movie?" \emph{Pues}\ldots{} (``Will it mean staying out late? Can I find a
baby-sitter? What good movies are around? Shouldn't I be working on
my taxes? Can I afford a movie these days?")\ldots{} \emph{sí, vámonos}.

You'll sometimes hear \emph{pues} reduced to a shortened, or just
more mumbled, form: \emph{pos}, for instance, or \emph{pus} or \emph{pos'n}. In parts of
Spain you'll hear an emphatic \emph{¡pue!}; in parts of Mexico a barely audible \emph{pss}. When a foreigner talks this way, it can sound very natural
or very stupid, depending on the stigma (upper-class snob, lower-class
slob, rural hick, urban gang, etc.) that accompanies a form in each
given locale. Listen to how people around you speak, then contract
your \emph{pues} if you're so inclined.

\section{\emph{entonces}}

%ENTONCES
It means ``then," but which ``then"? Usually, \emph{entonces} crops
up in storytelling situations, as when you are relating what happened
in chronological order. \emph{Entonces yo llamé a la policía}. \emph{Entonces Juan
escondió la tu-ya-sabes-que}. \emph{Entonces llegó la senora de los rifles}\ldots{}
But you will also run across it in present-time situations where it
translates more like ``so" or ``so then." \emph{Entonces, ¿qué hacemos?} = ``So
what do we do?" \emph{Entonces, yo voy par la Carretera 1} = ``So then (what
you're saying is that) I take Highway 1." \emph{¿Entonces no van a querer
pizza?} = ``So you're not going to want any pizza?" \emph{¿Entonces?} and \emph{¿Entonces qué?} as questions mean ``And next?" or ``Then what?" Both are
very common and, when reduced in a slangy context, can come out
sounding like \emph{¿'tons qué?} (``So what'll it be?" or ``Now what?").
You
may even hear \emph{¿'tons?} by itself, but remember that these lazy shortcuts
have their enemies among the more refined speakers of the language.

\section{\emph{luego}}

%LUEGO
\emph{Luego} is a lot like \emph{entonces} in that it means ``then" and is interchangeable with \emph{entonces} in many cases. \emph{Luego} generally tends
to be reserved more for time constructions and often works better as
``later" than as ``then." A subtle distinction: \emph{entonces} implies ``and
then (as a result)," whereas \emph{luego} is often simply ``and then (in sequence)." \emph{Primero me voy a bañar, luego voy a la tienda, y luego a tu
casa} = ``First I'm going to take a bath, then go to the store, and then
to your house." It works as ``later" when you're putting something off
into an indistinct future. The most familiar example of this is \emph{Hasta
luego} for ``See you later." You can also say \emph{Luego te hablo} (``I'll call
you later"), \emph{Termínalo luego} (``Finish it later"), \emph{Luego voy} (``I'll go
later"), \emph{Luego te digo} (``I'll tell you later"), and so on. Note that regionally \emph{luego luego} (paradoxically) means ``right away" or ``immediately,"
and not ``later, later."

\section{\emph{a propósito}}

%A PROPÓSITO
``By the way." As in English, this usually announces a change
of subject or indicates that the speaker has been reminded of something else that needs relating. \emph{Son las nueve. A propósito, ¿no ibas a
llamar a Juan?} = ``It's nine o'clock. By the way, weren't you going to
call Juan?"

\section{\emph{por cierto}}

%POR CIERTO
This also means ``by the way." If there is a difference, this
phrase is better for changing the subject when no apparent cause for
it exists. In other words, if someone is talking about the philosophical
implications of modern cinema, and you're hungry and want to talk
about sandwiches, you could say \emph{Por cierto, ¿nadie tiene hambre?}
You'll still get a funny look, but you'll be more correct than if you use
\emph{a propósito,} which denotes ``in reference to that."

\section{\emph{es que}}

%ES QUE
This is an extremely useful sentence starter, and one you
should learn and prepare to embellish. You won't appreciate the importance of \emph{es que} unless you stop and think of all of the utterly useless
words you tack on to the beginning of your conversational speech in
English. Rarely, for instance, do we say ``He doesn't want to go to the
store." We say something like, ``No, well, as it happens, he doesn't
want to go to the store." Okay, I'm exaggerating, but pay close attention to how you speak and you'll see I'm not exaggerating much. Are
you hungry? ``No, it's just that I've got this craving for a Milky Way."
Did you sleep well? ``No, well, sort of, but the thing was, there were
so damn many dogs howling that I kept waking up." It may not read
pretty, but this is the way we speak much of the time. In Spanish you
can add useless words and phrases to your speech, too, and sound more
fluent in the process. The best all-purpose add-on is \emph{es que}, which can
be used in all of the comments used above. \emph{No, es que no quiere ir a
la tienda. No, es que se me antoja un Milky Way. No, bueno, mas o
menos sí, pero es que estaban ladrando tantos malditos perros que
me seguía despertando}.

In these examples, \emph{es que} introduces a negation or qualification of a preceding statement. It doesn't always, but it does require a preceding statement of some sort to build from. Once you get the hang of \emph{es
que}, you will find that just about any preceding statement at all will do:

\bsk

\inda ``Why is Juan unhappy?"

\indu \emph{Es que está enamorado de María}.

\inda ``Why doesn't he tell her?"

\indu \emph{Es que ella está hablando con ese tipo}.

\inda ``And when she's finished?"

\indu \emph{Es que Juan tiene miedo}.

\inda ``Of what?"

\indu \emph{Es que ese tipo está muy grandote}.

\inda ``So?"

\indu \emph{Es que es el esposo de María}.

\inda ``Ah. So he'd better forget about Maria\ldots{}"

\indu \emph{Es que no puede}.

\inda ``\ldots{}and find someone else."

\indu \emph{Es que no le gustan las otras}.

\inda ``So he's got a problem."

\indu \emph{Es que está loco}.

\inda ``And unhappy."

\indu \emph{Es que está enamorado de Maria}\ldots{}

\bsk

And so on.

\section{\emph{lo que pasa es que}}

%LO QUE PASA ES QUE
Just like \emph{es que}, only more so. This is a very typical, very
natural, very unnecessary phrase to add on at the beginning of your
thought to make yourself sound more fluent. It can be used in the
same instances that \emph{es que} can be used-which is to say practically
anytime. Why is Juan unhappy? \emph{Lo que pasa es qué está
enamorado}\ldots{} and so on.

\section{\emph{la verdad es que}}

%LA VERDAD ES QUE
This means something like ``the truth of the matter is," but
you'll hear it---and need to use it---more often than that. In fact, this is
often a good translation for ``actually," which many English speakers
use to start a sentence and which, as beginning Spanish speakers, they
tend to translate as \emph{actualmente}. Sadly, \emph{actualmente} doesn't mean
``actually" but ``at present" or ``currently." So you're left with \emph{la verdad es que} or some other substitute. ``Everything okay?" \emph{Bueno, la
verdad es que la sopa es un poco salada} (``Well, actually, the soup is a
little salty"). ``Can I drop by later tonight?" \emph{Pues, la verdad es que pensaba ir al cine} (``Well, actually, I was thinking of going to a movie").
Note that \emph{la verdad es que}, like ``actually," is frequently used as a polite way to break some bad news or indicate disagreement.

\section{\emph{resulta (que)}}

%RESULTA (QUE)
Here's the word you've been wanting for ``it turns out (that)."
\emph{Fuimos al cine, pero resulta que cierran los viernes} = ``We went to
the movies, but it turns out that they close on Fridays," \emph{Creía que
Juan era un sinvergüenza, pero resulta que es muy amable} = ``I
thought Juan was a jerk, but it turns out he's a nice guy." Often it is
used to start off a sentence, especially in response to questions and
quizzical expressions. You walk in, empty-handed, after going to the
store. Your mother looks at you strangely. You say, prompted or unprompted and a little peeved, \emph{Resulta que ya no aceptan efectivo en
las tiendas} (``It turns out that they don't accept cash anymore in the
stores"). \emph{Resultar que} can also be used for ``to work out" in the sense
of a mathematical result. \emph{Resulta que nos debes dos mil pesos} = ``It
works out that you owe us 2,000 pesos."

\emph{Resulta} without a \emph{que} clause and with an adjective or a noun,
and sometimes with \emph{ser} thrown in for good measure, is also a common
construction. Usually, it still translates well as ``to turn out." \emph{Resultó
(ser) un desastre su fiesta} = ``His party turned out to be a disaster."
\emph{Resultó (ser) agradable la visita} = ``The visit turned out to be quite
pleasant."

\section{\emph{que}}

%QUE
You know of course that \emph{que} is a key element of speaking
Spanish, and you even know where and when to use it---most of the
time. But if you are like most native English speakers, you will have
to train yourself to use \emph{que} when answering questions that start with
\emph{¿que?} ``That" is not always used this way in English, though sometimes it is. Say your Spanish professor calls you at home, and after the
call your brother asks, \emph{¿Que pasó?} (``What's up?"). \emph{Que mañana va a
haber un examen} (``We're having an exam tomorrow"), you would answer. Or say the terrorists have taken hostages and sent a list of demands. Trying to read over the police chief's shoulder, you ask, \emph{¿Qué
quieren?} He would answer, \emph{Que liberemos a cien prisioneros} (``For us
to free a hundred prisoners").

Another case in which ``that" is rarely used in English but que
almost always is in Spanish is when you repeat something:

\bsk

\indu \emph{Juan no está en su casa}.

\indu \emph{¿Qué dices?}

\indu \emph{Que Juan no está en su casa}.

\bsk

In some cases, as when repeating a direct command, you'll have to
change from the imperative to the subjunctive (indirect command) the
second time around. This extra hurdle, in fact, may explain why many
foreigners don't take the trouble to learn this use of \emph{que}:

\bsk

\indu \emph{Vete a tu casa}. = ``Go home."

\indu \emph{¿Qué dices?} = ``What did you say?"

\indu \emph{Que te vayas a tu casa}. = ``I said for you to go home."

\bsk

Note that the police chief also used the subjunctive as he transformed
a direct command into an indirect command above.

\section{\emph{así que}}

%ASÍ QUE
Generally, this is the phrase you need to translate ``so" at the
start of a sentence. ``So you wanna be a rock `n' roll star?" would be
expressed as \emph{¿Así que quieres ser una estrella de rock?} ``So you're
really leaving me?" would be \emph{¿Así que de verdad me vas a dejar?} And
so on. Note that \emph{así que} does not, however, mean ``So what?" For that,
use \emph{¿Y qué?}

\section{\emph{fíjate (que)}}

%FÍJATE (QUE)
Again, not a phrase that will boost the intellectual content of
your comment, but one that will help you sound more fluent. It means
something like ``Look, \ldots{}" and is used to call someone's attention to
something. For some reason, I associate it with reading a newspaper
and passing on bits of news to someone sitting nearby. It's used more
than that, of course, but that's a good example of how it should be
used. \emph{Fíjate que van a cerrar la autopista mañana} = ``Look, they're
going to close the freeway tomorrow." \emph{Fíjate que Robert Redford usa
los mismos zapatos que tú} = ``How about that! Robert Redford wears
the same shoes as you." Sometimes it introduces a note of skepticism
into what you are relating. \emph{Fíjate que el presidente dice que ya no hay
crisis económica} = ``Whaddya know? The president says there's no
more economic crisis."

\emph{Fíjate que} plus the subjunctive means ``make sure" and can be
used instead of the barbarism \emph{checar} to say ``to check." \emph{Fíjate que esté
bien cerrada la puerta} = ``Make sure the door's closed all the way."
\emph{Fíjate que no tenga gusanos antes de comerlo} = ``Check to make sure
it doesn't have any worms before you eat it."

\emph{Fíjate} without the \emph{que} is simply ``watch out" or ``pay attention." Parents tend to use this phrase to scold children who mindlessly
wander into puddles, for instance. \emph{¡Fíjate!} or \emph{¡Fíjate donde caminas!}
They also might use it to get a child to concentrate on an explanation
of a math problem, say. \emph{Fíjate, la dos pasa para allá} = ``Look, the two
goes over there." Finally, if you want to be as rude as the klutz who
just bumped into you on the sidewalk, you can say \emph{¡Fíjate, imbécil!}

\section{\emph{mira}}

%MIRA
Meaning ``Look!" or ``Look here/' this imperative can be used
literally (\emph{¡Mira! ¡No trae ropa!}) or figuratively (\emph{Mira, yo no quiero
problemas}). In the figurative sense, it almost always is used to set the
record straight. \emph{Mira, yo nunca dije que no pudieras irte} = ``Look,
I never said you couldn't go," If you're keen on sounding macho, it's
also a good way to lead off a verbal assault, much like ``Look here, \ldots{}"
\emph{Mira, o te largas o llamo a la policía} = ``Look here, either you get
lost or I call the police."

\section{\emph{haz de cuenta que}}

%HAZ DE CUENTA QUE
This phrase is mostly to be found in the world of kids, but
grown-ups can use it, too. The best translation among kids is ``Make
believe\ldots{} ," as in ``Make believe I'm Spiderman and you're Godzilla."
Since we grown-ups don't say things like that, in our mouths \emph{haz de
cuenta} could translate as ``Imagine\ldots{}" \emph{Haz de cuenta que sales del
baño y hay un alacrán en tu toalla}, = ``Imagine that you get out of the
bath and there's a scorpion on your towel" \emph{Haz de cuenta que eres el
presidente, ¿qué harías?} = ``Say you were the president. What would
you do?" It can also be used to cover a lot of uses of ``to pretend,"
which pretender is notoriously unfit to do. ``Pretend I'm not here" =
\emph{Haz de cuenta que no estoy}. The formal form of this phrase, for the
record, is \emph{haga de cuenta}, while in the plural it is \emph{hagan de cuenta}.
Note also that sometimes it is used reflexively: \emph{hazte de cuenta, hágase de cuenta, háganse de cuenta}.

\section{\emph{ni modo que}}

%NI MODO QUE
This expression isn't used everywhere in the Spanish-speaking
world, but in Mexico, for instance, it's so common that you'll wonder
how the other countries survive without it. It's a very handy expression, and one that can be considered ``advanced" for beginners---especially since it calls for the subjunctive to follow it. But if you're feeling
ready to flex some Spanish muscles, give it a go. \emph{Ni modo que}, more or
less literally, means ``no way that." I say this reluctantly, because \emph{ni
modo} definitely does not mean ``No way!" as in ``No way, Jose" (use
\emph{para nada} for that), But \emph{ni modo que} as a sentence starter can translate as ``No way that\ldots{}" Better translations might be ``Like hell\ldots{}"
or ``You can't expect\ldots{}"

An example will help here. On your tax return you claim that
you made only two dollars the previous year, despite working full time
as an investment banker. A friend questions the ethics of your claim
and you say, \emph{¡Ni modo que 1es diga la verdad!} (``No way I'm going to
tell them the truth!"). A few months later you get a call from an IRS
agent who says she found some irregularities in your return and wants
to talk to you. You grudgingly make an appointment. \emph{Ni modo que me
vaya a México} (``Well, I can't just run off to Mexico"), you shrug. The
agent chats with you, then a prosecutor chats with you, and then all of
a sudden a judge is chatting with you. \emph{Ni modo que te dejemos libre}
(``Well, we can't just let you go free"), he says. You make bail and decide to visit a travel agency, where you buy a one-way ticket to Cancun. \emph{Ni modo que vaya a prisión} (``Like hell I'm going to jail"), you
say. Months later, on the beach in Cancun, someone asks you why you
came to Mexico. \emph{Pues ni modo que me quedara ahí, con tanto crimen
que hay} (``Well, you couldn't expect me to stay there, with all that
crime"), you respond.

\section{\emph{menos mal que}}

%MENOS MAL QUE
Here's an extremely useful sentence starter that you should
learn and use. The best translation is ``good thing" or ``it's just as
well," and you'll find yourself needing it more and more as you become
comfortable with it. \emph{Menos mal que trajiste paraguas, porque va a
llover} = ``Good thing you brought an umbrella, because it's going to
rain." \emph{Menos mal que no fuiste, estuvo muy aburrida} = ``Just as well
you didn't go, it was very boring." Often it is used by itself, just like
the English comments ``Good thing" or ``Just as well." \emph{Juan no pudo
venir. Menos mal. Me cae mal}. = ``Juan couldn't come." ``Just as well.
I don't like him." \emph{Este tren no para hasta San José. Menos mal}.
``This train doesn't stop until San Jose." ``Good,"

\section{\emph{lo bueno, lo malo, lo increíble, lo peor}, etc.}

%LO BUENO, LO MALO, LO INCREÍBLE, LO PEOR, ETC.
In English, it is common to begin a sentence by referring to an
unspecified ``thing." It may not be glamorous, but we do it. The good
thing is that in Spanish there is an equivalent construction; the bad
thing is that it is done differently; the only thing to remember is the
word \emph{lo}.

La plus an adjective is the formula in Spanish for ``the good
thing," ``the bad thing," ``the only thing," and similar constructions:
\emph{Lo bueno, lo malo, lo único}, etc. This formula saves you from ungainly constructions such as \emph{la buena cosa} or \emph{la cosa mala}, which
aren't natural to Spanish. Once you get the hang of this formula, I
promise that you will wonder how you spoke Spanish so long without
it. The usual formula is \emph{lo} + adjective + \emph{es que}. Here are some of the
common adjectives to make use of this formula:

\bsk

\indu \emph{lo bueno} = ``the good thing"

\indu \emph{lo malo} = ``the bad thing"

\indu \emph{lo único} = ``the only thing"

\indu \emph{lo difícil} = ``the hard thing"

\indu \emph{lo peor} = ``the worst thing"

\indu \emph{lo mejor} = ``the best thing"

\indu \emph{lo raro, lo extraño} = ``the strange thing"

\indu \emph{lo chistoso} = ``the funny thing"

\indu \emph{lo increíble} = ``the amazing thing"

\indu \emph{lo (más) absurdo} = ``the crazy thing"

\bsk

The formula \emph{es} + \emph{lo} + adjective is also a handy one for quick and
fluent comebacks, of the sort that are dealt with in the next chapter.
For example, \emph{Es lo raro} = ``That's the strange thing (about what has
just happened)."

\section{\emph{a ver}}

%AVER
Either by itself or followed by a clause, \emph{a ver is} a useful sentence starter that translates well as ``Let's see\ldots{}" By itself, it is used
almost as an interjection. ``My computer is acting up." \emph{A ver} (``Let
me have a look"). ``I can turn water into wine." \emph{A ver} (``Let's see you
do it"). ``Do you have change for a dollar?" \emph{A ver} (``Hold on, let me
check"). The clauses that usually follow \emph{A ver} begin with \emph{qué} or \emph{si}
and translate as ``Let's see if" or ``Let's see what." Often it provides an
easy, painless translation for ``to check," similar to \emph{fíjate} (see above)
but usually implying that the speaker will help with the checking. \emph{A
ver si la puerta está bien cerrada} = ``Let's check and see if the door's
closed." \emph{A ver si ya regresaron} = ``Let's see if they're back yet." \emph{A
ver qué hacen los niños} = ``Let's see (check on) what the kids are doing." When writing \emph{a ver}, take care not to confuse it with its homonym
\emph{haber}.

\section{\emph{con razón}}

%CON RAZÓN
This is an excellent sentence starter that can be used, like
\emph{a ver}, either by itself or to introduce a clause. In both cases, its best
translation is ``No wonder!" or ``Little wonder." ``As a child, John was
attacked by a Great Dane." \emph{Con razón les tiene miedo ahora} (``No
wonder he's afraid of them now"). ``My car broke down." \emph{Con razón
llegaste caminando} (``No wonder you arrived on foot"). ``I ate three
chili dogs and now I'm feeling a little sick" \emph{Pues con razón} (``Little
wonder" or ``Well, whaddya expect?"). This notion can also be rendered
with \emph{No es para menos}.

\section{\emph{por eso}}

%POR ESO
This phrase means ``that's why" and frequently comes in
handy, by itself or with a clause, to draw a connection that may not be
immediately obvious. ``I hear you're going to marry my daughter." \emph{Sí
señor, por eso quiero hablar con usted} (``Yes sir, that's why I want to
talk with you"). ``There's a lot of crime in this neighborhood." \emph{Por eso
tenemos triple candado en la puerta} (``That's why we have a triple
lock on the door"). ``You want to go outside? But it's pouring rain!" \emph{Por
eso. Quiero estrenar mi nuevo paraguas} (``For that very reason. I want
to try out my new umbrella").

\section{\emph{en fin}}

%EN FIN
A close fit for ``So anyhow, \ldots{}" and very useful for leading off
a summation or a conclusion or for steering a conversation back to its
original point. \emph{En fin, no puedo visitar ahora} = ``So anyhow, I can't
visit now." The phrase is usually used as a sentence starter after you've
been talking for a while. For instance, after recounting all the different
plates you sampled at the restaurant, you might finish your description
with \emph{En fin, comí bien} (``So anyhow, I ate well"). If you use it after
someone else has been talking awhile, you're saying, in effect, ``Yes,
very interesting, but back to the subject at hand\ldots{}"

\section{\emph{total}}

%TOTAL
\emph{Total} can also be used for ``So anyhow, \ldots{}" and can be paired
with \emph{que} to lead off a clause. \emph{Total que nadie fue a su fiesta} = ``So (as
it turned out) no one went to his party." Or, slangily, \emph{Total que fuimos
y no había nadie} = ``So anyhow we went and there was no one there."
Both \emph{total} and \emph{en fin} sound more natural than the standard translations that dictionaries give for ``anyhow" and ``anyway": \emph{de todos modos, de todas maneras, de cualquier modo, en todo caso}---at least at
the beginning of sentences. These translations are best saved for cases
where ``anyway" or ``anyhow" is used at the end of the sentence:
``He stole your car." \emph{Pero lo quiero de todos modos} (``But I love him
anyway").

\section{\emph{ya} and \emph{todavía}}

%YA AND TODAVÍA
Beginning students often have trouble distinguishing these
two invaluable sentence starters. As a rule, \emph{ya} means ``already" and \emph{todavía} means ``still" or ``yet." \emph{Ya} in the negative can equal ``no longer"
or ``not\ldots{} anymore."

\bsk

\indu \emph{Ya estoy comiendo}. = ``I'm already eating."

\indu \emph{Todavía estoy comiendo}. = ``I'm still eating."

\indu \emph{Ya quiero comer}. = ``I want to eat already (now)."

\indu \emph{Todavía quiero comer}. = ``I still want to eat."

\indu \emph{Ya no quiero comer}. = ``I don't want to eat anymore," ``I no longer want to eat."

\indu \emph{Todavía no quiero comer}. = ``I still don't want to eat," ``I don't want to eat yet."

\bsk

Let's say you're having a dinner party and the guest of honor is half an
hour late. You go to call his house and return with one of the following
messages:

\bsk

\indu \emph{Ya viene}. = ``He's on his way," ``He's about to leave."

\indu \emph{Todavía viene}. = ``He's still (planning on) coming (though not immediately)."

\indu \emph{Ya no viene}. = ``He's not coming anymore."

\indu \emph{Todavía no viene}. = ``He still hasn't left."

\bsk

To learn \emph{ya} well, keep in mind that it is often used for emphasis with
present-tense verbs. Regionally in American English ``already" is used
in the same way: ``all right already," for instance. \emph{Ya voy} = ``I'm coming already." Sometimes ``now" is the better translation. \emph{Ya nos vamos}
= ``We're going now." \emph{Ya está la comida} = ``The food's ready now."
Sometimes the added emphasis is in the tone. \emph{Ya veras} = ``You'll see."
\emph{Ya} intensifies imperatives as well. \emph{Cómete la fruta} would be ``Eat your
fruit"; \emph{Ya cómete la fruta} would be more like ``Come on, eat your
fruit." Note that in most cases, \emph{ya} indicates a change in the way things
have been up to that point.

\emph{Ya que} is also a common sentence starter that students seem
to shy away from because ``already that" doesn't ring any bells in their
head. It means ``since" or, even better, ``seeing that" or ``now that." \emph{Ya
que terminaste este capítulo, ¿qué vas a hacer?} = ``Now that you've
finished this chapter, what are you going to do?"

