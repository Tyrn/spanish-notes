\chapter{The Twilight Zone}

First, an explanation. Why title this chapter, which is about
the subjunctive mode in Spanish, ``The Twilight Zone?" The answer is
simple and twofold. First, the concept of a hazy, ephemeral Twilight
Zone accurately conveys the spirit of the subjunctive. And, second, if
it were called ``The Subjunctive Mode," no one would read it.

What is it about the subjunctive that inspires such fear and
loathing in students of Spanish? Mostly, it is the task of retraining the
mind to recognize a concept that has no readily obvious equivalent in
English. After all, it's bad enough that Spanish puts different endings
on its verbs to denote mode, tense, and person. But to invent a whole
new mode outright---one that needs endings all its own---is nearly
criminal.

Spanish, of course, did not invent the subjunctive. In fact, the
subjunctive is widely used in English, though not nearly as frequently
as it is in Spanish. Take, for example, a sign hanging in the Sears restrooms in Waco, Texas: ``It is important to us that our restrooms be
clean." A nicer, neater subjunctive was never seen.

But in Spanish it's often hard to get a grasp of why the subjunctive is needed and when. Thus ``The Twilight Zone." For that, essentially, is what the subjunctive is: the Twilight Zone of the verb universe. The subjunctive gets the job of describing ``could-have-beens,"
``might-bes," and ``maybe-never-weres.'~Anything that has happened,
is happening, or may happen on the borders of our consciousness gets
handled by the subjunctive. Without the subjunctive, Garcia Marquez
would read like Hemingway. The subjunctive is more than a verb
mode; it is a complete separate reality.

Although beginning and intermediate students of Spanish find
it difficult to believe, many English speakers who have learned to live
with the Spanish subjunctive will tell you that it can actually be quite
enjoyable. With the mere flick of a verb ending you can cast doubt or
aspersions, relegating a simple occurrence to a different realm of understanding. It is a realm you can flirt with and explore, avoid when
you want (sometimes), and revel in at will. Almost certainly you will
find yourself wanting it more and more as you explore the Spanish-speaking world. Magical realism finds its home here, as does the seeming surrealism of much of daily life in the Spanish-speaking world.

So how do you go about learning it? Basically, there are two
equally important approaches. One approach is to learn the
rules---especially with the Twilight Zone concept in mind. A second
approach is to learn the common cues for the use of the subjunctive.
Practice both, and the subjunctive will appear to you one day---perhaps
even in a dream. From that day forward, it will be your constant companion, your escape hatch into the unreal. Signpost up ahead: Subjunctive Mode!

\section{Indirect commands (shallow twilight)}

This group covers giving orders, asking others to do things,
and engaging in other bossy behavior. Thus in a sentence like ``Tell the
mariachis to go away," the English infinitive ``to go away" must be rendered in the subjunctive in Spanish. Why? Because the action of the
mariachis going away doesn't become a reality until they actually go
away. Until then, it must be considered an entirely suspect notion,
lurking off in the unknown: Will the mariachis go away? Will they
stay? How about if we pay them to go away?

We can lump implicit indirect commands in this category, including ``wishing" and ``hoping" that the mariachis will go away. ``I
hope the mariachis go away" is, after all, nothing more than a cowardly version of ``Tell the mariachis to go away." And, as in the first
example, simply hoping that they will go away is no guarantee that
they will actually do so. They may want to sing ``La Bamba" again.
You have no way of knowing, so you have to rely on the subjunctive. In
passing, note that you won't always have an obvious ``telling" or ``hoping" verb directing the action. Sometimes, the order is implicit or impersonal, and it often begins with que all by itself: \emph{Que se vayan los
mariachis} = ``Have the mariachis go away." Similarly, \emph{ojalá} will often
initiate a wishing construction: \emph{Ojalá que se vayan los mariachis} =
``I sure hope that the mariachis will leave."

\section{The eternal mystery (deep twilight)}

For this category, we must venture even further into the Twilight Zone. This is the realm of doubt, uncertainty, suspicion, and
downright disbelief. For example, \emph{Es posible que los mariachis se
vayan} (``It's possible that the mariachis will go"), \emph{Dudo que se vayan}
(``I doubt that they will go"), \emph{No creo que se vayan jamás} (``I don't
think they are ever going to go"), and so on. In each of these cases, we
return to the elemental problem of the mariachis' departure as a mystery, an eternal uncertainty, an action belonging to a separate realm.

Statements of negation also lurk in this shadowy world. At
first, their presence here confounds us: aren't they statements of fact
and thus perfect candidates for the indicative? On closer examination,
however, we can see why they are here. Negations are declarations of
something that never happened, actions that only exist in somebody's
mind. Here, with a little study, we can see the careful distinction that
turns a harmless indicative statement into an unruly, ethereal subjunctive. \emph{No asalté el banco} = ``I didn't rob the bank" Straightforward and
indicative: I didn't do it. But when the sentence structure forces us to
make the action of bank robbery stand alone, it acquires its true character---that of an action that was not. \emph{Niego que haya robado el banco}
= ``I deny that I robbed the bank." Here ``I robbed the bank" is a nonevent, an untrue claim, a load of nonsense. It simply didn't happen---I swear! And since it didn't happen, it must be exiled to that world
where all the things that never happened---the ``could-have-beens,"
``might-bes," and ``maybe-never-weres"---reside. In short, it must go
to the Twilight Zone.

A similar treatment awaits things you ``don't believe" or
``don't think." Since you don't believe them, you certainly don't have
to consider them real. \emph{No creo que esté aquí} = ``I don't think she's
here." Her presence here is something that for you, the speaker,
doesn't belong in your universe of hard facts. Thus into the subjunctive
it goes.

\section{\emph{Que} cues}

By now, being a sharp reader, you will have noticed that every
use of the subjunctive so far has been preceded by a certain word: \emph{que}.
And you're thinking, ``Hey, maybe I'm on to something." In fact, you
are---sort of. Que is a good cue for using the subjunctive, though not
an entirely reliable one. That is, almost every time the subjunctive
appears, there will be a \emph{que} preceding it. But \emph{que} will not always be
followed by the subjunctive every time it appears. Still, if you pay
attention to when you use \emph{que}, you will be on your way to spotting
opportunities for showing off your subjunctive.

The context surrounding \emph{que} is the deciding factor in whether
the subjunctive should indeed follow. And often this context is little
more than the proper combination of words with \emph{que}. Thus certain
impersonal expressions followed by \emph{que} almost invariably take the
subjunctive:

\bsk

\indu \emph{Es posible (probable, factible, concebible) que}

\indu \emph{Es mejor (conveniente, preferible, oportuno) que}

\indu \emph{Es importante (necesario, preciso, urgente, obligatorio,
forzoso) que}

\bsk

As a matter of fact, only a handful of common adjectives can
be placed between es and que and still produce the indicative. Some of
these exceptions are \emph{claro, obvio, evidente}, and the like, which stress
that a fact is a fact is a fact. \emph{Es obvio que estoy aquí} = ``It's obvious
that I'm here." So obvious a fact certainly has no business in the Twilight Zone.

Many of the adjectives in the \emph{es} + adjective constructions
used above also have verb forms. With our old friend \emph{que} these verbs
also take the subjunctive:

\bsk

\indu \emph{Urge que} (it's urgent that)

\indu \emph{Conviene que} (it suits/appropriate)

\indu \emph{Precisa que} (it states/specifies that)

\indu \emph{Prefiero que} (I'd rather)

\bsk

Along these lines are other impersonal expressions that take
the subjunctive:

\bsk

\indu \emph{Más vale que} ([you/we'd] better)

\indu \emph{Lástima que} (it's a pity/shame/too bad)

\bsk

So do certain ``impersonalized" or reflexive constructions:

\bsk

\indu \emph{Se espera que} (always; it is expected)

\indu \emph{Se cree que} (sometimes; it is believed that)

\bsk

If you change these constructions from impersonal to personal, in
most cases you will still need the subjunctive. Some common verbs
that, when followed by \emph{que}, usually require the subjunctive include
\emph{esperar, sentir, querer, pedir, mandar, dejar}, and \emph{permitir}.

Note that when there is no change in subject, the infinitive
can be substituted for the subjunctive clause, as it is in English. These
are the constructions you won't have any trouble with:

\bsk

\indu \emph{Quiero ir}. = ``I want to go."

\indu \emph{Esperan ganar}. = ``They hope to win."

\bsk

Changing the subject of the second clause will require the subjunctive,
however:

\bsk

\indu \emph{Quiero que vayas}. = ``I want you to go."

\indu \emph{Espero que gane ella}. = ``I hope she wins."

\bsk

With some indirect command verbs, especially \emph{mandar, permitir}, and \emph{dejar}, the imperative can also be rigged together with the
infinitive to avoid the subjunctive altogether:

\bsk

\indu \emph{Manda traer el dinero}. = ``Send for the money to be brought."

\indu \emph{Déjale traer el dinero}. = ``Let him bring the money."

\bsk

A more natural-sounding construction in these and other cases of indirect commands is simply starting your sentence with \emph{que} and following it with the subjunctive, as in the earlier example \emph{Que se vayan los
mariachis}. This equates with the English ``Have\ldots{}," which is one of
the most common indirect command forms in English.

\bsk

\indu \emph{Que traiga el dinero}. = ``Have him bring the money."

\indu \emph{Que venga a las seis}. = ``Have her come at six."

\bsk

\emph{Que} is also a reliable cue for the subjunctive when paired with
other words to form certain conjunctions. Most textbooks will give
you a laundry list of these conjunctions, half of which you will probably never need. Here are the important ones to remember:

\bsk

\indu \emph{para que} = ``so that," ``in order that"

\indu \emph{a menos que} = ``unless"

\indu \emph{a pesar de que} = ``despite," ``even though"

\indu \emph{antes de que} = ``before"

\section{Non-\emph{que} cues}

Another common use of the subjunctive is generally not introduced by que, so you'll have to be alert for it. Instead, it uses cuando,
donde, como, and other adverbs. The best guide in this case is the English translation. When you could substitute ``-ever," as in ``whenever," ``wherever," or ``however," follow the adverb with the subjunctive
in Spanish. If it helps you remember, memorize one of the classic
lines used to challenge someone to a fight in Spanish: \emph{Cuando quieras,
donde quieras, y como quieras} (``Whenever you want, wherever you
want, however you want"). (The expression is reputedly in use as well
as a ``pick up" line, so make sure the person you use it on knows
whether you're a lover or a fighter!) Most likely, though, you will be
called upon to use this subjunctive construction in these more mundane situations:

\bsk

\indu \emph{¿Cuándo quieres ir? Cuando tú quieras}. = ``When do you
want to go?" ``Whenever you want."

\indu \emph{¿Adónde vamos? Donde quieras}. = ``Where are we going?"
``Wherever you want."

\indu \emph{¿Cuándo me vas a dar el dinero? Cuando yo quiera}, = ``When
are you going to give me the money?" ``Whenever I
feel like it."

\section{The subjunctive with \emph{ser}: \emph{sea}}

\emph{Ser} is also commonly used in ``-ever" constructions, and expressions with \emph{sea} are good to slip into your conversational Spanish.
\emph{Cuando sea, como sea, donde sea}, and \emph{quien sea} are equivalents for
``whenever," ``however," ``wherever," and ``whoever" when used alone.
Often more common in English is to use an ``any-" word---``anywhere," ``anyhow," and so forth:

\bsk

\indu \emph{¿Con quién quieres ir al cine? Can quien sea}. = ``Whom
do you want to go to the movies with?" ``With
whomever."

\indu \emph{¿Dónde quieres comer? Donde sea}. = ``Where do you want to
eat?" ``Wherever (anywhere)."

\indu \emph{¿Cómo quieres la carne: con salsa, sin salsa, can papas, sin
papas, término medio, bien cocida? Como sea}. =
``How do you want your meat: with sauce, without
sauce, with potatoes, without potatoes, medium, well
done?" ``Any ol' way will do."

\bsk

Often, the best English translation of expressions using \emph{sea}
would be a slangy expression like ``It's up to you," ``You name it," ``I
don't care," or ``It doesn't matter." All of these can be conveyed by the
Spanish subjunctive.

Although \emph{ser} and \emph{querer} are the two commonest verbs used in
``-ever" expressions, virtually any verb can be used:

\bsk

\indu \emph{¿Cuándo vas a llegar a la fiesta? Cuando pueda}. = ``When are
you going to get to the party?" ``Whenever (as soon as)
I can."

\indu \emph{Yo quiero salir ahora. Bueno, lo que tú digas}. = ``I want to
leave now." ``Okay, whatever you say."

\bsk

Once you get a feel for the \emph{cuando quiera- donde quiera- como quiera complex}, you'll be close to mastering one of the trickiest uses of the subjunctive: a clause containing the adverb plus the
subjunctive to refer to the future. Here you should keep the Twilight
Zone idea in mind. In Spanish, for instance, you would say \emph{Cuando
termine el libro, te llamaré} for ``When I finish the book, I'll call you."
In this case, \emph{cuando} is followed by the subjunctive because it refers to
an event in the future that may never happen. A meteor could strike
the reader one page from the book's end, so the notion of ``when I finish the book" must be considered uncertain.

Only when you are referring to a habitual action should you
use the indicative. In these cases, note that you are not so much referring to the future as to the past. \emph{Cuando termino de leer en las mañanas, voy a la tienda} = ``When I finish reading in the morning, I go
to the store." Here the indicative is safe because presumably you have
done this sort of thing before and thus know that it can happen and has
happened.

\section{The traveler's subjunctive}

A final note on the subjunctive is especially useful for those
traveling in the Spanish-speaking world. A common question format
for lost, bewildered, or just-curious travelers goes as follows: ``Is there
a such-and-such near here that does such-and-such?" For instance, you
might want to ask, ``Is there a store near here that sells wine?" or ``I'm
looking for place where I can leave my luggage." In all cases like these
you must use the subjunctive in Spanish, since the place you are seeking mayor may not exist. Or, put another way, it won't exist until your
question is answered, ``Yes, there is such a place." Until such an answer is given, the place belongs in the never-never world of the Twilight Zone.
\emph{¿Hay una tienda por aquí que venda vino? Busco un lugar
donde pueda dejar mi equipaje}. The same logic applies when you ask
about people. ¿Hay alguien aquí que hable inglés? = ``Is there anyone
here who speaks English?"

When the answer is in the negative, the place remains nonexistent and therefore must still be referred to in the subjunctive. \emph{No hay
una tienda cerca que venda vino} = ``There is no store nearby that
sells wine." This is, after all, but a simple statement of negation, like
the ones we saw above. Ditto for nonexistent people: \emph{¡Na hay nadie en
esta ciudad que me entienda!} = ``There's no one in this city who understands me!"

